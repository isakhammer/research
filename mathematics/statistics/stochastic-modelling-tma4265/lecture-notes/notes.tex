\documentclass{article}



\usepackage[utf8]{inputenc}

\title{Stochastic Modelling}
\author{isakhammer }
\date{2020}

\usepackage{natbib}
\usepackage{graphicx}
\usepackage{amsmath}
\usepackage{amsthm}
\usepackage{amsfonts}
\usepackage{mathtools}
\usepackage{enumerate}
\usepackage{todonotes}


\usepackage{hyperref} 
\hypersetup{
  colorlinks=true, %set true if you want colored links
  linktoc=all,     %set to all if you want both sections and subsections linked
  linkcolor=blue,  %choose some color if you want links to stand out
} 
\hypersetup{linktocpage}


% inscape-figures
\usepackage{import}
\usepackage{pdfpages}
\usepackage{transparent}
\usepackage{xcolor}
\newcommand{\incfig}[2][1]{%
\def\svgwidth{#1\columnwidth}
\import{./figures/}{#2.pdf_tex} } \pdfsuppresswarningpagegroup=1

% Box environment
\usepackage{tcolorbox}
\usepackage{mdframed}
\newmdtheoremenv{definition}{Definition}[section]
\newmdtheoremenv{theorem}{Theorem}[section]
\newmdtheoremenv{lemma}{Lemma}[section]

\theoremstyle{remark}
\newtheorem*{remark}{Remark}
%\newtheorem{example}{Example}


\begin{document}
\maketitle
\tableofcontents
\newpage

\newpage
\section{Lecture 1}%
\label{sec:lecture_1}

\subsection{Practical Information}%
\label{sub:practical_information}

Two projects 
\begin{itemize}
  \item The projects count $20 \%$ and exam $80 \%$.
  \item Must be done with two people.
  \item If you want to do statistics is it worth learning $R$.
\end{itemize}

\textbf{Course Overview} 
\begin{itemize}
  \item Markov chains for discret time and discrete outcome.
    \begin{itemize}
      \item Set of states and discrete time points.
      \item Transition between states
      \item Future depends on the present, but not the past.
    \end{itemize}
  \item Continious time Markoc chains. (continious time and discrete toutcome.
  \item Brownian motion and Gaussian processes (continionus time and continious outcome.)
\end{itemize}


\subsection{Mathematical description}%
\label{sub:mathematical_description}
 \begin{definition}
   A \textbf{stochastic process} $\{ x\left( t \right), t \in T\} $ is a family of random variables, where $T$ is a set of indicies, and $X\left( t \right)$ is a random variable for each value of $t$.
 \end{definition}

\subsection{Recall from Statistics Course}%
\label{sub:recall_from_statistics_course}

A random experiment is perfomed the outcome of the experiment is random.
\begin{itemize}
  \item THe set of possible outcomes is the \textbf{sample space}  $\omega $ 
    \begin{itemize}
      \item An \textbf{event}  $A \subset \omega $  if the outcome is contained in $A$
      \item The \textbf{complement}  of an event $A$ is  $A^{c} = \omega  \setminus A$ 
      \item The \textbf{null event} $\emptyset$ is the empty set $\emptyset = \omega \setminus \omega $ 
    \end{itemize}
\end{itemize}

\subsubsection{Combining Event}%
\label{ssub:combining_event}

Let $A$ and B be events 
\begin{itemize}
  \item The \textbf{union} $A \cup  B$ is the event that at least one of $A$ and $B$ occur.
  \item the \textbf{intersection}  $A \cap B$ is the event that both $A$ and $B$ occur.
\end{itemize}

The events $A_{1}, A_{2}, \ldots$ are called disjoint (or \textbf{mutually exclusive} ) if $A_{i} \cap A_{j} = \emptyset$ for $i \neq j$

\subsubsection{Probability}%
\label{ssub:probability}

$Pr$ is called a probability on $\omega $ if 

\begin{itemize}
  \item Pr $\{ \omega \} = 1  $ 
  \item $0 \le P\left\{ A \right\} \le 1$ for all events $A$ 
  \item For $A_{1}, A_{2} , \ldots$ that are mutually exclusive \[
  P \left\{ \bigcup_{i = 1}^{\infty}A_{i}  \right\} = \sum_{i=1}^{\infty} P \left\{ A_{i} \right\}
  \] 
\end{itemize}
We call $P\left\{ A \right\}$ the probability of $A$.


\subsubsection{Law of total probability}%
\label{ssub:law_of_total_probability}

Let $A_{1}, A_{2}, \ldots$ be a partition of $\omega $ ie 
\begin{itemize}
  \item $\omega  = \bigcup_{i=1}^{\infty} A_{i}$
  \item $A_{1}, A_{2}, A_{3}, \ldots$ are mutually exclusive.
\end{itemize}

Then for any event $B$ \[
  P\left\{ B \right\} = \sum_{i=1}^{\infty} P\left\{ B \cap A_{i} \right\}
\] 

\textbf{This concept is very important.} 

\subsubsection{Independence}%
\label{ssub:independence_2s}
Event $A$ and $B$ are independent of \[
P\left\{ A\cap B \right\} = P\left\{ A \right\}P\left\{ B \right\}
\] 
Events $A_{1}, \ldots, A_{n}$ are independent if for any subset \[
P\left\{ \bigcap_{j=1}^{k} A_{i_j} \right\} = \prod_{j=1}^{k} P \left\{ A_{i_j} \right\} 
\] 

In this case $P\left\{ \bigcap_{i = 1}^{n} A_{1} \right\} =  \prod_{i = 1}^{n} P\left\{ A_{i} \right\} $


\newpage
\subsubsection{Random Variables}%
\label{ssub:random_variables}

\begin{definition}
  A \textbf{random variable}  is a real-vaued function on the sample space. Informally:  A random variable is a real valued variable that takes on its value by chance.
\end{definition}


\begin{tcolorbox}
  \textbf{Example.} 
  \begin{itemize}
    \item Throw two dice. $X = \text{sum of the two dice}$
    \item Throw a coin.  $X$ is $1$ for heads and $X$ is $0$ for tails.
  \end{itemize}
\end{tcolorbox}


\subsubsection{Notation for random variables}%
\label{ssub:notation_for_random_variables}

We use 
\begin{itemize}
  \item upper case letters such at $X$, $Y$ and $Z$  to represent random variables.
  \item lower case letters as $x$, $y$, $z$ to denote the real-valued realized value of a the random variable.
\end{itemize}

Expression such as $\left\{ X \le x \right\}$ denators the event that $X$ assumes a valye less than or earl to the real number x.

\subsubsection{Discrete random variables}%
\label{ssub:discrete_random_variables}

The random variable $X$ is \textbf{discrete}  if it has a finite or countablle number of possible outcomes $x_{1}, x_{2}, \ldots$ \par
\begin{itemize}
  \item The \textbf{probability mass function } $p_{x} \left( x \right) $ is given by \[
  p_{x}\left( x \right) = P \left\{ X = x \right\}
  \] and satisfies \[
  \sum_{i=1}^{\infty} p_{x}\left( x_{i} \right) = 1 \quad  \text{and} \quad  0\le p_{x} \left( x_{i} \right) \le  1 
  \] 
\item The \textbf{cumulative distribution function} (CDF) a of $X$ can be written \[
F_{x}\left( x \right) = P\left\{ X \le x \right\} = \sum_{i: x_{i} \le x}^{} p_{x}\left( x_{i} \right) 
\]  
\end{itemize}

\subsubsection{CFD}%
\label{ssub:cfd} 

The CDF of $X$ may also be called the \textbf{distrobution function}  of $X$ \par 
Let $F_{x}\left( x \right)$ be the CDF of $X$, then 
\begin{itemize}
  \item $F_{x}\left( x \right)$ is monetonaly increasing.
  \item $F_{x}$ is a stepfunction, which is a pieace-wise constant with jumps at $x_{i}.$
  \item $\lim_{x \to \infty} F_{x}\left( x \right) = 1$
  \item $\lim_{x \to - \infty} F_{x}\left( x \right) = 0$
\end{itemize}


\subsubsection{Continious random vairbales}%
\label{ssub:continious_random_vairbales}
 A \textbf{continious} random variables takes value o a continious scale.
 \begin{itemize}
   \item The CDF, $F_{x}\left( x \right) = P \left( X \le x \right)$ is continious.
   \item The \textbf{probability density function} (PDF) $f_{x}\left( x \right) = F_{x}' \left( x \right)$ can be used to calculate probablities \[
   \begin{split}
     Pr \left\{ a < X < b \right\} &=  Pr \left\{ a \le X < b \right\} = Pr\left\{ a < X \le b \right\} \\
     &=  Pr\left\{ a \le X \le b \right\} = \int_{a}^{b}  f_{x}\left( x \right)dx   
   \end{split} 
   \] 
 \end{itemize}


 \subsubsection{Important properties}%
 \label{ssub:important_properties}

 \begin{itemize}
   \item CDF:
     \begin{itemize}
       \item Monotonely increaing
       \item continious
        \item $\lim_{x \to \infty} F_{x} = 1$ and $\lim_{x \to - \infty} F_{x}\left( x \right) = 0$
     \end{itemize}
   \item PDF
     \begin{itemize}
       \item $f_{x}\left( x \right) \ge 0$ for $x \in\mathbb{R} $
       \item $\int_{-\infty}^{\infty} f_{x}\left( x \right)dx = 1$
     \end{itemize}
 \end{itemize}


\subsubsection{Expectation}%
\label{ssub:expectation}

Let $g: \mathbb{R}  \to \mathbb{R} $ be a function and $X$ be a random variable.
\begin{itemize}
  \item If $X$ is discrete, the expected value of $g\left( X \right) $ is \[
  E\left[ g\left( X \right) \right] =  \sum_{x: p_{x}\left( x \right)> 0}^{} g\left( x \right) p_{x}\left( x \right)  
  \] 
\item If $X$ is continous, the expected value of $g\left( X \right) $ is  \[
E\left[ g\left( X \right) \right] = \int_{-\infty}^{\infty} g\left( x \right)f_{x}\left( x \right) dx 
\] 
\end{itemize}

\subsubsection{Variance}%
\label{ssub:variance}

The variance of the random variable $X$ is \[
  Var\left[ X \right] =  E \left[( X - E\left[ X \right])^{2} \right] =  E\left[ X^2 \right] - E\left[ X \right]^2 
\] 
Important properties of expectation and variance.
\begin{itemize}
  \item Expectations is linear \[
  E\left[ aX + bY +c \right] = aE\left[ X \right] + bE\left[ Y \right] + c.
  \] 
\item Variance scales quadratically and is invaraient to the addition of constants \[
Var\left[ aX + b \right] = a^2 Var \left[ X \right] 
\] 
\item fir independent stochastic variables.\[
    Var \left[ X + Y \right] = Var \left[ X \right] + Var\left[ Y \right]
\] 
\end{itemize}

\subsubsection{Joint CDF}%
\label{ssub:joint_cdf}

If $\left( X,Y \right)$ is a pair for random variables, their \textbf{joint comulative distribution function } is given by \[
F_{X,Y} = F\left( x,y \right) =  Pr\left\{ X \le x \cap Y \le y \right\}
\]. 
\subsubsection{Joint distrubution for discrete random variables}%
\label{ssub:joint_distrobution_for_discrete_random_variables}
If $X$ and $Y$  are discrete, the \textbf{joint probability mass function } $ p_{x,y} = Pr\left\{ X = x, Y =y \right\} $. can be used to compute probabilities \[
Pr\left\{ a < X < b, c < Y \le d \right\} =  \sum_{a < x \le b}^{}  \sum_{c < y \le d}^{} p_{X,Y}   \left( x,y \right)
\] 

\subsubsection{Joint distrubution for continous random variables}%
\label{ssub:joint_distrobution_for_continous_random_variables}

If $X$ and $Y$ are continious the \textbf{joint probability density function}  \[
.f_{X,Y} \left( x,y \right) = f\left( x,y \right) = \frac{\partial ^2}{\partial x \partial y } F\left( x,y \right)   
\]  can be used to compute probabilities \[
Pr\left\{ a < X \le b,  \quad  c < Y \le d  \right\} = \int_{a}^{b} \int_{c}^{d} f\left( x,y \right)dxdy    
\] 

\subsubsection{Independence}%
\label{ssub:independence_3}

The random variables $X$ and Y are independent if \[
Pr\left\{ X \le a , Y \le b \right\} =  Pr\left\{ X \le a \right\} \cdot  Pr\left\{ Y \le b \right\}, \quad  \forall a,b \in  \mathbb{R}  
\] 
In terms of CDFs:  $F_{X,Y}(a,b ) =  F_{X}\left( a \right)\cdot F_{Y}\left( b \right) \quad  \forall a,b \in \mathbb{R}  $
\par
Thus we have 
\begin{itemize}
  \item $p_{X,Y} \left( x,y \right) = p_{X}\left( x \right) \cdot  p_{Y}\left( Y \right)$ for discrete random variables
  \item $f_{X,Y}\left( x,y \right) = f_{X}\left( x \right) \cdot  f_{Y}\left( Y \right)$ for continuous random variables.
\end{itemize}






 
 




\newpage
\section{Lecture 2}%
\label{sec:lecture_2}

\begin{definition}
  Let $A$ and $B$ be events.  The conditionally pprobability of $A$ fiven $B$ is defined by \[
  Pr \left\{ A  \mid  B \right\} = \begin{cases}
    \frac{Pr \left\{ A \cap B \right\}}{ Pr\left\{ B \right\} } ,  &  \quad  Pr\left\{ B \right\} > 0,   \\
    \text{Not defined} \quad  Pr\left\{ B \right\} = 0 
  \end{cases}
  \] 
\end{definition}

\todo{ Check if Lecture two is in google cal }

\begin{tcolorbox}
  \textbf{Example.} Throw one die and let $X$ denote the number of eyes. Find $Pr \left \{ X \ge 5  \mid X \ge 3 \right \} $.  \[
    \begin{split}
      Pr \left \{ X \ge 5     \mid  \ge 3 \right \} &= \frac{Pr \left \{ X \ge 5 \right \} }{ Pr \left \{   X \ge 3\right \} }  \\
  &=  \frac{\frac{2}{6}}{ \frac{4}{6}}  = \frac{1}{2} 
    \end{split} 
  \]  
\end{tcolorbox}

\begin{definition}
  \textbf{Conditional EMF} (Conditionally probability mass function PMF).  Assume $X$ and $Y$ are jointly distributed random variables.  The Conditional PMF.  $p_{x \mid y} $ of $X$ given $Y$ given by \[
  p_{X \mid  Y} \left( x  \mid y \right) = \frac{Pr \left \{ X = x , Y = y \right \} }{Pr \left \{ Y = y \right \} }  = \frac{p_{X,Y}\left( x,y \right)}{p_{Y}} ,  \quad p_{y} \left( y \right) > 0  
  \] 
\end{definition}

\begin{remark}
  $\left\{ X = x, Y =x \right\}$ is shorthand for $\left\{ \left( X = x \right) \cap \left( Y = y \right) \right\}$
\end{remark}
\begin{remark}
  \par
  \begin{itemize}
    \item $ p _{X \mid  Y} \left( x  \mid y \right) $ is a pmf for \[
    x  \implies  \sum_{x}^{} p_{X \mid Y} \left( x  \mid  y \right) = 1 \quad \forall  y   
    \] 
  \item $P_{X \mid  Y} \left( x|y \right) $ is not a pmf for \[
  y \implies  \sum_{X \mid Y}^{} \neq 1 \quad \text{In General}  
  \] 
  \end{itemize}
\end{remark}

\begin{tcolorbox}
  \textbf{Example.} Throw die and let \[
  \begin{split}
    X &=  \text{Number of eyes} \\
    Y &= \begin{cases}
      0,  &  \quad  \text{if} \quad  X \ge 2 \\
      1,   &  \text{if}   \quad X \ge 3
    \end{cases}  
  \end{split} 
  \] 
  Find the conditionally \textbf{PMF}  $p _{X  \mid  Y}$

  \par
  \textbf{Solution}. For $y = 0$ \[
  p_{x \mid y} \left( x  \mid y \right) = \begin{cases}
    \frac{1}{2} ,  &  x = 1,2 \\
    0, & x = 3,4,5,6  
  \end{cases} 
  \] 
  For $y = 1$ \[
  p_{x  \mid  y} \left( x  \mid  y \right) = \begin{cases}
    0,  &  x= 1,2 \\
    \frac{1}{4} ,  &  x=3,4,5,6
  \end{cases}
  \] 

\end{tcolorbox}
  \todo[inline]{ Didnt quite understand this example }

  \begin{tcolorbox}
    \begin{itemize}
      \item $\sum_{x}^{}  p_{X \mid Y} \left( X  \mid 0 \right) = 1$
      \item \textbf{Noob mistake}  \[
          p_{X \mid Y} = \left( 1  \mid 0 \right) + p_{X \mid Y } \left( 1 \mid 1 \right) = \frac{1}{2} + 0 \neq 1
      \] 
    \end{itemize}
  \end{tcolorbox}


  \subsection{Joint Distrobution}%
  \label{sub:joint_distrobution}
  
  THe conditional \textbf{PMF}  is essential to us because we can siplify the joint \textbf{PMF}  as \[
    \begin{split}
  p_{X \mid Y}  \left( x,y \right)   & = Pr \left \{ X =x , Y = y \right \}  \\
    &  = Pr \left \{ U = y   \right \}  Pr \left \{ X = x  \mid  Y =y \right \}   \\
    &= p_{Y} \left( y \right) p_{X \mid Y} \left( x  \mid  y \right)
    \end{split} 
  \] 
  \begin{remark}
    if $X $ and $Y$ are independent then is \[
    \begin{split}
      p_{X \mid Y}  \left( x  \mid y \right) &= p_{X} \left( x \right) \quad \text{if} \quad  p_{Y} \left( y \right) > 0 \\
       &  \implies  p_{X \mid Y}  \left( x  \mid y \right) = p_{X}  \left( x \right) \cdot  p_{Y} \left( Y \right) 
    \end{split} 
    \] 
  \end{remark}

  \subsection{Simplified Notation}%
  \label{sub:simplified_notation}
  
  Unless it will cause confusion, we typically write 
  \begin{itemize}
    \item $p\left( x \right) $ instead of $p_{X} \left( x \right)$ 
    \item $p \left( y \right) $ instead of $p_{Y} \left( y \right)$ 
    \item $p\left( x,y \right) $ instead of $p_{X  ,Y} \left( x,y \right) $
    \item $p\left( x  \mid  Y =y \right)$ instead of $ p_{X \mid Y} \left( X  \mid  y \right)$
  \end{itemize}

  \subsection{Marginalization}%
  \label{sub:marginalization}
  
  THe law of total probability gives \[
  \begin{split}
    Pr \left \{ X = x \right \}  &=  \sum_{y}^{}  Pr \left \{ X =x, Y =y \right \} \\
    &=  \sum_{y}^{}  Pr \left \{ Y=y  \right \} Pr \left \{ X =x  \mid  Y =y \right \}   
  \end{split} 
  \] 
   \begin{tcolorbox}
     \textbf{Example.} A hunter ecounter $N$ birds. For each burd, he gets one shot and either or misses. Assume the probability of hitting is $p$ for each bird and that the shots are independent. Additionally, assume that the number of birds ancountered in Poission distrubuted with mean $\lambda$ I.e $N \sim \text{Passion}\left( \lambda  \right)$. Find the \textbf{EMF}  of the number of birds hit.
     \par \textbf{Solution}. 
     \begin{itemize}
       \item[i] Notation. \[
       \begin{split}
         \text{Let} \quad    &  I_{i} = \begin{cases}
           0,  &  \quad  \text{Miss bird}\quad   i \\
          1 ,   & \quad     \text{Hit bird} \quad   i 
         \end{cases} \quad \text{for} \quad i = 1,2,3,4, \ldots  
       \end{split} 
       \] 
       Let $X = \text{Number of birds hit}$
       Target is $p\left( x \right), x = 0,1,2, \ldots$
       \item[i] Condition on $N$ \[
       \left( X | N=n \right) = \begin{cases}
         0,  &  \quad  n =0 \\
         \sum_{i=1}^{n}  I_{i} ,  &  \quad  n >0  
       \end{cases}
       \] .  
       We know \[
         \begin{split}
       \left( X  \mid  N =n \right)   & \sim Binomial\left( n, p \right)  \\
       \implies  &  Pr \left \{ X =x  \mid  N =n \right \}  \\
       &= \begin{pmatrix}
       n \\
       x
       \end{pmatrix} 
       p ^{x} \left( 1-p \right)^{n-x} \quad  x = 0,1, \ldots, n   
         \end{split} 
       \]   . 
     \item[iii)]  \[
         \begin{split}
     Pr \left \{ X=x \right \} &=    \sum_{n=0}^{\infty}  Pr \left \{ X =x, N =n \right \}  \\
     &= \sum_{n=x}^{\infty}  Pr \left \{ N =n \right \} Pr \left \{ X=x  \mid  N =n \right \}     \\
     &=  \sum_{n = x}^{ \infty}  \frac{\lambda ^{n} e^{-\lambda }}{ n! }  \frac{p^{x} \left( 1-p \right) ^{n-x} n !}{ x! \left( n -x \right)!}  \\
     &=   \lambda ^{x}\frac{e^{- \lambda } p^{x} }{x!}   \sum_{n=x}^{\infty}  \frac{\left( 1-p \right)^{n-x}}{\left( n-x \right)!} \lambda ^{n-x}, \quad \text{hint } \quad \sum_{k=0}^{\infty} \frac{a^{k}}{k!}  = e^{a} \\
        &=  \left( \lambda p \right)^{x} e^{-\lambda } \frac{1}{x!}  \sum_{n=x}^{\infty}  \frac{\left[ \lambda \left( 1-p \right) \right]^{n-x}}{\left( n-x \right)!}   \\
        &= \left( \lambda p \right)^{x} e^{-\lambda p} \frac{1}{x!}  , \quad x = 0,1,2,3 \ldots  \\
        &\implies  \sim Possion\left( \lambda p \right)  
         \end{split} 
     \]  
     \end{itemize}

   \end{tcolorbox}

   \subsection{Conditional Expectation}%
   \label{sub:conditional_expectation}

   Let $X$ and $Y$ be random variables and $g$ a real function. The \textbf{Conditional expected value}  of $g\left( X \right) $ given $Y =y$ is \[
   E \left[ g\left( X \right)  \mid  Y =y \right] = \sum_{x}^{}  g\left( x \right) Pr \left \{ X =x  \mid  Y =y \right \} , \quad \text{ if} \quad  Pr \left \{ Y =y \right \} > 0  
   \] 
   \begin{remark}
     \begin{itemize}
       \item
     Note that $E\left[ g\left( X \right)  \mid  Y \right]$ is a stochastic variable variable! 
   \item $E\left[ g\left( X \right)  \mid  Y =y \right]$ has probability $Pr \left \{ Y =y \right \} $
     \end{itemize}
   \end{remark}

   \begin{theorem}[Law of iterated expectations]
     Let $X$ and $Y$ be random variables such that $E\left[ \left| g\left( X \right) \right| \right] < \infty$, and let $g$ be a real function. then \[
     E\left[ g\left( X \right)  \right] = E\left[ E\left[ g\left( x \right)  \mid  Y \right] \right] .
     \] 
   \end{theorem}
   \begin{proof}
     \[
       \begin{split}
     E\left[ E\left[ g\left( X \right)  \mid  Y \right] \right]   &=  \sum_{y}^{} E\left[ g\left( X \right)  \mid  Y =y \right] \\
     &=   \sum_{y}^{ }  \left\{ \sum_{x}^{} g\left( x \right) \cdot Pr \left \{ X = x  \mid  Y = y \right \} \right\} x Pr \left \{ Y =y \right \}  \\
     &=  \sum_{y}^{}  \sum_{x}^{}  g\left( x \right) \cdot  Pr \left \{ X =x , Y= y \right \}   \\
     &= \sum_{x}^{} g\left( x \right) \sum_{y}^{}  Pr \left \{ X =x, Y =y \right \}   \\
&= \sum_{x}^{}  g\left( x \right) Pr \left \{ X =x \right \}  \\
&= E\left[ g\left( X \right) \right]  \\
       \end{split} 
     \] 
   \end{proof}

   \begin{theorem}[Law of total variance]
     Let $X$ and $Y$ be random variables such that $E\left[ X^2 \right] < \infty$, then \[
     Var\left[ X \right] = E\left[ Var\left[ X  \mid  Y \right] \right] + Var \left[ E\left[ X  \mid Y \right] \right]
     \] 
   \end{theorem}
   
   \begin{tcolorbox}
     \textbf{Revisited Example.} A hunter ecounter $N$ birds. For each burd, he gets one shot and either or misses. Assume the probability of hitting is $p$ for each bird and that the shots are independent. Additionally, assume that the number of birds ancountered in Poission distrubuted with mean $\lambda$ I.e $N \sim \text{Passion}\left( \lambda  \right)$. Find the expected value and the variance of the number of birds hit. \[
    \begin{split}
      E\left[ X \right] &=  E \left[ E\left[ X  \mid  Y \right] \right] \\
      Var \left[ X \right]   & =  E\left[ var\left[ X  \mid Y \right] \right] + Var \left[ E\left[ E  \mid Y \right] \right]  \\
    \end{split}  
     \] 
     \textbf{Solution} .
   \[
       \begin{split}
       E\left[ X  \mid N =n  \right] &=  np \\
   Var\left[ X  \mid  N =n \right]   &  =  np\left( 1-p \right) \\
   &\implies  Var\left[ X  \mid  N \right] = N p\left( 1-p \right) \\
       \end{split} 
   \] 
   We get 
   \begin{itemize}
     \item \[
         \begin{split}
     E\left[ X \right] &=  E\left[ E\left[ X  \mid  N \right] \right] \\
     &=  E\left[ Np \right] = p E\left[ N \right] \\
     &=  p \lambda    
         \end{split} 
     \] 
   \item \[
       \begin{split}
   Var\left[ X \right]  & = E\left[ Var\left[ X  \mid N \right] \right] +  Var \left[ E\left[ X  \mid N \right] \right] \\
   &=  E \left[ Np\left( 1-p \right) \right] +  Var\left[ Np \right] \\
   &=  p\left( 1-p \right) \lambda + p^2 Var\left[ N \right] \\
   &=  p\left( 1-p \right) \lambda + p^2 \lambda   \\
   &=  \lambda p    
       \end{split} 
   \] 
     \end{itemize}

   \end{tcolorbox}
   


\newpage

\newpage
\section{Lecture 3}%
\label{sec:lecture_3}

\subsection{Randoms sum}%
\label{sub:randoms_sum}

Building on the hunter example from last week. we can more generally consider random sums \[
  X = \begin{cases}
    0,  &  \quad  N = 0 \\
    \zeta_{1} + \zeta _{2} + \ldots + \zeta_N , \quad  N >0  
  \end{cases}
\] 
where 
\begin{itemize}
  \item $N$ is a discrete random variable with values $0,1, \ldots$ 
  \item $\zeta _{1}, \zeta _{2}, \ldots $ are independent random variables
  \item $N$ is independent of $\zeta _{1}, \zeta _{2} + \ldots + \zeta _{N}$ 
  \item \textbf{Notation}  $X = \sum_{i=1}^{N} \zeta _{i} = \zeta _{1} + \zeta _{2} + \ldots + \zeta _{N}$ 
\end{itemize}

\begin{tcolorbox}
  \textbf{Example.} 
  \begin{enumerate}
    \item Insurance company \[
    N: \text{ Number of claims.} 
    \] 
  \[
    \zeta _{1} , \zeta _{2} , \ldots \quad  : \quad \text{Sizes of the claims} 
  \] 

  Total liabilility: \[
  X = \zeta _{1}+ \zeta _{2} + \ldots + \zeta _{N}
  \] 
\item  Be careful! \[
    \begin{split}
      \overbrace{E\left[ \sum_{i=1}^{N} \zeta _{i} \right]}^{\neq \sum_{i=1}^{N} E\left[ \zeta _{i} \right]}   & = E\left[ E\left[ \sum_{i=1}^{N} \zeta _{i}  \mid N \right] \right]\\
&= E\left[ \sum_{i=1}^{N} E\left[ \zeta _{i}  \mid  N \right] \right] 
    \end{split} 
\] 
  \end{enumerate}
\end{tcolorbox}

\subsection{Self Study}%
\label{sub:self_study}

Section 2.2, 2.3, 2.4

\subsection{Stochastic process in descrete time}%
\label{sub:stochastic_process_in_descrete_time}
\begin{definition}
  A \textbf{discrete-time stochastic process}  is a family of random variables $\left[ X_{t} : t \in  T \right]$ where $T$ is discrete.
  \begin{itemize}
    \item We use $T = \left\{ 0,1,2,.. \right\}$ and write $X_{n}$ instead of $X_{t}$
    \item  we call $X_{n}$ the \textbf{state}  at time $n =  0,1,2,3, \ldots$
    \item We call the set of all possible states the \textbf{state space} 
  \end{itemize}
\end{definition}

\begin{table}[htpb]
  \centering
  \caption{Table for example}
  \label{tab:label}
  \begin{tabular}{l|cccc}
    Day & $n =0$ & $n=1$ & $n=2$ & \ldots \\ 
    Random Variable  & $X_{0} $ & $X_{1}$ & $X_{2}$ & \ldots \\
    Realization  1& $x_{0} = 0$ & $x_{1} =1$ &  $x_{2} = 1 $ & \ldots \\
    Realization 2 & $x_{0} = 1$ & $x_{1} =1$ &  $x_{2} = 1 $ & \ldots \\
  \end{tabular}
\end{table}
\begin{tcolorbox}
  \textbf{Example.}  \[
  X_{n} = \begin{cases}
    1 ,  &  \quad \text{if it rains on day } n \\
    0,   &  \quad     \text{no rain on day } n
  \end{cases}
  \] 
  State space $= \left\{ 0,1 \right\}$
  \par
  \textbf{We have a problem.} Need \[
  Pr \left \{ X_{n} = x_{n}  \mid  X_{n-1} = x_{n} , X_{n-2} = x_{n-2}, \ldots, X_{0} = x_{0} \right \}.
  \]    for all $n = 0,1,2,\ldots$

\end{tcolorbox}

\subsection{Markov chain}%
\label{sub:markov_chain}


\begin{definition}[Discrete time Markov Chain]
  A \textbf{ Discrete time markoc chain}  is a discrete time stochastic process $\left\{ X_{n} : n = 0,1,\ldots \right\}$ that statisfied the \textbf{markov property}  such that \[
  \begin{split}
       & Pr \left \{ X_{n-1} = j  \mid  X_{n} = i ,    X_{n-1} = i_{n-1} , \ldots, X_{0} = i_{0} \right \}  \\
    &=  Pr \left \{ X_{n+1} = j  \mid  X_{n} = i \right \}  
  \end{split} 
  \] 
  for $n = 0,1,2,3, \ldots$ and for all states $i$ and $j$
\end{definition}







\section{References}%
\label{sec:references}



\bibliographystyle{plain}
\bibliography{references}
\end{document}

