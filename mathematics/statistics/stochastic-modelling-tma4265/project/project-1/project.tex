\documentclass{article}
\usepackage[utf8]{inputenc}

\title{Project 1}
\author{isakhammer }
\date{2020}

% 
%%%% DEPENDENCIES v1.3 %%%%%%

\usepackage{natbib}
\usepackage{graphicx}
\usepackage{amsmath}
\usepackage{amsthm}
\usepackage{amsfonts}
\usepackage{mathtools}
%\usepackage{enumerate}
\usepackage{enumitem}
\usepackage{todonotes}
\usepackage{esint}
\usepackage{float}


\usepackage{hyperref} 
\hypersetup{
  colorlinks=true, %set true if you want colored links
  linktoc=all,     %set to all if you want both sections and subsections linked
  linkcolor=blue,  %choose some color if you want links to stand out
} 
\hypersetup{linktocpage}


% inscape-figures
\usepackage{import}
\usepackage{pdfpages}
\usepackage{transparent}
\usepackage{xcolor}
\newcommand{\incfig}[2][1]{%
\def\svgwidth{#1\columnwidth}
\import{./figures/}{#2.pdf_tex} } \pdfsuppresswarningpagegroup=1

% Box environment
\usepackage{tcolorbox}
\usepackage{mdframed}
\newmdtheoremenv{definition}{Definition}[section]
\newmdtheoremenv{theorem}{Theorem}[section]
\newmdtheoremenv{lemma}{Lemma}[section]

% \DeclareMathOperator{\span}{span}

\theoremstyle{remark}
\newtheorem*{remark}{Remark}
%\newtheorem{example}{Example}

\newcommand{\newpara}
  {
  \vskip 0.4cm
  }

%%%%%%%%%%%%%%%%%%%%%%%%%%%%%%%%%%%%%%%%%%%%%%%%%%%%%%%%%%%%

%

\begin{document}
\maketitle
\tableofcontents
\newpage

\newpage
\section{Problem 1: Modelling the outbreak of measles }%
\label{sec:problem_1_modelling_the_outbreak_of_measles_}


\subsection{Answer a}%
\label{sub:answer_a}

\begin{definition}
  \label{def:dmarkov}
  Consider a stochastic process $\left\{ X_{n} \right\}_{n \in  \mathbb{N}_0  }$, where $n$ is a random variable. Then is the process said to be a \textbf{discrete-time markov chain} if, and only if \[
  Pr \left \{ X_{n+1} = i_{n+1}   \mid X_{n}=i_{n},  \ldots,  X_{0} = i_{0}  \right \}  = Pr \left \{ X_{n+1}  \mid  X_{0} \right \} 
\] 
for all states $i_{0},  \ldots,  i_{n}, i_{n+1} $. 
  
\end{definition}

Using the fact that $\left\{ X_{n}: n=0,1,\ldots \right\}$ has well defined probabilities which does not depend on time, and makes it possible to determine the state $X_{n}$ at any time $n$ makes this problem equivalent to the definition of a discrete markov chain \ref{def:dmarkov}. Using this fact, can we conclude that it exists a common transition matrix such that \[
\mathbf{P} = \begin{bmatrix} 
  p_{00}  & p_{01}  &  p_{12} \\
  p_{10}  &  p_{11}  &  p_{10} \\
  p_{20}  & p_{21}  & p_{22}
\end{bmatrix} 
\] 
where $p_{ij}$ is the stationary transition probability from state $i $ to state $j$ in one trial for all $n$ states. Note that that the sum of rows has to be $\sum_{j}^{}  p_{ij} = 1$. However, given the transition information in the problem description, 1), 2) and 3), can this be simplified to \[
\mathbf{P} = \begin{bmatrix} 
1- \beta   &  \beta  & 0\\
0  &  1-\gamma  & \gamma \\
\alpha   &  0  & 1-\alpha 
\end{bmatrix} 
.
\]  

Where $\beta , \alpha$ and $  \gamma $ are real constants which satisfies the conditions mentioned above.

\subsection{Answer b}%
\label{sub:answer_b}

\begin{itemize}
  \item Is this a reducible or irreducable Markov chain? 
    \newpara
    \textbf{Answer.}  
    If all state

  \item Determine the euivalent classes and determine wherher they are recurrent of transient.
  \item What is the period of each state? 
\end{itemize}





\newpage
\section{References}%
\label{sec:references}

\bibliographystyle{plain}
\bibliography{references}
\end{document}

