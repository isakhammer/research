\documentclass{article}
\usepackage[utf8]{inputenc}

\title{Problem Sets Linear Methods 2020}
\author{isakhammer }
\date{2020}

\usepackage{natbib}
\usepackage{graphicx}
\usepackage{amsmath}
\usepackage{amsthm}
\usepackage{amsfonts}
\usepackage{mathtools}
\usepackage{enumerate}
\usepackage{todonotes}


\usepackage{hyperref} 
\hypersetup{
  colorlinks=true, %set true if you want colored links
  linktoc=all,     %set to all if you want both sections and subsections linked
  linkcolor=blue,  %choose some color if you want links to stand out
} 
\hypersetup{linktocpage}


% inscape-figures
\usepackage{import}
\usepackage{pdfpages}
\usepackage{transparent}
\usepackage{xcolor}
\newcommand{\incfig}[2][1]{%
\def\svgwidth{#1\columnwidth}
\import{./figures/}{#2.pdf_tex} } \pdfsuppresswarningpagegroup=1

% Box environment
\usepackage{tcolorbox}
\usepackage{mdframed}
\newmdtheoremenv{definition}{Definition}[section]
\newmdtheoremenv{theorem}{Theorem}[section]
\newmdtheoremenv{lemma}{Lemma}[section]

\theoremstyle{remark}
\newtheorem*{remark}{Remark}
%\newtheorem{example}{Example}


\begin{document}
\maketitle
\tableofcontents
\newpage

\newpage
\section{Exercise Set 1}%
\label{sec:exercise_set_1}

\subsection{Problem 1}%
\label{sub:problem_1}

  \todo[inline]{ Dodo: check solutions.   }
Let $X,Y$ and $Z$ be sets
\begin{itemize}
  \item Show that $X\cap \left( Y\cup Z \right) = \left( X\cap Y \right) \cup  \left( X\cap Z \right)$ 
    \begin{tcolorbox}
      \textbf{Answer.}  
      Recall the definitions
      \begin{definition}
        For two sets $X$ and $Y$ is \[
        X \cap  Y = \left\{ x \in  X \quad   \text{and } \quad  x \in  Y  \right\}
        \] 
        
         \[
        X \cup Y = \left\{ x \in X \quad \text{or} \quad   x \in  Y \right\}
        \] 
      \end{definition}

      Let $ x \in  X$ and  $x \in  \left( Y \cup  Z \right)$. Then is $x \in  X \cap Y$ and $x \in  X \cap  Z$ which ratifies $x \in  \left( X \cap Y \right) \cup  \left( X \cap Z \right)$.

    \end{tcolorbox}

  \item Show that $X\setminus \left( Y\cup Z \right) = \left( X \setminus Y \right) \cap  \left( X \setminus Z \right)$
    \begin{tcolorbox}
      \textbf{Answer.} 
      Recall the definition 
      \begin{definition}
        For two sets $X$ and $Y$, then is \[
        X \setminus Y  = \left\{ x \in X \quad \text{ and } \quad x \not\in Y   \right\}
        \]  
      \end{definition}
      We can then use the same argumentation as in the previous proof. Let  $x \in  X$ and not in $x \in \left( Y \cup   Z  \right)$. Then is $x \in  X \setminus  Y$  and $ x \in X \setminus  Z$
    \end{tcolorbox}
\end{itemize}

\subsection{Problem 2}%
\label{sub:problem_2} 

Let $f:  X \to  Y$ be a function, let $B$ be a subset of $Y$, and let $\left\{ B_{i} \right\}_{i \in  I}$ be a family of subsets of $Y$.

\begin{itemize}
  \item Prove that \[
      f^{-1} \left( \bigcap_{i \in  I}^{} B_{i} \right)  =   \bigcap_{i \in  I}^{}  f^{-1} \left( B_{i} \right)
  \] 
  \begin{tcolorbox}
    \textbf{Answer.} 
    
    Since $f^{-1} $ is mapped $Y \to  X$ and $B_{i}$ is a subset of $Y$. Obviously is \[
      \bigcap_{i \in  I}^{} B_{i} \to  B
    \] 

    And since 
  \end{tcolorbox}
\item Prove that $f\left( f^{-1}\left( B \right) \right) \subseteq  B$ and if $f$ is surjective then equality holds. Show by example that equality need not to hold if $f$ is surjective. 
  \begin{tcolorbox}
    \textbf{Answer.} 
    $f$ is surjective  if $f: X \to  Y$ and there exist one $x \in  X$ does it exists at least one $y \in Y$. Since  by the definition of an inverse function is $f^{-1}  \left( f\left( x \right) \right)= x$ for every $x \in  X$, 
  \end{tcolorbox}
\end{itemize}



\newpage
\section{References}%
\label{sec:references}

\bibliographystyle{plain}
\bibliography{references}
\end{document}

