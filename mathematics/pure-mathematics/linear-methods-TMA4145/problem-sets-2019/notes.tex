\documentclass{article}
\usepackage[utf8]{inputenc}

\title{Problem Sets 19}
\author{isakhammer }
\date{2020}

% 
%%%% DEPENDENCIES v1.1 %%%%%%

\usepackage{natbib}
\usepackage{graphicx}
\usepackage{amsmath}
\usepackage{amsthm}
\usepackage{amsfonts}
\usepackage{mathtools}
\usepackage{enumerate}
\usepackage{todonotes}
\usepackage{float}


\usepackage{hyperref} 
\hypersetup{
  colorlinks=true, %set true if you want colored links
  linktoc=all,     %set to all if you want both sections and subsections linked
  linkcolor=blue,  %choose some color if you want links to stand out
} 
\hypersetup{linktocpage}


% inscape-figures
\usepackage{import}
\usepackage{pdfpages}
\usepackage{transparent}
\usepackage{xcolor}
\newcommand{\incfig}[2][1]{%
\def\svgwidth{#1\columnwidth}
\import{./figures/}{#2.pdf_tex} } \pdfsuppresswarningpagegroup=1

% Box environment
\usepackage{tcolorbox}
\usepackage{mdframed}
\newmdtheoremenv{definition}{Definition}[section]
\newmdtheoremenv{theorem}{Theorem}[section]
\newmdtheoremenv{lemma}{Lemma}[section]

% \DeclareMathOperator{\span}{span}

\theoremstyle{remark}
\newtheorem*{remark}{Remark}
%\newtheorem{example}{Example}

\newcommand{\newpara}
  {
  \vskip 0.4cm
  }

%%%%%%%%%%%%%%%%%%%%%%%%%%%%%%%%%%%%%%%%%%%%%%%%%%%%%%%%%%%%

%

\begin{document}
\maketitle
\tableofcontents
\newpage

\newpage
\section{Exercise 1}%
\label{sec:exercise_1}

\subsection{Problem 2}%
\label{sub:problem_2}

Define functions $\mathbb{R} $ with values in $\mathbb{R} $ .
\begin{enumerate}
  \item A function that is not left invertible.
  \item A function that is not right invertible.
\end{enumerate}


Show that the given functions have their respective properties.

\newpara
\begin{tcolorbox}
  
 function is left inverrtible if there exists a function $f^{-1}_{l}$ such that \[
x = f\left( f_{l}^{-1}\left( x \right) \right) 
\] 
or formally \[
id_{x} = f \circ f_{l}^{-1}
\] 
Same for right invertible function which can be written as \[
id_{x} = f_{r}^{-1} \circ  f 
\] 
\end{tcolorbox}

\newpara
A function $h=x^2$ is a function that does not support both right and left invertible. 


\subsection{Problem 3}%
\label{sub:problem_3}

Given the linear mapping $T: \mathbb{R} ^2 \to  \mathbb{R} ^3$ given by $Tx = Ax$ with \[
A = \begin{pmatrix}
3  &  -4 \\
1  &  6 \\
1  &  1
\end{pmatrix} 
\]  
\begin{enumerate}
  \item Show that the matrix \[
  A_{l}^{-1} = \frac{1}{9}  \begin{pmatrix}
  -11  &  -10  &  16 \\
  7  &  8  &  -11
  \end{pmatrix} 
  \] 
  Is inducing a left invere $T^{-1} _{ l}  $ of $T$ . 
  This left inverse is not unique. show that \[
  \frac{1}{ 2}  \begin{pmatrix}
  0  &  -1  &  6\\
  0  &  1  &  -4
  \end{pmatrix} 
  \] 
  gives another left inverse.
  \begin{enumerate}
    \item We can show it by  computing $T \circ  T^{-1} _{ l}$ such that \[
    A \cdot  A^{-1} _{ l} = I
    \] 
  \item THe right inverse can be computed be analysing the transpose of $A$. \[
      A A^{-1}_{l} = I \quad  \leftrightarrow  \quad I=I^{T} = \left( A A^{-1}_{l} \right)^{T}   = A^{-1}_{lT} A^{T}
  \] 
  At least this is the solution given. Not sure since finding an right inverse to $A^{T}$ answer the question.
  \end{enumerate}
\end{enumerate}
\subsection{Problem 4}%
\label{sub:problem_4}
Show that cartesion product of two (infinite) countable sets is countable.

\newpara
\textbf{Solution}. A set is countable if it exist a integer which can be allocated for every $\mathbb{N}^{+} $ . Let $A = \left\{ a_{1} , a^{2} ,a^{3}, \ldots \right\}$  and $B = \left\{ b_{1}, b_{2}, \ldots \right\}$ be two infinite countable sets. Let us define the product $C = B \times A $ such that \[
C = \left\{ a_{1} b_{1}, a_{2} b_{2}, \ldots \right\}
\] 
If we compare it with $N^{+}$ can we observe that \[
\begin{split}
C  & = \left\{ a_{1} b_{1}, a_{2} b_{2}, \ldots \right\} \\
N^{+} &=  \left\{ 1,2, \ldots \right\} \\
\end{split} 
\] 
Which means that there exists one element in $\mathbb{N} $  for every element in $C$, which shows that $C$ has to be countable.

\subsection{Problem 5}%
\label{sub:problem_5}
Show that the sets $\mathbb{Z} $ of integeres and $\mathbb{Q} $ of rational numbers are countable.

\begin{tcolorbox}
  Important to define \textbf{surjective}, \textbf{injective} and \textbf{bijective} . Let $f: X \to  Y$  be a function,
  \begin{enumerate}
    \item  \textbf{Injective} if it exists at least one $Y$ for every $Y$ 
    \item \textbf{surjective} (or onto)  if there exists $ran \left( f \right) = Y$, or it exists an $Y$ for every $X$ in 
    \item \textbf{Bijection} is when both injective and surjective is conserved. We also call bijection as \textbf{one-to-one}.
  \end{enumerate}
\end{tcolorbox}

\textbf{Solutions}.
\begin{itemize}
  \item To show that $\mathbb{Z} $ is countable can we describe the set such that \[
  \mathbb{Z}  = \left\{ \ldots, -2, -1, 0, 1, 2 \ldots \right\}
  \] 
  By comparing every element in $N^{+}$ such that \[
    \begin{split}
      N^{+} &=  \left\{ 1,2, \ldots \right\}  \\
  \mathbb{Z}   & = \left\{ \ldots, -2, -1, 0, 1, 2 \ldots \right\}
    \end{split} 
  \] 
  Lets every odd element in $N^{+}$  be $N_{\text{ODD}} $ and every even element be $N_{\text{EVEN}}$, then can we make  \[
      N^{ODD} &=  \left\{ 1,3, \ldots \right\}  \\
  \mathbb{Z}^{-}   & = \left\{ \ldots, -2, -1, 0 \right\}
  \] 
  and
  \[ 
      N^{EVEN} &=  \left\{ 2,4,6, \ldots \right\}  \\
  \mathbb{Z}^{+}   & = \left\{ 1,2,3 , \ldots\ldots \right\}
  \] 
  We have then showed it exists a element in $\mathbb{N} ^{+}$ for every element in $\mathbb{Z} $, which makes it countable.
\item For the rational numbers $\mathbb{Q} $ such that $\frac{a_{1}}{a_{2}}  \in  \mathbb{Q} $ where $a_{1}, a_{2} \in  \mathbb{Z}   $ and $a_{2} \neq 0$. We can then use the fact that $\mathbb{Z} $ is countable such that both the nominator and demonitor is countable. In practice can we write the rational numbers as a set such that \[
    \begin{split}
\mathbb{Q} &= \left\{ \frac{-1}{}  \right\} \\.
    \end{split} 
\] 

    
\end{itemize}



\newpage
\section{References}%
\label{sec:references}

\bibliographystyle{plain}
\bibliography{references}
\end{document}

