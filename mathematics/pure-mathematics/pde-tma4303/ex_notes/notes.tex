
\documentclass{article}
\usepackage[utf8]{inputenc}

\title{Notes}
\author{isakhammer }
\date{2020}

% 
%%%% DEPENDENCIES v1.3 %%%%%%

\usepackage{natbib}
\usepackage{graphicx}
\usepackage{amsmath}
\usepackage{amsthm}
\usepackage{amsfonts}
\usepackage{mathtools}
%\usepackage{enumerate}
\usepackage{enumitem}
\usepackage{todonotes}
\usepackage{esint}
\usepackage{float}


\usepackage{hyperref} 
\hypersetup{
  colorlinks=true, %set true if you want colored links
  linktoc=all,     %set to all if you want both sections and subsections linked
  linkcolor=blue,  %choose some color if you want links to stand out
} 
\hypersetup{linktocpage}


% inscape-figures
\usepackage{import}
\usepackage{pdfpages}
\usepackage{transparent}
\usepackage{xcolor}
\newcommand{\incfig}[2][1]{%
\def\svgwidth{#1\columnwidth}
\import{./figures/}{#2.pdf_tex} } \pdfsuppresswarningpagegroup=1

% Box environment
\usepackage{tcolorbox}
\usepackage{mdframed}
\newmdtheoremenv{definition}{Definition}[section]
\newmdtheoremenv{theorem}{Theorem}[section]
\newmdtheoremenv{lemma}{Lemma}[section]

% \DeclareMathOperator{\span}{span}

\theoremstyle{remark}
\newtheorem*{remark}{Remark}
%\newtheorem{example}{Example}

\newcommand{\newpara}
  {
  \vskip 0.4cm
  }

%%%%%%%%%%%%%%%%%%%%%%%%%%%%%%%%%%%%%%%%%%%%%%%%%%%%%%%%%%%%

%

\begin{document}
\maketitle
\tableofcontents
\newpage

\newpage
\section{Exercise Week 35}%
\label{sec:exercise_1}

\subsection{B 3.6}%
\label{sub:b_3_6}
\begin{tcolorbox}
Hva skjer når $a\to 0$
\end{tcolorbox}


 Burger equation 
 \begin{equation}
 \label{eq:burger}
 \frac{\partial u}{\partial t}  + u \frac{\partial u}{\partial x} = 0
 .\end{equation}

 \begin{enumerate}[label=(\alph*)]
   \item Use the method of characteristics as described in Sect 3.4 to find a formula for the solution $u\left( t,x \right)$ given the inital condition \[
   u\left( 0, x \right) = \begin{cases}
     0,  &  \quad x\le 0\\
     \frac{x}{a} ,  &  \quad  0 < x < a \\
     1, &  \quad    x\ge a
   \end{cases}
   \] 
  \item Suppose that $a > b$ and \[
  u\left( 0,x \right) = \begin{cases}
    a,  &  \quad  x\le 0, \\
    a\left( 1-x \right) + bx ,  &  \quad    0 < x < 1, \\
    b,  &  \quad  x\ge 1 
  \end{cases}
  \] 
  Show that all of th characteristics originating from $x_{0} \in  \left[ 0,1 \right]$ meet at the same point.
 \end{enumerate}

 \subsubsection{Answer a}%
 \label{ssub:answer_a}
 
 \subsubsection{Answer b}%
 \label{ssub:answer_a}
 
 

\subsection{X1}%
\label{sub:x1}

Gitt en PDE med initaldata \[
u_{t} + u^2 u_{x} = 0, \quad  u\left( 0,x \right) = \frac{1}{1+x^2} 
\] 
Hva er største verdi at $T$ slik at problemet har en klassisk løsning for $x \in \mathbb{R} $ og $t \in [0,T)$

\subsubsection{Answer}%
\label{ssub:answer}







% \subsection{ B 3.6}%
% \label{sub:problem_b3_}

% Burger equation 
% \begin{equation}
% \label{eq:burger}
% \frac{\partial u}{\partial t}  + u \frac{\partial u}{\partial x} = 0
% .\end{equation}

% \begin{enumerate}[label=(\alph*)]
%   \item Use the method of characteristics as described in Sect 3.4 to find a formula for the solution $u\left( t,x \right)$ given the inital condition \[
%   u\left( 0, x \right) = \begin{cases}
%     0,  &  \quad x\le 0\\
%     \frac{x}{a} ,  &  \quad  0 < x < a \\
%     1, &  \quad    x\ge a
%   \end{cases}
%   \] 
%  \item Suppose that $a > b$ and \[
%  u\left( 0,x \right) = \begin{cases}
%    a,  &  \quad  x\le 0, \\
%    a\left( 1-x \right) + bx ,  &  \quad    0 < x < 1, \\
%    b,  &  \quad  x\ge 1 
%  \end{cases}
%  \] 
%  Show that all of th characteristics originating from $x_{0} \in  \left[ 0,1 \right]$ meet at the same point.
% \end{enumerate}


% \subsection{ B 3.7}%
% \label{sub:problem_1}

% \begin{theorem}
%   \label{th1}
%   Suppose that $u \in  C^{1} \left( \left[ 0,T \right] \times  \Omega   \right)$ is a solution of \[
%   \frac{\partial u}{\partial t}  + \mathbf{a}\left( u \right) \cdot  \nabla u = 0
%   \] 
%   For some region $\omega  \subset  \mathbb{R}  ^{n}$ with $\mathbf{a} \in  C^{1} \left( \mathbb{R} ; \mathbb{R} ^{n} \right)$.  Then for each $\mathbf{x}_{0} \in  \Omega $ , $u$ is a constant along the characteristic line defined by \[
%   \mathbf{x}\left( t \right) = \mathbf{x}_{0} + \mathbf{a} \left( u\left( 0, \mathbf{x}_{0} \right) \right)t
%   \] 
% \end{theorem}


% Let the Hamilton equation be 
% \begin{equation}
% \label{eq:1}
% \frac{\partial u}{\partial t}  + \frac{1}{2} \left( \frac{\partial u}{\partial x}  \right)^2 = 0
% .\end{equation}
% Assume that $u \in C^{1} \left( \left[ 0, \infty \right] \times  \mathbb{R} ^{n}  \right)$ is a solution.  By analogy with Theorem \ref{th1}, a characteristic of the equation is defined as a solution of 
%   \begin{equation}
%   \label{eq:2}
% \frac{d x}{d t} \left( t \right) = \frac{\partial u}{\partial x}  \left( t, x\left( t \right) \right), \quad  x\left( 0 \right) = x_{0} 
%   .\end{equation}
%   \begin{enumerate}[label=(\alph*)]
%     \item Assuming that $x\left( t \right)$ solves \eqref{eq:2}, use the chain rule to compute $\frac{d ^2 x}{d t^2} $.
% \begin{tcolorbox}
%   \textbf{Answer.} 
% \textbf{Haralds Solution}. If we let \[
%   u_{t} + \frac{1}{2} u_{x}^2 = 0
% \] 
% and \[
% \dot{x} = u_{x}
% \] 
% Then can we write \[
% \ddot{x} = u_{xt} + \dot{x}u_{xx} = u_{xt} + u_{x}u_{xx} = u_{xt} + \frac{1}{2 } \left( u_{x}^2 \right)_{x}
% \] 
%  I did not get this derivation. 
% \end{tcolorbox}
%     \item Differentiate \eqref{eq:1}  with respect to $x$ and then restrict the results to $\left( t, x\left( t \right) \right)$ where $x\left( t \right)$ solves \eqref{eq:2}. Conclude from (a) that to \[
%     \frac{d ^2 x}{d t^2}  = 0
%     \] 
%     Hense, for some constant $v_{0}$ (which depends on the characteristic) ,\[
%     x\left( t \right) = x_{0} + v_{0} t
%     \] 
% \begin{tcolorbox}
%   \textbf{Answer.} 
% \textbf{Haralds Solution.}  Derivation of \eqref{eq:1} with $x$  gives \[
% u_{xt} + u_{xx} u_{x} = 0.
% \] 
% Since $u \in  C^{2}$ is $u_{t x} \approx u_{xt} \approx 0$. So (a) gives us $\ddot{x} = 0$, and that is why $x\left( t \right) = x_{0} + v_{0}t$ der ($x_{0}$, $v_{0}$

% \end{tcolorbox}
%   \item Show that the Lagrangian derivative of $u$ along $x\left( t \right)$ satisfies \[
%   \frac{Du}{Dt}  = \frac{1}{2} v_{0}^2
%   \] 
%   Implying that \[
%   \left( t, x_{0} + v_{0} t \right) = u\left( 0, x_{0}  \right) + \frac{1}{2} v_{0}^2 t
%   \] 
% \begin{tcolorbox}
%   \textbf{Answer.} 
% \textbf{Harald solution.}  \[
%   \begin{split}
% \frac{Du}{Dt}   & = \frac{d }{d t}  u\left( t,x\left( t \right) \right) = \frac{d }{d t}  u\left( t, x_{0} + v_{0} t \right) \\
% &= u_{t} + v_{0} u_{x} = -\frac{1}{2} u_{x}^2 + u_{x}^2  \\
% &= \frac{1}{2} u_{x}^2 = \frac{1}{2} v_{0}^2 \\
%  &  \implies  u\left( t, x\left( t \right) \right) = u\left( 0, x_{0} \right) + \frac{1}{2} v_{0} ^2 t
%   \end{split} 
% \] 
% Nb! $v_{0} = u_{x}$ evaluated in $t=0$ given $v_{0} = u_{x} \left( 0,x_{0} \right)$
% \end{tcolorbox}
% \item Use this approach to find the solution $u\left( t,x \right)$ under the inital condition \[
% u\left( 0,x \right) = x^2
% \] 
% (For the characteristic starting at $\left( 0,x_{0} \right)$, note that you can compute $v_{0}$ by evaluation \eqref{eq:2} 
% \begin{tcolorbox}
%   \textbf{Answer.}  Derivation

% \end{tcolorbox}
%   \end{enumerate}

%   \[
%    x     \approx \lambda  \frac{}{}  \\
%    \text{let alt,so}
%   \] 
  
  
%   \subsection{ B 4.1}%
%   \label{sub:pb14}
%   \begin{theorem}
%     \textbf{Wave Equation }is on the form 
%     \begin{equation}
%     \label{eq:wf}
%     \frac{\partial ^2 u}{\partial t^2}  - c^2 \frac{\partial ^2 u}{\partial x^2}  = 0.
%     .\end{equation}
    
%   \end{theorem}

%   Suppose $u\left( t,x \right)$ satisfies \eqref{eq:wf} for $x \in \mathbb{R} $. Let $\mathcal{P} $ be a parallellogram in the $\left( t,x \right)$ plane whose sides are characteristic lines. Show that the valie  of $u$ at each vertex $\mathcal{P} $ is determined y the values at the other three vertices.
  

% \subsection{B 4.2}%
% \label{sub:problem_4_2}

% \begin{equation}
% \label{eq:nmb}
% u\left( 0,x \right) = g\left( x \right) , \quad  \frac{\partial u}{\partial t}  \left( 0,x \right) = h\left( x \right) 
% .\end{equation}

% \begin{equation}
% \label{eq:nm2}
%  u\left( t,x \right) = \frac{1}{2} \left[ g\left( x + ct \right) 0 g\left( x- ct \right) \right] + \frac{1}{2c} \int_{x -ct}^{ x+ct}  h\left( \tau  \right) d\tau   
% .\end{equation}


% The weave equation \eqref{eq:wf} is an appropriate model for the longitudal vibrations of a spring. In this application $u\left( t,x \right)$ represents displacement parallel to the spring. Suppose that spring has length $l$ and is free at the ends. This corresponds to the Neumann boundary conditions \[
% \frac{\partial u}{\partial x} \left( t, 0 \right) = \frac{\partial u}{\partial x} \left( t, l \right) = 0 , \quad  \forall t \ge 0 
% \] 
% Assume the inital conditions are $g$ and $h$ as in \eqref{eq:nmb}, which also satisfu Neumann boundary condition on $\left[ 0,l \right]$. Determine th appropriate  extension of $g$ and $h$ from $\left[ 0,l \right]$ to $\mathbb{R} $  so that the solution $u\left( t,x \right)$ given by \eqref{eq:nm2}  will satisfy Neumann boundary problem for all $t$.

  
  





% \newpage
\section{Exercise Week 36}%
\label{sec:exercise_week_36}

\subsection{B 3.7 }%
\label{sub:problem_b_3_7_}

\begin{tcolorbox}
 (you may need to assume that $u \in  C^{2}$). Additionally, note that $w=u_x$ satisfies Burgers' equation!
\end{tcolorbox}


\subsection{ B 4.1}%
\label{sub:problem_b_4_1}

\subsection{ B 4.5}%
\label{sub:problem_b_4_5}

\subsection{ X2}%
\label{sub:problem_x2}

Løs initialproblemet \[
u u_{x} + y^2 u_{y} = yu, \quad  u\left( x,1 \right) = x 
\] 
Hva er det største området i planet som tillater en klassisk løsning?


\newpage
\section{Exercise Week 37}%
\label{sec:exercise_week_37}

\subsection{B 4.2}%
\label{sub:b_4_2}

\subsection{B 4.4}%
\label{sub:b_4_4}

\subsection{X3}%
\label{sub:problem_x3}

Benytt løsningen til B 4.1 til å vise at den homogene bølgeligningen på et område gitt ved $a_{0} < x + ct < a_{1}$, $b_{0} < x -ct < b_{1} $ har generell løsning $u\left( t,x \right) = f_{1}\left( x-ct \right) 0 f_{2} \left( x + ct\right)$ for funksjoner $f_{1}$ of $f_{2}$. Hvordan kan du utvide resultatet til $x \in  \mathbb{R} , t >0$ ?


\subsection{X4}%
\label{sub:problem_x4}

En alternativ utledning av D'Alemberts løsning: Fyll ut de manglende detaljene nedenfor. 


\newpara

Start med ligningen $u_{tt} - c^2 u_{xx} = 0$. Anta at $u$ er en løsning, of definer de to funksjonene $u_{t} \pm cu_{x}$. Disse oppfyller enkle transportligninger, så ver av dem er en bølge med hastight $\pm x$. Med andre ord finned funksjoner $w_{\pm}$ slik at \[
\begin{split}
  u_{t} - cu_{x} &=  -2c w'_{+} \left( x - ct \right) , \quad \text{(høyrebølge)}  \\
  u_{t} + cu_{x} &=  2cw' _{-} \left( x + ct \right), \quad \text{venstrebølge}  \\
\end{split} 
\] 
(Faktorene $\pm2c$ of derivasjonen på høyre side er ikke vesentlige; de er bare for å forenkle regningen videre.) Adder de to ligningene og integrer mhp $t$, of subtraher dom of integrer mhp. $x$. Du trenger to integrasjonskosntanter, $C_{1}\left( x \right), \quad C_{2}\left( t \right) $. Konkluder at de integrasjonskonstantene må være like, og derfor en virkelig konstant $C$.  Konkludr at \[
u\left( t,x \right) = w_{+} \left( x- ct \right) + w_{-} \left( x+ct \right) + C
\] 
(Men vi kan like godt inkorporere $C$ i en av de to funksjonene $w_{\pm}$.)


\newpara
Til slutt, sett inn i initaldataene \[
u\left( 0,x \right) = g\left( x \right) , \quad  u_{t}\left( 0,x \right) = h\left( x \right) 
\] 

og utled D'Alemberts løsning.



\newpage
\section{Exercise Week 38}%
\label{sec:exercise_week_38}

\subsection{B 4.7}%
\label{sub:b_4_7}

\subsection{B 4.9}%
\label{sub:b_4_9}

\subsection{X5}%
\label{sub:x5}


Bjelkeligningen har formen $u_{tt}+ u_{xxxx} = f\left( t,x \right)$. FInn en tilhørende energitetthet og fliks, og bruk disse til å vise entydighet av øsninger for et inital-og randverdiproblem på intervallet $\left( 0,1 \right)$. Det er en del av oppgaven å finne egnede initalverdier of randbetingelser som sikrer entydighet.

\newpage
\section{Exercise Week 39}%
\label{sec:exercise_week_39}

\subsection{B 6.1}%
\label{sub:b_6_1}

\subsection{6.3}%
\label{sub:6_3}

\subsection{6.4}%
\label{sub:6_4}








\newpage







\newpage

\section{Prewritten Exercises}%
\label{sec:prewritten_exercises}



\section{References}%
\label{sec:references}

\bibliographystyle{plain}
\bibliography{references}
\end{document}

