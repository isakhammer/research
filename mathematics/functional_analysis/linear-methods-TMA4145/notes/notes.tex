\documentclass{article}
\usepackage[utf8]{inputenc}

\title{Template}
\author{isakhammer }
\date{July 2020}

\usepackage{natbib}
\usepackage{graphicx}
\usepackage{amsmath}
\usepackage{amsthm}
\usepackage{amsfonts}
\usepackage{mathtools}
\usepackage{enumerate} 


% inscape-figures
\usepackage{import}
\usepackage{pdfpages}
\usepackage{transparent}
\usepackage{xcolor}

\newcommand{\incfig}[2][1]{%
    \def\svgwidth{#1\columnwidth}
    \import{./figures/}{#2.pdf_tex} } \pdfsuppresswarningpagegroup=1

% Box environment
\usepackage{tcolorbox} 
\usepackage{mdframed}
\newmdtheoremenv{definition}{Definition}[section]
\newmdtheoremenv{theorem}{Theorem}[section]
\newmdtheoremenv{lemma}{Lemma}[section]


%% Swap the definition of \abs* and \norm*, so that \abs
%% and \norm resizes the size of the brackets, and the 
%% starred version does not.
%\makeatletter
%\let\oldabs\abs
%\def\abs{\@ifstar{\oldabs}{\oldabs*}}
%%
%\let\oldnorm\norm
%\def\norm{\@ifstar{\oldnorm}{\oldnorm*}}
%\makeatother


\theoremstyle{remark}
\newtheorem*{remark}{Remark}



\begin{document}
\maketitle

\section{Problem 1}%
\label{sec:Problem 1}

Determine whether the followeing statements are true or false. if the statement are true, no further explanation is required. If the statement is false, give a counter example.

  \begin{enumerate}
    \item The Kerner of a bounded linear operator $T: X \mapsto Y$ between normed spaces $X$ and $Y$ is closed.
      \begin{tcolorbox}
        \textbf{Answer.}  
        A subset is only closed if and only if its complement $A=X^{c}$ is open.   

      \end{tcolorbox}
    \item The range of a bounded linear operator $T: X \to Y$ between normed spaces $X$ and $Y$ is closed.
      \begin{tcolorbox}
        \textbf{Answer.} 
      \end{tcolorbox}
    \item The dual space $X^{'}$ of a normed space is a Banach Space.
      \begin{tcolorbox}
        \textbf{Answer.} 
      \end{tcolorbox}

    \item A closed subspace of a Banach Space is itself a Banach Space.
      \begin{tcolorbox}
        \textbf{Answer.} 
        Let $(X,\|.\| )$ be a Banach space and let $Y \subset X$ be a closed subset. Then is 
      \end{tcolorbox}
  \end{enumerate}



\section{Appendix}%
\label{sec:Notes}

\subsection{Sequences in metric spaces and normed spaces}%
\label{sub:sequences_in_metric_spaces_and_normed_spaces}

\begin{definition}[Sequence]
  Let $\left( X,d \right) $ be a metric space. A sequence $\left( x_{n} \right) _{n \in \mathbb{N}}$ in $X$ is said to \textbf{converge to} $x \in X$ for every $\epsilon > 0$ one can find $N=N(\epsilon) \in \mathbb{N}$ such that \[
    d\left( x_n, x \right) <  \epsilon 
  .\] whenever $b \ge N$. The element $x$ is called the \textbf{limit} of the sequence $\left( x_{n} \right) _{n \in \mathbb{N}}$. In particular, in $\left( X,\|.\| \right) $ is a normed space. then $\left( x_{n} \right) _{n \in \mathbb{N}}$ converge to $x \in X$ for every $\epsilon > 0$ one can find $N = N\left( \epsilon \right) \in \mathbb{N}$ such that \[
  \|x - x_{n} \| < \epsilon
  .\] whenever $n \ge N$..
\end{definition}



\subsection{Cauchy Sequence}%
\label{sub:Cauchy Sequence}

\begin{definition}
  
  Let $(x_n)_{n \in \mathbb{N}}$ be a sequence in the metric space $(X, d)$. We say that $\left( x_n \right)_{n \in \mathbb{N}} $ is \textbf{Cauchy Sequence} if for any $ \epsilon > 0$ there exist an $N \in \mathbb{N}$ such that \[  
    d(x_{n}, x_{m}) < \epsilon      
  .\]  
  In particular if $\left( x_n \right)_{n \in \mathbb{N}}$ is a sequence in the normed space $\left( X,\|.\| \right) $, then $\left( x_n \right) _{n \in\mathbb{N}}$ is Cauchy if for any $\epsilon > 0$ there exist an $N \in \mathbb{N}$ such that

  \[
  \|x_{n}  - x_{m}\| < \epsilon,\quad \textrm{s.t.} \quad  n,m \ge N
  .\] 
  In an inner product space $(X, \left< .,. \right> )$, we say that a sequenxe $\left( x_n \right)_{n \in \mathbb{N}}$ is Cauchy if the sequence is Cauchy with respect to the indeuced norn $\|x\| := \left< x,x \right>^{ \frac{1}{2} }$ . 
\end{definition}

\begin{lemma}
  Any Cauchy sequence in $\left( X,d \right)$ is bounded.
\end{lemma}

\begin{proof}
  Let $\left( x_n \right) _{n \in \mathbb{N}}$ be a Cauchy sequence. Then there exist $N \in \mathbb{N}$ such that for all $ m,n \ge N$ we have \[
    d(x_{m}, x_{n}) < 1
  .\] In particular, we have \[
  d(x_{N},x_{m})  < 1 \quad \forall \quad m \ge N      
  .\] 
  Or equivalently $x_{m} \in B_1(x_N)$ for all $m \ge N$. Now let \[
    r = max \{ 1, d(x_1, x_N), d(x_2, x_N), \ldots, d(x_{N-1}, x_N)\} 
  .\] 
  Then for any $n \in \mathbb{N}$ we have $x_n \in B_{r+1}\left( x_N \right) $\, so $\left( x_n \right) _{n \in \mathbb{N}}$
is bounded. 

\end{proof}
\subsection{Common}%
\label{sub:common}

\begin{definition}[Range]
  A range of a function $f: X \mapsto Y$, is denoted by $range\left( f \right) $ or $f\left( X \right) $, is the set of all $y \in Y$ that are the image of some $x \in X$. More compact can this be written. \[
    range\left( f \right) = \{y \in Y  \mid \text{there exist } x \in X \text{ such that } f\left( x \right) = y\} 
  \] 
\end{definition}

\begin{definition}
  Let $f: X \mapsto Y$ be a function. 
  \begin{enumerate}
    \item We call $f$ injective or one-to-one if $f\left( x_1 \right)  = f\left( x_2 \right) $ implies $x_1=x_2$, i.e, no two elements of the domain have tha same image. Equivalently, if $x \neq x_2$ then $f\left( x_1 \right) \neq f\left( x_2 \right) $. 
    \item We call $f$ surjective or onto if $range\left( f \right) = Y$, i.e each $y \in Y$ is the image of at least one $x \in X$ .
    \item We call $f$ bijective if f is both injective and surjective.
      
  \end{enumerate}
\end{definition}

\begin{definition}[Closed Set]
\end{definition}

\subsection{Linear Operator}%
\label{sub:linear_operator}

\begin{definition}
  A linear operator $T$ is an operator such that 
  \begin{enumerate}
    \item the domain $\mathbb{D}\left( T \right) $ of $T$ is a vector space and the range $R\left( T \right) $ lies in a vector space over the same field. 
    \item  $\forall x,y \in \mathbb{D}\left( T \right) $ and scalars $\alpha$ 
      \begin{equation}
      \label{eq:linear-operator}
      T\left( x + y \right) = Tx + Ty \quad \text{and} \quad T\left( \alpha x \right) = \alpha Tx
      .\end{equation}
  \end{enumerate}

\end{definition}

\begin{definition}[Bounded Linear Operator]
  An linear operator $T: X \mapsto Y$ is bounded if $\forall x \in X$ and $c > 0$ such that $\|Tx\|= \|T\|\|x\| \le c \|x\|$
\end{definition}

\begin{theorem}
  Let $T: \mathbb{D} \mapsto Y$ be a linear operator where $\mathbb{D} \subset X  $ and $X, Y$ are normed spaces, then
  \begin{enumerate}
    \item $T$ is continous if and only if T is bounded. 
    \item If $T$  is continous at a single point, $T$ is continious.  
  \end{enumerate}
\end{theorem}

\begin{proof}
  \begin{enumerate}
    \item  For $T = 0$ the statement is trivial. Let $T \neq 0$. Then $\|T\| \neq 0$. We Assume $T$ To be bounded and consider any $x_0 \in \mathbb{D}\left( T \right) $. Let any $\epsilon >  0$. Then, since $T$ is linear, for every $x \in \mathbb{D}\left(  T\right) $ such that 
      \[
      \|x - x_0\| < \delta \quad where \quad \delta = \frac{\epsilon}{\|T\|} 
      \] we obtain \[
      \|Tx- Tx_0\| = \|T\left( x - x_0 \right) \| \le \|T\| \|x - x_0\| < \|T\|\delta = \epsilon
    \]. Since $x_0 \in \mathbb{D}\left( T \right) $ was arbitary, this shows that $T$ is continous. 
    \par Conversely, assume that $T$ is continous at an arbitary $x_0 \in \mathbb{D}\left( T \right) $ then, given any $\epsilon > 0$, there is a $\delta > 0$ such that 
    \begin{equation}
    \label{eq:1}
      \|Tx- Tx_0\| \le \epsilon \quad \text{for all } x \in \mathbb{D}\left( T \right) \text{satisfying} \quad \|x- x_0\|\le \delta.       
    .\end{equation}
     We now take any $y \neq 0$ in $\mathbb{D}\left( T \right)  $ and set \[
    x = x_0+ \frac{\delta}{\|y\|} y. \quad \text{then} \quad x - x_0 = \frac{\delta}{\|y\|} y. 
  \]  Hence $\|x- x_0\| = \delta$,  so that we may use the result in \eqref{eq:1} . Since $T$ is linear we have  \[
  \| Tx_0 - Tx\| = \|T\left( x-x_0 \right)  \| =  \|T\left( \frac{\delta}{\|y\|}y \right) \| = \frac{\delta}{\|y\|} \|Ty\|
  \] and this implies \[
  \frac{\delta}{\|y\|}\|Ty\| \le \epsilon. \quad \text{Thus} \quad \|Ty \le \frac{\epsilon}{\delta}\|\|y\| 
  .\] This can be written $\|Ty\| \le  \|y\|$, where $c = \frac{\epsilon}{\delta}$ and shows that $T$ is bounded.  
\item Continuity of T at a point implies boundedness of $T$ by the second part of the proof of (a), which in turn implies boundedness of $T$ by (a).

  \end{enumerate}
\end{proof}

\bibliographystyle{plain}
\bibliography{references}
\end{document}


