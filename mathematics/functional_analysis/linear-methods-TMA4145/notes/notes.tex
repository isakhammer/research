\documentclass{article}
\usepackage[utf8]{inputenc}

\title{Linear Methods Exams}
\author{isakhammer }
\date{July 2020}

\usepackage{natbib}
\usepackage{graphicx}
\usepackage{amsmath}
\usepackage{amsthm}
\usepackage{amsfonts}
\usepackage{mathtools}
\usepackage{enumerate} 


\usepackage{hyperref}
\hypersetup{
    colorlinks=true, %set true if you want colored links
    linktoc=all,     %set to all if you want both sections and subsections linked
    linkcolor=blue,  %choose some color if you want links to stand out
}
\hypersetup{linktocpage}


% inscape-figures
\usepackage{import}
\usepackage{pdfpages}
\usepackage{transparent}
\usepackage{xcolor}

\newcommand{\incfig}[2][1]{%
    \def\svgwidth{#1\columnwidth}
    \import{./figures/}{#2.pdf_tex} } \pdfsuppresswarningpagegroup=1

% Box environment
\usepackage{tcolorbox} 
\usepackage{mdframed}
\newmdtheoremenv{definition}{Definition}[section]
\newmdtheoremenv{theorem}{Theorem}[section]
\newmdtheoremenv{lemma}{Lemma}[section]



\theoremstyle{remark}
\newtheorem*{remark}{Remark}



\begin{document}
\maketitle
\tableofcontents
\newpage

\section{Exam 18h}%
\label{sec:exam_18h}

\subsection{Problem 1}%
\label{sub:problem_1}


Determine whether the followeing statements are true or false. if the statement are true, no further explanation is required. If the statement is false, give a counter example.

  \begin{enumerate}
    \item The Kerner of a bounded linear operator $T: X \mapsto Y$ between normed spaces $X$ and $Y$ is closed.
      \begin{tcolorbox}
        \textbf{Answer.}  
        True
      \end{tcolorbox}
    \item The range of a bounded linear operator $T: X \to Y$ between normed spaces $X$ and $Y$ is closed.
      \begin{tcolorbox}
        \textbf{Answer.} 
      False. Lets assume that $X$ and $Y$ is closed. Then is this true. 
      \end{tcolorbox}
    \item The dual space $X^{'}$ of a normed space is a Banach Space.
      \begin{tcolorbox}
        \textbf{Answer.} 
        True.
      \end{tcolorbox}

    \item A closed subspace of a Banach Space is itself a Banish Space.
      \begin{tcolorbox}
        \textbf{Answer.} 
          True
      \end{tcolorbox}
  \end{enumerate}

\subsection{Problem 2}%
\label{sub:problem_2}

Let $\left( x_{k} \right)_{k \in \mathbb{N}} $ be a sequence in a normed space $\left( X, \|.\| \right) $. 

\begin{enumerate}
  \item[a)] Prove that $\left( x_{k} \right)_{k \in \mathbb{N}}$ is a Cauchy sequence, then $\left( x_{k} \right)_{k \in \mathbb{N}}$ is bounded. 
      \par
      \begin{tcolorbox}
        \textbf{Answer.} 
        We need to show that it exist $d\left( x_{m}, x_{n} \right)< \epsilon$.
        First let $x_{n} \mapsto x $, then does is this true $d\left( x_{n}, x \right) < \frac{\epsilon}{2}$ for an $n \ge N$. Using the triangle equality can we determine \[
        d\left( x_{n}, x_{m} \right) = d\left( x,x_{m} \right) + d\left( x, x_{n} \right) < \epsilon
        \] 
        This is then true. 
      \end{tcolorbox}
  \item[b)]  Let $\|.\|_{a}$ and $\|.\|_{b}$ be equivalent norms on $X$ and let $x \in X$. Prove that  $\left( x_{k} \right)_{k \in \mathbb{N}}$ converges to $x$ in $\left( X, \|.\|_a \right)$ if and only if $\left( x_{k} \right)_{ k \in \mathbb{N}}$ converges to $x$ in $\left( X, \|.\|_{b} \right)$.
      \begin{tcolorbox}
        \textbf{Answer.} 
        
        \begin{proof}
          Let $x_{n} \mapsto x$ and $x_{m} \mapsto x$. Then is $\|x_{n} - x\|_{a} < \frac{\epsilon}{2}$ for an $n > N_{a}$.  This also holds for $x_{m}$ such that $\|x_{m} - x\|_{b} < \frac{\epsilon}{2}$ for an $m > N_{b}$. If we let $m,n > \text{max}\left( N_{a}, N_{b} \right)$ then can we conclude that \[
          \|x_{n} - x\|_{a} + \|x_{m} - x\|_{b} < \epsilon
          .\] 
          Which proves that if $\|.\|_b$ is converging does this hold for $\|.\|_a$ for all $\left( x_n \right)_{n \in \mathbb{N}}$ 
        \end{proof}

      \end{tcolorbox}


\end{enumerate}


\subsection{Problem 3}%
\label{sub:problem_3}

Let $\left( \ell^{2}, \left<.,. \right> \right)$ be the inner product space of complex-valued sequences $x \in \left( x_{k} \right)_{k \in \mathbb{N}}$ equipped with the standard inner product 
\begin{equation}
\label{eq:3a}
\left<x,y \right> = \sum_{k=1}^{\infty} x_k \overline{y_k}  \quad \text{for} \quad  x,y \in \ell^{2}.
\end{equation}
and let $T: \ell^{2} \mapsto \ell^{2}$ be the multiplication operator given by \[
Tx = \left( i^{k} x_{k}/k \right)_{k \in \mathbb{N}}
\] 

where $i =\sqrt{-1} $.

\begin{enumerate}
  \item[a)] Show that $T $  is a bounded linear operator on $\ell^{2}$, and determine the operator norm $\|T\|$. 
    \begin{tcolorbox}
      \textbf{Answer.} 
      We want to show that $T$ is Cauchy. Let $\left( x_n \right)_{n \in \mathbb{N}}$ be a Cauchy sequence and let $\epsilon > 0$ such that $\| x_{n} - x\|< \frac{\epsilon}{2}$ for a $N$. By observing that \[
      Tx_{m} \mapsto Tx
      \] 
      can we use the argument such that $\|Tx_{m} - Tx\| = \|T\left( x_{m} - x \right)\| < \frac{\epsilon}{2} $ if $m > M$. Applying the triangle in equality can it be shown that \[
        \|Tx_{m} - Tx_{n}\| \le \|Tx_{m} - Tx\| + \|Tx_{n} - Tx\| < \epsilon \quad n,m = \text{max}\left( N,M \right)
      \]  
      And then shows that $T$ is bounded.
       \par
      The operator norm of $T$  is \[
      \|T\| = \text{sup}_{\substack{x_{k} \in X \\ \|x_{k}\| = 1} }  \frac{\|Tx_{k}\|}{\|x_{k}\|} = \|\frac{i^{k}}{k}\| = \frac{1}{k}
      \] 
    \end{tcolorbox}
  \item[b)] Determine the adjoint operator $T^{*}$. State what it means for an operator to be normal, and determine whether or not $T$ is normal. 
    \begin{tcolorbox}
      \textbf{Answer.} The adjoint operator should have this property, $\left<T^{*}y, y \right> = \left<y,Tx \right>$. 
    \end{tcolorbox}
  \item[c)] Show that the range of $T$ is dense in $\ell^{2}$. 


\end{enumerate}

\subsection{Problem 4}%
\label{sub:problem_4}
Let 
\[
A =
\begin{bmatrix} 
2  & 2 \\
2 & 2 \\
-1  & -1
\end{bmatrix} 
\] 
\begin{enumerate}
  \item[a)]  Find a singular value decomposition of $A$.
  \item[b)] Find the pseudoinverse $A^{+}$ of $A$ and use it to find the best approximation for a solution of the inconsistent system.
    \begin{align*}
      2x_1 + 2x_2 &= 3 \\
      2x_1 + 2x_2 &= 4 \\
      -x_1 - x_2 &= -4
    .\end{align*}
\end{enumerate}


\subsection{Problem 5}%
\label{sub:problem_5}

Find $a,b \in \mathbb{C}$ such that \[
\int_{0}^{2\pi} \left\lvert t - a\sin\left( t \right) - b  \sin\left( 2t \right) \right\rvert ^{2} dt 
\] 
Tip: You might find the formula $\left( \sin \left( t \right) \right)^2 = \frac{1- \cos\left( 2t \right)}{2}$ useful.

\subsection{Problem 6}%
\label{sub:problem_6}

\begin{enumerate}
  \item[a)] Show that if $X \neq \O$ is a complete metric space, and $T: X \to X$ is a mapping such that \[
  T^{k} = T \cdot T\cdot \ldots \cdot T
\] Is a contraction for some natural number $k> 1 $, then $T$ has a unique fixed point. 
\item[b)]  Consider the space of continuous functionsa $C\left[ 0,1 \right]$ equppised with the metric induced by the supremenum norm \[
    d\left( f,g \right) = \|f-g\|_{\infty} \sup_{0\le t \le 1} \left| f\left( t \right) - g\left( t \right)  \right|
\] 
and let $T: C\left[ 0,1 \right] \to C\left[ 0,1 \right]$ be given by \[
\left( Tf \right)\left( t \right) = 1 - \int_{0}^{t} f\left( s \right) ds , \quad 0 \le t \le 1. 
\] 
Show that $T$ has a unique fixed point, and use iteration to find it starting with $f_0\left( t \right) = 1 $ 
\par 
\textit{Tip: You can use the results from  a) even if you did not solve this problem.} 
\end{enumerate}

\newpage
\section{Appendix}%
\label{sec:Notes}

\subsection{Sequences in metric spaces and normed spaces}%
\label{sub:sequences_in_metric_spaces_and_normed_spaces}

\begin{definition}[Norm]
  Criterias for norms
  \begin{enumerate}[(i)]
    \item $\|cx\| = c\|x\|$
    \item $\|xy\| \le \|x\| \|y\|$
    \item $\|x + y\| \le \|x\| +  \|y\|$
    \item $\|x\| = 0 \quad \text{only if} \quad x = 0 $ 
  \end{enumerate}

\end{definition}


\begin{definition}[Sequence]
  Let $\left( X,d \right) $ be a metric space. A sequence $\left( x_{n} \right) _{n \in \mathbb{N}}$ in $X$ is said to \textbf{converge to} $x \in X$ for every $\epsilon > 0$ one can find $N=N(\epsilon) \in \mathbb{N}$ such that \[
    d\left( x_n, x \right) <  \epsilon 
  .\] whenever $b \ge N$. The element $x$ is called the \textbf{limit} of the sequence $\left( x_{n} \right) _{n \in \mathbb{N}}$. In particular, in $\left( X,\|.\| \right) $ is a normed space. then $\left( x_{n} \right) _{n \in \mathbb{N}}$ converge to $x \in X$ for every $\epsilon > 0$ one can find $N = N\left( \epsilon \right) \in \mathbb{N}$ such that \[
  \|x - x_{n} \| < \epsilon
  .\] whenever $n \ge N$..
\end{definition}

\begin{definition}
  Given a point $x_0 \in X$ and a real number $r>0$, we define three types of sets:
  \begin{enumerate}[(i)]
    \item $B\left( x_0; r \right)  = \{x\in X  \mid d\left( x,x_0 \right) < r\} $ \textbf{(Open ball)} 
    \item $\hat{B}\left( x_0; r \right)  = \{x\in X  \mid d\left( x,x_0 \right) \le r \} $ \textbf{(Closed ball)} 
    \item $S\left( x_0; r \right)  = \{x\in X  \mid d\left( x,x_0 \right) = r\} $ \textbf{(Sphere)} 
  \end{enumerate}

  Here is $x_0$ called the center and $r$ the radius. Remark that $S\left( x_0, r \right)  = \hat{B}\left( x_0,r \right) - B\left( x_0,r \right) $.
\end{definition}


\begin{definition}[Open and Closed Set]
  A subset $M$ of a metric space $X$ is said to be open if it contains a ball around each of its points. A subset $K$ of $X$ is said to be closed if its complement (in $X$ ) is open, that is, $K^{c} = X - K$ is open.  
\end{definition}

\begin{remark}
  A complement set is defined such that $A^{c} = U \setminus A $ or more formally $A^{c} = \{x \in U  \mid x \not\in A \} $
\end{remark}

\begin{lemma}
  A convergent sequence in a metric space $\left( X,d \right) $ is bounded. 
\end{lemma}

\begin{definition}[Dense Set]
  Formally, $S \subset X$ is dense in $X$ if, for any $\epsilon > 0$ and $x \in X$, there is some $s \in S$ such that $\|x-s\| < \epsilon$. An equivalent definition is that $S$ is dense in $X$ if, for any $x \in X$, there is a sequence $\{x_{n}\} \subset S$ such that \[
  \lim_{n \to \infty} x_{n} = x
  \] 
  
\end{definition}


\subsection{Linear Operator}%
\label{sub:linear_operator}

\begin{definition}
  A linear operator $T$ is an operator such that 
  \begin{enumerate}
    \item the domain $\mathbb{D}\left( T \right) $ of $T$ is a vector space and the range $R\left( T \right) $ lies in a vector space over the same field. 
    \item  $\forall x,y \in \mathbb{D}\left( T \right) $ and scalars $\alpha$ 
      \begin{equation}
      \label{eq:linear-operator}
      T\left( x + y \right) = Tx + Ty \quad \text{and} \quad T\left( \alpha x \right) = \alpha Tx
      .\end{equation}
  \end{enumerate}

\end{definition}

\begin{definition}[Bounded Linear Operator]
  An linear operator $T: X \mapsto Y$ is bounded if $\forall x \in X$ and $c > 0$ such that $\|Tx\|= \|T\|\|x\| \le c \|x\|$
\end{definition}
\begin{remark}
  What is the smallest possible $c$ such that $\|Tx\| \le c \|x\|$ still hold for all non-zero $x \in \mathbb{D}\left( T \right) $? (We can leave out $x=0$ since $Tx=0$ for $x=0$) By division, \[
  \frac{\|Tx\|}{\|x\|} \le c.
\] and this shows that  $c$ must be at least as big as the supremum of the expression on the left taken over the range $\mathbb{D}\left( T \right) - \{0\} $. Hense the answer to our question is that the smallest possible c is that supremum. This quantity denoted by $\|T\|$, thus \[
\|T\| = \sup_{\substack{ x \in \mathbb{D}\left( T \right) \\ x \neq 0}}  \frac{\|Tx\|}{\|x\|}
\] 
$\|T\|$ is called the \textbf{norm} of the operator $T$. If the range $\mathbb{D}\left( T \right) = \{0\} $, we define $\|T\| = 0$. Note that with $c = \|T\|$ is \[
\|Tx\| \le  \|T\| \|x\| 
\] which is a quite frequently used formula. 

\end{remark}

\begin{lemma}
  Let $T$ be a bounded linear operator. Then is this true,
  \begin{enumerate}[(i)]
    \item 
\[
.
        \|T\| = \sup_{\substack{x \in \mathbb{D}\left( T \right) \\ x \neq 0 } }  \frac{\|Tx\|}{\|x\|}= \sup_{\substack{x \in \mathbb{D}\left( T \right) \\ x = 1 } } \|Tx\| 
\] 
    \item The norm satisfy general norm aksioms.
  \end{enumerate}
\end{lemma}

\newpage
\begin{proof}
  \begin{enumerate}[(i)]
    \item Let $\|x\| = a $ and define $y = \frac{x}{a}$. Using this definition can we see that $ \|y\|= 1$. Hense can we rewrite the definition. \[
        \sup_{\substack{x \in \mathbb{D}\left( T \right) \\ x \neq 0} } \frac{\|Tx\|}{\|x\|} = \sup_{\substack{x \in \mathbb{D}\left( T \right) \\ x \neq 0} } \frac{\|Tx\|}{a} = \sup_{\substack{x \in \mathbb{D}\left( T \right) \\ x \neq 0} } \|\frac{Tx}{a}\| = \sup_{\substack{y \in \mathbb{D}\left( T \right)  \\ y = 1 }  }  \|Ty\|
    \] 

    \item We need to prove that it satisfy the criteria $\|cT\| = c\|T\|$ and $\|T_1 + T_2\| \le \|T_1\| + \|T_{2}\| $. 
      \begin{align*}
        \|cT\| &= \sup_{\substack{ y \in \mathbb{D}\left( T \right) \\ \|y\| = 1 } }  \|Tcy\|  = \sup_{\substack{ y \in \mathbb{D}\left( T \right) \\ \|y\| = 1 } }  c\|Ty\|   \\ &=  c \|T\| 
      .\end{align*}

      \begin{align*}
        \label{eq:label}
        \|T_{1} + T_{2}\| &= \sup_{x \in \mathbb{D}\left( T \right), \|x\| = 1} \|\left(  T_1x + T_2x \right)\|  \le \sup_{x \in \mathbb{D}\left( T \right), \|x\| = 1} \| T_1x \|  +\|T_2x\| \\ &= \|T_1\| + \|T_2\| 
      .\end{align*}
  \end{enumerate}
\end{proof}


\begin{theorem}
  Let $T: \mathbb{D} \mapsto Y$ be a linear operator where $\mathbb{D} \subset X  $ and $X, Y$ are normed spaces, then
  \begin{enumerate}
    \item $T$ is continous if and only if T is bounded. 
    \item If $T$  is continous at a single point, $T$ is continious.  
  \end{enumerate}
\end{theorem}

\begin{proof}
  \begin{enumerate}
    \item  For $T = 0$ the statement is trivial. Let $T \neq 0$. Then $\|T\| \neq 0$. We Assume $T$ To be bounded and consider any $x_0 \in \mathbb{D}\left( T \right) $. Let any $\epsilon >  0$. Then, since $T$ is linear, for every $x \in \mathbb{D}\left(  T\right) $ such that 
      \[
      \|x - x_0\| < \delta \quad where \quad \delta = \frac{\epsilon}{\|T\|} 
      \] we obtain \[
      \|Tx- Tx_0\| = \|T\left( x - x_0 \right) \| \le \|T\| \|x - x_0\| < \|T\|\delta = \epsilon
    \]. Since $x_0 \in \mathbb{D}\left( T \right) $ was arbitary, this shows that $T$ is continous. 
    \par Conversely, assume that $T$ is continous at an arbitary $x_0 \in \mathbb{D}\left( T \right) $ then, given any $\epsilon > 0$, there is a $\delta > 0$ such that 
    \begin{equation}
    \label{eq:2}
      \|Tx- Tx_0\| \le \epsilon \quad \text{for all } x \in \mathbb{D}\left( T \right) \text{satisfying} \quad \|x- x_0\|\le \delta.       
    .\end{equation}
     We now take any $y \neq 0$ in $\mathbb{D}\left( T \right)  $ and set \[
    x = x_0+ \frac{\delta}{\|y\|} y. \quad \text{then} \quad x - x_0 = \frac{\delta}{\|y\|} y. 
  \]  Hence $\|x- x_0\| = \delta$,  so that we may use the result in \eqref{eq:2} . Since $T$ is linear we have  \[
  \| Tx_0 - Tx\| = \|T\left( x-x_0 \right)  \| =  \|T\left( \frac{\delta}{\|y\|}y \right) \| = \frac{\delta}{\|y\|} \|Ty\|
  \] and this implies \[
  \frac{\delta}{\|y\|}\|Ty\| \le \epsilon. \quad \text{Thus} \quad \|Ty \le \frac{\epsilon}{\delta}\|\|y\| 
  .\] This can be written $\|Ty\| \le  \|y\|$, where $c = \frac{\epsilon}{\delta}$ and shows that $T$ is bounded.  
\item Continuity of T at a point implies boundedness of $T$ by the second part of the proof of (a), which in turn implies boundedness of $T$ by (a).

  \end{enumerate}
\end{proof}

\subsection{Banach Spaces}%
\label{sub: Banach Spaces}

\begin{definition}[Cauchy Sequence]
  
  Let $(x_n)_{n \in \mathbb{N}}$ be a sequence in the metric space $(X, d)$. We say that $\left( x_n \right)_{n \in \mathbb{N}} $ is \textbf{Cauchy Sequence} if for any $ \epsilon > 0$ there exist an $N \in \mathbb{N}$ such that \[  
    d(x_{n}, x_{m}) < \epsilon      
  .\]  
  In particular if $\left( x_n \right)_{n \in \mathbb{N}}$ is a sequence in the normed space $\left( X,\|.\| \right) $, then $\left( x_n \right) _{n \in\mathbb{N}}$ is Cauchy if for any $\epsilon > 0$ there exist an $N \in \mathbb{N}$ such that

  \[
  \|x_{n}  - x_{m}\| < \epsilon,\quad \textrm{s.t.} \quad  n,m \ge N
  .\] 
  In an inner product space $(X, \left< .,. \right> )$, we say that a sequenxe $\left( x_n \right)_{n \in \mathbb{N}}$ is Cauchy if the sequence is Cauchy with respect to the indeuced norn $\|x\| := \left< x,x \right>^{ \frac{1}{2} }$ . 
\end{definition}

\begin{lemma}
  Any Cauchy sequence in $\left( X,d \right)$ is bounded.
\end{lemma}

\begin{proof}
  Let $\left( x_n \right) _{n \in \mathbb{N}}$ be a Cauchy sequence. Then there exist $N \in \mathbb{N}$ such that for all $ m,n \ge N$ we have \[
    d(x_{m}, x_{n}) < 1
  .\] In particular, we have \[
  d(x_{N},x_{m})  < 1 \quad \forall \quad m \ge N      
  .\] 
  Or equivalently $x_{m} \in B_1(x_N)$ for all $m \ge N$. Now let \[
    r = max \{ 1, d(x_1, x_N), d(x_2, x_N), \ldots, d(x_{N-1}, x_N)\} 
  .\] 
  Then for any $n \in \mathbb{N}$ we have $x_n \in B_{r+1}\left( x_N \right) $\, so $\left( x_n \right) _{n \in \mathbb{N}}$
is bounded. 

\end{proof}

\begin{remark}
A set is  \textbf{closed} if the set contains all of its boundary points ( the closure of the set is equal to the set). There are some other definitions for closed also.  A set is \textbf{bounded} if the distance between any two points in the set is less then some finite constant. A set in $\mathbb{R}^{n}$ is bounded if all of the points are contained within a disc of finite radius.
\end{remark}



\begin{definition}[Completeness]
  A sequence $\left( x_{n} \right)_{n \in \mathbb{N}}$ in a metric space $X=\left( X,d \right)$ is said to be Cauchy (or fundemental) if for every $\epsilon> 0$ there is an $N = N\left( \epsilon \right)$ such that $d\left( x_{m}, x_{n} \right) < \epsilon$  for every $m,n \ge N$. The space $X$ is said to be complete if every Cauchy sequence in $X$ converges (that is, has a limit which is an element of $X$). 
\end{definition}

\begin{remark}[Procedure for Completeness proofs]
  
  To prove completeness do we choose an arbitary Cauchy sequence $\left( x_n \right)_{n \in \mathbb{N}}$ in $X$ and show that it does converge in $X$. They often have the same pattern.

  \begin{enumerate}
    \item Contruct an element $x$ (to be used as an limit).
    \item Prove that x is in the space considered.
    \item Prove convergence $x_{n} \mapsto x$
  \end{enumerate}
\end{remark}

\begin{theorem}[Convergent sequences]
  Every convergent sequences in a metric space is a Cauchy Sequence.
\end{theorem}
\begin{proof}
  Let $x_{n} \mapsto x$ for $x \in X$, then is for an $N = N\left( \epsilon \right)$\[
  d\left( x_{n},x \right) < \frac{\epsilon}{2} \quad \text{for any} \quad  n > N 
  .\] 
  To prove that this is Cauchy can we use the triangulation theorem such that \[
  d\left( x_{n}, x_{m} \right) \le d\left( x,x_{n} \right) + d\left( x, x_{m} \right) < \epsilon \quad \text{such that} \quad m,n \ge N\left( \epsilon \right)
  \]  
  This proves that $\left( x_{n} \right)_{n \in \mathbb{N}} $ is Cauchy.
  
\end{proof}

\begin{definition}[Banach Space and Hilbert Space]
  A metric space $\left( X,d \right) $ is said to be complete if every Cauchy sequence $\left( x_n \right) _{x_{n}\in\mathbb{N}} \in X$ converges to a limit $x \in X$. A complete normed space $\left( X,\|.\| \right) $ is classed a Banach Space. Similary, a complete inner product space $\left( X, \left<.,. \right> \right) $ is called a Hilbert space. 
\end{definition}

\begin{theorem}
  Let $\left( f_n \right) $ be a sequence of continious functions on $[a,b]$ which converges uniformly to a limit function $f$. Then $f$ is continious on $[a,b]$.
\end{theorem}

\begin{proof}
  We want to show that for any fixed $y \in [a,b]$ and $\epsilon > 0$ we can find a $\delta > 0 $ such that \[
    \|x-y\| < \delta \quad \implies \quad \|f\left( x \right) - f\left( y \right) \| < \epsilon 
  \] By the uniformly convergence $\left( f_n \right) $ to $f$, there exist an $N$ such that \[
  \|f_n\left( x \right) - f\left( x \right) \| < \epsilon \quad \text{for all} \quad x \in [a,b], n\ge N. 
  \] 
  Moreover, the function $f_n$ is continious, so there exist a $\delta >0 $ such that \[
    \|x-y\| < \delta \quad \implies \quad \|f_N\left( x \right) - f_N\left( y \right) \| < \frac{\epsilon}{3}. 
  \] 
  It follow that 
  \begin{align*}
    \|f\left( x \right) - f\left( y \right) \| &\le \|f\left( x \right) - f_N\left( x \right) \| - \|f_N\left( x \right) - f_N\left( y \right) \| + \|f_N\left( y \right) - f\left( y \right) \|
                                               < \frac{\epsilon}{3} + \frac{\epsilon}{3} + \frac{\epsilon}{3} = \epsilon
  .\end{align*}

  whenever $\|x-y\| < \delta$
\end{proof}


\begin{theorem}
  $\left( C[a,b] , \|.\|_{\infty} \right) $ is a Banach Space 
\end{theorem}

\begin{proof}
  \begin{enumerate}[(i)]
    \item \textbf{Find a candidate for the limit}  
    \par  
    Fix $x \in [a,b]$ and note that \[
      \|f_{n}\left( x \right)  - f\left( x \right) \| \le \|f_n - f_m\|_{\infty} = \text{max}_{a \le x \le b } \|f_n\left( x \right) - f_m\left( x \right) \|.
    \] 

    This if $\left( f_n \right) $ is a Cauchy sequence in $\left( C[a,b], \|.\|_{\infty} \right) $, then $\left( f_n\left( x \right)  \right)_{n \in \mathbb{N}} $ is a Cauchy Sequence in $\left( \mathbb{R}, \|.\| \right) $. Since $\left( \mathbb{R}, \|.\| \right) $ is complete, there exist a point $f\left( x \right) \in \mathbb{R}$ such that $f_{n}\left( x \right) \mapsto f\left( x \right) $. A reasonable candidate for the limit is the function $f$ given by the pointwise limits. 

  \item \textbf{Show that $f \in C[a,b]$} 
      \par
      We observe that the convergence of $f_n$ to $f$ is not onlypointwise, but in fact uniform; Since $\left( f_n \right) $ is Cauchy, there is for every $\epsilon > 0 $ an integer $N$ such that \[
        \|f_n - f\|_{\infty} =  \text{max}_{a \le x \le b} \|f_n\left( x \right) -  f_m\left( x \right) \| < \frac{\epsilon}{2}, \quad n,m \ge N \quad  
      \] 
      In particular, this hold as $m \mapsto \infty$, and we get 
      \begin{equation}
      \label{eq:1}
      \text{max}_{a \le x \le b} \|f_n\left( x \right)  - f\left( x \right) \| \le \frac{\epsilon}{2} < \epsilon, \quad n\ge N \quad  
      .\end{equation}

      Thus, $f_n$ converges uniformly to $f$ on the interval $[a,b]$, and it follows by Theorem 3.13 (linear method lecture notes) that $f \in C[a,b]$ .

    \item \textbf{Show that $f_n \mapsto f$} 
        \par
        Follows from \eqref{eq:1} 

  \end{enumerate}
\end{proof}

\subsection{Banach Fixed Point}%
\label{sub:banach_fixed_point}

\begin{definition}[Contraction]
  Let $X = \left( X,d \right)$ be a metric space. A mapping $T: X \mapsto X$ is called a \textbf{contraction}  on X if there is a positive real number $\alpha < 1$ such that for all $x,y \in X$ \[
    d\left( Tx,Ty \right) \le \alpha d\left( x,y \right) \quad , \alpha  < 1 
  \] 
  Geometrically this means that any point $x$ and $y$ have images that are closer together than those points $x$ and $y$; more precisely, the ratio  \[
\frac{d\left( Tx, Ty \right)}{d \left( x,y \right)}
  \] 
  does not exceed a constant $\alpha $ which is strictly less than $1$.
\end{definition}

\begin{theorem}[Banach Fixed Point Theorem]
  Consider a metric space $X=\left( X,d \right)$, where $X \neq \O$. Suppose that $X$ is complete and let $T: X \mapsto X$ be a contraction on $X$. Then $T$ has precisely one fixed point.
\end{theorem}

\begin{proof}
  We constrict a sequence $\left( x_{n} \right)$ and show that it is Cauchy so that it converges in the complete space $X$, and then we prove that its limit $x$ is a fixed point on $T$ and $T$ has no further fixed points. This is the idea of the proof. 
  \par
  We choose any $x_0 \in X$ and define the "iterative sequence " $\left( x_{n} \right)$ by 
  \begin{equation}
  \label{eq:fp_1}
    x_0, \quad  x_1 = Tx_0, \quad  x_2 =Tx_1 = T^{2}x_0 \quad \ldots \quad x_{n} = T^{n} x_0, \quad   \ldots    
  .\end{equation}
  Clearly, this is the sequence of the image of $x_0$ under repeated application of $T$. We show that $\left( x_{n} \right)$ is Cauchy by the contraction definition and \eqref{eq:fp_1} , 

  \begin{align}
    d\left( x_{m+1}, x_{m} \right) &= d\left( Tx_{m}, Tx_{m-1} \right) \\
    &\le \alpha d\left( x_{m}, x_{m-1} \right)  \\
    &= \alpha d\left( Tx_{m-1}. Tx_{m-2} \right) \\
    &\le \alpha ^{2} d\left( x_{m-1}, x_{m-2} \right) \\
    \ldots&= \alpha ^{m} d\left( x_1, x_0 \right) \\
  .\end{align}
  Hense by the triangle inequality and the formula for the sum of a geometric progression we obtain for $n\ge m$.
  \begin{align*}
    \left( x_{m}, x_{n} \right) &\le d\left( x_{m}, x_{m+1}  \right) d\left( x_{m+1}, x_{m+2} \right) +  \ldots  + d\left( x_{n-1}, x_{n} \right) \\
  &\le \left( \alpha ^{m} + \alpha ^{m+1} + \ldots + \alpha ^{n-1}  \right) d\left( x_0, x_1 \right)  \\
  &= \alpha ^{m} \frac{1- \alpha ^{n-m}}{1- \alpha } d\left( x_0, x_1 \right) \\
  .\end{align*}
  Since $0 < \alpha  <1 $, in the numerator we have $1 - \alpha ^{n-m} < 1 $. Consequently
  \begin{equation}
  \label{eq:fp_3}
  d\left( x_{m}, x_{n} \right) \le \frac{\alpha ^{m}}{1- \alpha } d\left( x_0, x_1 \right), \quad n>m. 
  .\end{equation}
  On the right is $0< \alpha  < 1$and $d\left( x_0,x_1 \right) $ is fixed, so that we can make the right-hand side as small as we please by taking $m$ sufficiently large (and $n>m$). This proves that $\left( x_{m} \right)$ is Cauchy. Since $X$ is complete, $\left( x_{m} \right)$ converges, say, $x_{m} \mapsto x$. We show hat this limit $x$ is a fixed point of the mapping $T$.
  \par
  From the triangle inequality and the contraction theorem  we have
  \begin{align}
    d\left( x,Tx \right) &= d\left( x,x_{m} \right) +  d\left( x_{m}, Tx \right) \\
    &\le d\left( x, x_{m} \right) +  \alpha d\left( x_{m-1}, x \right) 
  .\end{align}
  and can make the sum in the second line smaller than any preassigned $\epsilon > 0$ because $x_{m} \mapsto x$. We conclude that $d\left( x,Tx \right) = 0$, so that $x=Tx$. This shows that $x$ is a fixed point of $T$ . 
  \par
  $x $ is the only fixed point of $T$ because from $Tx= x$ and $T\hat{x} = \hat{x}$ we obtain by \[
    d\left( \hat{x}, x \right) = d\left( T \hat{x}, Tx \right) \le \alpha d\left( \hat{x}, x \right)
  \] 
  Which implies $d\left( \hat{x}, x, \right) = 0$ since $\alpha  < 1 $. Hense $x = \hat{x}$ and the theorem is proved. 
   
  


\end{proof}


\subsection{Hilber Spaces}%
\label{sub:hilber_spaces}
\begin{definition}[Separable]
  A metric space is said to be \textbf{separable} if it contains a countable dense set
  \[
 X \quad \text{separable} \leftrightarrow \left( x_{n} \right)_{n \in \mathbb{N}} \mathbb{C} \quad \text{such that} \quad   \overline{\left( x_{n} \right)_{n \in\mathbb{N}} } = X  .
  \] 
\end{definition}



\begin{definition}[Inner product space, Hilbert space]
  An inner product space (or pre-Hilbert space) is a vector space $X$ with an inner product defined on  $X$. A Hilbert space is a complete inner product space. Here, an \textbf{inner product} on $X$ is a mapping from $X \times X $ into the scalar field $K$ of $X$ ; that is,  with every pair of vectors $x$ and $y$ there is assiciated a scalar which is written \[
  \left<x,y \right>
  \] and is called the inner product of $x$ and  $y$ such that for all vectors $x,y,z$ and scalars $\alpha$ we have 
  \begin{enumerate}
    \item[IP1)] $\left< x +y ,z\right> = \left<x,z \right> + \left<y,z \right>$ 
    \item[IP2)] $\left<\alpha x , y \right> = \alpha \left<x,y \right> $
    \item[IP3)] $ \left<x,y \right> =\overline{\left<  y,x\right>} $
    \item[IP4)] $ \left<x,x \right> \ge 0 \quad \text{and} \quad \left<x,x \right> = 0 \quad \implies \quad x= 0    $
  \end{enumerate}
  
\end{definition}

\begin{definition}[Hilbert-adjoint operator ]
  Let $T: H_1 \mapsto H_2$ be a bounded linear operator, where $H_1$ and $H_2$ then the Hilbert adjoint operator $T^{*}$ of $T$ is the operator \[
  T^{*}: H_2 \mapsto H_1 .
  \] 
  Such that for all $x \in H_1$ and $y \in H_2$ 
  \begin{equation}
  \label{eq:ajoint}
  \left<Tx,y \right> = \left<x,  T^{*}y \right>
  .\end{equation}
  
\end{definition}

\begin{theorem}[Properties of Hilber adjoint operators]
  Let $H_1$ and $H_2$ be hilbert spaces, $S: H_1 \mapsto H_2$ and $T: H_1 \mapsto H_2$ bounded linear operators and $\alpha$ any scalar. Then we have
  \begin{align}
    \left<T^{*}y,x \right> &= \left<y,Tx \right> \\
    \left( S +T \right)^{*} &= S^{*} + T^{*} \\
    \left( \alpha T \right)^{*} &= \hat{\alpha} T^{*} \\
    \left( T^{*} \right)^{*} &=   T^{*}\\  
    \|T T^{*}\| &=  \|T^{*} T\| =  \|T\|^{2} \\
    T^{*}T &= 0 \quad \implies \quad T=0   \\
    \left( ST \right)^{*} &= T^{*}S^{*} 
  .\end{align}
  
\end{theorem}

\begin{definition}[Self, Adjoint, unitary and normal operators]
  A bounded linear operator $T: H \mapsto H$ on a Hilbert space $H$ is said to be 
  \begin{align*}
    \text{Self adjoint or Hermition} \quad T^{*} &= T ,\\
    \text{Unitary if $T$ is bijective and} \quad T^{*} &= T^{-1} ,\\
    \text{Normal if} \quad TT^{*} &= T^{*}T. \\ 
  .\end{align*}
  
\end{definition}


\subsection{Series and Normes}%
\label{sub:series_and_normes}

\begin{definition}[Hamel Basis]
  We call a linearly independent set $S$ of a vector space $X$ a \textbf{Hamel basis} if $S$ spans $X$, i.e. if any $x \in X$ has a unique and finite representation. \[
  x  = a_1 x_1 + \ldots + a_{n} x_{n}, \quad x_{j} \in S, a_{j} \in \mathbb{F} 
  \] 
  
\end{definition}

\begin{theorem}[Finite-dimensional norm equivalence]
  On a finite-dimensional vector space $X$, all norms are equivalent. For instance, all norms are quivalent on $\mathbb{R}^{n}$
\end{theorem}


\subsection{Common}%
\label{sub:common}

\begin{definition}[Range]
  A range of a function $f: X \mapsto Y$, is denoted by $range\left( f \right) $ or $f\left( X \right) $, is the set of all $y \in Y$ that are the image of some $x \in X$. More compact can this be written. \[
    range\left( f \right) = \{y \in Y  \mid \text{there exist } x \in X \text{ such that } f\left( x \right) = y\} 
  \] 
\end{definition}

\begin{definition}
  Let $f: X \mapsto Y$ be a function. 
  \begin{enumerate}
    \item We call $f$ injective or one-to-one if $f\left( x_1 \right)  = f\left( x_2 \right) $ implies $x_1=x_2$, i.e, no two elements of the domain have tha same image. Equivalently, if $x \neq x_2$ then $f\left( x_1 \right) \neq f\left( x_2 \right) $. 
    \item We call $f$ surjective or onto if $range\left( f \right) = Y$, i.e each $y \in Y$ is the image of at least one $x \in X$ .
    \item We call $f$ bijective if f is both injective and surjective.
  \end{enumerate}
\end{definition}

\begin{definition}[Testing]
  I am a big test 
  
\end{definition}

\begin{definition}[Closed Set]

  Let $X$ be a subset of a set $Y$.
  If $X$ is closed is this true.
  \begin{enumerate}[(i)]
    \item The compliment $X^{c}$ is an open set.
    \item $X$ is it own set closure.
    \item Sequences/nets/filters in $X$ that converge do so in $X$.
    \item Every point outside  $X$ has a neightbourhood disjoint from $X$
  \end{enumerate}

\end{definition}

\bibliographystyle{plain}
\bibliography{references}
\end{document}

