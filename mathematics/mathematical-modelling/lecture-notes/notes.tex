\documentclass{article}
\usepackage[utf8]{inputenc}

\title{Mathemathical Modelling}
\author{isakhammer }
\date{2020}

\usepackage{natbib}
\usepackage{graphicx}
\usepackage{amsmath}
\usepackage{amsthm}
\usepackage{amsfonts}
\usepackage{mathtools}
\usepackage{enumerate}
\usepackage{todonotes}


\usepackage{hyperref} 
\hypersetup{
  colorlinks=true, %set true if you want colored links
  linktoc=all,     %set to all if you want both sections and subsections linked
  linkcolor=blue,  %choose some color if you want links to stand out
} 
\hypersetup{linktocpage}


% inscape-figures
\usepackage{import}
\usepackage{pdfpages}
\usepackage{transparent}
\usepackage{xcolor}
\newcommand{\incfig}[2][1]{%
\def\svgwidth{#1\columnwidth}
\import{./figures/}{#2.pdf_tex} } \pdfsuppresswarningpagegroup=1

% Box environment
\usepackage{tcolorbox}
\usepackage{mdframed}
\newmdtheoremenv{definition}{Definition}[section]
\newmdtheoremenv{theorem}{Theorem}[section]
\newmdtheoremenv{lemma}{Lemma}[section]

\theoremstyle{remark}
\newtheorem*{remark}{Remark}
\newtheorem{example}{Example}


\begin{document}
\maketitle
\tableofcontents
\newpage
\newpage
\section{Lecture 1}%
\label{sec:lecture_1}

\subsection{Practical Information}%
\label{sub:practical_information}

You need to know 
\begin{itemize}
  \item Separable 1. order equations.
  \item Linear 1. order equations.
  \item 2. order linear equations with constant coefficients.
\end{itemize}

\subsection{Dimensional Analysis}%
\label{sub:dimensional_analysis}

Basic facts
\begin{itemize}
  \item Any physical relation has to make sense dimensionally.
  \item Any physical relation must be valid for any choice of fundamental units.
\end{itemize}
\begin{remark}
  \todo{ Make sure remark looks better }
  \begin{itemize}
    \item
  \textbf{Forbidden} $3m + 2kg = ?$ 
\item $m = f\left( x,t \right)$ is legal
\item  $e^{-t}$ and $s = 5t ^{2}$ , is nonsense
  \end{itemize}
\end{remark}

\begin{itemize}
  \item
\textbf{Dimension}  is length, mass , energy, etc.
\item
\textbf{Unit} is meter, feet, year, etc
\end{itemize}
Given a variable $R$, we write $R =\overbrace{v\left( R \right)}^\text{numerical value}   \underbrace{\left[ R \right]}_\text{unit}$.   
\par
If we have a physical relation that is dimensionall correct that \[
f\left( R_{1}, R_{2}, \ldots, R_{n} \right) = 0 \quad  \to \quad f\left( v\left( R_{1} \right), v\left( R_{2} \right), \ldots, v\left( R_n \right) \right)   = 0
\] 
\subsection{Fundamental Units}%
\label{sub:fundamental_units}

Given units $F_{1}, F_{2}, \ldots , F_{m}$ for fundamental if \[
  F_{1}^{\alpha_1}, F_{2}^{\alpha_2}, \ldots , F_{m}^{\alpha m} = 0 \quad  \to \quad  \alpha_1 = \alpha_{2} = \ldots = 0   
\] 
This units are then independent.
\begin{example}
  The units $kg, m, s$ are independent.  
\end{example}
\begin{example}
  In a right angle triangle with angle $\alpha $ and hypothenus $c$. We know the area $A$ is uniquely determined by $\alpha $ and $c$ \[
  A = f\left( c,\alpha  \right)
  \] 
  $\alpha $ is dimensialless since  $\alpha  = \frac{s}{r}$. Since $A$ scales as the square of the length, then is \[
  f\left( ac, \alpha  \right) = a^2f\left( c,\alpha  \right)
  \]   
  \[
    c = 1 \to f\left( a, \alpha  \right) = a^2f\left( 1,\alpha  \right) = a^2h\left( \alpha  \right)
  \] 
  Which then ends up with the relation \[
  A = a^2h\left( \alpha  \right)
  \] 
\end{example} 

\todo[inline]{ Make corollary environmet }


Lets derive $A = a ^2 h\left( \alpha   \right)$ somwhat differently. We know there is a relation $f\left( A, c, \alpha  \right) = 0$ . We want to introduce new variables.\[
\Pi_1 = \frac{A}{c^2}, \quad  c = c_1, \quad \alpha = \alpha _1   
\] 
which means $f\left( c^2 \Pi_1, c, \alpha   \right) = 0$  and $h\left( \Pi_1, \alpha , c \right) = 0$. $h$ must be dimensially consistent $\to$ $h$ must be independent of $c$. 
\begin{equation*}
  \begin{split}
    h\left( \Pi_1, \alpha  \right) &= 0 \leftrightarrow \Pi_1 = k\left( \alpha  \right) \\
    \to  \frac{A}{c^2} &= k\left( \alpha  \right) \quad   \leftrightarrow \quad A = c^2k\left( \alpha  \right) 
  \end{split}
.\end{equation*}

\subsection{Trinity of the first atomic blast}%
\label{sub:trinity_of_the_first_atomic_blast}

We assume there is a relation \[
f\left( E, \rho, r , t \right) = 0
\] 

\begin{itemize}
  \item Energy: $E$, $\left[ E \right] = kg m^2 s^{-2}$ 
  \item Mass density of air: $\rho$, $\left[ \rho \right] = kg ^{-3}$
  \item Radius: $r$ , $\left[ r \right] = m$
  \item Time: $t$, $\left[ t \right] =s$ 
\end{itemize}

We choose 3 independent variables, say $r, t, \rho$. Also we call $r, t, \rho  $ \textbf{core variables}.  Let is define a dimensionalless number $\Pi_{1} $ such that \[
\left[ \Pi _{1} \right] = 0
\] 

The relation is now given by $h\left( \Pi , t, r, \rho \right) = 0$, where $h$ is independent of $t$ , $r$ and $\rho$. Which in fact is $h\left( \Pi  \right) = 0$, where $\Pi _{1} = c$ s.t. $\left[ c \right] = 1$.  
\par
Given by the definitnion is \[
\frac{E t^{2}}{\rho r ^{5}} = c \quad  \to \quad  E = \frac{c \rho r^{5}}{t^{2}}  
\] 
Using $\rho = 12 kgm^{-3}$, $r = 110m$ , $t = 6 \cdot 10^{-3}$ do we end up with the relation \[
E = c \cdot  7.5 \cdot  10 ^{13} J
\] 

\subsection{Steady-state single phase flow in a uniform straight pipeline}%
\label{sub:steady_state_single_phase_flow}
\todo[inline]{ Figure of a pipe }

Pipe with flow $u$, length $L$ and pressure drop $\Delta p$ Then there is a relation between 
\begin{itemize}
  \item $L$ : length, $\left[ L \right] = m$ 
  \item $D$: diameter $\left[ D \right] = m$
  \item $u$: flow rate $\left[ u \right] = ms^{-1}$
  \item $\Delta p$: Pressure drop, $\left[ \Delta kg m^{-1} s^{-2} \right]$
  \item $\mu$: (Shear) viscousity $\left[ \mu \right] = kg m^{-1} s^{-1}$ 
  \item $\rho$: mass density: $\left[ \rho \right] = kg m^{-3}$
  \item $E$: Wall roughness: $\left[ E \right] = m$
\end{itemize}

We have to choose $3$ core variables and they are not unique.  Since we have $3$ independent units $\rho , u, D$ are independent such that it can be a core variable: \[
\Pi _{1} = \frac{L}{D} \quad , \quad  \Pi _{2} = \frac{\Delta p}{\rho u^2} \quad , \quad \Pi _{3} = \frac{\rho}{\mu} \quad , \quad  \Pi _{4} = \frac{E}{D}    
\] 
THen the relation is \[
  \begin{split}
    f\left( \Pi _{1}, \Pi _{2} , \Pi _3, \Pi ^{4},\rho, D, u \right) &= 0  \quad  \Pi _{2} = h\left( \Pi _{1}, \Pi _{3}, \Pi _{4} \right) \leftrightarrow \frac{\Delta p}{\rho u^2} = h\left( \Pi _{1}, \Pi _3 , \Pi _{4} \right) \\
    \to  & \frac{\Delta p}{ u^2 \rho} = \Pi _{1} k\left( \Pi _{3}, \Pi _{4} \right)\\
    \Delta p &= u^2 \rho \frac{L}{D} k \left( \frac{\rho D u}{\mu}, \frac{E}{D} \right)   \\
    \text{measure} \quad & \frac{\rho D \mu}{ \mu} \quad ,\quad    k= \frac{\Delta p D }{u^2\rho}
  \end{split}
\]
  



\bibliographystyle{plain}
\bibliography{references}
\end{document}
