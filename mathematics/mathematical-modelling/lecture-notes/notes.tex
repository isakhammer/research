\documentclass{article}
\usepackage[utf8]{inputenc}

\title{Mathematical Modelling}
\author{isakhammer }
\date{2020}

\usepackage{natbib}
\usepackage{graphicx}
\usepackage{amsmath}
\usepackage{amsthm}
\usepackage{amsfonts}
\usepackage{mathtools}
\usepackage{enumerate}
\usepackage{todonotes}


\usepackage{hyperref} 
\hypersetup{
  colorlinks=true, %set true if you want colored links
  linktoc=all,     %set to all if you want both sections and subsections linked
  linkcolor=blue,  %choose some color if you want links to stand out
} 
\hypersetup{linktocpage}


% inscape-figures
\usepackage{import}
\usepackage{pdfpages}
\usepackage{transparent}
\usepackage{xcolor}
\newcommand{\incfig}[2][1]{%
\def\svgwidth{#1\columnwidth}
\import{./figures/}{#2.pdf_Tex} } \pdfsuppresswarningpagegroup=1

% Box environment
\usepackage{tcolorbox}
\usepackage{mdframed}
\newmdtheoremenv{definition}{Definition}[section]
\newmdtheoremenv{theorem}{Theorem}[section]
\newmdtheoremenv{lemma}{Lemma}[section]

\theoremstyle{remark}
\newtheorem*{remark}{Remark}


\begin{document}
\maketitle
\tableofcontents
\newpage

\newpage
\section{Test}%
\label{sec:test}

\begin{equation}
\label{eq:n1}
  \begin{split}
    m \frac{d}{dt^{*}} x^{*} &=  gm - k \frac{d}{dt ^{*}} x^{*} , \\
     & x^{*} \left( 0 \right) = 0  \quad  \frac{d}{dt ^{*}} x^{*}\left( 0 \right) =V
  \end{split}
.\end{equation}



Let \[
T_{0} = \frac{L}{v_{0}} = \frac{Lk}{mg} 
\] 
and

\begin{equation*}
  \frac{d ^2 x ^{*}}{d t^{*}^2} = \frac{- R^2 g}{\left( R + x^{*} \right)^2} = - \frac{-g}{ \left( 1 + \frac{x^{*}}{R} \right)^2} \approx  -g 
.\end{equation*}

this equation may be solved given \[
x^{*}\left( t^{*} \right) \approx -  \frac{1}{2} g t^{*}^2 + V t^{*}
\] 

The height can be approximated by \[
\frac{d x^{*}}{d t^{*}}  \approx gt^{*} + V = 0 
\] 

or \[
t_{\text{Max}} \approx  \frac{V}{g}
\] 

Thus \[
X = \frac{V^2}{g}, \quad  T = \frac{V}{g} 
\] 

Inserted does this lead to 

\begin{equation*}
  \frac{d ^2 \left( \frac{V^2x / g}{} \right)}{d \left( \frac{Vt}{g} \right)^2} = - \frac{R^2g}{ \left( R + \frac{V^2}{g}x \right)^2} 
.\end{equation*}

and 
\begin{equation*}
  \frac{V^2}{g}x\left( 0 \right) = 0 \quad  \frac{d \left( \frac{V^2}{g}x \right)}{d \frac{V}{g}t} = V 
.\end{equation*}

Let us repeat 

\begin{equation*}
  \frac{d ^2 x^{*}}{d t^2^{*}}  = - \frac{R^2}{\left( R + x^{*} \right)^2} g  = \frac{-g}{\left( 1 + \frac{x^{*}}{R} \right)^2} \approx -g
.\end{equation*}


\begin{equation*}
  x^{*} \approx -g t^{*} + V = 0
.\end{equation*}


Inserted to the equation does it lead to 
\begin{equation*}
  \frac{d ^2 \left( \frac{V^2}{g}x \right)}{d \left( \frac{Vt}{g} \right)^2}  = - \frac{R^2g}{\left( R + \frac{V^2}{g}x \right)^2}
.\end{equation*}
and
\begin{equation*}
  \frac{V^2}{g}x\left( 0 \right) = 0 \quad  , \quad  \frac{d \left( \frac{V^2}{g}x \right)}{d \left( \frac{V}{g}t \right)}   
.\end{equation*}

\begin{align*}
  \ddot{x} &= \frac{-1}{\left( 1 + \epsilon x \right)^2} \\
   & x\left( 0 \right) = 0   \quad  \dot{x}\left( 0 \right) = 1  \quad  \epsilon = \frac{V^2}{Rg} 
.\end{align*}



We should study 
\begin{equation*}
  x\left( t \right) = x_{0}\left( t \right) + \epsilon x_{1}\left( t \right) + \epsilon ^{2} x_{1}\left( t \right) + \ldots
.\end{equation*}

into the equation


\begin{equation*}
  \left[ T \right] = \frac{\left[ \sigma \right]}{ \left[ \frac{\partial ^2 \eta}{\partial x^2}  \right]} = \left( \frac{kgm}{s^2 m ^2} \right)  \cdot  \frac{m^2}{m} = \frac{kg}{s^2}
.\end{equation*}

which often ends up with 
\begin{equation*}
  \omega  = \left( a, k ,T V \right)
.\end{equation*}

\subsubsection{Equation}%
\label{ssub:equation}

\begin{align}
  \label{eq:label}
  \frac{d v^{*}}{d x^{*}} &= \frac{Pc}{A} \left( C^{*}\left( x^{*} \right) - C_0 \right) , \quad  0 \le x^{*} \le L   \\
  C^{*} v^{*} - D \frac{d C^{*}}{d x^{*}}  &=  \begin{cases}
    \frac{N_0 c}{A} x^{*}  &  \quad  x^{*} < \delta \\
    \frac{N_0 c}{A}  \delta  & \quad  \delta \le x^{*} 
  \end{cases} 
.\end{align}

Boundary conditions
\begin{align*}
  v^{*}\left( 0 \right) &=  0 \quad    C^{*}\left( L \right) = 0  \\
                        & v^{*}, C^{*} \quad   \text{is continious for} \quad  x^{*} = \delta 
.\end{align*}

Determine \[
O s^{*} = \frac{F^{*}\left( L \right)}{v^{*}\left( L \right)} = \frac{c N_{0}\delta /A }{v^{*}\left( L \right)}
\] 
\bibliographystyle{plain}
\bibliography{references}
\end{document}


