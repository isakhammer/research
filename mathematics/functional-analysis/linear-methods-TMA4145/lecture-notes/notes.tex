\documentclass{article}
\usepackage[utf8]{inputenc}

\title{Linear Methods Lecture}
\author{isakhammer }
\date{2020}

\usepackage{natbib}
\usepackage{graphicx}
\usepackage{amsmath}
\usepackage{amsthm}
\usepackage{amsfonts}
\usepackage{mathtools}
\usepackage{enumerate}
\usepackage{todonotes}


\usepackage{hyperref} 
\hypersetup{
  colorlinks=true, %set true if you want colored links
  linktoc=all,     %set to all if you want both sections and subsections linked
  linkcolor=blue,  %choose some color if you want links to stand out
} 
\hypersetup{linktocpage}


% inscape-figures
\usepackage{import}
\usepackage{pdfpages}
\usepackage{transparent}
\usepackage{xcolor}
\newcommand{\incfig}[2][1]{%
\def\svgwidth{#1\columnwidth}
\import{./figures/}{#2.pdf_tex} } \pdfsuppresswarningpagegroup=1

% Box environment
\usepackage{tcolorbox}
\usepackage{mdframed}
\newmdtheoremenv{definition}{Definition}[section]
\newmdtheoremenv{theorem}{Theorem}[section]
\newmdtheoremenv{lemma}{Lemma}[section]

\theoremstyle{remark}
\newtheorem*{remark}{Remark}
\newtheorem{example}{Example}


\begin{document}
\maketitle
\tableofcontents
\newpage

\newpage
\section{Lecture 1}%
\label{sec:lecture_1}

\subsection{Set Theory}%
\label{sub:set_theory}

\begin{definition}
  A \textbf{set} is a collection of distinct objects, its elements. \[
  x \in X \quad  x \text{ is a element of the set } X 
  \] 
  and similary 
  \[
  x \not\in X \quad  \text{ x is not an element of X} 
  \] 

  \par
   Two sets are identical $X=Y$ , if \[
   x \in X \leftrightarrow x \in Y
   \] 
   for any element  $x$ .
\end{definition}

\begin{definition}
  $Y$ is a subset of $X$, $Y \mathbb{C}  X$ if for all $y \in X$. If $Y \subset X $ and $Y \neq X$, we write $y \subset X$ (or $Y \not \subset X$). $Y$ is then a proper subset of $X$ .
  Showing to sets are equal, 
  \begin{itemize}
    \item $x \in X \leftrightarrow x \in Y$
    \item $x \subset Y$ and $ y \subset X$
  \end{itemize}
  The empty set are denoted by null.
\end{definition}

\begin{example}
  \begin{itemize}
    \item $\mathbb{N}  = \left\{ 1,2,3,4,5, \ldots \right\}$
    \item $\mathbb{Z}  = \left\{ \ldots, -1,0,1,\ldots \right\}$
    \item $\mathbb{Q}  = \left\{ \frac{p}{q}: p,q \in \mathbb{Z} , q \neq0 \right\}$ 
    \item $\mathbb{R}  = \text{reals}$ 
    \item $\mathbb{C}: \text{Complex numbers} \quad  a + ib  $ 
    \item Finite set $\left\{ 3,4,5,6 \right\}$ 
    \item Intervals in $\mathbb{R} $ For real numbers $a < b < \infty$
      \begin{align*}
        & (a,b)\\
        & \left[ a,b \right] \\
        & (a,b] , \quad  [a,b) 
      .\end{align*}
  \end{itemize} 
\end{example}


\begin{definition}
  Let $X$ and $Y$ be two sets then
  \begin{itemize}
    \item Union. $X \cup Y = \left\{ z  \mid z \in X \quad \text{or} \quad  z \in Y     \right\}$ 
      \[
      \bigcup_{i \in  I}  X_{i} = \left\{ z  \mid z \in X_{i} \quad   \text{ for some } \quad   i \in I \right\}
      \] 
    \item Intersection if $\bigcap_{i \in  I}  = \{ z  \mid z \in X_i \quad \text{For every} \quad  i \in I  \} $
    \item Complement if $S$ is a subset of $X$ , then the complement of $S$ is \[
    X \setminus S = S^{c} = \{ x \in X: x \not\in S\} .
    \] 
  \item Cartesian product \[
  X \times  Y = \{ \left( x,y \right) : x \in X , \quad   y \in Y\}  
  \] 
  \end{itemize}
\end{definition}


\begin{lemma}
  \begin{itemize}
    \item $x \cap \left( Y \cup Z \right) = \left( X \cap Y \right) \cup \left( X \cap Z \right) \quad   $  and \[
    X \cup \left( Y \cap Z \right) = \left( X \cup Y \right) \cap \left( X \cup Z \right)
    \] 
  \item $\left( X \cup Y \right)^{c} = X^{c} \cap Y^{c}$
  \item $\left( X \cap Y \right)^{c} = X^{c} \cup Y^{c}$
  \item Demo organs law 

    \begin{align*}
      X \setminus \left( Y \cup Z \right) &= \left( X \setminus Y\right) \cap \left( X \setminus Z \right) \\
    .\end{align*}
  \item $\left( X^{c} \right)^{c} = X$
  \end{itemize}
\end{lemma}

\begin{proof}
  Proof of $\left( X \cup Y \right)^{c} ) =  X^{c} \cap Y^{c}$
  \begin{equation*}
    \begin{split}
      x \in \left( X \cup Y \right)^{c} & \to x \in X \cup U\\
       & x \not\in X \quad \text{and} \quad  x \not\in Y \\
       & x \in X^{c} \quad \text{and} \quad  x \in Y \\
       & x \in X^{c} \cap Y^{c}
    \end{split}
  .\end{equation*}
\end{proof}

\subsection{Functions}%
\label{sub:functions}

Let $X,Y$ be sets. A function $f$ from $X$ to $Y$, denoted $f: X \to Y$ , is defined by a set $G$ of ordered pairs $\left( x,y \right) $, where $x \in X, \quad  y \in Y  $ and with the property that;
\par
For each set is there a unique $y \in Y \quad  \text{ s.t.} \quad  \left( x,y \right) \in G $. We write $f\left( x \right) = y$. 
\begin{itemize}
  \item We say that $X$ is the domain and $Y$ is the codomain.
  \item The (direct) image of a set $ A \subset X$ under f is \[
  f\left( A \right) = \{f\left( t \right): t \in A\} \subset Y
  \] 
\item The \textbf{inverse image}  of a set $B \subset Y$ under f is \[
    f^{-1} \left(  B \right)  = \{x \in X  \mid f\left( x \right) \in B\} \subset X
\] 
\item The \textbf{range} if $f$ is the image of its domain $X$ is \[
    ran\left( f \right) = f\left( X \right) = \{f\left( t \right): t \in X\} 
\] 
\end{itemize}

\begin{example}
  Let $f: \mathbb{R}  \to \mathbb{R} $ given by \[
  f\left( x \right) = max\left\{ x,0 \right\} = x^{+}
  \] 
  Then is the $ran\left( f \right) = [0, \infty)$.  The inverse is $f^{-1} \left( \left\{ y \right\} \right) = \left\{ y \right\}$ and $f^{-1}\left( \left\{ 0 \right\} \right) = (- \infty , 0]$  and \[
  f^{-1} \left( \left\{ y \right\} \right) = \text{NULL} \quad \text{if} \quad  y < 0  
  \] 
\end{example}


\begin{definition}
  Let $f: X \to Y$ be a function
  \begin{itemize}
    \item $f$ is \textbf{injective}  or \textbf{one-to-one}  if $f\left( x_{1} \right) \to x_{1} = x_{1}$ 
    \item $f$ is \textbf{surjective}  or \textbf{onto}  if $ran\left( f \right) = y$ 
    \item $f$ is \textbf{bijective}  if it is both surjective and injective.
  \end{itemize}
\end{definition}

\begin{example}
  Lets continue the example.
  \begin{itemize}
    \item
  Let $f: \mathbb{R} \to \mathbb{R} $ , $f\left( x \right) = max \left\{ x,0 \right\}$. Injective? No; $f\left( x_{1} \right) = \underbrace{f\left( x_{2} \right)}_\text{$= 0$}  $ for any two $x_{1}, x_{1} < 0$ .
\item  A \textbf{bijection}  $f:  X \in Y$ has a \textbf{inverse}  function $f^{-1} : Y \to X$, defined by $f^{-1} \left( y \right) = x$ if $f\left( x \right) = y$ . \par
  THe inverse function $f^{-1} $ is also a bijection. 
  \end{itemize}
\end{example}
\begin{remark}
  Not to be confused with the inverse image of a set $f^{-1} \left( B \right) $ introduced earlier.
\end{remark}

\newpage
\section{Lecture 2}%
\label{sec:lecture_2}

\subsection{Recall}%
\label{sub:recall}

Let $f: X \to Y$ then is 
\begin{itemize}
  \item[i)] Inective: $f\left( x_{1} \right) = f\left( x_{2} \right) \to \quad x_{1} = x_{2} $ 
  \item [ii)] Surjective:  For all $y$ in $Y$ there is a $x$ in $X$ s.t. $f\left( x \right) = y$. 
  \item [iii)] Bijective if i) and ii) holds.
\end{itemize}


\begin{itemize}
  \item If $F: X \to Y$ is a bijective then there is an inverse \[
      f^{-1} : Y \to X
  \] 
  Given by \[
  f^{-1} \left( y \right) = x \quad  \text{if} \quad  f\left( x \right)   = y
  \] 
\item Identify function/map 
  \begin{itemize}
    \item id: $X \to X$
    \item $id_{x}\left( x \right) = x$ for all $x \in X$
  \end{itemize}
\item The composition of a function \[
    g: Y \to Z \quad  \text{with} \quad  f: X \to X  
\] is the function $g\cdot f : X \to Y$   defined by \[
\left( g\cdot f \right)\left( x \right) = g\left( f\left( x \right) \right) \quad  \text{for} \quad  x \in X  
\] 
\end{itemize}

\begin{definition}
  Anternative version. Given a bijection $f: X\to Y$ the inverse function $f^{-1}: Y \to X$ is the unique function satisftying $f^{-1} \cdot  f = id_{x}$ and $f\cdot f^{-1} = id_{y} $
\end{definition}
\begin{example}
  $\frac{d }{d x}: C^{1}\left( \mathbb{R} , \mathbb{R}  \right) \to C \left( \mathbb{R} ,\mathbb{R}  \right) $. Inverse? no. 
  \par
  Let $g \in C^{1} \left( \mathbb{R} ,\mathbb{R}  \right)$. Then is \[
  \frac{d \left( g + c \right)}{d x} = \frac{d g}{d x} \quad \text{where c is the constant.} 
  \] 
  It is surjective because given any $f \in C\left( \mathbb{R} ,\mathbb{R}  \right)$
  we can define $F \in C^{1}\left( \mathbb{R} ,\mathbb{R}  \right)$ by \[
  F: X \to \int_{0}^{x} f\left( t \right)dt
  \] 
  and \[
  \frac{d F}{d x} = f \quad  \text{fundamental theorem of calculus.} 
  \] 
\end{example}

\subsection{Cardinality}%
\label{sub:cardinality}

Cardinality is a tool for comparing the sizes of sets. 
\begin{definition}
  We say that two sets $A$ and $B$ has the same cardinality if there exist a bijection between $A$ and $B$. 
\end{definition}

\begin{tcolorbox}
  \textbf{Example.}
  \par
  \begin{itemize}
    \item [i)] The two inervals $\left[ 0,2 \right]$ and $\left[ 0,1 \right]$ have the same cardinality. \[
    \begin{split}
       &  f:\left[ 0,2 \right] \to \left[ 0,1 \right]  \\
        &  f\left( t \right) = \frac{t}{2}
    \end{split}
    \] 
  \item [ii)] Let $\mathbb{N}  = \{ 1,2,3,4, \ldots \} $ and $\mathbb{N} \setminus \{1\} = \{2,3,4,5, \ldots\}  $ have the same cardinality \[
  f\left( n \right) = n+1
  \] 
\item [iii)] $n$ is finite integer. Then there is no bijection \[
    f: \{1,2,3, \ldots , n\}  \to \mathbb{N} 
\] 
These two sets \textbf{do not}  have the same cardinality. 
  \end{itemize}

\end{tcolorbox}

\begin{definition}
  Let $X$ be a set. We say $X$ is \textbf{finite} if either $X = \text{NULL} $ or there exist $n \in \mathbb{N} $ s. T. $X$ has the same cardinality as $\{ 1,2,3,4, \ldots, n\} $ if \[
 \text{ There exist} \quad  f: \{1,2,3, \ldots m b\}  \to X \quad \text{for some} \quad  n    
  \] 
  $X$ is \textbf{infinite }  if it is not finite.
\end{definition}

\newpage 
\begin{definition}
  A set $X$ is 
  \begin{itemize}
    \item Countable infinite if it has the same cardinality as $\mathbb{N} $. \[
    \exists \text{bijection} \quad  f: X \to \mathbb{N}  
    \] 
  \item Countable if it is either countably infinite or finite. or equivalently 
    \begin{itemize}
      \item if $\exists$ injection $f: X \to \mathbb{N} $ 
      \item $\exists$ surjection $f: \mathbb{N}  \to X$
    \end{itemize}
  \item Uncountable if it is not countable.
  \end{itemize}
\end{definition}

\begin{tcolorbox}
  \textbf{Example.} 
  \begin{itemize}
    \item Any finitie set is, e.g. $\{ 2,5,9\} $ 
    \item $X = \{ 1,4,9,16, \ldots , n^2 , \ldots\} $  such that \[
    f: \mathbb{N}  \to X, \quad  f\left( n \right) = n^2 
    \] 
  \item $\mathbb{N} \times  \mathbb{N}  $ is countable ; \par
    We arrange $N\times N $ in a table. \[
    \begin{split}
         f: \mathbb{N}  &\to \mathbb{N} \times  \mathbb{N}  \\
        f\left( 1 \right) &=  \left( 1,1 \right) \\
        f\left( 2 \right) &=  \left(2,1  \right)  \\
        f\left( 3 \right) &=  \left( 1,2 \right) \\
        f\left( 4 \right) &=  \left( 3,1 \right) \\
        \vdots   
    \end{split} 
    \]  
  \item $\mathbb{Z} $ and $\mathbb{Q} $ are countable (Prob set 1).
  \item If $X$ and $Y$ are countable, then so is $X \cup  Y$ .
  \end{itemize}
\end{tcolorbox}

\subsection{Schroeder Bernstein Theorem}%
\label{sub:schroeder_bernstein_theorem}

Let $X$ and $Y$ by two be two sets. Suppose there are injective maps $f: X \to Y$ and $g: Y \to X$. Then there exists a bijection between $X$ and $Y$. 

\begin{tcolorbox}
  \textbf{Example.} The interval $\left( 0,1 \right) \subseteq  \mathbb{R} $. Claim it is uncountable.  
  \begin{proof}
    The Cantor diagonalization argument. Suppose that $\left( 0,1 \right) $ is countable. \[
      \begin{split}
        \left( 0,1 \right) = &\{x_{1}, x_{2}, x_{3}, x_{4} , \ldots\} \\
         & f\left( 1 \right), f\left( 2 \right) , f\left( 3 \right), \ldots\\
         \\
      f: \mathbb{N}  & \to \left( 0,1 \right) \\
      x_{i} = 0,  & x_{i1}, x_{i2} , x_{i3}, \ldots 
      \end{split} 
    \] 
    Now let \[
    a = 0, a_{1}, a_{2}, a_{3}, a_{4}, a_{5}, \ldots
    \] 
    where \[
    a_{i} = \begin{cases}
      3 &  \text{if} \quad   x_{ii} \neq 3 \\
      1  & \text{if} \quad   x_{ii} =3  
    \end{cases}
    \] 
    Then $a_{i} \neq x_{ii} $ , so by construction $a \neq x_{i}$ for all $i$. Moreover, we must have $a \in \left( 0,1 \right)$. This is a contradiction, so $\left( 0,1 \right) $ cannot be countable. 
  \end{proof}
\end{tcolorbox}


\begin{tcolorbox}
  \textbf{Example.} The set of all binary sequences $ X = \left\{ \left( x_{1}, x_{2}, x_{3} , \ldots \right) \right\} : \quad  x_{i} \in \left\{ 0,1 \right\}  $
  is uncountable . 

  \begin{proof}
    Problem set 2.
  \end{proof}
\end{tcolorbox}

\newpage
  \begin{lemma}
    Let $X$ and $Y$ be sets. Then 
    \begin{itemize}
      \item If $X$ is countable and $ Y \subseteq  X$ , then $Y$ is also countable. \[
      \left\{ 1,2,3,4,5, \ldots \right\} \to \{x_{1}, x_{2} , x_{3}, x_{4} , \ldots\} 
      \] 
    \item If $X$ is uncountable and $X \subseteq  Y$, then $Y$ is uncountable. 
    \item If $X$ is countable and there is an injection \[
    f: Y \to X   
    \] 
    then $Y$ is countable.
  \item If $X$ is uncountable and \[
  \exists \quad  \text{injective} \quad  f: X \to Y,   
  \] 
  then $Y$ is uncountable.
    \end{itemize}
  \end{lemma}
   \begin{tcolorbox}
     \textbf{Example.} Have proed formally that $\left( 0,1 \right) \subseteq  \mathbb{R} $ is countable $\overbrace{\to}^\text{ii)}  \mathbb{R} $ must be uncountable \[
       R \subset \mathbb{C}  \overbrace{\longrightarrow}^\text{ii)} \mathbb{C} \quad \text{is uncountable} 
     \] 
   \end{tcolorbox}

   \begin{tcolorbox}
     \textbf{Example.} $R = \mathbb{Q}  \cup \mathbb{I}$. Know: $\mathbb{Q} $ countable.  \par
     Assume $\mathbb{I}$ countable. Then $R \cup \mathbb{I}$ which is a contradiction. So $\mathbb{I}$ is uncountable
   \end{tcolorbox}


\bibliographystyle{plain}
\bibliography{references}
\end{document}

