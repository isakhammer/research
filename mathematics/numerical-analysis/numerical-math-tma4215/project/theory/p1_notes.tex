\documentclass{article}
\usepackage[utf8]{inputenc}

\title{Project 1 Notes}
\author{isakhammer }
\date{2020}

%%%% DEPENDENCIES v1.2 %%%%%%

\usepackage{natbib}
\usepackage{graphicx}
\usepackage{amsmath}
\usepackage{amsthm}
\usepackage{amsfonts}
\usepackage{mathtools}
%\usepackage{enumerate}
\usepackage{enumitem}
\usepackage{todonotes}
\usepackage{float}


\usepackage{hyperref} 
\hypersetup{
  colorlinks=true, %set true if you want colored links
  linktoc=all,     %set to all if you want both sections and subsections linked
  linkcolor=blue,  %choose some color if you want links to stand out
} 
\hypersetup{linktocpage}


% inscape-figures
\usepackage{import}
\usepackage{pdfpages}
\usepackage{transparent}
\usepackage{xcolor}
\newcommand{\incfig}[2][1]{%
\def\svgwidth{#1\columnwidth}
\import{./figures/}{#2.pdf_tex} } \pdfsuppresswarningpagegroup=1

% Box environment
\usepackage{tcolorbox}
\usepackage{mdframed}
\newmdtheoremenv{definition}{Definition}[section]
\newmdtheoremenv{theorem}{Theorem}[section]
\newmdtheoremenv{lemma}{Lemma}[section]

% \DeclareMathOperator{\span}{span}

\theoremstyle{remark}
\newtheorem*{remark}{Remark}
%\newtheorem{example}{Example}

\newcommand{\newpara}
  {
  \vskip 0.4cm
  }

%%%%%%%%%%%%%%%%%%%%%%%%%%%%%%%%%%%%%%%%%%%%%%%%%%%%%%%%%%%%



\begin{document}
\maketitle
\tableofcontents
\newpage

\newpage
\section{Problem 1}%
\label{sec:problem_1}
\subsection{Problem Description}%
\label{sub:problem_describtion}

Let normal matrices, those with diagonalization be on the form \[
A = U \Lambda U^{H} 
\] 
Where $\Lambda $ is a diagonal complex $n\times n $ matrix and $U$ a unitary (complex) matrix such that $U ^{H} U = I$ (recall that $U^{H}$ is the complex conjugate of $U^{T}$ ).
\newpara
Show that for any such matrix, one has $\|A\|_{2} = \rho \left( A \right)$, where $\rho \left( A \right) $ is the spectral radius of $A$ .

\subsection{Answer 1a}%
\label{sub:proof}

\begin{proof}
  Starting with the definition of a subordinate matrix norm given in Mayers \cite{sul} can we let \[
  \|A\|_{2}^{2} = \sup_{x \neq 0} \frac{\left<Ax, Ax \right>}{ \left<x,x \right>} .
  \] 
   Indeed, by using the assumption that $U^{H} U = I$ and substituting $Uy =  x$ can we show that \[
    \|A\|_{}^{2} = \sup_{x \neq 0} \frac{\left<Ax, Ax \right>}{\left<x,x \right>}  = \sup_{y\neq 0}  \frac{\left<AU y, A U y \right>}{ \left<U y, Uy \right>} = \sup_{y \neq 0} \frac{\left<U^{H} A^{H} A U y,y \right>}{\left<y,y \right>} 
  \] 
    % \textbf{Kinda sketchy argument, given in Quartentoni page 41/664. In fact, I do not believe it is true to assume A is hermetian/unitary.}  
  \newpara
  Recall the property $A = U \Lambda  U^{H}$ and thus
  \[
    \begin{split}
  A^{H} A   & = U \Lambda ^{H} U^{H} U \Lambda U^{H}  \\
   & = U \Lambda ^{H} \Lambda U^{H}.   \\
    \end{split} 
  \] 
  As a consequence do we end up with \[
    \begin{split}
  \sup_{y \neq 0} \frac{\left<U^{H} A^{H} A U y, y  \right>}{\left<y,y \right>}   & = \sup_{y\neq 0}  \frac{\left<U^{H} U \Lambda ^{H} \Lambda  U^{H} U  y, y\right>}{ \left<y,y \right>}  \\
  &=  \sup_{y \neq 0} \frac{\left<\Lambda ^{H} \Lambda  y, y \right>}{\left<y,y \right>}  \\  
  &=   \sup_{ y \neq 0}  \frac{\sum_{i=1}^{n}  \left\lvert \lambda _{i} \right\rvert ^2  \left| y_{i} \right|^2}{ \sum_{i=1}^{n}  \left| y_{i} \right|^2}  = \max _{i} \left( \left\lvert \lambda_{i}  \right\rvert ^{2}  \right)  \\
    \end{split} .
  \] 
  Given from Sacco\cite{qu}, the definition of a spectral radius is characterized by \[
  \rho \left( A \right) = \max_{i} \left\lvert \lambda_{i}  \right\rvert    .
  \] 
  Which completes the proof of $ \|A\|_{2} = \rho \left( A \right)$.
   
\end{proof}

% \subsection{Note}%
% \label{sub:note}

% Let say $A = U\Lambda U^{H}$. Then is it not possible to get any useful answers. 
%     \begin{enumerate}[label=(\roman*)]
%       \item Lets compute $AA$ \[
%       A A = U \Lambda U^{H} U \Lambda  U^{H} = U \Lambda  \Lambda  U ^{H}
%       \] 
%     \item Lets compute $A^{H} A$ 
%       \[
%       A^{H} A = U \Lambda ^{H} U^{H} U \Lambda U^{H} = U \Lambda ^{H} \Lambda U^{H}
%       \] 
%       sec         
%     \item Lets compute  $A A^{H}$ \[
%     A A^{H} = U \Lambda U^{H}  U \Lambda ^{H} U^{H}  = U \Lambda  \Lambda ^{H} U^{H}.
%     \] 
%     \end{enumerate}

\newpage

\section{Problem 2}%
\label{sec:problem_2}

\subsection{Problem Describtion}%
\label{sub:problem_describtion}



Consider the $n \times n $ matrix $A$ whise nonzero elements are located on its unit subdiagonal , i.e. $A_{i+1, i} = 1$ for $i = 1, \ldots, n-1$\[
A = \begin{bmatrix} 
  0  &  \ldots  &  \ldots  & 0 \\
  1  &  0  &    & \vdots  \\
  \vdots   & \ddots   & \ddots   &  \vdots   \\
  0  &  \ldots     & 1  & 0 
\end{bmatrix} 
\] 

\begin{enumerate}[label=(\alph*)]
  \item  What are the eigenvalues of $A$ ? What would the Gershgorin theorem tell us about the location of the eigenvalues of $A$.
  \item  Now construct the matrix $\hat{A}$ by adding a small number $\epsilon $ in the $\left( 1, n \right)$- element of $A$ (so that $\hat{A} = A  +  \epsilon e_{1} e_{n}^{T} $). Show that \[
      \rho \left( \hat{A} \right) = \epsilon ^{\frac{1}{n} }
  \] 
  And find an expression for the eigenvalues and eigenvectors of $\hat{A}$. 
\item Derive an exact expression for the condition number \[
K_{2} \left( \hat{A} \right) =  \|\hat{A}\|_{2}^{} \cdot  \|\hat{A}^{-1}\|_{2}^{}.  
\] 

  \end{enumerate}

  \subsection{Answer 2a}%
  \label{sub:answer_a}
  

\newpara
The eigenvalues can be computed such that \[
  \begin{split}
  det \left( A - \lambda   \right) &= \begin{vmatrix} 
  -\lambda &   &   &      \ldots  & 0 & 0 \\
  1  &  -\lambda  & \ldots  &   &   & 0 \\
  0  & 1  & - \lambda   &  \ldots  &   & 0 \\
  \vdots  &   &  &   \ddots  \\
  0  &  \ldots  &  &    & 1    & - \lambda      \\
  \end{vmatrix} 
    \\
   &= -\lambda  \begin{vmatrix} 
   -\lambda   &  \ldots  &      & 0 \\
  1  & - \lambda  &      \\
  \vdots  & &    \ddots  \\
  0  &  \ldots & 1    & -\lambda      \\
  \end{vmatrix} 
    \\
   &=\left( -1 \right)^{n}   \lambda^{n}  = 0 \quad   \implies  \quad \quad \lambda  = 0
  \end{split} 
\] 
Which concludes that all eigenvalues are zero. Recall the Gershgoring Theorem, given in Mayers \cite{sul}.
\begin{definition}[Gerschgorin Discs]
  \label{G1}
  Suppose that $n\ge2 $ and $A \in  \mathbb{C} ^{n \times n }$. The \textbf{Gerschgoring Discs} $D_{i},  i = 1,2,3 \ldots, n$ of the matrix $A$ are defined as the closed circular regions \[
  D_{i} = \left\{ z \in  \mathbb{C}  : \quad    \left\lvert z - a_{ii} \right\rvert \le R_{i} \right\}
  \] 
  In the complex plane, where \[
  R = \sum_{\substack{j = 1 \\ j \neq i} }^{n}  \left\lvert a_{ij} \right\rvert 
  \] 
  is the radius of $D_{i}$.
  
\end{definition}

\begin{theorem}[Gerschgorin Theorem]
  \label{G2}
  Let $n\ge2$ and $A \in  \mathbb{C} ^{n \times  n}$ . All eigenvalues of the matrix $A$ lie in the region $D =  \bigcup_{i=1}^{n} D_{i}$ , where $D_{i}, \\\ i = 1,2,\ldots,n $, are discs defined by in the definition \ref{G1}.
\end{theorem}

Using the Gerschgorin Theorem \ref{G2} on the matrix $A$, can we establish that every eigenvalue must be inside the union of gerschgorin discs. Note that in this example is all discs centered around the origin with a radius of $0$ or $1$. It is worth commenting that even though this is true, because all eigenvalues is zero, can it is worth mentioning that $r=1$ is a fairly inaccurate estimate even though the theorem has powerful statements.

    \subsection{Answer 2b}%
\label{sub:answer_b_}


    All eigenvalues $\lambda $ of matrix $\hat{A}$ requires that \[
    det\left( \hat{A} - \lambda I \right) = 0.
    \]  
    It is true that
    \[
    \hat{A} = A + \varepsilon e_{1}e_{n}^{T} = \begin{pmatrix}
    0    &   &  \ldots           &    &  \varepsilon  \\
    1  & 0  &  \ldots \\
    0  &  1  & \ddots  \\
    \vdots   &   &  \ddots   &  \ddots   & \\
     0  &  \ldots  &   &  1  & 0
    \end{pmatrix}, 
    \] 
    and therefore can obtain
     \[
      \begin{split}
    det \left(  \hat{A} - \lambda I \right)  & = \begin{vmatrix}
    -\lambda     &   &  \ldots           &    &  \varepsilon  \\
    1  & -\lambda   &  \ldots \\
    0  &  1  & \ddots  \\
    \vdots   &   &  \ddots   &  \ddots   & \\
     0  &  \ldots  &   &  1  & -\lambda 
    \end{vmatrix}  \\
    &=  \left( -1 \right)^{n} \lambda ^{n}  + \left( -1 \right)^{n+1}\varepsilon 
     \begin{vmatrix}
    1  & -\lambda   &  \ldots \\
    0  &  1  & \ddots  \\
    \vdots   &   &  \ddots   &  \ddots   & \\
     0  &  \ldots  &   &  1  & -\lambda 
    \end{vmatrix}  \\
     \\
     &=  \left( -1 \right)^{n} \lambda ^{n} + \left( -1 \right)^{n+1}\varepsilon  = 0 .  \\
      \end{split} 
    \] 
    We can see that the eigenvalues $\lambda $ have several complex solutions depending on the value $n$. However, it is clear that all $\lambda $ will satisfy $\left\lvert \lambda  \right\rvert = \varepsilon ^{\frac{1}{n}}$, thus
     \[
    \rho \left( \hat{A} \right) = \varepsilon ^{\frac{1}{n}}.
    \] 
    The general expression of the eigenvalues is \[
     \lambda_k = \varepsilon^{\frac{1}{n}} e^{i2\pi k /n}, \quad    k = \left\{ 0,1,\ldots,n -1 \right\} .
    \] 
    The procedure to compute the eigenvectors is then to find all solutions  which satisfies \[
      \hat{A} v_{k}    = \lambda _{k} v_{k}. \\
    \] 
    Using recursion do we end up with the equation,
        \[
      \begin{split}
         \quad   \begin{bmatrix} 
        \varepsilon v_{k,n} \\
        v_{k,0}\\
        \vdots  \\
        v_{k,n-2} \\
        v_{k,n-1}
      \end{bmatrix} 
       & = \begin{bmatrix} 
       v_{k,0} \lambda _{k} \\
       v_{k,1} \lambda _{k} \\
       \vdots  \\
       v_{k,n-1} \lambda _{k} \\
       v_{k,n} \lambda _{k}
       \end{bmatrix} 
       = \begin{bmatrix} 
         v_{k,n} \lambda _{k}^{n}\\
         v_{k,n} \lambda _{k}^{n-1}\\
         \vdots \\
         v_{k,n } \lambda _{k} ^{2} \\
         v_{k,n } \lambda _{k} ^{1}
       \end{bmatrix} 
      \end{split} .
    \] 
    Observe that for the $n$-th element is \[
      \begin{split}
      \varepsilon v_{k,n}   & = v_{k,n}\lambda _{k} ^{n} = v_{k,n} \left( \varepsilon ^{\frac{1}{n}} e^{\frac{i 2\pi k}{ n} } \right)^{n} = v_{k,n} \varepsilon \cdot  e^{i2\pi  k }  \\
      \end{split}. 
    \] 
    Choosing the scale of all eigenvectors such that the element $v_{k,n} = 1$ , can we determine a general expression of the eigenvectors to be \[
    v_k=  \begin{bmatrix} 
      v_{k,0} \\
      v_{k,1} \\
      \vdots  \\
      v_{k,n-1} \\
      v_{k,n} 
      \end{bmatrix} 
      = \begin{bmatrix} 
        \lambda_k ^{n} \\
        \lambda_k ^{n-1} \\
        \vdots  \\
        \lambda _k \\
        1
      \end{bmatrix} 
      = \begin{bmatrix} 
      \left( \varepsilon ^{\frac{1}{n}} e^{\frac{i 2 \pi k}{n} } \right)^{n} \\
      \left( \varepsilon ^{\frac{1}{n}} e^{\frac{i 2 \pi k}{n} } \right)^{n-1} \\
      \vdots \\
\varepsilon ^{\frac{1}{n}} e^{\frac{i 2 \pi k}{n} } \\
      1
      \end{bmatrix} 
      ,\quad k = \left\{ 0, 1,2, \ldots, n-1 \right\} .
    \] 

\subsection{Answer 2c}%
\label{sub:answer_c}
 We recall the theorem given in Mayers \cite{sul}. 
\begin{theorem}
  Let $ A \in  \mathbb{R} ^{n \times n }$ and denote the eigenvalues of the matrix $B = A^{T} A$ by $\lambda _{k}, k = 0,1,\ldots,n-1$. Then, \[
    \|A\|_{2}^{} = \max_{k }  \lambda _{k} ^{\frac{1}{2}}.
  \] 
\end{theorem}
Note that $\hat{A}^{T} = \hat{A}^{-1}$, so by computing the matrix \[
\hat{A}^{-1} \hat{A} = \begin{bmatrix} 
1  & 0  &  \ldots  &  0 \\
0   &  1  &  \ldots  \\
\vdots   &   &  \ddots   &  \\
0  &  \ldots  &   &   \varepsilon ^{2} 
\end{bmatrix} ,
\] 
 can we determine the eigenvalues.\[
   \begin{split}
det \left(  \hat{A}^{-1} \hat{A} - \lambda I \right)  &= 
\begin{vmatrix} 
1- \lambda   & 0  &  \ldots  &  0 \\
0   &  1 - \lambda   &  \ldots  \\
\vdots   &   &  \ddots   &  \\
0  &  \ldots  &   &   \varepsilon ^2  - \lambda  
\end{vmatrix}  \\
   &=  \left( \varepsilon ^2 - \lambda  \right) \left( 1- \lambda  \right)^{n-1} = 0 \\
   \end{split}  
 \] 
 Which shows that $\left\lvert \lambda_{\text{max}}  \right\rvert^{\frac{1}{2}}   = 1  $ and $\left\lvert \lambda_{\text{min}}  \right\rvert^{\frac{1}{2}}   = \varepsilon   $, since $\varepsilon   < 1$. And therefore can conclude that $\|\hat{A}\|_{2}^{} = 1 $. Similary, for the inverse matrix $\hat{A}^{-1}$ is $\|\hat{A}^{-1}\|_{2}^{} = \varepsilon ^{-1} $. 
 We therefore end up with \[
 K_{2}\left( \hat{A} \right) = \frac{1}{\varepsilon } .
 \] 

 \newpage
 \section{Problem 3}%
 \label{sec:problem_3}
 
 

\newpage

\section{References}%
\label{sec:references}
\bibliographystyle{plain}
\bibliography{references}

\end{document}

