\documentclass{article}
\usepackage[utf8]{inputenc}

\title{Project 1 Notes}
\author{isakhammer }
\date{2020}

\usepackage{natbib}
\usepackage{graphicx}
\usepackage{amsmath}
\usepackage{amsthm}
\usepackage{amsfonts}
\usepackage{mathtools}
\usepackage{enumerate}
\usepackage{todonotes}
\usepackage{float}


\usepackage{hyperref} 
\hypersetup{
  colorlinks=true, %set true if you want colored links
  linktoc=all,     %set to all if you want both sections and subsections linked
  linkcolor=blue,  %choose some color if you want links to stand out
} 
\hypersetup{linktocpage}


% inscape-figures
\usepackage{import}
\usepackage{pdfpages}
\usepackage{transparent}
\usepackage{xcolor}
\newcommand{\incfig}[2][1]{%
\def\svgwidth{#1\columnwidth}
\import{./figures/}{#2.pdf_tex} } \pdfsuppresswarningpagegroup=1

% Box environment
\usepackage{tcolorbox}
\usepackage{mdframed}
\newmdtheoremenv{definition}{Definition}[section]
\newmdtheoremenv{theorem}{Theorem}[section]
\newmdtheoremenv{lemma}{Lemma}[section]

\theoremstyle{remark}
\newtheorem*{remark}{Remark}
%\newtheorem{example}{Example}

\newcommand{\newpara}
  {
  \vskip 0.4cm
  }


\begin{document}
\maketitle
\tableofcontents
\newpage

\newpage
\section{Problem 1}%
\label{sec:problem_1}

Let normal matrices, those with diagonalization be on the form \[
A = U \Lambda U^{H} 
\] 
Where $\Lambda $ is a diagonal complex $n\times n $ matrix and $U$ a unitary (complex) matrix such that $U ^{H} U = I$ (recall that $U^{H}$ is the complex conjugate of $U^{T}$ ).
\newpara
Show that for any such matrix, one has $\|A\|_{2} = \rho \left( A \right)$, where $\rho \left( A \right) $ is the spectral radius of $A$ .


\begin{tcolorbox}
  \textbf{Answer.} 
  \newpara Ideas
  \begin{itemize}
    \item Complex conjugate \[
     A^{H} = \overline{A^{T}} 
    \] 
    \item  Definition of spectral radius is denoted by \[
        \rho \left( A \right) = \max_{\lambda _{i} \in  \sigma \left( A \right)} \left\{ \lambda _{i} \right\}
    \] 
    of the eigenvalues $\lambda _{i}$ in the eigenvalue spectrum $\sigma \left( A \right)$.
  
  \end{itemize}
\end{tcolorbox}




\newpage
\section{References}%
\label{sec:references}

\bibliographystyle{plain}
\bibliography{references}
\end{document}

