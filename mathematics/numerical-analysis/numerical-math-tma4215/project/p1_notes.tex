\documentclass{article}
\usepackage[utf8]{inputenc}

\title{Project 1 Notes}
\author{isakhammer }
\date{2020}

%%%% DEPENDENCIES v1.0 %%%%%%


\usepackage{natbib}
\usepackage{graphicx}
\usepackage{amsmath}
\usepackage{amsthm}
\usepackage{amsfonts}
\usepackage{mathtools}
\usepackage{enumerate}
\usepackage{todonotes}
\usepackage{float}


\usepackage{hyperref} 
\hypersetup{
  colorlinks=true, %set true if you want colored links
  linktoc=all,     %set to all if you want both sections and subsections linked
  linkcolor=blue,  %choose some color if you want links to stand out
} 
\hypersetup{linktocpage}


% inscape-figures
\usepackage{import}
\usepackage{pdfpages}
\usepackage{transparent}
\usepackage{xcolor}
\newcommand{\incfig}[2][1]{%
\def\svgwidth{#1\columnwidth}
\import{./figures/}{#2.pdf_tex} } \pdfsuppresswarningpagegroup=1

% Box environment
\usepackage{tcolorbox}
\usepackage{mdframed}
\newmdtheoremenv{definition}{Definition}[section]
\newmdtheoremenv{theorem}{Theorem}[section]
\newmdtheoremenv{lemma}{Lemma}[section]

%\DeclareMathOperator{\span}{span}

\theoremstyle{remark}
\newtheorem*{remark}{Remark}
%\newtheorem{example}{Example}

\newcommand{\newpara}
  {
  \vskip 0.4cm
  }

%%%%%%%%%%%%%%%%%%%%%%%%%%%%%%%%%%%%%%%%%%%%%%%%%%%%%%%%%%%%


\begin{document}
\maketitle
\tableofcontents
\newpage

\newpage
\section{Problem 1}%
\label{sec:problem_1}

Let normal matrices, those with diagonalization be on the form \[
A = U \Lambda U^{H} 
\] 
Where $\Lambda $ is a diagonal complex $n\times n $ matrix and $U$ a unitary (complex) matrix such that $U ^{H} U = I$ (recall that $U^{H}$ is the complex conjugate of $U^{T}$ ).
\newpara
Show that for any such matrix, one has $\|A\|_{2} = \rho \left( A \right)$, where $\rho \left( A \right) $ is the spectral radius of $A$ .



\newpara
  \textbf{Answer.} 
  \newpara Ideas
  \begin{itemize}
    \item Complex conjugate \[
     A^{H} = \overline{A^{T}} 
    \] 
    \item  Definition of spectral radius is denoted by \[
        \rho \left( A \right) = \max_{\lambda _{i} \in  \sigma \left( A \right)} \left\{ \lambda _{i} \right\}
    \] 
    of the eigenvalues $\lambda _{i}$ in the eigenvalue spectrum $\sigma \left( A \right)$.
  
  % \item As we remember one can denote the norm such that \[
  %     \begin{split}
  %     \|A\|_{2}  & = \sup_{x \neq 0}  \sqrt{\frac{\|Ax\|_{2}^{2}}{\|x\|^{2}_2}    } = \sup _{x \neq0}  \sqrt{\frac{ \left<Ax, Ax \right> }{\|x\|_{2}^{2}}}    \\
  %      & = \sup _{x \neq0}  \sqrt{\frac{ \left<A^{H}Ax, x \right> }{\|x\|_{2}^{2}}} \\
  %      & = \sup _{x \neq0}  \sqrt{\frac{ \left<U^{H}\Lambda ^{H} U U^{H} \Lambda U^{H}x, x \right> }{\|x\|_{2}^{2}}}\\
  %      & = \sup _{x \neq0}  \sqrt{\frac{\left<U^{H} \Lambda ^2 U^{H}x, x \right>}{\|x\|_{2}^{2}}}\\
  %     \end{split} 
  % \] 
\item $UAU^{H}  = \Lambda = diag\left( \sigma _{1} , \ldots , \sigma _{n} \right) $ 
\item  Let \[
\|A\|_{2}^{2} = \sup _{x\neq 0} \sqrt{\frac{\|A \|_{2}^{2}}{\|x\|_{2}^{2}}}  = \sup _{x \neq 0} \sqrt{\frac{\left< Ax , Ax \right>}{\|x\|_{2}^{2}}}   = \sup _{x \neq 0} \sqrt{\frac{\left<A^{H} Ax ,x \right>}{\|x\|_{2}^{2}} }  
\]  
If we use the fact that $A = U^{H} \Lambda  U$, can we substitue $y = U^{H} x$ , such that  \[
  \begin{split}
\|A\|_{2}^{2}  & = \sup _{x \neq 0} \sqrt{\frac{\|Ax \|_{2}^{2}}{  \|x\|_{2}^{2}}}  = \sup _{y \neq 0} \sqrt{\frac{\left< AU y, AU y\right>}{ \|y\|_{2}^{2}}}  \\
 \implies  \|y\|_{2}^{2}  & = \left<U^{H} x, U^{H} x \right> =  \left< U U^{H} x,x \right>  = \|x\|_{2}^{2} \\
\implies  \left<AUy, AUy \right> &=  \left<\left( AU \right)^{H} AU y , y \right> = \left<U^{H} A^{H} AU y, y \right> \\
  \end{split} 
\] 
Since $A^{H} A$ is unitary can we write $U^{H} A^{H} A U = diag (\mu _{1} , \mu _{2} , \ldots, \mu _{n} )$.   Which then ends up with the relationship \[
\|A\|_{2}^{2} = \sup_{y \neq 0 }  \frac{\sum_{i=1}^{n} \mu _{i} \cdot \left| y_{i} \right|^{2} }{ \sum_{i=1}^{}  \left\lvert y_{i} \right\rvert^{2} } =  \max_{i} \left( \mu _{i} \right) 
\] 
  
  \end{itemize}




\newpage
\section{References}%
\label{sec:references}

\bibliographystyle{plain}
\bibliography{references}
\end{document}

