\documentclass{article}
\usepackage[utf8]{inputenc}

\title{Mathemathical Modelling}
\author{isakhammer }
\date{2020}

\usepackage{natbib}
\usepackage{graphicx}
\usepackage{amsmath}
\usepackage{amsthm}
\usepackage{amsfonts}
\usepackage{mathtools}
\usepackage{enumerate}
\usepackage{todonotes}


\usepackage{hyperref} 
\hypersetup{
  colorlinks=true, %set true if you want colored links
  linktoc=all,     %set to all if you want both sections and subsections linked
  linkcolor=blue,  %choose some color if you want links to stand out
} 
\hypersetup{linktocpage}


% inscape-figures
\usepackage{import}
\usepackage{pdfpages}
\usepackage{transparent}
\usepackage{xcolor}
\newcommand{\incfig}[2][1]{%
\def\svgwidth{#1\columnwidth}
\import{./figures/}{#2.pdf_tex} } \pdfsuppresswarningpagegroup=1

% Box environment
\usepackage{tcolorbox}
\usepackage{mdframed}
\newmdtheoremenv{definition}{Definition}[section]
\newmdtheoremenv{theorem}{Theorem}[section]
\newmdtheoremenv{lemma}{Lemma}[section]

\theoremstyle{remark}
\newtheorem*{remark}{Remark}
\newtheorem{example}{Example}


\begin{document}
\maketitle
\tableofcontents
\newpage
\newpage
\section{Lecture 1}%
\label{sec:lecture_1}

\subsection{Practical Information}%
\label{sub:practical_information}

You need to know 
\begin{itemize}
  \item Separable 1. order equations.
  \item Linear 1. order equations.
  \item 2. order linear equations with constant coefficients.
\end{itemize}

\subsection{Dimensional Analysis}%
\label{sub:dimensional_analysis}

Basic facts
\begin{itemize}
  \item Any physical relation has to make sense dimensionally.
  \item Any physical relation must be valid for any choice of fundamental units.
\end{itemize}
\begin{remark}
  \todo{ Make sure remark looks better }
  \begin{itemize}
    \item
  \textbf{Forbidden} $3m + 2kg = ?$ 
\item $m = f\left( x,t \right)$ is legal
\item  $e^{-t}$ and $s = 5t ^{2}$ , is nonsense
  \end{itemize}
\end{remark}

\begin{itemize}
  \item
\textbf{Dimension}  is length, mass , energy, etc.
\item
\textbf{Unit} is meter, feet, year, etc
\end{itemize}
Given a variable $R$, we write $R =\overbrace{v\left( R \right)}^\text{numerical value}   \underbrace{\left[ R \right]}_\text{unit}$.   
\par
If we have a physical relation that is dimensionall correct that \[
f\left( R_{1}, R_{2}, \ldots, R_{n} \right) = 0 \quad  \to \quad f\left( v\left( R_{1} \right), v\left( R_{2} \right), \ldots, v\left( R_n \right) \right)   = 0
\] 
\subsection{Fundamental Units}%
\label{sub:fundamental_units}

Given units $F_{1}, F_{2}, \ldots , F_{m}$ for fundamental if \[
  F_{1}^{\alpha_1}, F_{2}^{\alpha_2}, \ldots , F_{m}^{\alpha m} = 0 \quad  \to \quad  \alpha_1 = \alpha_{2} = \ldots = 0   
\] 
This units are then independent.
\begin{example}
  The units $kg, m, s$ are independent.  
\end{example}
\begin{example}
  In a right angle triangle with angle $\alpha $ and hypothenus $c$. We know the area $A$ is uniquely determined by $\alpha $ and $c$ \[
  A = f\left( c,\alpha  \right)
  \] 
  $\alpha $ is dimensialless since  $\alpha  = \frac{s}{r}$. Since $A$ scales as the square of the length, then is \[
  f\left( ac, \alpha  \right) = a^2f\left( c,\alpha  \right)
  \]   
  \[
    c = 1 \to f\left( a, \alpha  \right) = a^2f\left( 1,\alpha  \right) = a^2h\left( \alpha  \right)
  \] 
  Which then ends up with the relation \[
  A = a^2h\left( \alpha  \right)
  \] 
\end{example} 

\todo[inline]{ Make corollary environmet }


Lets derive $A = a ^2 h\left( \alpha   \right)$ somwhat differently. We know there is a relation $f\left( A, c, \alpha  \right) = 0$ . We want to introduce new variables.\[
\Pi_1 = \frac{A}{c^2}, \quad  c = c_1, \quad \alpha = \alpha _1   
\] 
which means $f\left( c^2 \Pi_1, c, \alpha   \right) = 0$  and $h\left( \Pi_1, \alpha , c \right) = 0$. $h$ must be dimensially consistent $\to$ $h$ must be independent of $c$. 
\begin{equation*}
  \begin{split}
    h\left( \Pi_1, \alpha  \right) &= 0 \leftrightarrow \Pi_1 = k\left( \alpha  \right) \\
    \to  \frac{A}{c^2} &= k\left( \alpha  \right) \quad   \leftrightarrow \quad A = c^2k\left( \alpha  \right) 
  \end{split}
.\end{equation*}

\subsection{Trinity of the first atomic blast}%
\label{sub:trinity_of_the_first_atomic_blast}

We assume there is a relation \[
f\left( E, \rho, r , t \right) = 0
\] 

\begin{itemize}
  \item Energy: $E$, $\left[ E \right] = kg m^2 s^{-2}$ 
  \item Mass density of air: $\rho$, $\left[ \rho \right] = kg ^{-3}$
  \item Radius: $r$ , $\left[ r \right] = m$
  \item Time: $t$, $\left[ t \right] =s$ 
\end{itemize}

We choose 3 independent variables, say $r, t, \rho$. Also we call $r, t, \rho  $ \textbf{core variables}.  Let is define a dimensionalless number $\Pi_{1} $ such that \[
\left[ \Pi _{1} \right] = 0
\] 

The relation is now given by $h\left( \Pi , t, r, \rho \right) = 0$, where $h$ is independent of $t$ , $r$ and $\rho$. Which in fact is $h\left( \Pi  \right) = 0$, where $\Pi _{1} = c$ s.t. $\left[ c \right] = 1$.  
\par
Given by the definitnion is \[
\frac{E t^{2}}{\rho r ^{5}} = c \quad  \to \quad  E = \frac{c \rho r^{5}}{t^{2}}  
\] 
Using $\rho = 12 kgm^{-3}$, $r = 110m$ , $t = 6 \cdot 10^{-3}$ do we end up with the relation \[
E = c \cdot  7.5 \cdot  10 ^{13} J
\] 

\subsection{Steady-state single phase flow in a uniform straight pipeline}%
\label{sub:steady_state_single_phase_flow}
\todo[inline]{ Figure of a pipe }

Pipe with flow $u$, length $L$ and pressure drop $\Delta p$ Then there is a relation between 
\begin{itemize}
  \item $L$ : length, $\left[ L \right] = m$ 
  \item $D$: diameter $\left[ D \right] = m$
  \item $u$: flow rate $\left[ u \right] = ms^{-1}$
  \item $\Delta p$: Pressure drop, $\left[ \Delta kg m^{-1} s^{-2} \right]$
  \item $\mu$: (Shear) viscousity $\left[ \mu \right] = kg m^{-1} s^{-1}$ 
  \item $\rho$: mass density: $\left[ \rho \right] = kg m^{-3}$
  \item $E$: Wall roughness: $\left[ E \right] = m$
\end{itemize}

We have to choose $3$ core variables and they are not unique.  Since we have $3$ independent units $\rho , u, D$ are independent such that it can be a core variable: \[
\Pi _{1} = \frac{L}{D} \quad , \quad  \Pi _{2} = \frac{\Delta p}{\rho u^2} \quad , \quad \Pi _{3} = \frac{\rho}{\mu} \quad , \quad  \Pi _{4} = \frac{E}{D}    
\] 
Then the relation is \[
  \begin{split}
    f\left( \Pi _{1}, \Pi _{2} , \Pi _3, \Pi ^{4},\rho, D, u \right) &= 0  \quad  \Pi _{2} = h\left( \Pi _{1}, \Pi _{3}, \Pi _{4} \right) \leftrightarrow \frac{\Delta p}{\rho u^2} = h\left( \Pi _{1}, \Pi _3 , \Pi _{4} \right) \\
    \to  & \frac{\Delta p}{ u^2 \rho} = \Pi _{1} k\left( \Pi _{3}, \Pi _{4} \right)\\
    \Delta p &= u^2 \rho \frac{L}{D} k \left( \frac{\rho D u}{\mu}, \frac{E}{D} \right)   \\
    \text{measure} \quad & \frac{\rho D \mu}{ \mu} \quad ,\quad    k= \frac{\Delta p D }{u^2\rho}
  \end{split}
\]
\newpage
\section{Lecture 2}%
\label{sec:lecture_2}
\subsection{Practical Information }%
\label{sub:practical_information_2}

Ask for zoom meeting. ola.mahlen@ntnu.no, wednesday 13-14.

\subsection{Recall}%
\label{sub:repetition}

Last time did we consider steady-state single phase in a flow  in a pipe.
\begin{itemize}
  \item Assuming $f\left( L, \Delta p, u, \mu , D ,E , \rho \right) = 0$ we arrive with this formula \[
  \frac{\Delta p D}{u^{2} \rho L}  = k \left( \overbrace{\frac{\rho u D}{\mu }}^\text{Reynhold number}  , \underbrace{\frac{E}{D}}_\text{Relative wall roughness}  \right)
  \] 
\item Dimensionless numbers are often called \textbf{dimensionless groups}. Such numbers are independent of choice of fundamental units. They have real physical meaning. \textbf{Reynholds number}  $R_{e} $ essentially define what type of flow. Usually $R_{e} < 2000$ is it laminar flow and $R_{e} > 4000$ turbulent flow.
\end{itemize}

\subsection{Scaling}%
\label{sub:scaling_2}

Let a pipe have diameter $D$ and flow rate $u$ such that $ t_{v} = \frac{D}{ u} $. Then can we describe \[
t_{\alpha } =  \frac{D^{2}}{ \frac{ \mu }{e} } 
\] 
where $\mu $ is the kinematic viscosity. Then is $R_{e} $ defined such that \[
R_{e} = \frac{t_{\alpha }}{ t_{v}} 
\]
\par
Assume we have the relation \[
R_{1} = f\left( R_{2}, \ldots, R_{m} \right)
\] 
Such that it exist an \[
\Pi _{1} = g\left( \Pi _{2}, \Pi _{2} , \ldots, \Pi _{m-k}  \right).
\] 
\subsection{Buckinghams Pi-Theorem}%
\label{sub:buckinghams_pi_theorem}

Assume we have a dimensionally valid relation $f\left( R_{1}, \ldots, R_{m} \right) = 0$ and a set of fundemental units $F_{1}, F_{2}, \ldots, F_{n}$ such that \[
\left[ R_{j} \right] = F_{1} ^{a_{j1}} F_{2} ^{a_{j2}} \ldots F_{n}^{a_{jn} } \quad  j = 1,2, \ldots, m 
\] 
This then defines the dimension matrix $A$ given by 
 % \begin{table}[]
 % \begin{tabular}{lllll}
 %  &  $F_{1}$ & $F_{2}$ & $\ldots$  & $F_{n}$  \\
 % $R_1$ &$a_{11}$  &$a_{11}$  &     & $a_{1n}$ &  \\
 % $R_2$ &$a_{21}$  &$a_{21}$  &     &  $a_{2n}$ &  \\
 % $\vdots$  &  & $\ddots$  &  &  \\
 % $R_n$ &  $a_{m1}$ & $\ldots$   &  &  $a_{mn}$&
 % \end{tabular}
 % \end{table}
\todo[inline]{ Fix better table environment. }

Let $rank\left( A \right) = dim \left( row\left( A \right) \right) = k$. This translates to that we have $k$ dimensionally  independent variables. Choosing $k$ linearly independent row vectors, corresponds to choosing core variables. Let this basis be $\mathbf{a}_{i1}, \mathbf{a}_{i2}, \ldots , \mathbf{a}_{ik}$. Let the rest of the row vectors be \[
\mathbf{a}_{j_{1}} , \mathbf{a}_{j_{2}}, \ldots, \mathbf{a}_{j_{m-k}}
\] 
Then is $\mathbf{a}_{j_{r}} =  \sum_{s=1}^{k} C _{j_{r}, s} \mathbf{a_{i_{s}}}  $ where $r=  1, \ldots, m-k$. We end up with the equation \[
  \Pi _{r} = \frac{R_{j_{r}}}{ R_{i_{1}} ^{ r_{j_{r}, 1}} R_{j_{2}}^{a_{j_{r}, 2}} \ldots R_{j_{k}} ^{ a_{j_{r}, k}} } 
\] 
Are dimensionally numbers. \par Our relation becomes \[
g\left( \Pi _{1} , \ldots, \Pi _{m-k} \right) = 0, \quad \begin{cases}
  i_{1}, i_{2} , \ldots, i_{k}  &  \\
  j_{1} , \ldots, j_{m-k}
\end{cases} 
\] 

\begin{tcolorbox}
  \textbf{Example.} Swinging pendulum 

Assume there is a relation \[
f\left( w, \alpha _{0} , L, M , g,  \right) = 0
\] 
where $w$ is the frequenxy, $g$ gravitational acceleration, $M$ mass, $\alpha _{0}$ the swinging angle.
We can set $L, M, g$ as core variables such that \[
  \begin{split}
    \left[ \frac{L}{g} \right] &= s^2 \quad  \to  \quad  \left[ \frac{L}{g} w^2 \right]  = 1 \\
    f\left( w, \alpha _{0} , L, M, g \right) &=  0 \implies  \quad g\left( \alpha _{0} , \frac{L w^2}{g}  \right)   = 0
  \end{split} 
\] 
\end{tcolorbox}

\subsection{Scaling}%
\label{sub:scaling}
We have a problem at hand, usually differential equations. Then we tru to find representative scales for the various variables, and then write the equation on so-called fimensionless form. This has several advantages 
\begin{itemize}
  \item Our dimensionless variables are of order $1$ .
  \item  We get rid of a lot of physical constants.
  \item It makes us able to see what terms are "small" in the equation.
    The idea is to introduce dimensionless variables by introducing appropiate scales.  If we have a stick of lenght $L$, we choose $L$ as length scale i.e \[
    x^{*} = Lx \quad  \text{Where} \quad   x \quad   \text{is dimenionless} 
    \] 
\end{itemize}

\begin{tcolorbox}
  \textbf{Example.} Heat flow in a rod with length $L$. Let $u^{*} \left( x^{*}, t^{*} \right)$ be the temperatur with the boundary conditions \[
  u^{*}\left( 0, t^{*} \right) = 0 \quad  u^{*}\left( L, t^{*} \right) = 0 
  \] 

  If we let the model be \[
    \begin{split}
    \frac{\partial u^{*}}{\partial t^{*}}   & = D \cdot  \frac{\partial ^2 u^{*}}{\partial {x^{*} }^2},  \quad   u^{*}\left( 0, t^{*} \right) = 0 \quad  u^{*}\left( L, t^{*} \right) = 0 \\
 u^{*}\left( x^{*}, 0 \right)   & = u_{0} \sin \left( \pi  \frac{x^{*}}{ L}  \right) 
    \end{split} 
  \] 
  We fund the tune scale $T$ by scales \textbf{balancing the equation} . 
  \par Let $x^{*} = Lx$ , and $t^{*} = Tt$, where $T$ is to be determined $u^{*} = u_{0} u$. If we find $u\left( x,t \right) $,  then the physical temperature is given by \[
    u^{*}\left( x^{*} , t^{*} \right) = u_{0} u\left( \frac{x^{*}}{L} , \frac{t^{*}}{T}  \right)
  \]  
  We have $u\left( 0,t \right) = u\left( 1,t \right) = 0$ \[
    \begin{split}
  \frac{\partial u^{*}}{\partial t^{*}}   & = D \frac{\partial ^2 u^{*}}{\partial {x^{*}}^2}  \quad  \implies  \quad  \frac{u_{0}}{T}  \frac{\partial u}{\partial t}  = \frac{u_{0}}{L^2} D \frac{\partial ^2}{\partial x^2}    \\
    & \leftrightarrow  \frac{\partial u}{\partial  t}   = \left( \frac{T D}{L^2}  \right) \frac{\partial ^2 u}{\partial x ^2}  \quad  \text{Balancing the equation}   \\
      \frac{TD}{ L^2}    & = 1 \quad \implies  T = \frac{L^2}{D}   \\
     u^{*}\left( x^{*}, 0 \right) &=  u_{0} \sin \left( \pi \frac{x^{*}}{L}  \right)  \\
     u\left( x, 0 \right) &=  \sin \left( \pi x \right) 
    \end{split} 
  \] 
  which fulfills the condition \[
  \frac{\partial u}{\partial t}  = \frac{\partial ^2 u}{\partial  x^2}  \quad  , u\left( 0,t \right) = u\left( 1,t \right) = 0 
  \] 
\end{tcolorbox}



\newpage
\section{References}%
\label{sec:references}


  
\bibliographystyle{plain}
\bibliography{references}
\end{document}
