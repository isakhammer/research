\documentclass{article}
\usepackage[utf8]{inputenc}

\title{Mathemathical Modelling}
\author{isakhammer }
\date{2020}

%%%% DEPENDENCIES v1.3 %%%%%%

\usepackage{natbib}
\usepackage{graphicx}
\usepackage{amsmath}
\usepackage{amsthm}
\usepackage{amsfonts}
\usepackage{mathtools}
%\usepackage{enumerate}
\usepackage{enumitem}
\usepackage{todonotes}
\usepackage{esint}
\usepackage{float}

\usepackage{amssymb}

\usepackage{hyperref} 
\hypersetup{
  colorlinks=true, %set true if you want colored links
  linktoc=all,     %set to all if you want both sections and subsections linked
  linkcolor=blue,  %choose some color if you want links to stand out
} 
\hypersetup{linktocpage}


% inscape-figures
\usepackage{import}
\usepackage{pdfpages}
\usepackage{transparent}
\usepackage{xcolor}
\newcommand{\incfig}[2][1]{%
\def\svgwidth{#1\columnwidth}
\import{./figures/}{#2.pdf_tex} } \pdfsuppresswarningpagegroup=1

% Box environment
\usepackage{tcolorbox}
\usepackage{mdframed}
\newmdtheoremenv{definition}{Definition}[section]
\newmdtheoremenv{theorem}{Theorem}[section]
\newmdtheoremenv{lemma}{Lemma}[section]

% \DeclareMathOperator{\span}{span}

\theoremstyle{remark}
\newtheorem*{remark}{Remark}
%\newtheorem{example}{Example}

\newcommand{\newpara}
  {
  \vskip 0.4cm
  }

%%%%%%%%%%%%%%%%%%%%%%%%%%%%%%%%%%%%%%%%%%%%%%%%%%%%%%%%%%%%



\begin{document}
\maketitle
\tableofcontents
\newpage
\newpage
\section{Lecture 1}%
\label{sec:lecture_1}

\subsection{Practical Information}%
\label{sub:practical_information}

You need to know 
\begin{itemize}
  \item Separable 1. order equations.
  \item Linear 1. order equations.
  \item 2. order linear equations with constant coefficients.
\end{itemize}

\subsection{Dimensional Analysis}%
\label{sub:dimensional_analysis}

Basic facts
\begin{itemize}
  \item Any physical relation has to make sense dimensionally.
  \item Any physical relation must be valid for any choice of fundamental units.
\end{itemize}
\begin{remark}
  \todo{ Make sure remark looks better }
  \begin{itemize}
    \item
  \textbf{Forbidden} $3m + 2kg = ?$ 
\item $m = f\left( x,t \right)$ is legal
\item  $e^{-t}$ and $s = 5t ^{2}$ , is nonsense
  \end{itemize}
\end{remark}

\begin{itemize}
  \item
\textbf{Dimension}  is length, mass , energy, etc.
\item
\textbf{Unit} is meter, feet, year, etc
\end{itemize}
Given a variable $R$, we write $R =\overbrace{v\left( R \right)}^\text{numerical value}   \underbrace{\left[ R \right]}_\text{unit}$.   
\par
If we have a physical relation that is dimensionall correct that \[
f\left( R_{1}, R_{2}, \ldots, R_{n} \right) = 0 \quad  \to \quad f\left( v\left( R_{1} \right), v\left( R_{2} \right), \ldots, v\left( R_n \right) \right)   = 0
\] 
\subsection{Fundamental Units}%
\label{sub:fundamental_units}

Given units $F_{1}, F_{2}, \ldots , F_{m}$ for fundamental if \[
  F_{1}^{\alpha_1}, F_{2}^{\alpha_2}, \ldots , F_{m}^{\alpha m} = 0 \quad  \to \quad  \alpha_1 = \alpha_{2} = \ldots = 0   
\] 
This units are then independent.
\textbf{Example.} 
   The units $kg, m, s$ are independent.  

   \textbf{Example.} 
   In a right angle triangle with angle $\alpha $ and hypothenus $c$. We know the area $A$ is uniquely determined by $\alpha $ and $c$ \[
   A = f\left( c,\alpha  \right)
   \] 
   $\alpha $ is dimensialless since  $\alpha  = \frac{s}{r}$. Since $A$ scales as the square of the length, then is \[
   f\left( ac, \alpha  \right) = a^2f\left( c,\alpha  \right)
   \]   
   \[
     c = 1 \to f\left( a, \alpha  \right) = a^2f\left( 1,\alpha  \right) = a^2h\left( \alpha  \right)
   \] 
   Which then ends up with the relation \[
   A = a^2h\left( \alpha  \right)
   \] 

\todo[inline]{ Make corollary environmet }


Lets derive $A = a ^2 h\left( \alpha   \right)$ somwhat differently. We know there is a relation $f\left( A, c, \alpha  \right) = 0$ . We want to introduce new variables.\[
\Pi_1 = \frac{A}{c^2}, \quad  c = c_1, \quad \alpha = \alpha _1   
\] 
which means $f\left( c^2 \Pi_1, c, \alpha   \right) = 0$  and $h\left( \Pi_1, \alpha , c \right) = 0$. $h$ must be dimensially consistent $\to$ $h$ must be independent of $c$. 
\begin{equation*}
  \begin{split}
    h\left( \Pi_1, \alpha  \right) &= 0 \leftrightarrow \Pi_1 = k\left( \alpha  \right) \\
    \to  \frac{A}{c^2} &= k\left( \alpha  \right) \quad   \leftrightarrow \quad A = c^2k\left( \alpha  \right) 
  \end{split}
.\end{equation*}

\subsection{Trinity of the first atomic blast}%
\label{sub:trinity_of_the_first_atomic_blast}

We assume there is a relation \[
f\left( E, \rho, r , t \right) = 0
\] 

\begin{itemize}
  \item Energy: $E$, $\left[ E \right] = kg m^2 s^{-2}$ 
  \item Mass density of air: $\rho$, $\left[ \rho \right] = kg ^{-3}$
  \item Radius: $r$ , $\left[ r \right] = m$
  \item Time: $t$, $\left[ t \right] =s$ 
\end{itemize}

We choose 3 independent variables, say $r, t, \rho$. Also we call $r, t, \rho  $ \textbf{core variables}.  Let is define a dimensionalless number $\Pi_{1} $ such that \[
\left[ \Pi _{1} \right] = 0
\] 

The relation is now given by $h\left( \Pi , t, r, \rho \right) = 0$, where $h$ is independent of $t$ , $r$ and $\rho$. Which in fact is $h\left( \Pi  \right) = 0$, where $\Pi _{1} = c$ s.t. $\left[ c \right] = 1$.  
\par
Given by the definitnion is \[
\frac{E t^{2}}{\rho r ^{5}} = c \quad  \to \quad  E = \frac{c \rho r^{5}}{t^{2}}  
\] 
Using $\rho = 12 kgm^{-3}$, $r = 110m$ , $t = 6 \cdot 10^{-3}$ do we end up with the relation \[
E = c \cdot  7.5 \cdot  10 ^{13} J
\] 

\subsection{Steady-state single phase flow in a uniform straight pipeline}%
\label{sub:steady_state_single_phase_flow}
\todo[inline]{ Figure of a pipe }

Pipe with flow $u$, length $L$ and pressure drop $\Delta p$ Then there is a relation between 
\begin{itemize}
  \item $L$ : length, $\left[ L \right] = m$ 
  \item $D$: diameter $\left[ D \right] = m$
  \item $u$: flow rate $\left[ u \right] = ms^{-1}$
  \item $\Delta p$: Pressure drop, $\left[ \Delta kg m^{-1} s^{-2} \right]$
  \item $\mu$: (Shear) viscousity $\left[ \mu \right] = kg m^{-1} s^{-1}$ 
  \item $\rho$: mass density: $\left[ \rho \right] = kg m^{-3}$
  \item $E$: Wall roughness: $\left[ E \right] = m$
\end{itemize}

We have to choose $3$ core variables and they are not unique.  Since we have $3$ independent units $\rho , u, D$ are independent such that it can be a core variable: \[
\Pi _{1} = \frac{L}{D} \quad , \quad  \Pi _{2} = \frac{\Delta p}{\rho u^2} \quad , \quad \Pi _{3} = \frac{\rho}{\mu} \quad , \quad  \Pi _{4} = \frac{E}{D}    
\] 
Then the relation is \[
  \begin{split}
    f\left( \Pi _{1}, \Pi _{2} , \Pi _3, \Pi ^{4},\rho, D, u \right) &= 0  \quad  \Pi _{2} = h\left( \Pi _{1}, \Pi _{3}, \Pi _{4} \right) \leftrightarrow \frac{\Delta p}{\rho u^2} = h\left( \Pi _{1}, \Pi _3 , \Pi _{4} \right) \\
    \to  & \frac{\Delta p}{ u^2 \rho} = \Pi _{1} k\left( \Pi _{3}, \Pi _{4} \right)\\
    \Delta p &= u^2 \rho \frac{L}{D} k \left( \frac{\rho D u}{\mu}, \frac{E}{D} \right)   \\
    \text{measure} \quad & \frac{\rho D \mu}{ \mu} \quad ,\quad    k= \frac{\Delta p D }{u^2\rho}
  \end{split}
\]
\newpage
\section{Lecture 2}%
\label{sec:lecture_2}
\subsection{Practical Information }%
\label{sub:practical_information_2}

Ask for zoom meeting. ola.mahlen@ntnu.no, wednesday 13-14.

\subsection{Recall}%
\label{sub:repetition}

Last time did we consider steady-state single phase in a flow  in a pipe.
\begin{itemize}
  \item Assuming $f\left( L, \Delta p, u, \mu , D ,E , \rho \right) = 0$ we arrive with this formula \[
  \frac{\Delta p D}{u^{2} \rho L}  = k \left( \overbrace{\frac{\rho u D}{\mu }}^\text{Reynhold number}  , \underbrace{\frac{E}{D}}_\text{Relative wall roughness}  \right)
  \] 
\item Dimensionless numbers are often called \textbf{dimensionless groups}. Such numbers are independent of choice of fundamental units. They have real physical meaning. \textbf{Reynholds number}  $R_{e} $ essentially define what type of flow. Usually $R_{e} < 2000$ is it laminar flow and $R_{e} > 4000$ turbulent flow.
\end{itemize}

\subsection{Scaling}%
\label{sub:scaling_2}

Let a pipe have diameter $D$ and flow rate $u$ such that $ t_{v} = \frac{D}{ u} $. Then can we describe \[
t_{\alpha } =  \frac{D^{2}}{ \frac{ \mu }{e} } 
\] 
where $\mu $ is the kinematic viscosity. Then is $R_{e} $ defined such that \[
R_{e} = \frac{t_{\alpha }}{ t_{v}} 
\]
\par
Assume we have the relation \[
R_{1} = f\left( R_{2}, \ldots, R_{m} \right)
\] 
Such that it exist an \[
\Pi _{1} = g\left( \Pi _{2}, \Pi _{2} , \ldots, \Pi _{m-k}  \right).
\] 
\subsection{Buckinghams Pi-Theorem}%
\label{sub:buckinghams_pi_theorem}

Assume we have a dimensionally valid relation $f\left( R_{1}, \ldots, R_{m} \right) = 0$ and a set of fundemental units $F_{1}, F_{2}, \ldots, F_{n}$ such that \[
\left[ R_{j} \right] = F_{1} ^{a_{j1}} F_{2} ^{a_{j2}} \ldots F_{n}^{a_{jn} } \quad  j = 1,2, \ldots, m 
\] 
This then defines the dimension matrix $A$ given by 

\begin{table}[htpb]
  \centering
  \caption{}
  \label{tab:label}
  \begin{tabular}{c | c c c c c}
  &  $F_{1}$ & $F_{2}$ & $\ldots$  & $F_{n}$ & \\ \hline
  $R_1$ &$a_{11}$  &$a_{11}$  &     & $a_{1n}$ &  \\
  $R_2$ &$a_{21}$  &$a_{21}$  &     &  $a_{2n}$ &  \\
  $\vdots$  &  & $\ddots$   &  &  \\
  $R_n$ &  $a_{m1}$ & $\ldots$   &  &  $a_{mn}$&
  \end{tabular}
\end{table}

\todo[inline]{ Fix better table environment. }

Let $rank\left( A \right) = dim \left( row\left( A \right) \right) = k$. This translates to that we have $k$ dimensionally  independent variables. Choosing $k$ linearly independent row vectors, corresponds to choosing core variables. Let this basis be $\mathbf{a}_{i1}, \mathbf{a}_{i2}, \ldots , \mathbf{a}_{ik}$. Let the rest of the row vectors be \[
\mathbf{a}_{j_{1}} , \mathbf{a}_{j_{2}}, \ldots, \mathbf{a}_{j_{m-k}}
\] 
Then is $\mathbf{a}_{j_{r}} =  \sum_{s=1}^{k} C _{j_{r}, s} \mathbf{a_{i_{s}}}  $ where $r=  1, \ldots, m-k$. We end up with the equation \[
  \Pi _{r} = \frac{R_{j_{r}}}{ R_{i_{1}} ^{ r_{j_{r}, 1}} R_{j_{2}}^{a_{j_{r}, 2}} \ldots R_{j_{k}} ^{ a_{j_{r}, k}} } 
\] 
Are dimensionally numbers. \par Our relation becomes \[
g\left( \Pi _{1} , \ldots, \Pi _{m-k} \right) = 0, \quad \begin{cases}
  i_{1}, i_{2} , \ldots, i_{k}  &  \\
  j_{1} , \ldots, j_{m-k}
\end{cases} 
\] 

\begin{tcolorbox}
  \textbf{Example.} Swinging pendulum 

Assume there is a relation \[
f\left( w, \alpha _{0} , L, M , g,  \right) = 0
\] 
where $w$ is the frequenxy, $g$ gravitational acceleration, $M$ mass, $\alpha _{0}$ the swinging angle.
We can set $L, M, g$ as core variables such that \[
  \begin{split}
    \left[ \frac{L}{g} \right] &= s^2 \quad  \to  \quad  \left[ \frac{L}{g} w^2 \right]  = 1 \\
    f\left( w, \alpha _{0} , L, M, g \right) &=  0 \implies  \quad g\left( \alpha _{0} , \frac{L w^2}{g}  \right)   = 0
  \end{split} 
\] 
\end{tcolorbox}

\subsection{Scaling}%
\label{sub:scaling}
We have a problem at hand, usually differential equations. Then we tru to find representative scales for the various variables, and then write the equation on so-called fimensionless form. This has several advantages 
\begin{itemize}
  \item Our dimensionless variables are of order $1$ .
  \item  We get rid of a lot of physical constants.
  \item It makes us able to see what terms are "small" in the equation.
    The idea is to introduce dimensionless variables by introducing appropiate scales.  If we have a stick of lenght $L$, we choose $L$ as length scale i.e \[
    x^{*} = Lx \quad  \text{Where} \quad   x \quad   \text{is dimenionless} 
    \] 
\end{itemize}

\begin{tcolorbox}
  \textbf{Example.} Heat flow in a rod with length $L$. Let $u^{*} \left( x^{*}, t^{*} \right)$ be the temperatur with the boundary conditions \[
  u^{*}\left( 0, t^{*} \right) = 0 \quad  u^{*}\left( L, t^{*} \right) = 0 
  \] 

  If we let the model be \[
    \begin{split}
    \frac{\partial u^{*}}{\partial t^{*}}   & = D \cdot  \frac{\partial ^2 u^{*}}{\partial {x^{*} }^2},  \quad   u^{*}\left( 0, t^{*} \right) = 0 \quad  u^{*}\left( L, t^{*} \right) = 0 \\
 u^{*}\left( x^{*}, 0 \right)   & = u_{0} \sin \left( \pi  \frac{x^{*}}{ L}  \right) 
    \end{split} 
  \] 
  We fund the tune scale $T$ by scales \textbf{balancing the equation} . 
  \par Let $x^{*} = Lx$ , and $t^{*} = Tt$, where $T$ is to be determined $u^{*} = u_{0} u$. If we find $u\left( x,t \right) $,  then the physical temperature is given by \[
    u^{*}\left( x^{*} , t^{*} \right) = u_{0} u\left( \frac{x^{*}}{L} , \frac{t^{*}}{T}  \right)
  \]  
  We have $u\left( 0,t \right) = u\left( 1,t \right) = 0$ \[
    \begin{split}
  \frac{\partial u^{*}}{\partial t^{*}}   & = D \frac{\partial ^2 u^{*}}{\partial {x^{*}}^2}  \quad  \implies  \quad  \frac{u_{0}}{T}  \frac{\partial u}{\partial t}  = \frac{u_{0}}{L^2} D \frac{\partial ^2}{\partial x^2}    \\
    & \leftrightarrow  \frac{\partial u}{\partial  t}   = \left( \frac{T D}{L^2}  \right) \frac{\partial ^2 u}{\partial x ^2}  \quad  \text{Balancing the equation}   \\
      \frac{TD}{ L^2}    & = 1 \quad \implies  T = \frac{L^2}{D}   \\
     u^{*}\left( x^{*}, 0 \right) &=  u_{0} \sin \left( \pi \frac{x^{*}}{L}  \right)  \\
     u\left( x, 0 \right) &=  \sin \left( \pi x \right) 
    \end{split} 
  \] 
  which fulfills the condition \[
  \frac{\partial u}{\partial t}  = \frac{\partial ^2 u}{\partial  x^2}  \quad  , u\left( 0,t \right) = u\left( 1,t \right) = 0 
  \] 
\end{tcolorbox}



\newpage
\section{Lecture 3}%
\label{sec:lecture_something}

\subsection{Recall}%
\label{sub:recall_2}

\[
  \begin{split}
\frac{\partial u^{*}}{\partial t^{*}}   & = D \frac{\partial ^2 u^{*}}{\partial x^{*2} } \\
0 &  \le x^{*} \le L \\
x^{*}  &  = Lx \\
t^{*} &=   Tt \\
u^{*} &=   u_{0}
  \end{split} 
\] 
We can also recall \[
  \begin{split}
u^{*}\left( x^{*} , t^{*}  \right) &=  u_{0} u\left( \frac{x^{*}}{ L} , \frac{t^{*} }{T}  \right) \\
\frac{u_{0}}{T}  \frac{\partial u}{\partial t}  &=  D \frac{u_{0}}{L^2} \implies \frac{\partial u}{\partial t}  = \frac{TD}{L^2}  \frac{\partial ^2 u}{\partial x^2}    \\
\text{Require} \quad  \frac{TD}{L^2}  = 1 \implies  T = \frac{L^2}{D}  
  \end{split} 
\] 
This can be generelized to \[
\frac{\partial u}{\partial t}  = \frac{\partial ^2 u}{\partial x^2}  , \quad  0\le x \le 1 
\] 



\subsection{Sinking Ball}%
\label{sub:sinking_ball}

\begin{figure}[ht]
    \centering
    \incfig{sinkingball}
    \caption{sinkingball}
    \label{fig:sinkingball}
\end{figure}

Let 
\begin{itemize}
  \item $\rho _{b}$ e mass density of ball
  \item $ \rho _{f}$mass density of fluid
  \item $V$ Volume of ball
\end{itemize}

Then is the equation \[
  \begin{split}
\rho _{b} V g - \rho _{f} V g &=  V g \rho _{b} \left( 1- \frac{\rho _{f}}{ \rho _{b}}  \right) \\
&=  m \hat{g} \implies  \hat{g} = g \left( 1 - \frac{\rho _{f}}{ \rho _{b}}  \right)  \\
  \end{split} 
\] 

  And we then end up with the newtions law \[
  m \frac{d x^{*2}}{d t^{*2}}  = m \hat{g} - k \frac{d x^{*}}{d t}  , \quad  \text{Friction coefficient}\quad k  
  \] 
  where \[
 x^{*}\left( 0 \right) = 0 , \quad  \frac{d x^{*}}{d t^{*}}  \left( 0 \right) = V  
  \] 

  The cases can be described as follows

\begin{figure}[ht]
    \centering
    \incfig{highv}
    \caption{highV}
    \label{fig:highv}
\end{figure}

\begin{figure}[ht]
    \centering
    \incfig{frefall}
    \caption{frefall}
    \label{fig:frefall}
\end{figure}

\begin{figure}[ht]
    \centering
    \incfig{sinking}
    \caption{sinking}
    \label{fig:sinking}
\end{figure}

  \begin{enumerate}
    \item High friction, not so high $V$. Ball will sink at constant speed most of the time.
    \item Friction is low, andC not "too high". ( "Free fall with V=0")
\item High V, and high friction  
$m \frac{d ^2 x^{*}}{d t^{*2} } = m \hat{g} - k  \frac{d x^{*}}{d t^*} $
  \end{enumerate}

  For this problem there is $3$ characteristic speeds 
  \begin{enumerate}
    \item $V$ : initial velocity
    \item $v_{0} $ : equilibrium  speed in case A $v_{0} = \frac{m \hat{g}}{k} $
    \item $v_{f} $ :  free fall $v_{f} = \sqrt{2 \hat{ g} L} $
  \end{enumerate}
   Let us put \[
   \begin{split}
     \frac{d ^2 x ^{*}}{d t^{*2}} = 0  &  \implies  k \frac{d x^{*}}{d t}  = \hat{g}m \\
      & \implies  \frac{d x^{*}}{d t^{*}}  = \hat{g} \frac{m}{k}  = v_{0} 
   \end{split} 
   \] 
   and put \[
   \begin{split}
     x^{*} \left( 0 \right) = &  \frac{d x^{*}}{d t^{*}}  \left( 0 \right) = 0 \\
     k = 0  &  \quad   
   \end{split} 
   \] 
   \subsubsection{Scaling}%
   \label{ssub:scaling}
   
   \begin{enumerate}
     \item Case A:  The ball sinks at constant speed "most" of the time.
       \begin{enumerate}
         \item Length scale $L$ : $x^{*} = L x$. Since the ball falls with speed most of the time, a timescale would be $T = \frac{L}{v_{0}} $. $v$ is not much larger than $v_{0}$ $\implies $ it is not so that $v \gg  v_{0}$
            \[
              \begin{split}
            m \frac{L}{T^2}  x^{''} &=   m \hat{g} - k \frac{L}{T}  x^{'} \quad  \text{Divide by } L \\
           \implies m \frac{1}{kT}  x^{''} &=   \frac{Tm \hat{g}}{KL}  - x^{'}\\
           \frac{m}{k \frac{L}{v_{0}} }  x^{''} &=   \frac{ \frac{k}{v_{0}} m \hat{g}}{kL}  - x^{'} \\
           \implies  \frac{m v_{0}}{Lk}  x^{''} &=   \frac{L m  \hat{g}}{ K L v_{0}}  - x^{ '} \\
              \end{split} 
            \] 
            We can then derive \[
            \begin{split}
              \frac{m \frac{m \hat{g}}{k} }{Lk}  x^{''} &=   1 - x^{'} \\
              \implies  \frac{m^2 \hat{g}}{Lk^2}  x^{''} &=   1- x^{'} \\
              \implies  \frac{m^2 \hat{g}^2}{\hat{g}L k^2 }  x^{''} &=    1 - x^{'}  \\
              \epsilon x ^{''} &=    1- x^{'}  \quad  \text{Where} \quad \epsilon = 2\left( \frac{v_{0}}{v_{f}}  \right)^2  
            \end{split} 
            \] 
            The condition are $x\left( 0 \right) = 0$, $\frac{L}{T}  x^{'} \left( 0 \right) = V$ which can be rewritten to \[
              x^{'} \left( 0 \right) = \frac{T V}{L}  ) \frac{ \frac{L}{ v_{0} V} }{L}  = \frac{V}{v_{0}}  = \mu 
            \] 
       \end{enumerate}
   \end{enumerate}

   \subsection{Let Analyze The equation}%
   \label{sub:let_analyze_the_equation}
   In case A is the  \[
   \epsilon \ddot{x} = 1 - \dot{x} 
   \] 
   An approximation  we can do is to put $\epsilon  = 0$ such that \[
   0 = 1- \dot{x} \quad x\left( 0 \right) = 0,\quad   \dot{x} \left( 0 \right) = \mu  \quad \dot{x} = 0 
   \] 
   unless $\mu = 1$, we cant find a solution.
   
   \subsubsection{Case B}%
   \label{ssub:case_b}
   Small friction,  $V$ is not too high. Let the lengthscale be $L$. \[
     \begin{split}
   \frac{d ^2}{d t^{*2}} x^{*2} &=   \hat{g} , \quad  x^{*} \left( 0 \right) = \frac{d x^{*}}{d t^{*}}  \left( 0 \right) = 0 \\
   x^{*} \left( t^{*} \right) &=  \frac{1}{2} \hat{g} \left( t^{*} \right)^2 
     \end{split} 
   \] 
   Hit the bottom with speed $V_{f}$ . We can choose time scale $T$ such that \[
   \begin{split}
     T &=  \frac{L}{v_{f}}   
   \end{split} 
   \] 
   So gain \[
   \begin{split}
     \frac{mL}{T^2}  \ddot{x} = m \hat{g} - \frac{kL}{T}  \dot{x}
   \end{split} 
   \] 
   What you can observe is that gravity dominates so we modify the equation to be \[
     \begin{split}
   \frac{L}{\hat{g} T^2} \ddot{x} &=   1 - \frac{kL}{gmT} \dot{x} \\
   \implies  2 \ddot{x} &=   1 - \left( \frac{v_{F}}{ v_{0}}  \right) , \quad  \frac{K}{T}  \dot{x}\left( 0 \right) = 0 \\
   2\ddot{x} &=   1- \epsilon \dot{x} \quad \dot{x}\left( 0 \right) = \frac{V}{v_{f}}  = \mu  
     \end{split} 
   \] 
   
\subsubsection{Case C:  High V and high friction}%
\label{ssub:j}

Let us consider \[
  m \frac{d^2 x^{*}}{d t^{*2}}  = -kV \quad \frac{d x^{*}}{d t^{*}}  = V - \frac{kV}{m} t^{*} = 0
\] 
Where we choose the scales $t^{*} = \frac{m}{k}  = T$, $L = \frac{V m}{k} $, where  $TV = L$. \[
\implies \ddot{x} = \epsilon  - \dot{x}, \quad x\left( 0 \right) = 1, \quad \dot{x}   = 1, \quad    \epsilon  = \frac{v_{0}}{V} 
\] 
 \begin{tcolorbox}
   \textbf{Example.}  Let \[
   \begin{split}
     a \frac{d ^2 x^{*}}{d t^{*2}} + b \frac{d x^{*}}{d t^{*} }  + c x^{*} &=  0\\
     x^{*} \left( 0 \right) = x_{0} , \quad  \frac{d x^{*}}{d t^{*}} \left( 0 \right) = 0 
   \end{split} 
   \] 
   Three waus to scale by balancing the equation. Last term "small" \[
   x^{*} = x_{0} x, \quad  t^{*} = Tt 
   \] 
   Where $T$ is to be determined. \[
   a \frac{x_{0}}{T^2}  \ddot{x} + b \frac{x_{0}}{T}  \dot{x} + c x_{0}  = 0
   \] 
   \[
   \ddot{x} + \frac{bT}{a}  \dot{x} + \frac{cT^2}{a}  = 0
   \] 
   If we are smart can we choose the timescale $T = \frac{a}{b}$  then we get \[
     \begin{split}
   \ddot{x} + \dot{x} + \frac{c a^2}{b^2 a}  &=   0.  \\
   \implies  \ddot{x} + \dot{x} + \left( \frac{ca}{b^2}  \right)x &=   0
     \end{split} 
   \] 
 \end{tcolorbox}

\subsection{Turbulence}%
\label{sub:turbulence}

Reynold number \[
  R_{e} = \frac{u \rho L}{\mu }  = \frac{uL}{\frac{mu}{\rho } }  =  \frac{uL}{\mathcal{V}   }  
\] 
Then we have \[
\frac{\partial v}{\partial t}  = \mathcal{V}  \frac{\partial ^2 v}{\partial x^2} 
\] 


\newpage
\section{Lecture 31/08}%
\label{sec:lecture_31_08}

\subsection{Turbulence }%
\label{sub:turbulence_}
\textbf{Kolmogorvs Microscales} .

\[
\rho \frac{d u}{d t}  = \mu \frac{\partial ^2 u}{\partial x^2} 
\] 
Time svale for convitive flow over a distance $L$ \[
t_{c} = \frac{L}{U}  , \quad  U \quad  \text{is velocity.}  
\] 

This can be rearranged such that \[
\frac{\partial u}{\partial t}  = \left( \frac{\mu }{\rho } \frac{\partial ^2 u}{\partial x ^2}  \right). 
\] 
We also define $\mathcal{V}  = \frac{\mu }{\rho }  $ where $\left[ \mathcal{V}  \right] = m^2 s^{-1}$, which is the time for dispersion of velocity. 

\newpara
Let $t_{d} = \frac{L^2}{\mathcal{V} }$ such that the Reynolds number can be written \[
  R_{e} = \frac{v \rho  L}{\mu }  = \frac{UL}{\left( \frac{\mu }{\rho }  \right) }  = \frac{UL}{\mathcal{V} }   =   \frac{ t_{d}}{ t _{0}} 
\] 


\newpara
For water is $\mathcal{V}  = 10^{-6} m^2 s^{-1}$. So for a river , put $L = 100m$ with $U = 1 m s^{-1}$ \[
R_{e} = \frac{1 m s^{-1} \cdot  100m}{ 10 ^{-6} m ^2 s^{-1 }}  = 10 ^{8}
\] 
Assume the generation of new whrils stops when $ t_{d} \approx t_{c} \to  R_{e} \approx 1$ . Let \[
\begin{split}
  E  &=  \frac{\text{Energy}}{ \text{time per unit mass}}  \\
  \left[ E \right] &=  kg m^2 s^{-2} s^{-1} kg
\end{split} 
\] 
Let $l$ be bthe scale of the smallest whirls and $u$ the unit velocity then is \[
E = E\left( l,u , \mathcal{V}  \right). 
\] 
We assume that $E$ is proportional to $u^2$. \[
f\left( \frac{E}{u^2} , l , \mathcal{V}  \right) = 0
\] 

\begin{table}[htpb]
  \centering
  \caption{}
  \label{tab:jsjksja}
  \begin{tabular}{l|ccc}
    & $m$   & $s$ &  \\
   $\frac{E}{n^2}$ &  1 & 0  &  \\
   $l$ & 1 & 0 \\
   $v$ & 2 & -2
  \end{tabular}
\end{table}
   
\[
  \begin{split}
    \left[ \frac{\frac{E}{u^2}}{\mathcal{V} }  \right] &= m^{-2} \\
    \Pi  &= \frac{\frac{E}{u^2}}{\mathcal{V} } l^2 \\
    \text{choose } \Pi &= 1 \\
    \to E &=  \mathcal{V}  (\frac{u^2}{l})^2  \\
    ul &= \mathcal{V}  \\
    \implies  k &= \left( \mathcal{V}^{3} \frac{1}{E}   \right)^{\frac{1}{4} }, \quad  u = \left( V E \right)^{\frac{1}{4}}  \\
  \end{split} 
\] 

\textbf{Example} . Let us have $1 kg$ what in a mixmaster and apply $100W$ power. then is \[
\begin{split}
  l = \left( \frac{\left( 10^{-6} m^2 s ^{-1} \right)^{3}}{100 m^2 s^{-3}}  \right) ^{\frac{1}{4}} =  0.01mm
\end{split} 
\] 

\newpage
\subsection{Regular Perturbation Theory }%
\label{sec:regular_perturbation_theory_}

Assume we have an equation s.t. \[
D\left( x, \varepsilon  \right) = 0 \quad \text{where} \quad \varepsilon \ll 1  
\] 
meaning that $\varepsilon $ is small. 

\newpara
We have a solution $x\left( \varepsilon  \right)   $ to the problem $D\left( x, \varepsilon  \right)$. The perturbation problem is regular if $\lim_{\varepsilon \to 0} x\left( \varepsilon  \right)$ is a solution to $D\left( x, 0 \right) = 0$ . The idea is 
\begin{enumerate}
  \item Put $x\left( \varepsilon  \right) = x_{0} + \varepsilon  x_{1} + \varepsilon ^2 x_{2} + \ldots$ \[
  \begin{split}
    x\left( \varepsilon  \right)  & \approx x_{0} \quad \text{ in 0. order}   \\
    x\left( \varepsilon  \right)  & \approx x_{0}  +  \varepsilon x_{1 } \quad \text{to 1. order}  \\
  \end{split} 
  \] 
\item Insert $ x\left( \varepsilon  \right) = x_{0} + \varepsilon x_{1} + \ldots$ into $D\left(x, \varepsilon  \right)$ . 
\item  Collect all terms of order 0, all terms of order 1 so that \[
D\left( x, \varepsilon  \right) = 0 \leftrightarrow  \overbrace{\left( \right)}^{= 0}  + \overbrace{\left(  \right) \varepsilon ^2} ^{ = 0} + \ldots = 0 
\] 
\end{enumerate}

\textbf{Example}. Let \[
x^3 + x^2 + \varepsilon x - 2 = 0 , \quad  \varepsilon  \ll  1 
\] 
For $\varepsilon  = 0$ we have $x =1 $ as a solution. To find a solution "close to" $1$ when $\varepsilon  \neq 0$ we put \[
x = 1 + \varepsilon  x_{1} + \varepsilon ^2 x_{2} + O\left( \varepsilon  \right)
\] 
Want an approximation to 2. order. We get \[
  \begin{split}
\left( 1 + \varepsilon  x_{1} + \varepsilon ^2 x_{2}  \right)^{3} + \left( 1 + \varepsilon x_{1} + \varepsilon ^2 x_{2} \right)^{2} + \varepsilon \left( 1 + \varepsilon  x_{1} + \varepsilon ^2 x_{2} \right)  -2  & = 0 \\
\implies  \varepsilon \left( 5x_{1} +1 \right) + \varepsilon ^2 \left( \ldots \right) &=  0 \\
x\left( \varepsilon  \right)   &  \approx 1  - \frac{\varepsilon }{5} + \frac{\varepsilon ^2}{ 125}  
  \end{split} 
\] 

\subsection{The Projectile Problem}%
\label{sub:the_projectile_problem}

Let $v_{0}$ be the vertical velocity and $v_{e}$ be escape velocity such that $v_{0} \ll  v_{e}$ . 

\newpara
Newton gravitational law \[
\mathbf{F} = -m \frac{R^2 g}{ \left( R+ x^{*} \right)^2} 
\] 
Where $g$ is the gravitational constand at $x^{*} = 0$. 

\newpara
Energy to move to $x^{*} = \infty$ \[
  \begin{split}
- \int_{0}^{\infty}  \mathbf{F} dx^{*} &= mg R^2 \int_{ 0}^{ \infty}  \frac{dx^{*}}{\left( R + x^{* }  \right)^2}   \\ 
 &=  mg R^2 \left[ - \frac{1}{\left( R + x^2 \right)}  \right]_0^{\infty} \\
 &=  mg R  = \frac{1}{2} mv_{e}^2 \\
 \implies   &  v_{e} = \sqrt{2gR} 
  \end{split} 
\] 
We have \[
m \frac{d ^{2} x^{*}}{d t^{*2} } = -m \frac{g R^2}{ \left( R + x^{* } \right) ^2}  
\] 
Such that \[
  \frac{d ^2}{d t^{*2}}  =  - \frac{R^2 g }{\left( R \ x^{* }  \right)^2} , \quad  x^{*} \left( 0 \right) = 0, \quad  \frac{d x^{*}}{d t^{*}}  \left( 0 \right)  = v_{0}  
\] 

and $v_{0}\ll  v_{e}$ , when $x^{* } \ll  R$  (a consequence of $v _{0} \ll  v_{e}$ ) \[
\frac{d ^2 x^{* }}{d t^{*2}}  \approx -g \quad  \frac{d x^{* }}{d t^{*} }  = v_{0} - t^{* } g  = 0      \quad  \leftrightarrow  t^{* } = \frac{v_{0}}{ g}  = T = \text{timescale}  
\] 
\[
X^{* } = v_{0} t^{* } - \frac{1}{ 2}  t^{* } g \quad x^{* } \left( T \right) = \frac{v_{0}^2}{g}  - \frac{1}{2}  \frac{v_{0} ^2 }{g}  = \frac{1}{2} \frac{v_{0}^2}{g}   
\] 
Let $L = \frac{v_{0}^2}{g} $ and scale the equation $\left( \frac{L}{T}  \right) = v_{0}$ and $x^{*} = Lx$ . \[
  \begin{split}
  \frac{L}{T^2}  \ddot{x} = \frac{-g R ^2}{\left( R + Lx \right) ^2} \leftrightarrow  \frac{L}{T^2}   \ddot{x}  & =  -\frac{g R^2}{ R^2 \left( 1  + \frac{L}{R} x \right)^2}  \\
  \to  \ddot{x} = \frac{-T^2 \frac{g}{L} }{\left( 1 + \frac{L}{R} x^2 \right)} \to  \ddot{x} &=  \frac{-1}{ \left( 1+ \varepsilon  x \right)^2}  \\
  \end{split} 
\] 
Where \[
\varepsilon  = \frac{L}{R} = \frac{v_{0} ^2}{Rg} = 2 \frac{2 v_{0}^2}{v_{e}^2}   
\]  

\newpara
Following problem \[
\ddot{x} = \frac{-1}{\left( 1 + \varepsilon x \right)^2 }  , \quad  x\left( 0 \right) = 0 , \quad  \dot{x} \left( 0 \right) = 1  
\] 
\begin{tcolorbox}
  Recall that \[
    \begin{split}
  f\left( u \right) = \frac{1}{\left( 1 + u \right) ^2}   &  \to  \int_{}^{}  f\left( u \right) = \frac{1}{1+u} + C \\
  &= C - \left(  1- u + u^2 - u ^3 + \ldots \right) \\
  \implies   &  f\left( u \right) = 1- 2 u 0 3u ^2 + O\left( u^3 \right) \\
    \end{split} 
  \] 
\end{tcolorbox}

Then to second order \[
  \ddot{x} = - \left( 1 - 2 \varepsilon  x + 3 \varepsilon  x ^2 \right) , \quad  x\left( 0 \right) = 0 , \quad  \dot{x}\left( 0 \right) =q  
\] 
Next et \[
x\left( t \right) = x_{0} \left( t \right) + \varepsilon x_{1} \left( t \right) + \varepsilon x_{2} \left( t \right) + O\left( \varepsilon  \right)
\] 
So let \[
  \begin{split}
x_{j} \left( 0 \right)    & = 0 \quad  \text{for} \quad  j = 0,1,2 \\
\ddot{x_{0}} \left( 0 \right)  = 1 ,  &  \quad  \dot{x_{1} }\left( 0 \right) = \dot{x_{2}}\left( 0 \right) = 0  \\
\to  \ddot{x_{0}} + \varepsilon \ddot{x_{1}} + \varepsilon ^2 \ddot{x_{2}}   & = -1 + 2\varepsilon  \left( x_{0} 0 \varepsilon  x_{1} \right)  - 3 \varepsilon ^2 x_{0} ^2 \\
\to  \left( \ddot{x_{0}} + 1 \right) + \varepsilon \left( \ddot{x_{1}} - 2 x_{0} \right)  + \varepsilon ^2 \left( \ddot{x_{2}} + 2 x_{1}  + 3 x_{0} ^{2} \right) &=  0 \\
\ddot{x_{0}} = -1 \quad  x_{0} \left( 0 \right) = 0 , \quad  \dot{x_{0}} = 1  &  \\
\ddot{x_{1}} = 2x_{0} , \quad  \dot{x1 } \left( 0 \right) = \dot{x_{i}}\left( 0 \right) = 0  &  \\
\ddot{x_{2}} =  2 x_{1} - 3 x_{0}^2   & , \quad  x_{2} \left( 0 \right) = \dot{x_{2}} \left( 0 \right) = 0 \\  
  \end{split} 
\] 
\[
  \begin{split}
\to  x_{0} \left( t \right) &=  t- \frac{1}{2}tst \\
\ddot{x_{1}}\left( t \right) &=  2t - t^2 \\
\dot{x_{1}}\left( t \right) &=  t ^2 - \frac{1}{3} t^3 \\
x_{1} \left( t \right) = \frac{1}{3} t^3 - \frac{1}{12} t ^{4}
  \end{split} 
\] 
Where \[
  \begin{split}
\ddot{x_{2}} &=   \frac{2}{3} t^3 - \frac{1}{6} t ^{4} - 3 \left(t ^2 - t ^3 +  \frac{1}{4} t ^{4}  \right) \\
x_{2} &=  -\frac{1}{4} t ^{4} + \frac{11}{60} t^{5} - \frac{11}{360 } t ^{6} \\
  \end{split} 
\] 
Which end up with \[
x\left( t \right) = t -\frac{1}{2} t^{2} 0 \varepsilon  \left( \frac{1}{3} t^3 - \frac{1}{12} \right) + \varepsilon ^2 \left( - \frac{t ^{4}}{4}  0 \frac{11}{60} t ^{5} - \frac{11}{360 } t ^{ 6} \right)
\] 
Gives the diea of how to approx the time to the macimum height. $\dot{x}\left( t \right) = 0$ is a 5. degree equation containing $\varepsilon $ . 

\newpara
Lets put \[
t = 1+ \varepsilon  t_{2}  \varepsilon ^2 t_{2}
\] 
Into the 5. degree edition and to regular perturabation \[
\to  t =  1 + \frac{2}{3} \varepsilon  + 2 /5 \varepsilon ^2 + O\left( \varepsilon  \right)
\]  
such that \[
  \begin{split}
\ddot{x}  &=  \frac{-1}{ \left(  1+ \varepsilon  x \right)^2 }  \to  \ddot{x} \dot{x} = \frac{\dot{x}}{\left( 1+ \varepsilon  x \right)^2}  \\
\to  \frac{d }{d t}  \left( \frac{1}{2} \dot{x}^2 \right) &=  \frac{d }{d t} \left( \frac{-1}{\varepsilon }  \frac{1}{ 1+ \varepsilon  x}  \right)  \\
\frac{1}{2}  \dot{x}^2 &=  \frac{-1}{\varepsilon }  \frac{1}{ 1+ \varepsilon x}  + C  \\
\frac{1}{2} &=  \frac{-1}{\varepsilon }   \\
  \end{split} 
\] 
\[
C = \frac{1}{2} + \frac{1}{\varepsilon }
\] 
where \[
\frac{1}{2} \dot{x}^2 = \frac{-1}{\varepsilon }  \frac{1}{1+ \varepsilon  x}  + \frac{1}{2} + \frac{1}{\varepsilon }
\]  
At maximum height $\dot{x} = 0$
\[
 0 = -\frac{1}{\varepsilon } .
\] 


\newpage
\section{Lecture 02/09}%
\label{sec:lecture_02_09}

Let Newtons Law be \[
  \begin{split}
   \frac{d ^2 s ^{*}}{d t^{*2}}   & = g \sin \left( \alpha ^{*} \right) \implies   \frac{d ^2 \alpha ^{*}}{d t^{*2}}  = -\frac{g}{L} \sin\left( \alpha ^{*} \right) \\
  \end{split} 
\] 
scaling: 
\[
  \begin{split}
   \alpha ^{*}  & = \varepsilon \alpha , \quad  t^{* }   = Tt   \\
   \frac{\varepsilon }{T^2} \ddot{\alpha }  & = \frac{-g}{L}  \sin \left( \varepsilon \alpha  \right) \implies  \ddot{\alpha } = - \left( T^2 g \frac{1}{ L}  \right) \frac{\sin \left( \varepsilon \alpha  \right)}{ \varepsilon } \\
   T &= \sqrt{\frac{L}{g} }  \implies  \ddot{\alpha }   = - \frac{\sin \left( \varepsilon \alpha  \right)}{\varepsilon }  \\ 
   \alpha \left( 0 \right) = 1  &  \quad  \dot{\alpha } \left( 0 \right) = 0 
  \end{split} 
\] 

Let put $\alpha  = \alpha _{0} \left( t \right) + \varepsilon ^2 \alpha _{2} \left( t \right) + O\left( \varepsilon ^{4} \right)$. where $\alpha \left( t \right)$ is an even function of $\varepsilon $ due to symmetry. \[
  \begin{split}
\alpha _{0} \left( 0 \right) &=  1, \quad  \dot{\alpha }_{0} \left( 0 \right) = 0, \quad  \alpha _{2} \left( 0 \right) = \dot{\alpha _{2} } \left( 0 \right) = 0  
  \end{split} 
\] 
Inserted into the equation \[
  \begin{split}
    \ddot{\alpha _{0}} + \varepsilon ^2 \ddot{\alpha _{2}}  & = - \frac{\sin \left( \varepsilon \left( \alpha _{0} + \varepsilon ^2 \alpha _{2} \right) \right)}{\varepsilon }  \implies \quad \ddot{\alpha _{0}} + \varepsilon ^2 \ddot{\alpha _{2}} \\
     & = \frac{-1}{3}  \left( \varepsilon \underbrace{\left( \alpha _{0} + \varepsilon ^2 \alpha _{2} \right)} _{u} \frac{\varepsilon ^2}{6}  \left( \alpha _{0}  + \alpha \varepsilon ^2 \right) \right)
  \end{split} 
\] 


Let \[
  \begin{split}
\alpha _{0} \left( t \right)  & = A\cos t + B \sin t \\
\alpha _{0} \left( 0 \right)  &  = 1 , \quad  \dot{\alpha } \left( 0 \right) = 0 \quad  \implies  \alpha _{0} \left( t \right) = \cos t   \\
\alpha _{2} \left( t \right) &=  A \cos t + B \sin  t + \alpha _{2, f} \left( t \right)  \\
\cos ^3 t  & = \left( \frac{1}{2} \left( e^{it} - e^{it} \right) \right)^3 = \frac{1}{8} \left( e^{i3t} + 3e^{it} 0 3 e^{-i3t} \right) \\
&= \frac{1}{4} \left( \cos 3t + 3\cos t \right) \\
  \alpha _{20} \left( t \right) &=  A \cos3t + B \sin 3t + Ct \cos t + Dt \sin t \\
  \alpha _{2}\left( t \right) &=  \frac{1}{192} \left( \cos t + \cos 3t \right) +\frac{1}{16} t \sin t \\
  \alpha \left( t \right)  & = \alpha _{0} \left( t \right) + \varepsilon ^2 _{2} \left( t \right) \quad  \text{ is not periodic}  
  \end{split} 
\] 

\begin{tcolorbox}
  \textbf{Poincare-Lin Stel Method} .
Instead let \[
\alpha \left( t \right) = \alpha _{0} \left( \omega \left( \varepsilon  \right)t \right) + \alpha _{2} \left( \omega \left( \varepsilon  \right)t \right)\varepsilon ^2 + O\left( \varepsilon ^{4} \right)
\] 
Where $\omega \left( \varepsilon  \right)  = 1 + \omega _{2} \varepsilon ^2 \ O\left( \varepsilon ^{4} \right)$ . See exercise.
\end{tcolorbox}

\subsection{Modelling how the kidney disposes salt and water.}%
\label{sub:modelling_how_the_kidney_disposes_salt_and_water_}



\begin{figure}[ht]
    \centering
    \incfig{watermodell}
    \caption{watermodell}
    \label{fig:watermodell}
\end{figure}


\textbf{Assumptions}  
\begin{enumerate}
  \item Secondary channel is fed water by osmosis from the sorrouinding tissue.
  \item Ions are transported down the channel by connection and diffusion.
  \item Ions are fed into the channel be a chemical ppump-
\end{enumerate}
We want the steady-state profiles of ion concenstration $C^{*} \left( x^{*} \right) $ and the velocity $v^{*} \left( x^{*} \right) $ of the ion water solution.


\newpara
The ion concentration is written as \[
\left[ C^{*} \right] = \frac{ions}{m^3} = \frac{osmol}{m^3} 
\] 

One mole salt give two moles ions

\begin{figure}[ht]
    \centering
    \incfig{molefig}
    \caption{molefig}
    \label{fig:molefig}
\end{figure}


\textbf{Osomosis} : \[
J^{* } = P\left( c^{* } - c_{0} \right)
\] 
Is flux density of water entering the secondary channel. $J^{*}$ is volume water in per area per time. $c_{0}$ ion concentration is tissue and main channel. $P$ is called membrance permeability. \[
\left[ P \right] = \frac{\left[ J^{*} \right]}{ \left[ c^{*} \right]}  = \frac{m s^{-1}}{ osmol \cdot m ^{-3}}  = \frac{m^{4}}{ s\cdot  osmol} 
\] 
Ion flux density \[
N^{*} = \begin{cases}
  N_{0},  &  \quad  0 \le x^{*} \le \delta  \\
  0,  &  \delta \le x^{* } \le L 
\end{cases}
\] 
Where $\left[ N_{0} \right] = \frac{osmol}{m^2 \cdot  s} $. The toal rate of salt entering the channel \[
N_{0} \cdot c\cdot \delta 
\] 
Where $c$ is the area of pump. 

\newpara
\begin{itemize}
  \item
 The flux density of ions in the secondary channel \[
 F^{*} = F_{c}^{* } + F_{\alpha } ^{*}
 \] 
 \[
 \left[ F^{*} \right] = \frac{osmol}{m^2 \cdot s} 
 \] 
 \item
Convective flow \[
F^{*} _{c} = c^{*} v^{*}
\] 
\item 
Diffusion: \textbf{Ficus law}  \[
F_{1}^{*} = - D \frac{d c^{*}}{d x^{*}}  .
\] 
where $D$ is the diffusion of salt in water.
\end{itemize}

Conservation of water 
\begin{figure}[ht]
    \centering
    \incfig{conssswater}
    \caption{conssswater}
    \label{fig:conssswater}
\end{figure}

\[
Q^{*} _{\text{out}} = Q^{* } _{\text{in}} + Q^{*}_{\text{os}}
\] 
\[
\begin{split}
  v^{*} \left( x^{*} + \Delta x^{*} \right) &= v^{*}A + P\left( c^{*}\left( \hat{x} \right) - c_{0} \right) c \Delta x^{*}, \quad \\
   \text{where }  &   \hat{x^{*}} \in \left<x^{*}, x^{*} + \Delta x^{*}  \right>   \\
  \implies  \frac{v^{*} \left( x^{*} + \Delta x^{*}   \right) - v^{*} \left( x^{*} \right)}{\Delta x^{*}}  &= \frac{c}{A}  P\left( c^{*}\left( \hat{x^{*}}   \right) - c_{0} \right) \\
  \Delta  x^{*} \to  0  \quad   &  \implies  \frac{d v^{*}}{d x^{*}}  = \left( \frac{cP}{A}   \right) \left( c^{*} - c_{0} \right)
\end{split} 
\] 
COnservation of salt \[
F^{*} \left( x^{*} + \Delta x^{*} \right) A = F^{*}\left( x^{*} \right)A + N^{*}\left( \hat{x^{*}} \right) c \Delta x^{*}
\] 
This ends up with \[
\begin{split}
  \implies  \frac{d F^{*}}{d x^{*}}  &= \frac{c}{A} N^{*}\left( x^{*} \right) \\
  \text{or} \quad \frac{d F^{*}}{d x^{*}}   &= \frac{c}{A} \cdot \begin{cases}
    N_{0},  &  \quad  0 < x^{*} < \delta \\
    0,   & \quad \delta < x^{*} < L   
  \end{cases} \\
  F^{*}\left( 0 \right) = 0 \quad  &  \implies  F\left( x^{*} \right) = \begin{cases}
    \frac{N_{0} c}{A}  x^{*} , \quad  &  0 <  x ^{* } < \delta \\
    \frac{N_{0} \delta  c}{A} , \quad   &  \delta < x^{* } < L    
  \end{cases}  \\
  \implies  v^{*} c^{*} - D \frac{d c^{*}}{d x^{*}}  &=  F^{*}\left( x^{*} \right) \\
  \frac{d v^{*}}{d t^{*}}  &= \frac{cP}{A} \left( c^{*} - c_{0} \right) \\
  v^{*}\left( 0 \right) &=  0 \\
  c^{*}\left( L \right)  & = c_{0} \\
\end{split} 
\] 
Also same that $v^{*}$ and $c^{*}$ are continious at $x^{*} = \delta $ .

\subsubsection{Scaling the model}%
\label{ssub:scaling_the_model}
Two length scales $\delta $ and $L$. Choose $\delta $ as length svale. Natural to use $c_{0}$ as scale for $c^{*}$. The rate salt supplied is \[
N_{0} \delta  c = c_{0} UA
\] 
Ions supplied is convectiv flux with $c^{*}$ such that $U = \frac{N_{0} \delta c}{c_{0} A} $. 
\[
  \begin{split}
  x^{*}  & = \delta ,\\
    c^{*}   & = c_{0} c \\
v^{*} &=  U v \\
  \end{split} 
\] 

\begin{enumerate}
  \item $\left( Uc_{0} \right) cv - \frac{D c_{0}}{ \delta }  \dot{c} = F^{*}$ such that \[
  \implies  vc - \frac{Dc}{ \delta U c_{0}} \dot{c}  =  \frac{1}{Uc}  \cdot \begin{cases}
    \frac{N_{0} c \delta  x}{ AUc_{0} }  ,  &  \quad  0 < x \delta  < \delta \\
    \frac{N_{0} c \delta  }{Auc_{0}}  ,  &  \delta  < x\delta  < L 
  \end{cases}
  \] 
  \[
  vc - \varepsilon  \dot{c} = \begin{cases}
    x \quad   &  0 < x < 1 \\
    1 \quad   &  1 < x < \lambda   
  \end{cases}
  \] 
  where $\varepsilon  = \frac{D}{ \delta u} $ , and $\lambda  = \frac{L }{ \delta } $ \[
  \implies  U = \frac{N_{0} \delta  c}{ c_{0} A} 
  \] 
\item $\frac{U}{\delta } \dot{v}  = \frac{cP }{A}  c_{0}\left( c-1 \right)$ \[
\] 


\end{enumerate}



\newpage
\section{Lecture 07/09}%
\label{sec:lecture_07_09}

\subsection{Emergent Osmotic Concentration}%
\label{sub:emergent_osmotic_concentration}

\begin{enumerate}[label=(\roman*)]
  \item Total rate of salgt pumped per second $\delta  c  N_{0}$
  \item Water out per second  $v^{*} \left( L \right) A = U v\left( \lambda  \right) A $ , where $ \lambda  = \frac{L}{\delta } $
\end{enumerate}

\[
  \begin{split}
  \delta c N_{0}  & = C_{0} U \\
    &  \approx \text{Flow out of salt per sec} \\
    \implies  U &=  \frac{\delta c N_{0}}{ C_{0 } A}  \\
  \end{split} 
\] 
Measure of the efficiency \[
  \begin{split}
\frac{\text{Salt out}}{ \text{Water out}}   & = O s ^{*}\\
&= \frac{\delta  c N_{0}}{ U v\left( \lambda  \right) A}  = \frac{C_{0}}{ v\left( \lambda  \right)}  \\
  \end{split} 
\] 
Thus $v\left( \lambda  \right) > \frac{1}{4}$


\subsection{Boundary Value Problem}%
\label{sub:boundary_value_problem}

We know that \[
  \begin{split}
\sum_{}^{} v' \left( x \right)  & = C\left( x \right) - 1 \\
v\left( x \right)C\left( x \right) - \mu  C' \left( x \right)  & = f\left( x \right) = \begin{cases}
  x,  &  \quad  0 \le x \le \\
  1,  &  \quad  1 \le \le \lambda   
\end{cases}
    \end{split} 
\] 
Where  $v\left( 0 \right) = 0, \quad    C\left( \lambda  \right)  =1 $.
In addition $v$ and $C$ must be continuous. 

\newpara
Let assume $ 0 < \varepsilon  \ll  1$. Put $C = c_{0} + \varepsilon  C_{1} + O \left( \varepsilon ^2 \right)$ and $v = v_{0} + \varepsilon  v_{1} + Ø\left( \varepsilon ^2 \right)$ . Inserted into the equation \[
  \begin{split}
\varepsilon \left( v_{0}'   \right) &=  C_{0} + \varepsilon C_{1} - 1 + O\left( \varepsilon ^2 \right)  \\
\left( v_{0} + \varepsilon  v_{1} \right) \left( 1 + C_{1} \varepsilon  \right) ^2 - \mu \left(  \varepsilon  C_{1}'    \right)  &  = f\left( x \right) + O\left( \varepsilon ^2 \right)   \\
C_{0} - 1 &=  0 \leftrightarrow  \quad  C_{0} = 1  \\
C_{1} - v_{0}'  =  0  \implies  C_{1} = v_{0}'    & \implies \quad C_{1} = f\left( x \right) , \quad C_{1}\text{ is discontinuity }    \\
v_{0} - f\left( x \right) &=  0, \quad v_{0} = f\left( x \right)   \\
v_{1} + v_{0} C_{1} - \mu \varepsilon C_{1}'  &=  0 \\
  \end{split} 
\] 

Something is wrong. 
\[
\begin{split}
  \varepsilon  v'  &=  C- 1 \\
  \varepsilon  v C - \underbrace{\left( \varepsilon  \mu  \right)}_{ \text{not small} }  &=  \varepsilon f\left( x \right) \\
\end{split} 
\] 
For notation convenience let \[
\begin{split}
  \left( \varepsilon  \mu  \right) &= \omega ^{-1}   \\
  \varepsilon v'  &=  C - 1 \\
  \varepsilon vC - \frac{1}{\omega ^{2}} C'  &=  \varepsilon  f\left( x \right) \\
  \implies   &  \varepsilon \left( \omega ^2 v C \right) - C'  = \varepsilon  \omega ^2 f\left( x \right)
\end{split} .
\] 

We then get \[
\begin{split}
  v &=  v_{0} + \varepsilon  v_{1}  \\
  C &=  C_{0} + \varepsilon C_{1} \\
  \varepsilon  v_{0}'  &=  C_{0} + \varepsilon  C_{1} \implies  C_{0} = 1 , \quad  v_{0}'  = 1  \\
  \varepsilon \left( \omega ^2 \left( v_{0} C_{0} \right)  \right) - C_{0}'  - \varepsilon  C_{1}'   & = \omega ^2 \varepsilon  f\left( \varepsilon  \right) \\
  \omega ^2 v_{0} - v_{0}''  &=  \omega ^2 f\left( x \right) \\
  v_{0}''  - \omega ^2 v_{0}  & = - \omega ^2 f\left( x \right) \\
  v\left( 0 \right) = 0  &  \implies  v_{0}\left( 0 \right) = 0 \\
\end{split} 
\] 
Also 
\[
\begin{split}
C\left( \lambda  \right) = 1 = 1 + \varepsilon C_{1} \left( \lambda   \right) + O\left( \varepsilon  \right)  \\
\implies  C_{1} \left( \lambda  \right) = 0    \implies  v_{0} '  \left( \lambda  \right)  & = 0
\end{split} 
\] 
$v$ and $C$ is continuous . $v_{0}$ and $v_{0}' $ continuous. 

\newpara
For $0 \le x \le 1$  we have \[
v_{0}''  + \omega ^2   = - \omega ^2 x 
\] 
A solution to $ v_{0}''  + \omega  = 0$ \[
E e^{\omega x} + E e^{- \omega  x} = A \cosh \left( \omega  x \right) + B \sinh \left( \omega x \right)
\] 

\begin{tcolorbox}
\textbf{Identities.}  \[
  \begin{split}
\cosh u &=  \frac{1}{2} \left( e^{u} + e^{-u} \right) \\
\sinh u &=  \frac{1}{2} \left( e^{u} - e^{-u} \right) \\
\cosh' u  & = \sinh u \\
\sinh'  u  &  = \cosh u \\
\cosh u - v &=  \cosh u \cosh u  - \sinh u \sinh v \\
\cosh 0 &=  1 \\
\sinh 0  & = 0
  \end{split} 
\] 
\end{tcolorbox}

The solution is for $0 \le x \le 1$\[
v_{0} \left( x \right) = x + A \cosh \omega x + B \sinh \omega  x
\] 
In the same manner \[
  \begin{split}
v_{0} ^{+} \left( x \right) &= \overbrace{1 + C \cosh \omega x + D\sinh \omega x }^{0\le x \le \lambda = \frac{L}{\delta}  } \\
v_{0}\left( 0 \right) = 0  &  \implies  v_{0} ^{ -} = 0 \\
\implies  v_{0} ^{ -} \left( x \right) &=  x + B\sinh \omega x \\
\frac{d v_{0}^{+}}{d x}  \left( \lambda  \right) &=  0 \\
C \omega  \sinh \omega \lambda  +  D\omega \cosh \omega \lambda  &=  0 \\
  \end{split} 
\] 
The soution is \[
  \begin{split}
v_{0}\left( x \right) &=  E  \cosh \varepsilon \left( x- \lambda  \right) \\
  \end{split} 
\] 
Require continuity at $x = 1$  of  $ v_{0}\left( x \right)$ and $C_{1}\left( x \right) = \frac{d v_{0}}{d x}  \left( x \right)$\[
  \begin{split}
    v_{0}^{-} \left( 1 \right) &=  v_{0}^{+}\left( 1 \right) \\
    \frac{d v_{0}^{-}}{d x}  &=  \frac{d v_{0}^{+}}{d x}  \\
  \end{split} 
\] 
We get \[
  \begin{split}
v_{0}^{-} \left( x \right) &=  x - \frac{\cosh \left( \omega \left( \lambda  -1 \right) \right)}{ \omega  \cosh \left( \omega \lambda  \right)}  \sinh \omega \lambda  \quad  0\le x \le 1  \\
v_{0} ^{+} &=  1- \frac{\sinh \left( h \omega  \right)}{ \omega  \cosh \left( \omega \lambda  \right)} \cosh\omega \left( x-\lambda  \right) \\
  \end{split} 
\] 
\[
  \begin{split}
  Os ^{* }  &  = \frac{C_{0}}{v\left( \lambda  \right)}  \approx \frac{C_{0}}{v_{0}\left( \lambda  \right)}  \\
  &=  \frac{C_{0}}{ \left( 1 - \frac{\sinh \omega  }{\omega }  \frac{1}{\cosh \omega  \lambda }  \right)}  \\
  \end{split} 
\] 
$\varepsilon  \ll  1$, $Os ^{*}$ depends on $ \omega  $ and $ \lambda  \omega  = k$. 

\newpara
If $\omega $ is smak then is \[
\frac{\sinh \omega }{\omega }  \approx 1 + \frac{1}{6} \omega ^2 + \ldots
\] 
Let \[
  \begin{split}
Os ^{* }  & \approx \frac{C_{0}}{ 1- \frac{1}{\cosh k}}  = C_{0}\left( \frac{\cosh k}{ \cosh k - 1}  \right) = C_{0}\left( \frac{1+ \frac{1}{2} k^2 + O\left( k^{4} \right)}{ \frac{1}{2} k^2 + O\left( k^{4} \right)}  \right) \\
 & \approx \left( 1+ \frac{2}{k^2} \right) C^{*}
  \end{split} 
\] 

Argue that \[
\frac{2}{k ^{2}} \approx \frac{F^{*} _{\text{Diffusion}}}{ F^{*} _{\text{Convection}}} 
\] 
We can finally conclude that \[
Os ^{* } \approx C_{0}\left( 1+ \frac{F^{*} _{ \text{diff}}}{ F ^{*} _{\text{conv}}}  \right)
\] 

\section{Singular Perturbation}%
\label{sec:singular_perturbation}


\[
\varepsilon m^2 +  2m + 1 = 0, \quad  0< \varepsilon  \ll  1 , \quad m = .\frac{1}{2}  
\] 

If $\varepsilon m^2$ and $1$ are important \[
m \pm e \varepsilon ^{\frac{1}{2}} \quad  \implies  \varepsilon  m^2 + 2m \approx 0  
\] 
\[
\begin{split}
  \leftrightarrow  m\left( \varepsilon  m + 2  \right) &=  0 \\
  m  & \approx -\frac{2}{\varepsilon } \\
  \varepsilon  m^2 \approx -\frac{2}{\varepsilon } \\
  2m  &  \approx \frac{4}{3} \\
  \varepsilon m^2 + 2m + 1 &=  0 \\
  m &=  -\frac{1}{2} + \varepsilon  m_{1} \\
  m&=  -\frac{2}{3} \widetilde{m_{1}} \varepsilon  \\
\end{split} 
\] 

\subsection{Singular perturbation applied to differential equations}%
\label{sub:singular_perturbation_applied_to_differential_equations}

\[
\begin{split}
  \varepsilon y'' + 2 y'  + y &= 0 \\
  y\left( 0 \right) = 0,  &  \quad y\left( 1 \right) = 1   \\
  0\le x  &  \le 1
\end{split} 
\] 

Let $\varepsilon  = 0$ then is \[
\begin{split}
2 y'  + y &= 0 \implies  y = k  e ^{ -\frac{x}{2}} , k \in  \mathbb{R}  \\
y\left( 0 \right) &= 0 \implies  y := 0 \\
y\left( 1 \right) &= 1 \implies  y\left( x \right) = e^{\frac{1}{2}} e^{-\frac{x}{2}} \\
\end{split} 
\] 

Ther characteristic equation for \[
  \begin{split}
\varepsilon y''  + y'  + y &=  0 \\
\varepsilon r ^2 + 2 r + 1 &=  0, \quad  r_{1} \approx -\frac{1}{2} , r_{2} \approx-\frac{2}{3}  \\
y\left( x \right)  &  \approx A e^{-\frac{x}{2}} B e^{ -\frac{2x}{\varepsilon }}
  \end{split} 
\] 


For $y\left( 0 \right) = 0$ \[
y\left( x \right) =  A \left( e ^{ -\frac{x}{2}} - e^{-\frac{2x}{\varepsilon }} \right) \\
\] 
And for $y\left( 1 \right)=1$ \[
y\left( x \right) \approx e^{-\frac{1}{2}} \left( e^{-\frac{x}{2}} - e ^{ -\frac{2x}{\varepsilon }} \right)
\] 



\subsection{Further look at Singular Perturbation}%
\label{sub:further_look_at_singular_perturbation}

Our main equation \[
\varepsilon y''  + 2 y'   + y = 0, \quad  y\left( 0 \right) = 0,  \quad    y\left( 1 \right) = 1
\] 
\begin{enumerate}[label=(\roman*)]
  \item Find outer solution $y_{o}$ by setting $\varepsilon  = 0$. Since the solution $\varepsilon y_{0}\left( x \right) \approx y\left( x \right)$ for \[
      \begin{split}
  x > \delta \left( \varepsilon  \right), &  \quad \text{where}\quad \delta \left( \varepsilon  \right) \to 0 \text{ when } \varepsilon \to 0   \\
 y_{0}\left( x \right) = e ^{\frac{1}{2}} e^{-\frac{x}{2}} 
      \end{split} 
  \] 
  Characteristic equation for \[
    \begin{split}
  Y\left( \frac{x}{\delta \left( \varepsilon  \right)} \right) &= y\left( x \right) \quad \text{ is}  \\
  \zeta   & = \frac{x}{\delta \left( \varepsilon  \right)},    \quad  Y\left( \zeta  \right) = \frac{x}{y\left( \zeta \delta \left( \varepsilon  \right) \right)}
    \end{split} 
  \] 
  \[
  \varepsilon Y'' + 2 Y'  + Y = 0, \quad  \implies  \quad  \varepsilon \frac{1}{\delta ^2} Y''  + \frac{2}{\delta } Y' + Y =0  
  \] 
  Are of order of 1. \[
  \begin{split}
    \implies  \varepsilon \frac{1}{\delta ^2} , \frac{1}{\delta },  & 1 \quad \text{are} \\
  \end{split} 
  \] 
  the "size" of the terms. 
  
  \newpara
  CHossing $\delta =\varepsilon $ gives \[
  \frac{1}{3} Y'' + \frac{2}{3} Y'  = 0 \implies \quad  Y'' + 2Y'  + \varepsilon Y = 0
  \] 

\newpara
Let \[
Y\left( \frac{x}{\delta \left( \varepsilon  \right)} \right) = y\left( x \right), \quad y\left( 0 \right) = 0 \implies  Y\left( 0 \right) =  
\] 
  Which is called the \textbf{inner equation.} 
Putting $ \varepsilon  = 0$ and $Y'' + 2 Y'  = 0$ where \[
\implies  Y\left( \zeta  \right) = D + E e ^{2\zeta }
\] 
We see that \[
\begin{split}
Y\left( 0 \right)  & = 0 \implies  E = -D \\
Y\left( \zeta  \right) &= E\left( 1- e^{-2\zeta } \right) \\
\end{split} 
\] 

Let us match it with this equation \[
y_{0}\left( x \right) = e^{\frac{1}{2}} e^{-\frac{x}{2}}
\] 

We can try to match the solution at $x = \theta \left( \varepsilon  \right)$. Then we need to require that \[
  \begin{split}
\lim_{\varepsilon  \to  0^{+}}  \theta \left( \varepsilon  \right)   & = 0 \\
\lim_{\varepsilon \to 0^{+}} \frac{\theta \left( \varepsilon  \right)}{\delta \left( \varepsilon  \right)}  &= \infty \\
  \end{split} 
\] 
\begin{tcolorbox}
\textbf{Example.} 
\[
\delta  =\varepsilon , \quad  \theta = \varepsilon ^{\frac{1}{2}} 
\] 
\end{tcolorbox}



\newpara
We know that \[
Y\left( \frac{x}{\delta \left( \varepsilon  \right)} \right) = y_{I} \left( x \right)
\] 

Then can we start matching such that \[
y_{I}\left( \theta \left( \varepsilon  \right) \approx y_{0}\left( \theta \left( \varepsilon  \right) \right) \right) \implies  Y\left( \frac{\theta \left( \varepsilon  \right)}{\delta \left( \varepsilon  \right)}  \right) = y_{0}\left( \theta \left( \varepsilon  \right) \right)
\] 


Let $\varepsilon  \to  0$ and require equality since $\frac{\theta \left( \varepsilon  \right)}{\delta  \left( \varepsilon  \right) } \to \infty $ , $\theta \left( \varepsilon  \right) = 0$.  THen we obtain \[
\lim_{\zeta \to \infty}  Y\left( \zeta  \right) = \lim_{x\to 0} y_{0}\left( x \right)
\] 
the matching condition \[
\lim_{\zeta \to  \infty}  E\left( 1- e^{-2\zeta } \right) = \lim_{x \to  0}  e^{\frac{1}{2}} e^{-\frac{x}{2}} , \quad  \implies  E = e^{\frac{1}{2}} 
\] 
\[
y_{0}\left( x \right) + Y\left( \frac{x}{\varepsilon } \right) - \lim_{x\to 0} y_{0}\left( x \right) = y_{u}\left( x \right)
\] 
The uniform solution  \[
  \begin{split}
 y_{u}\left( x \right)  & = e^{\frac{1}{2}} e^{-\frac{x}{2}} + e^{\frac{1}{2}} \left( 1- e^{-\frac{2x}{\varepsilon }} \right) - e^{\frac{1}{2}} \\
 &= e^{\frac{1}{2}} \left( e^{-\frac{x}{2}} - e^{-\frac{2x}{\varepsilon }} \right) \\
  \end{split} 
\] 

\subsection{Biochemical reaction kinetics}%
\label{sub:biochemical_reaction_kinetics}

Let the differential equation be \[
\frac{d f^{*}\left( t^{*}  \right)}{d t^{*}}  = k a^{*} \left( t^{*} \right) b^{*} \left( t^{*} \right)
\] 
Where $s ^{*}, e ^{* } , c ^{*} , p ^{*}$ be molar concentrations of $S, E, C$ and P at time $t ^{*}$. 
\begin{align}
\frac{d s ^{*}}{d t ^{*}}  &= -k, \quad s ^{*}, e ^{*} + k_{-1}c ^{*}  \\
\frac{d e ^{*}}{d t^{*}}  & =  - k_{1} s ^{*} e ^{*} 0 k_{2}  \\
\frac{d c ^{*}}{d t ^{*}}  &= k_{1} s ^{*} e ^{*} - \left( k_{-1} + k_{2} \right) \\
\frac{d p ^{*}}{d t ^{*}} &= k_{2} c ^{*} 
.\end{align}

Add 2) and 3) we get \[
\begin{split}
  \frac{d }{d t ^{*}}  \left( e^{*} + c ^{*} \right) &=  0 \\
  e ^{*} \left( t ^{*} \right) +  c ^{* } \left( t ^{*} \right) &= k \\
\end{split} 
\] 
Inital conditions \[
  \begin{split}
s ^{*}\left( 0 \right) = \overline{s}   & , \quad e ^{*}\left( 0 \right) = \overline{e}  \\
c ^{*}\left( 0 \right)  & = p  ^{*}\left( 0 \right) =0 
  \end{split} 
\] 

We have that \[
e ^{*} \left( t \right) = \overline{e}  - c ^{*}\left( t ^{*} \right) \\
\] 
 1) gives out \[
   \begin{split}
 \frac{d s ^{*}}{d t ^{*}}  &=  - k_{1} s ^{*}\left( \overline{e}  -  c ^{*} \right) + k_{1} c ^{*} \\
  \frac{d s ^{*}}{d t ^{*} }  &=  - \left( k_{1} \overline{e}  \right) s ^{*} + \left( k_{1}  \right) s ^{*} c ^{* } + k_{-1} c ^{*} \\
 \implies  \frac{d s ^{*}}{d t ^{*}}  &= -\left( k_{1} \overline{e}  \right) s ^{*} + \left[ k_{1} s ^{*} + k_{-1} \right] c ^{*} \\
 \\
   \frac{d c^{*}}{d t ^{*} }  &=  k_{1} s ^{*} \left( \overline{e} - c ^{*}  \right) \left( k_{-1} + k_{2} \right) c ^{*}\\
 \implies \frac{d c^{*}}{d t^{*}}  &= \left( k_{1} \overline{e}  \right) - \left[ k_{1} s ^{*} + k_{-1} + k2 \right] c ^{*} \\
   \end{split} 
 \] 

 
 \newpara
 
 Let the scalars be $s ^{*} = \overline{s}  s$ , $c ^{*} = \overline{e}  c$, $t ^{* } = Tt$. \[
 \frac{\overline{s} }{T}  s'  = - \left( k_{1} \overline{e}  \right) \overline{s}  s + \left[ k_{1} \overline{s}  + k_{-1}  \right] \overline{e}  c
 \] 
 \[
   s'  = - \left(T k_{1} \overline{e}  \right)  s + \left[ Tk_{1} \overline{e} s  + k_{-1} \frac{\overline{e}  T}{\overline{s} }   \right]   c
 \] 
 Let $ \displaystyle Tk_{1} e = 1  \implies  T = \frac{1}{\overline{e}  k_{1}} $ \[
 s'  = -s + \left[ s 0 \left( \frac{k_{-1}}{k_{1}\overline{s} }  \right) \right]
 \] 
 
 


\newpage
\section{Lecture 2020-09-14}%
\label{sec:lecture_2020_09_14}

\subsection{Biochemical example, kinetics}%
\label{sub:biochemical_example_kinetics}

We start with the inital conditions \[
\begin{split}
  s ^{*} \left(  0 \right)  & = \overline{s}  \\
  e ^{* } \left( 0 \right) &=  \overline{e}  \\
  c ^{*} \left( 0 \right) &=  0 \\
  p^{*} \left( 0 \right) &=  0 \\
\end{split} 
\] 

With the reaction equations 
\begin{align}
  \label{eq:li}
  \frac{d s ^{*} }{d  t^{*}}  &=  -k_{1} e ^{*} s ^{*} + k_{-1} c ^{*} \\
  \frac{d e^{*}}{d t^{*}}  &=  -k_{1} e^{*} s ^{*} + k_{-1} c ^{*} + k_{2} c^{*} \\
  \frac{d c ^{*}}{d t^{*}}  &=  k_{1} e ^{*} s ^{* } - k_{1} c ^{*} - k_{2} c ^{*} \\
  \frac{d p ^{*}}{d t^{*} }  &=  -k_{2} c ^{*} 
\end{align}
 
We can eliminate \[
  \begin{split}
e ^{*} + c ^{*} &=  \overline{e}  \\
e ^{*}  & = \overline{e}  - c ^{*}
  \end{split} 
\] 

Inserter into (1) . \[
\begin{split}
  \frac{d s ^{*}}{d t ^{*}}  &=  -k_{1} s ^{*} \left( \overline{e}  -  c ^{*} \right) + k_{-1} c ^{*} \\
  \frac{d c ^{*}}{d t ^{*}}  &=  + k_{1} s ^{*}  \left( \overline{e}  -  c ^{*} \right)  -  \left( k_{-1} + k_{2} \right) c^{*} \\
  & \\ 
\end{split} 
\] 
Which can be transformed to 
\begin{align}
  \label{eq:labe}
  \frac{d s ^{*}}{d t^{*}}  &=  - \left( k_{1} \overline{e}  \right) s ^{*} +  \left[ k_{1} s ^{*} + k_{-1} \right] c ^{*} \\
  \label{eq:labe2}
  \frac{d c ^{*}}{d t^{*}}   & = \left( k_{1} \overline{e}  \right) s ^{*} -  \left[ k_{1} s ^{*} - \left( k_{-1} + k_{1}   \right) \right] c ^{*} 
.\end{align}

We can then scale such that \[
s ^{*} = \overline{s}  s, \quad  c ^{*} = \overline{e}  c , \quad  t ^{*} =  Tt    \\
\] 
Using \eqref{eq:labe}, \[
  \begin{split}
    \frac{\overline{s}  }{ T}  s'  &=  -T \left( k_{1} \overline{e}  \right) \overline{s}  s + \left[ Tk_{1} \overline{s} s + \frac{k_{-1} T}{ \overline{s} }  \right] \overline{e}  c \\
s'   & = - \left( T \overline{e} k_{1} \right)s + \left[ \left( Tk_{1} \overline{e}  \right)s + \frac{k_{-} T \overline{e}  }{ \overline{s} }  \right]c
  \end{split} 
\]  

Put $T = \frac{1}{\overline{e}  k_{1}}$ we find \[
\implies  s'  = - s + \left[ s + \frac{k_{-1}}{ k_{1} \overline{s} }  \right] c
\] 
Now seeing \eqref{eq:labe2} we get \[
\begin{split}
  \overline{e} \frac{c' }{ \overline{s} }  &=  \left( T k_{1} \overline{e}  \right) \overline{s}  s - \left[ k_{1} \overline{s}  s T + \frac{\left( k_{-1} + k_{2} \right) T}{ \overline{s}}  \right] \overline{e}  c \\
  \implies  \left( \frac{\overline{e} }{ \overline{s} }\right)  c' &= s - \left[ s + \frac{\left( k_{-1} + k_{2} \right)}{ \overline{s}  k_{1}}  \right] c \\     \\ 
  \frac{\overline{e} }{\overline{s} }  = \varepsilon  , \quad   &  \frac{k_{-1} + k_{2}}{ \overline{s}  k_{1}}  = k , \quad  \frac{k_{2}}{ \overline{s}  k_{1}}  = \lambda \\
\end{split} 
\]  

We then end up with \[
  \begin{split}
s'  &=  -s + \left[ s + k - \lambda  \right]c \\
\varepsilon c'  &=  s - \left[ s + k  \right] c \\ , 
s(0) &= 1 , c\left( 0 \right) = 0 \\
  \end{split} 
\] 
 Assume that \[
 \frac{\overline{e} }{\overline{s}  }  = \varepsilon  \ll  1
 \] 
\begin{figure}[ht]
    \centering
    \incfig{whenisgood}
    \caption{whines good}
    \label{fig:whenisgood}
\end{figure}


\textbf{Outer solution} : \[
  \begin{split}
 s  & = s_{0} + \varepsilon  s_{1} + \ldots \\
 c &=  c_{0} + \varepsilon  c_{1} + \ldots \\
  \end{split} 
\] 
Put $\varepsilon  = 0$ . This gives \[
\begin{split}
  0 &=  s - \left[ s+ k \right]c \\
  c&= \frac{s}{s+ k} \\
\end{split} 
\] 

C is then \[
  \begin{split}
s'  &=  -s + \left[ \left( s+ k \right) - \lambda  \right] \frac{s}{s+k}  \\
\implies  s'  &=  - \frac{\lambda s}{s + k}  \\
\implies  \left( \frac{s + k}{ s}  \right) ds &=  - \lambda  dt \\
 & \downarrow    \text{Integration} \\
\text{ Outer solution }  & 
\begin{cases}
s + k \ln s &=  -\lambda  t + K , \quad  K \text{ is constant.}  \\
c &=  \frac{s}{s + k} \\
\end{cases}
  \end{split} 
\] 

Let us introduce \[
  \begin{split}
S \left( \frac{t}{\delta } \right) = s\left( t \right) , \quad  \tau   & = \frac{t}{\delta }    \\
C\left( \frac{t}{\delta } \right) &=  c\left( t \right) \\ 
  \end{split} 
\] 
For the inner solution (now capital) . \[
\begin{split}
  \frac{1}{\delta } S'  &=  -S + \left[ S + k - \lambda  \right]C \\
  \frac{\varepsilon }{\delta }  C'  &=  S - \left[ S + k \right]C \\
\end{split} 
\] 
To retain $\left( \frac{\varepsilon}{\delta } C'   \right)$ we choose $\delta = \varepsilon $ . This gives \[
\begin{split}
  S'  &=  \varepsilon  \left( -S + \left[ S + k - \lambda  \right] C \right) \\
  C'  &= S - \left[ S + k \right]C \\
\end{split} 
\] 
If we let $\varepsilon  = 0: \quad   S' = 0 $. So we have that $S\left( \tau  \right) = L$, but $s\left( 0 \right) = 1$ , means $S\left( 0 \right) =1$ \[
S\left( \tau  \right) = 1
\] 
This gives $C' = 1 - \left[ 1 + k \right] C$, with the solution \[
  \begin{split}
    C\left( \tau  \right) &=  \frac{1}{1+ k} + M e ^{- \left( 1 + k \right) \tau } \\
    C_{I}\left( 0 \right) &=  0, \quad  \implies  \quad  C\left( \tau  \right) = \frac{1}{1+ k} \left[ 1 - e ^{ -\left( k +1 \right) \tau } \right]   \\
    S_{I} \left( \tau  \right) &=  1 \\
    C_{0} \left( t \right) &=  \frac{S_{0} \left( t \right)}{ S_{0} \left( t \right) + k}  \\
    S_{0} \left( t \right) + k \ln  S_{0} \left( t \right) &=  - \lambda t  K \\
  \end{split} 
\] 
\end{enumerate}


\textbf{Matching.} \[
  \begin{split}
\theta \left( \delta  \right) \to  0  & , \quad  \text{when} \quad  \delta \to  0   \\
\frac{\theta \left( \delta  \right)}{ \delta }  \to  \infty , \quad  \text{when} \quad  \delta  \to  0  
  \end{split} 
\] 
\[
  \begin{split}
\lim_{\delta \to 0}  \begin{bmatrix} 
S^{I} \left( \theta \left( \delta  \right) \frac{1}{ \delta }  \right) \\
C^{I} \left( \frac{\theta \left( \delta  \right)}{ \delta }  \right) 
\end{bmatrix} &=   
 \lim_{\delta  \to  0} \begin{bmatrix} 
   S_{0} \\
 C_{0} \left( \theta \left( \delta  \right) \right) 
\end{bmatrix}  \\
\implies   &  \lim_{\tau \to  0} \begin{bmatrix} 
S_{I} \left( \tau  \right) \\
C_{I}\left( \tau  \right)
\end{bmatrix} 
  \end{split} 
\] 


\[
\begin{split}
  \lim_{t\to  0}  \begin{bmatrix} 
  S_{0}\left( t \right) \\
  C_{0} \left( t \right) 
  \end{bmatrix} 
  &=  \lim_{ \tau  \to  \infty}  \begin{bmatrix} 
  1 \\
  \frac{1}{1+ k} \left( 1 - e ^{- \left( 1+ k  \right) \tau } \right)  
  \end{bmatrix} 
  =
  \begin{bmatrix} 
  1 \\
  \frac{1}{1+k}
  \end{bmatrix} 
   \\
\end{split} 
\] 


\textbf{ Uniform solution }   \[
  \begin{split}
\begin{bmatrix} 
  S_{u} \\
  C_{u} 
\end{bmatrix} 
 & = 
\begin{bmatrix} 
S_{0} \left( t \right) \\
C_{0} \left( t \right) 
\end{bmatrix} 
+ 
\begin{bmatrix} 
S_{I} \left( \frac{t}{\varepsilon } \right)  \\
C_{I} \left( \frac{t}{\varepsilon }  \right)
\end{bmatrix} 
- \begin{bmatrix} 
1 \\
\frac{1}{1+ k}
\end{bmatrix}  \\
&=   \begin{bmatrix} 
S_{0} \left( t \right)  \\
C_{0} \left( t \right) - \frac{1}{1+k} e^{-\left( 1+ k \right) \frac{t}{\varepsilon }}
\end{bmatrix} 
  \\
  &=  \begin{bmatrix} 
  S_{0} \left( t \right) \\
S_{0} \left( t \right) \frac{1}{ S_{0} \left( t \right) + k}  - \frac{1}{1+k} e^{\left( 1 + k \right) \frac{t}{\varepsilon }}
  \end{bmatrix} 
   \\
   \\
   S_{0}'  + k + k \frac{S_{0} ' }{ S_{0}} &=  - \lambda  \\ 
   S_{0} \left( 0 \right) = 1 \implies  \quad  S_{0}' \left( 0 \right) = \frac{-\lambda }{1+k}  
   S_{0} \left( t \right) &=  1 - \frac{\lambda  }{ 1+ k}  t + O\left( t^2 \right) \\
  \end{split} 
\] 
For large $\lambda t : k \ln S_{0}\left( t \right) \approx - \lambda  t$  \[
S_{0}\left( t \right) \approx e ^{ \frac{\lambda}{ k} t}
\] 



\subsection{Stability}%
\label{sub:hehehe}


\subsubsection{Dynamical Systems}%
\label{ssub:dynamical_systems}

Let \[
  \begin{split}
x'   & = f_1\left( x_{1} , x_{2} , \ldots , x_{n} \right) \\
x_{2}'  &=  f_{2}\left( x_{1}, x_{2} , \ldots , x_{n} \right) \\
 & \vdots  \\
x_{n}'  &=  f_{n}\left( x_{1}, x_{2} , \ldots , x_{n} \right) \\
  \end{split} 
\] 
Where $x_{j} \left( 0 \right) = x_{j}^{(0)}$ are given. Write this as \[
\mathbf{x} '  =  \mathbf{f}\left( \mathbf{x} \right), \quad  \mathbf{x}\left( 0 \right) = \mathbf{x}^{(0)}, \quad  \mathbf{x}\left( t  \right) \in \mathbb{R}    \\
\] 

\begin{tcolorbox}
  \textbf{Example.} \[
  \begin{split}
    x_{1}'  &=  -x_{2} , \quad  x_{1}\left( 0 \right) = 1   \\
    x_{2} ' &=  x_{1}, \quad  x_{2} \left( 0 \right) =  0   \\
  \end{split} 
  \] 
  An equilibrium point for \[
  \mathbf{x}'  = \mathbf{f}\left( \mathbf{x} \right) 
  \] 
  is a constant solution. I.e. $\mathbf{x}_{e}$ is an equilibrium point \[
  \implies  \mathbf{f}\left( \mathbf{x}_{e} \right) = 0
  \] 
\end{tcolorbox}

\begin{definition}
  An equilibrium point  $\mathbf{x}_{e}$ is \textbf{stable}  if for any $\varepsilon > 0 $ , there exist a $\delta  > 0$ such that if \[
  \|\mathbf{x}\left( 0 \right) - \mathbf{x_{e}}\|_{}^{ } < \delta  \implies  \|x\left( t \right) - \mathbf{x_{e}}\|_{}^{} < \varepsilon  , \quad  \text{ for } t > 0 
  \] 
\end{definition}


\begin{definition}
  If $\mathbf{x_{e}}$ is stable and, there exists a $\delta > 0 $ such that always \[
  \|x\left( 0 \right) - \mathbf{x_{e}}\|_{}^{} < \delta 
  \] 
  Implies \[
    \lim_{t\to \infty} \mathbf{x}\left( t_{1} \right) = \mathbf{x_{e}}
  \] 
  Then $\mathbf{x_{e}}$ is an asymptotically stable equilibrium point. 
  
  If $\mathbf{x_{e}}$ is not stable, it is  \textbf{unstable} .
\end{definition}

\begin{tcolorbox}
  \textbf{Example.}  \[
  x'  = -x , \quad  x_{e} = 0  \text{ is a equilibrium point.} 
  \] 
   Where the solution is \[
   x = C e^{-t} \quad  \to  \quad  \text{for any } C  
   \] 
\end{tcolorbox}

\subsubsection{Linearization}%
\label{ssub:linearization}

\[
x_{j}'  =  f_{j} \left( x_{1}, x_{2} , \ldots , x_{n} \right) , \quad  j = 1,2, \ldots, n  \\
\] 
Assume $\mathbf{x_{e}}$ is an equilibrium point.  If $f_{j}$ is differentiable we can write \[
.
  \frac{f_{j} \left( \mathbf{x_{0} } + \delta  \mathbf{\delta x} \right) - f_{j}\left( \mathbf{x} \right) + \sum_{i=1}^{n}  \frac{\partial f_{j}}{\partial x_{i}}  \Delta x_{i}}{ \|\Delta x\|_{}^{}}  \overbrace{\to}^{ \|\Delta x \to  0\|_{}^{}}   0
\] 

From matrix notation \[
\mathbf{f}\left( \mathbf{x_{1} } \Delta  \mathbf{x} \right) =  \mathbf{f}\left( \mathbf{x_{2}} \right) + J \left( \mathbf{x_{0}} \right) \Delta \mathbf{x_{1}} \\
\] 

Where the $n\times n $ matrix $J\left( \mathbf{x_{0}} \right)$ is given by \[
\left( J\left( \mathbf{x_{0}} \right) \right)_{ij} = \frac{\partial f_{j}}{\partial x_{i}}  \left( \mathbf{x_{0}} \right)
\] 
And is called the jacobian matrix of $\mathbf{f}\left( \mathbf{x} \right)$ at $\mathbf{x} = \mathbf{x_{0}}$


\newpage
\section{Lecture 2020-09-16}%
\label{sec:lecture_2020_09_16}


\subsection{Stability}%
\label{sub:stability}
Dynamic system \[
\mathbf{x' } = f\left( \mathbf{x}\left( t \right) \right)
\] 
Where $\mathbf{x}\left( t \right) \in \mathbb{R} ^{n}$. The equilibrium point $\mathbf{x}_{e} \implies  f\left( \mathbf{x}_{e} \right) = 0$ . \[
\mathbf{x}_{e} \text{ is either } \begin{cases}
  \text{stable} \\
  \text{asymptotically stable} \\
  \text{unstable}
  
\end{cases}
\] 

We can determinde the linear approximation if \[
\frac{\partial f_{i}}{\partial x_{i} \partial  x_{q}} .
\] 
is contiionus at  $\mathbf{x_{e}}$ . Then we have \[
  \begin{split}
f\left( \mathbf{x_{e}} + \delta  \mathbf{x} \right)  & = f\left( \mathbf{x}_{e} \right) + J \left( \mathbf{x_{e}} \right) \Delta x  + O\left( \|\Delta x\|_{}^{2} \right)  \\
\implies  f\left( \mathbf{x_{e}} + \Delta \mathbf{x} \right)  & \approx J\left( \mathbf{x_{e}} \right) \Delta \mathbf{x}, \quad  \mathbf{x_{e}} + \Delta \mathbf{x}   = \mathbf{x}  \\
 \Delta \mathbf{x}'   & = J\left( \mathbf{x_{e}} \right) \Delta \mathbf{x}, \quad \Delta \mathbf{x} \left( 0 \right) = 0 
  \end{split} 
\]

If $J\left( \mathbf{x}_{e} \right)$ has $n$ linearly independent eigenvectors $v_{1}, v_{2}, \ldots , v_{n} $ with corresonding  eigen values $\lambda _{1}, \lambda _{2}, \ldots, \lambda _{n}$ . Then the solution to \[
\Delta \mathbf{x}' = J\left( \mathbf{x_{0}} \right)  \mathbf{x}_{0} 
\] 
Is \[
\Delta \mathbf{x}\left( t \right) = c_{1} \mathbf{x_{1}}e^{\lambda _{1}t} + c_{2} \mathbf{x_{2}} e^{\lambda _{2} t} + \ldots + c_{n} \mathbf{x_{n}} e^{\lambda _{n} t}
\] 
Where $ c_{1}, \ldots, c_{n} $ is determined by $ \Delta \mathbf{x} \left( 0 \right)$ . $\mathbf{x_{e}}$ is an asymptotically stable eq. Point  if $Re \lambda _{j} < 0 $ for $ j = 1,2 ,\ldots, n$ for the system $\mathbf{\mathbf{x}' } = f\left( \mathbf{x} \right) $. If $Re \lambda _{k} > 0$ for one $k$ , then $\mathbf{x_{e}}$ is unstable. 

\begin{tcolorbox}
  \textbf{Example.}  \[
    \begin{split}
      x_{1}  & = x_{2} \\
      x_{2} '  &= -2x_{1} -2x_{2}   \\
    \end{split} 
  \] 
  \[
    \begin{split}
  \mathbf{x}  & = \begin{bmatrix}  
  0 \\
  0
  \end{bmatrix}, \quad   
  \mathbf{x}'  
    = \begin{bmatrix} 
  0 &  1 \\
  -2  &  -2
  \end{bmatrix} 
  \\
  &= A\mathbf{x}, \quad  \|A - \lambda I\|_{}^{} = 0  \\
    \implies   &  \begin{vmatrix} 
    -\lambda   & 1 \\
    -2  &  -2-\lambda 
    \end{vmatrix} 
    = 0 \\
    \end{split} 
  \] 


\end{tcolorbox}

  \begin{tcolorbox}
    \textbf{Example.}  \[
    \begin{split}
      x_{1}'   & = x_{1} ^2 - x_{2}  '  \\
      x_{2} '  &=  2x_{1} + x_{2} + 3 \\
    \end{split} 
    \] 
    Solve  \[
      \begin{split}
    x_{1} ^2 + x_{2} ^2  &=  0 \\
    2x_{1} + x_{2} + 3 &=  0 \\
    \implies   x_{2} &=  \pm x_{1} \\
      \end{split} 
    \] 
    when we get 
    \begin{itemize}
      \item $x_{2} = x_{1}$: \[
      3_{} x_{1} + 3 = 0 \quad  \implies  x_{1} = -1 
      \] 
      Which means $\left( -1, -1 \right)$ is a eq. point. 
    \item $x_{2} = -x_{1}$: $2x_{1} - x_{1} + 3 = 0$. Which means $x_{1} = -3$ and $\left( -3, 3 \right)$ is a equ. point.
    \end{itemize}
  \end{tcolorbox}

  Let \[
    \begin{split}
  A  & = \begin{bmatrix} 
    a  &  b \\
    x  &  d
  \end{bmatrix}  , \quad  \left\lvert A - \lambda I  \right\rvert   =  \left( a-\lambda  \right) \left( d - \lambda  \right) -bc \\
  &=  \lambda ^2 - \left( trA  \right) \lambda  + det A = \left( \lambda  - \lambda _{1}  \right) l\left( \lambda  - \lambda _{2} \right) \\
     trA &=   a +d , \quad  \lambda  = a + b  \\
     det A  &  < 0, \quad  \text{unstable}  \\
     det A  & >0 ,\quad   trA < 0  \implies  \text{asymptotic stable}
    \end{split} 
  \] 

  Back to the example  \[
    \begin{split}
  J\left( \mathbf{x} \right)  & = \begin{bmatrix} 
  2x_{1}  &  -2x_{2} \\
  2 &  1
  \end{bmatrix}  \\
  J\left( -1, -1 \right) &=  \begin{bmatrix} 
  -2  &  2 \\
  2  &  2
  \end{bmatrix} 
  \implies  \left\lvert J \right\rvert  = -6 , \quad  \mathbf{x_{e}} \text{ is unstable} 
   \\
   J\left( -3, -3 \right) &=  \begin{bmatrix} 
   -6  &  -6 \\
   2  &  1
   \end{bmatrix} 
   , \left\lvert J \right\rvert  = 6, \quad  tr A = -5 
    \\
    \end{split} 
  \] 
  $Re \lambda _{1} < 0,\quad   Re \lambda _{2} < 0, \implies  \quad  \left( -3, -3 \right)  $  asymptotically stable .

  \subsection{Amoebae and chemotaxis}%
  \label{sub:amoebae_and_cheotaxis}
  
  Let $\phi \left( x,t \right)$ be the amoebae concentration at position $x$ at time $t$. Let $A$ be the cross-sectional area of the tube. Then \[
    \begin{split}
  \left( \int_{x_{1}}^{x_{2}}  \phi \left( x,t \right) dA \right) A  & = \text{ nr amoebae in } \left[ x_{1}, x_{2} \right] \\
  \frac{d }{d t} \left( \int_{x_{1}}^{x_{2}}  \phi \left( x,t \right) dx \right)A  & = J\left( x_{1}, t \right) A - J \left( x_{2}, t \right) A 
    \end{split} 
  \]  
  \begin{itemize}
    \item Flux density: $\displaystyle J = - M \frac{\partial \phi }{\partial x} + E \frac{\partial c}{\partial x}  $, where $M>0$ motility and $E >0$  strength of chemotaxis.
    \item $c \left( x_{1}t \right)$ concentration of signaling substance. 
  \end{itemize}
 \[
   \begin{split}
 x_{2} - x_{1}  & = \Delta x, \quad \widetilde{x} \in  \left<x_{1}, x_{2} \right>  \\
 \implies \frac{\partial }{\partial t}  \left( \phi \left( \widetilde{x}, t  \right) \right) \Delta x  + J \left( x_{2}, t \right) - J\left( x_{1},t \right) &= 0 \\
 \Delta x \to  0  & \\
 \frac{\partial \phi }{\partial t } + \frac{\partial J}{\partial x}   &= 0   \\
   \end{split} 
 \]  
 We can rewrite such that \[
   \begin{split}
 \implies  \frac{\partial \phi }{\partial t}  + \frac{\partial }{\partial x}  \left( -M \frac{\partial \phi }{\partial x}  + E \phi \frac{\partial c}{\partial x}  \right)   &=  0 \\
 \implies  \frac{\partial \phi }{\partial t}  &=  \frac{\partial }{\partial x} \left( M \frac{\partial \phi }{\partial x}  - A \phi  \frac{\partial c}{\partial x}  \right) \\
   \end{split} 
 \] 
 Flux density for $c: \quad J_{c}= - D \frac{\partial c}{\partial x}  $  \[
   \frac{\partial c}{\partial x} + \frac{\partial J_{c}}{\partial x}  =  q_{1} \phi  - q_{2} c \\
 \] 
 \begin{itemize}
   \item $q_{1}$ strength of secretion.
   \item $q_{2}$ decay rate for $c_{1}$
 \end{itemize}
 \[
   \begin{split}
 \frac{\partial c}{\partial t}  + \left( -D \frac{\partial ^2 c}{\partial x ^2}  \right) &=  q_{1} \phi - q_{2} c \\
 \frac{\partial \phi }{\partial t}  &=  M \frac{\partial ^2 \phi }{\partial x ^2}  - E \frac{\partial }{\partial x} \left( \phi \frac{\partial c}{\partial x}  \right) \\
 \frac{\partial c}{\partial t} &=  D \frac{\partial ^2}{\partial x ^2} c + q_{1} \phi - q_{2} c \\
   \end{split} 
 \] 
 \[
 \begin{rcases}
   \phi \left( x,t \right)  & = \phi _{0} \\
   c\left( x,t \right) &= c_{0} \\
 \end{rcases}
 \text{a solution as long as: } q_{1} u_{0} - q_{2} c_{0} = 0
 \] 

 Let write \[
   \begin{split}
 \phi \left( x,t \right) &=  \phi _{0} +  \widetilde{\phi }\left( x,t \right) \\
 c\left( x,t \right) &=  c_{0} + \widetilde{c}\left( x,t \right) \\
 \implies  \frac{\partial \widetilde{\phi }}{\partial t}  &=  M \frac{\partial ^2 \widetilde{\phi }}{\partial  \widetilde{x}}   - E \left( \left( \phi _{0} + \widetilde{\phi } \right) \frac{\partial \widetilde{c}}{\partial x}   \right)_{x}  \\
 \frac{\partial \widetilde{c}}{\partial t}  &=  -D \frac{\partial \widetilde{c}}{\partial x ^2} + q_{1} \widetilde{c} - q_{2} \widetilde{c}  \\
 \implies  \frac{\partial \widetilde{\phi }}{\partial t}  &=  M \frac{\partial ^2 \widetilde{\phi }}{\partial  x ^2}   - E \phi _{0}  \frac{\partial ^2 \widetilde{c}}{\partial x ^2 } - E \left( \widetilde{\phi } \frac{\partial \widetilde{c}}{\partial x }  \right)_{x}  \\
   \end{split} 
 \] 

 Linearization \[
   \begin{split}
 \frac{\partial \widetilde{\phi }}{\partial t}  &=  M \frac{\partial ^2 \widetilde{ \phi }}{\partial x ^2 } - E \phi _{0} \frac{\partial ^2 \widetilde{c}}{\partial  x ^2}   \\
 \frac{\partial \widetilde{c}  }{\partial t}  &=  D \frac{\partial ^2 c \widetilde{c}}{\partial x ^2 } + q_{1} \widetilde{\phi } - q_{2} \widetilde{c}, \quad \widetilde{c} \left( x,0 \right) \text{ and } \widetilde{\phi } \left( x,0 \right) \text{ is geiven}\\
   \end{split} 
 \] 

 Let \[
 \begin{split}
   \widetilde{\phi }\left( x,t \right) &=  \sum_{}^{} \alpha _{n} \left( t \right) e^{i\beta _{n} x}  \\
   \widetilde{c}\left( x,t \right) &=  \sum_{}^{} \gamma _{n}\left( t \right) e^{\beta _{n} x n} \\
 \end{split} 
 \] 
 \begin{enumerate}[label=(\roman*)]
   \item \[
       \begin{split}
   \sum_{}^{}  \alpha _{n}'  \left( t \right) e^{i\beta _{n}x}  & = \sum_{}^{}  \left( -M \beta ^{2}_{n} \alpha _{n} \left( t \right) e^{i \beta _{n} x} + E \phi _{0} \gamma _{n} \left( t \right) e^{i \beta _{n} x} \right) \\
   \alpha _{n}'  \left( t \right)  & = -M \beta _{n}^{2} \alpha _{n} \left( t \right) + E \phi _{0} \gamma _{n}\left( t \right) \beta ^{2}_{n} \\
   \gamma _{n}' \left( t \right) &=  -D \beta _{n}^2 \gamma _{n} \left( t \right) + q _{1} \alpha _{n} \left( t \right) -  q_{2} \gamma _{n}\left( t \right) \\
   \implies \begin{bmatrix} 
   \alpha _{n} \\
   \gamma _{n}
   \end{bmatrix} 
   _{t}
   &= \begin{bmatrix} 
   M \beta _{n}^2  &  E \phi _{0} \beta _{n} ^2 \\
   q_{1}  &  -D \beta _{n}^2 - q_{1}
   \end{bmatrix} 
   \begin{bmatrix} 
   \alpha _{n } \\
   \beta _{n}
   \end{bmatrix} 
    \\
    trA < 0 , \quad  det A &=  M\beta _{n}^2 \left( D \beta _{n}^2 + q_{2} \right) - q_{1} E \phi _{0}\beta _{n} \begin{cases}
      <0 , \quad \text{stable}  \\
      <0 ,  \quad  text{unstable}
    \end{ases} \\ 
       \end{split} 
   \] 
   Unstable when \[
     \begin{split}
   det A  & < 0 \\
   \implies  M\left( D \beta _{n} ^{2} \right) + q_{1} \phi _{0} E  & < 0  \\
  q_{1}  & > \frac{M\left( D \beta _{n} ^2 + q_{2} \right)}{ \phi _{0} E}  \\
  \widetilde{\phi } &=  \sum_{}^{} \alpha _{n}\left( 0 \right) e^{i \beta _{n}x}  , \quad  \beta _{n}^2 \text{ is increasing.}\\
     \end{split} 
   \] 

 \end{enumerate}

 \newpage
 \section{Lecture 2020-09-21}%
 \label{sec:lecture_2020_09_21}
 

 \subsection{Bifurcation}%
 \label{sub:bifurcation}

 First, wee will only consider 1-D dynamical systems in general. Given \[
   \frac{d u}{dt  }  = f\left( \mu , u \right)
 \] 
 where $\mu  \in  \mathbb{R} $ is a parameter. For given $\mu $, we have equilibrium points when \[
 f\left( \mu ,u \right) = 0
 \] 
 If $u_{e} = u\left( \mu  \right)$ is an equilibrium  point, $u_{e}$ is asymptotically stable if \[
 \frac{\partial f}{\partial u}  \left( \mu ,u_{e} \right) < 0
 \]  
 and unstable if \[
 \frac{\partial f}{\partial u}  \left( \mu, u_{e}  \right) > 0
 \] 

 \textbf{Example.} 
 Let the problem be formulated as \[
 \begin{split}
   u'  &=  \left( u-1 \right)\left( \mu - u^2 \right) = f\left( \mu , u \right) \\
   f\left( \mu ,u \right) &=  0, \implies  u=1 \text{ or } u^2 = \mu  \\
 \end{split} 
 \] 

 \textbf{Example.}  \[
 \frac{d u}{d t}  = u \mu - u^2 = \underbrace{u\left( \mu  - u \right)}_{f\left( \mu ,u \right)} 
 \] 
 \[
 \begin{split}
   \frac{\partial f}{\partial u}  &=  \mu  -2u , \quad   \\
   \frac{\partial f}{\partial u} \left( \mu , u= 0 \right)  & = \mu \\
   \frac{\partial f}{\partial u} \left( \mu , u= \mu  \right) &=  -\mu  \\
 \end{split} 
 \] 

 \newpage

\begin{definition}
  \textbf{Implicit function theorem.} Let $\left( \mu ,u \right)$ have continious derivatives around $\left( \mu _{0}, u_{0} \right)$, where $f\left( \mu _{0}, u_{0} \right)= 0$.  Then there are constants $a>0, b> 0 $ such that $f\left( \mu ,u \right) = 0$ has a unique solution $u\left( \mu  \right)$ for \[
  \|\mu - \mu _{0}\|_{}^{} > a, \quad  \|u - u_{0}\|_{}^{} > b 
  \] 
  if $\displaystyle  \frac{\partial f}{\partial u} \left( \mu _{0}, u_{0} \right) \neq 0$ . Then   
  \[
  \frac{d u}{d \mu }  = - \frac{\frac{\partial f}{\partial \mu } }{\frac{\partial f}{\partial u} }  , \quad \frac{\partial f}{\partial \mu }  \left( \mu _{0}, u_{0} \right) \neq 0
  \] 
  We can find \[
  \mu \left( u \right)
  \] 
\end{definition}
 
 
\begin{remark}
  If \[
  \begin{split}
    f\left( \mu _{0}, u_{0} \right) &=  \frac{\partial f}{\partial \mu } \left( \mu _{0}, u_{0} \right) \\
    &=  \frac{\partial f}{\partial u}  \left( \mu _{0}, u_{0} \right) \\
  \end{split} .
  \] 
  Then $\left( \mu _{0}, u_{0} \right)$ is a singular point. Assume ass second derivatives are continuous, and not all zero. Then \[
  f\left( \mu _{0} + \Delta \mu _{1}, u_{0} + \Delta  u \right) \approx \frac{1}{2} f_{\mu \mu } \left( \mu _{0}, u_{0} \right) \Delta \mu ^2 + f_{\mu u } \left( \mu _{0}, u_{0} \right) \Delta \mu  \Delta u + \frac{1}{2} f_{uu} \left( \mu _{0} ,u_{0} \right) \Delta u^2 = 0
  \] 
  Assume $f_{uu}\left( \mu _{0} , u_{0} \right) \neq 0$. \[
    \begin{split}
  \implies  \underbrace{f_{\mu \mu }}_{c}  + \underbrace{2 f_{\mu u}}_{b}  \left( \frac{\Delta u}{\Delta \mu }  \right) + \underbrace{f_{uu}}_{a}  \left( \frac{\Delta u}{\Delta \mu }  \right)^2 &=  0 \\
   4 f^2_{\mu u}  - 4 f_{\mu \mu }f_{uu}   &  > 0, \quad \text{two solutions for } \frac{\Delta u}{\Delta \mu }   \\
    \end{split} 
  \] 
\end{remark}
 
\textbf{Example.}  Legislate: \[
f\left( \mu  , u \right) = \left( \mu ^2 + u^2 \right) ^2 - 2 \left( \mu ^2 - u^2 \right)
\] 
Where $\left( 0,0 \right)$ is a solution. \[
  \begin{rcases}
\frac{\partial f}{\partial \mu }   & = 2 \left( \mu ^2 + u^2 \right) \cdot 2 \mu  - 2 \left( 2 \mu   \right) \\
\frac{\partial f}{\partial u}  &=  2 \left( \mu ^2 + u^2  \right) 2u + 4 u  \\
  \end{rcases}
  = 0 \text{ at } \left( 0,0 \right)
\] 
Where $\left( 0,0 \right)$ is singular. \[
\begin{split}
  f_{\mu \mu } &=  12 \mu ^2 + 4 u^2 - 4 \\
  f_{\mu \mu }  & = 8 u \mu  \\
 f_{uu } &=  12  u ^2 + 4 \\ 
 4 * 0 ^2 - 4 \left( 4 \right) \left( -4 \right)   & \ge 0  \implies  \text{?????}
\end{split} 
\] 
\textbf{Example.} Tank Reactor.  Let $q $ be inflow rate, $\left[ q \right] = m^3 s^{-1}$. With a concentration $c_{in} $ and temperature $\theta _{in}$ and out $c ^{*} , \theta ^{*}$ . 
\begin{itemize}
  \item $c^{*}$ reactor concentration.
  \item  $\theta ^{*}$ Temperature. 
  \item $c^{*}\left( 0 \right) = c_{\text{in}}$
  \item $\theta ^{*}\left( 0 \right) = \theta _{in}$
\end{itemize}
Assume that $c ^{*}$ disappears at rate 
\[
kc ^{* } e ^{ -\frac{A}{\theta ^{*}}} .
\] 
The reaction generates heat given by \[
h\left( kc^{*}e^{\frac{A}{\theta ^{*}}} \right)
\] 
where $h$ is a constant,   $\left[ k \right] = s^{-1}$, $\left[ A \right]= Kelvin = K$ and  $\left[ h \right] = J mole ^{-1} $ . The consentration of reactant is as follows \[
  \begin{split}
\frac{d }{d t^{*}}  \left( V c^{*} \right) &=  q c_{in} - q c ^{*} - Vk c^{*} e^{- \frac{A }{ \theta ^{*}} } \\
\frac{d }{d t} \left( V C_{v} \theta ^{*} \right) &=  qC_{v} \theta _{in} - q C_{v} \theta ^{*} + V hl c^{*} e^{\frac{-A}{\theta ^{*}} } \\
  \end{split} 
\]  
Where $C_{v} $ is heat capacity .   Let start with scaling \[
  \begin{split}
  c^{* }   & = c_{in} c , \quad  \theta ^{*} = \theta _{in} \theta  \\
  t^{*} &=  Tt \\
  \end{split} 
\] 
Then is the equations \[
  \begin{split}
\frac{V}{T} c_{in} \frac{d c}{d t}  &=  q c_{in} - q c_{in} C - k V c_{in} e^{-\frac{A}{\theta _{in} \theta }}\\ 
 \implies  \frac{d c}{d t}   & = T\left( \frac{q}{v} \right) - T \left( \frac{q}{v} \right) -T kc\cdot e^{-\frac{A}{\theta _{in} \theta }}
  \end{split} 
\] 
If we choose $T = \frac{V}{q}$ then is \[
\implies  c'  = 1- c + \left( \frac{Vk}{q} \right)c e^{-\frac{A}{\theta _{in} \theta }}
\]  
To simplify, let us define $\mu  = \frac{q}{kv}$ and $\gamma = \frac{A}{\theta _{in} \theta }$. We then end up with the problem \[
  c'  = 1- c \frac{1}{\mu } e^{\gamma \frac{1}{\theta } }, \quad  c\left( 0 \right) = 1 
\] 
For the other problem is \[
  \begin{split}
\frac{1}{T} v C_{v} \theta _{in} \theta '  \left( t \right)  & = qC_{v} \theta _{in} \frac{1}{V }  - \frac{q C_{v} \theta _{in} \theta }{V}  + hk V c_{in}c e^{\frac{\gamma}{\theta } } \\
\implies  \theta '   & = 1 - \theta  - + \left( \frac{Vhkc_{in}}{qc_{v} \theta _{in}}  \right) c e^{\frac{\gamma }{\theta } } \\
\theta &=  1- \theta + \nu \frac{1}{\mu }  c e^{\gamma \frac{1}{\theta } }  \\
  \end{split} 
\]  
Where $\nu  = \frac{h c_{1}n}{c_{v} \theta _{in}} $ . We then have the equations 
\begin{align}
  \label{eq:eq11}
  c'  &=  1- c - \frac{1}{\mu } c e^{\gamma \frac{1}{\theta } } \\
  \label{eq:eq12}
  \theta '  &=  1- \theta  + \frac{\nu}{\mu }ce^{\frac{\gamma}{\theta } }   
.\end{align}
Multiply \eqref{eq:eq11}  by $\mu $ and add \[
  \begin{split}
\left( \mu c + \theta  \right)'  &=  \mu  - \mu  c + 1 - \theta  \\
\implies  \left( \nu c + \theta  \right) '   & = \nu + 1  \\
h'  + h = \nu +1
\nu c + \theta &=  \nu +1 \implies  \nu c = \nu + 1 - \theta  \\
  \end{split} 
\] 
Inserted into \eqref{eq:eq12}  given \[
  \begin{split}
\theta '  = 1 - \theta  + \frac{\left( \nu +1 - \theta   \right)}{ \mu }  e^{\frac{\gamma }{- \theta } }
\implies  \theta '  &=  1- \theta + \frac{\nu  + 1 - \theta }{\mu } e^{- \frac{\gamma }{\theta } } \\
\theta  & = u +1  \\
\implies  u'   & = -u + \frac{\nu  - u}{ \mu }  e^{\frac{\gamma}{u+1} }
  \end{split} 
\] 
Where $u\left( 0 \right) = 0$ since $\theta \left( 0 \right) = 1$ . Assume that $\gamma , \nu $ are constant while we vary $\mu = \frac{q}{Vk}$ physically \[
f\left( \mu , u \right)  = -u + \frac{\left( \nu - u \right)}{\mu }  e^{-\frac{\gamma }{u+1} }
\] 
Define $h\left( u \right) = \left( \nu - u \right) e^{- \frac{\gamma }{u+1} }$ . \[
f = 0 \implies  -u + \frac{h\left( u \right)}{\mu } = 0 \implies  \mu  u = h\left( u \right) 
\] 

Lets check the stability \[
  \begin{split}
f\left( \mu ,u \right) &=  -u h\left( u \right) \cdot \frac{1}{\mu } \\
\frac{\partial f}{\partial u}  &=  -1 + \frac{h' \left( u \right)}{\mu }  < 0 \\
\implies  h' \left( u \right)  & < \mu \quad  \implies \text{stability}
  \end{split} 
\] 


\newpage

\newpage
\section{Lecture 2020-09-23}%
\label{sec:lecture_2020_09_23}

\subsection{Dynamic population models}%
\label{sub:dynamic_population_models}

   Natural growth: $ \displaystyle  \frac{d v^{*}}{d t^{*}}  = r v^{*}$. Ok as long as there is food and space. But in reality there are limitiation to growth, so a model \[
  \frac{d v^{*}}{d t^{*}}  = r N^{*} \left( 1- \left( \frac{N^{*}}{K} \right)^{\alpha }  \right), \quad K>0 
  \] 
  when $ N^{*} \ll  K$ \[
  \frac{d N^{*}}{d t^{*}}  \approx r N^{*}
  \] 
  \textbf{Scaling.}  $\left[ r \right]=\frac{1}{s}$. One time scale is $T=\frac{1}{r}$.  $K$ natural scale for $N ^{*}$ \[
    \begin{split}
  t^{*}  & = \frac{1}{r} t , \quad  N^{*} = KN  \\
  rkN'&= KrN\left( 1 - \frac{KN}{K}  \right) \\
  \implies  N'  &=  N\left( 1-N \right) \\
  f\left( N \right) &=  0, \quad \implies  N = 0, N=1  \\
    \end{split} 
  \] 

  \textbf{Stability.}  \[
  \begin{split}
    f' \left( N \right) &=  1- 2N \\
    f' \left( 0 \right) &=  1 > 0 , \quad \text{unstable}    \\
    f' \left( 1 \right) = 1 - 2 = -1 < 0 , \text{totally stable}
  \end{split} 
  \] 
  The solution is \[
  N\left( t \right) = \frac{N\left( 0 \right)}{ N\left( 0 \right) + \left( 1 - N\left( 0 \right) e^{-t} \right)} 
  \] 
  \subsubsection{Whales and krill}%
  \label{ssub:whales_and_krill}
  Let $N^{*}\left( t^{*} \right) $ and $H^{*}\left( t^{*} \right)$ be the number of krill and whales respectively.  
  \begin{align}
    \label{eq:a1}
  \frac{d N^{*}}{d t^{*}}   & = rN^{*}\left( 1- \frac{N^{*}}{K_{N}}  \right) - \alpha _{2} N^{*}H^{*} - u_{N}F_{N} N^{*} \\
    \label{eq:a2}
  \frac{d H^{*}}{d t^{*}}  &=  q H^{*} \left( 1- \frac{H^{*}}{\alpha  N^{*}}  \right) - u_{H} F_{H} H^{*} , \quad  N^{*} \neq 0  \\
  r \ll  q  & \implies  T_{N} = \frac{1}{r} \ll  T_{H} = \frac{1}{q} 
  .\end{align}
  \textbf{Scaling.}  Choose $T = T_{H} $ as time scale. $K_{N}$ scale for $N^{*}$, $\alpha K_{N}$ scale for $H^{*}$. \[
  t^{*} = \frac{1}{q} t, \quad  N^{*} =  K_{N} N , \quad  H^{*} = \alpha  K_{N} H  
  \]  
  Using \eqref{eq:a1}    \[
    \begin{split}
  qN'   & = rK_{N} \left( 1- \frac{K_{N} N}{K_{N}}  \right) - \frac{\alpha _{2} K_{N}K_{r} NH}{r} - u_{N} F_{N} \frac{K_{N} N}{ r}    \\
  \implies  \left( \frac{q}{r} \right) N'  &=  N \left( 1 - N \right) - \left( \frac{\alpha \alpha _{2} K_{N}}{r}  \right) NH - \left( u_{N} F_{N} \frac{1}{r}  \right)N \\
    \end{split} 
  \] 
  Lets put \[
  \varepsilon  = \frac{q}{r} , \quad  \gamma  = \frac{\alpha \alpha _{2} K_{N}}{r}, \quad  f_{N} = \frac{u_{N} F_{N}}{r}   
  \] 
  We then end up with
  \[
    \varepsilon  N'  = N\left( 1- N \right) - \gamma  HN - f_{N} N
  .
  \] 
  For equation $\eqref{eq:a2} $ \[
  \alpha K_{N} q H'  = \alpha  K_{N} q H \left( 1 - \frac{\alpha  K_{N} H }{ \alpha  K_{N} N}  \right) - u_{H} F_{H} K_{N} \alpha H
  \] 

  Let $f_{H} = \frac{u_{N} F_{N}}{q} $ be \[
  H'  = H\left( 1 - \frac{H}{N} \right) - f_{H} H
  \] 
  such that \[
    \begin{split}
  \varepsilon N'  &=  N\left( 1 - N \right) - \gamma HN - f_{N} N \\
  H'  &=  H\left( 1 - \frac{H}{N} \right) - f_{H} H \\
    \end{split} 
  \] 
  \textbf{Stability Properties.}  \[
  \mathbf{f}\left( N,H \right) = \begin{bmatrix} 
  N\left(  1 - N  \right)  - \gamma  HN f_{N} N \\
  H\left( 1 - \frac{H}{N} \right) - f_{H} H
  \end{bmatrix} 
   = \begin{bmatrix} 
   0 \\
   0
   \end{bmatrix} 
  , \quad  N\neq0  
  \] 
  Fromt he first equation \[
    \begin{split}
  N\left( 1- N - \gamma  H - f_{N} \right)  &=  0 \\
  H\left( 1- \frac{H}{N} - f_{H} \right) &=  0 \\
    \end{split} 
  \] 
  If we llok at $H = 0$ it implies that \[
    \begin{split}
  1 - N - f_{N} &=  0 \\
 N  & = 1 - f_{N}
    \end{split} 
  \] 
  We have now one eq. point such that \[
    \left( 1 - f_{N}, 0 \right) = \left( N_{e}, H_{e} \right)
  \] 
  which can be considered as an unstable point if we introduce perturb by introducing , let say, one whale.  For $H\neq 0$ \[
  \begin{split}
    1 - N - \gamma  H f_{N} &=  0 \\
    1 - \frac{H}{N}  f_{H} &=  0 \\
  \end{split} 
  \] 
  \begin{align}
      \label{eq:a4}
      1- N - \gamma  H f_{N} &=  0  \\
      \label{eq:a5}
      N - H - f_{H} N = 0
  .\end{align}

  Insert \eqref{eq:a5}  we get $ H = \left( 1 - f _{H}  \right) N$. And inserting \eqref{eq:a4}  \[
    \begin{split}
  1 - N - \gamma  \left( 1 - f_{H} \right) N - f_{N} &=  0 \\
  \implies  N\left(  1 + \gamma  \left(  1 - f_{H} \right) \right)  & = 1- f_{N} \\
  \implies  N = \frac{1 - f_{N}}{ 1 + \gamma  \left(  1 - f_{H} \right)} 
    \end{split} 
  \] 
  In summary is the eq. point \[
  \left( N_{e}, H _{e} \right) = \left(  \frac{1 - f_{N}}{ 1 + \gamma  \left( 1 - f_{H} \right)}  \right) , \frac{\left( 1 - f_{H} \right) \left( 1 - f_{N} \right)}{ 1 + \gamma  \left( 1 - f_{H} \right)} 
  \] 

  Rate of krill being caught is \[
  N_{e} f_{N} = \frac{f_{N} \left( 1 - f_{N} \right)}{ 1 + \gamma  \left( 1 - f_{H} \right) }  = P_{N}
  \] 
  $P_{N}$ has maximum when $f_{N} = \frac{1}{2}$.  \[
  \mathbf{f}\left( N,H \right) =  
  \begin{bmatrix} 
  N\left( 1 - N \right) - \gamma  NH - f_{N}N \\
  H\left(  1 - \frac{H}{N} \right) - f_{H}H 
  \end{bmatrix} 
  \] 

  The jacobian matrix \[
  J\left( N, H \right) = \begin{bmatrix} 
  1 - 2N - \gamma   &  -\gamma N \\
  \frac{H^2}{N^2}   &  1- \frac{2H}{N} - f_{H}
  \end{bmatrix} 
  \] 
  The first eq. point we obtained $\left( N_{e}, H_{e} \right) = \left( 1 - f_{N}, 0 \right)$ \[
  J\left( N_{e}, H_{e} \right) = \begin{bmatrix} 
  \overbrace{1 - f_{N} - 2 \left( 1 - f_{N} \right)}^{- \left( 1 - f_{N} \right)}   &  - \gamma \left( 1 - f_{N} \right) \\
  0  & 1 - f_{H}
  \end{bmatrix} 
  \] 
  Observe that \[
    \begin{split}
  det\left( J \right) &=  - \left( 1 - f_{N} \right) \left( 1 - f_{H} \right) < 0 \\
  \implies  \left( 1 - f_{N}, 0 \right)  & , \text{ is in fact unstable.}
    \end{split} 
  \] 
  However,
  \[
   \begin{split}
    \begin{bmatrix} 
    \widetilde{N} \\
    \widetilde{H}
    \end{bmatrix} 
    '  & = J\left( N_{e}, 0 \right) \begin{bmatrix} 
    \widetilde{N} \\
    \widetilde{H}
    \end{bmatrix}  \\
    \begin{bmatrix} 
    \widetilde{N}\left( t \right) \\
    \widetilde{H}\left( t \right)
    \end{bmatrix}  
    &= c_{1}
    \begin{bmatrix} 
    1 \\
    0
    \end{bmatrix} 
    e^{- \left( 1 - f_{N} \right)t} \
   + c_{2} \begin{bmatrix} 
   1  \\
   \frac{- \left( 1 - f_{N} \right) - \left( 1 - f_{H} \right)}{\gamma \left( 1 - f_{H} \right)} e^{\left( 1 - f_{H} \right)t}
   \end{bmatrix} 
   \end{split}  
  \] 


  In the same manner we find \[
  J\left( N_{e}, H_{e} \right) 
  \] 
  for \[
  \left( N_{e} , H_{e} \right) = \left( \frac{1-f_{N}}{ 1 + \gamma  \left( 1 - f_{H} \right)}, \frac{\left( 1 - f_{N} \right) \left( 1 - f_{N} \right)}{ 1 + \gamma \left( 1 - f_{H} \right) } 1  \right)
  \] 
  And discover that $det\left( J \right) > 0$ and $trace\left( J \right) < 0$ 
\begin{itemize}
  \item Both eigenvalues have negative part
  \item  $\left( N_{e}, H_{e} \right)$ is asymptotic stable
\end{itemize}
Rate of whale catching \[
H_{e}f_{H} = P_{H}
\] 
$p_{N} $ and $p_{H}$ is the return value of each $N$ and $H$ \[
f\left( f_{N} , f_{H}  \right) = p_{N} P_{N} + p_{H}  P_{H}
\] 
  
\subsection{Conservation Laws}%
\label{sub:conservation_laws}

In general we can formulate everything in $n$- dimensions. But we stick to $n = 3 $.  Three main players .
 \begin{itemize}
  \item Conserved entity: $\phi $
  \item  Flux density: $\mathbf{J}$
  \item Sinks/sources : $Q$
\end{itemize}
  
  

  
  




\section{References}%
\label{sec:references}


  
\bibliographystyle{plain}
\bibliography{references}
\end{document}
