
\documentclass{article}
\usepackage[utf8]{inputenc}

\title{Partial Differential Equations}
\author{isakhammer }
\date{2020}

\usepackage{natbib}
\usepackage{graphicx}
\usepackage{amsmath}
\usepackage{amsthm}
\usepackage{amsfonts}
\usepackage{mathtools}
\usepackage{enumerate}
\usepackage{todonotes}


\usepackage{hyperref} 
\hypersetup{
  colorlinks=true, %set true if you want colored links
  linktoc=all,     %set to all if you want both sections and subsections linked
  linkcolor=blue,  %choose some color if you want links to stand out
} 
\hypersetup{linktocpage}


% inscape-figures
\usepackage{import}
\usepackage{pdfpages}
\usepackage{transparent}
\usepackage{xcolor}
\newcommand{\incfig}[2][1]{%
\def\svgwidth{#1\columnwidth}
\import{./figures/}{#2.pdf_tex} } \pdfsuppresswarningpagegroup=1

% Box environment
\usepackage{tcolorbox}
\usepackage{mdframed}
\newmdtheoremenv{definition}{Definition}[section]
\newmdtheoremenv{theorem}{Theorem}[section]
\newmdtheoremenv{lemma}{Lemma}[section]

\theoremstyle{remark}
\newtheorem*{remark}{Remark}
\newtheorem{example}{Example}


\begin{document}
\maketitle
\tableofcontents
\newpage

\newpage
\section{Lecture 1}%
\label{sec:lecture_1}
 \subsection{Praktiske Ting}%
 \label{sub:book}
 
 \begin{itemize}
   \item
 Borthwick, Introduction to Partial Differential Equations - Springer Link
 \item Ingen obligatoriske øvinger.

 \end{itemize}

 \subsection{Bevaring av Konserveringslov}%
 \label{sub:bevaring_av_konserveringslov}
 
 \begin{itemize}
   \item
\textbf{Konserveringslov}  \[
  \begin{split}
    \frac{\partial u}{\partial t} + \frac{\partial }{\partial x} f\left( u \right) &= 0\\
    u\left( t,x \right)  &  \quad \text{ukjent}  \\
    f \quad  & \text{er oppgit} 
  \end{split}
\] 
\item
\textbf{Hamilton Jacobi} \[
\frac{\partial u}{\partial t}  +  f \left( \frac{\partial u}{\partial x} \right)  = 0
\] 
\item \textbf{Bølgelingingen}  \[
\frac{\partial ^{2} u }{\partial t^{2}} - c^2 \frac{\partial ^2u}{\partial x^2}  = 0
\] 
\item \textbf{Varmeligningen} \[
\frac{\partial u}{\partial t} - \mathbb{H} \frac{\partial ^2}{\partial x^2}  = f\left( t,x \right) 
\] 
\item  \textbf{Possion lingingen} \[
    \begin{split}
- \frac{\partial ^2}{\partial x^2}  - \frac{\partial ^2}{\partial y^2}   & = f\left( x,y \right) \\
-\left( \frac{\partial ^2}{\partial x^2}  + \frac{\partial ^2}{\partial y^2}  \right)u &= f 
    \end{split}
\] 
\item \textbf{Korteweg - de vries} \[
\frac{\partial u}{\partial t}  + \frac{\partial ^{3}}{\partial x^{3}}  - 6u \frac{\partial u}{\partial x} = 0
\]  
\item \textbf{Navier Stokes} \[
\rho \left( \frac{\partial \mathbf{u}}{\partial t}  + \mathbf{u} \nabla \mathbf{u} \right) = - \nabla p + \mu \Delta \mathbf{u} + \rho \mathbf{g} 
\] 

 \end{itemize}

 \subsection{Notation}%
 \label{sub:notation}

 En generell pde kan beskrives som \[
 F\left( t, x, y, \ldots, u_{t}, u_{t} , \ldots , u_{y} , u_{xy}, \ldots  \right) = 0
 \] 

 og blir beskrevet som en partiell diffligning. \[
 u_{t} + f'\left( u \right) u_{x} = 0
 \] 

 En \textbf{klassisk løsning }  til en PDE av order $m$ er en $C^{m}$- funksjon som i ligningen. 

 \begin{example}
   Bølgeligningen $u_{tt} - c^2 u_{x x} = 0$ har en klassisk løsning $u\left( t,u \right) = f\left( x \pm ct \right)$ med $f \in C^2 $  der $\left[ f, f', f''  \right]$ er kontinuerlig. 
   \begin{align*}
     u_{t} &=  \pm c f'\left( x \pm ct \right) \\
     u_{t t} &=  f''\left( x \pm ct \right) \\
     u_{x x} &=  c^2 f''\left( x \pm ct \right) \\
     u_{t t} &=  f'' \left( x \pm ct \right) \\
     u_{tt} &=  c^2 u_{x x}     
   .\end{align*}
   Der av løsningen \[
   u\left( t,x \right) = f_{1}\left( x + ct \right) + f_{2}\left( x -ct \right)
   \] 
 \end{example}


 \subsection{PDE-Teori}%
 \label{sub:pde_teori}
 
 \begin{itemize}
   \item  Fine løsninger
   \item Analyse  
     \begin{itemize}
       \item Velstilt.
         \begin{itemize}
           \item Løsninger eksisteres
           \item De er entydige.
           \item De avhenger kontiuerlig av data. 
         \end{itemize}
       \item General opp oppførsel
         \[
           \begin{split}
         u_{t} - u_{x x}  &  \quad  t > 0 \\
         u\left( 0,x \right) &= u_{0} \left( x \right)
           \end{split}
         \] 
       \item Tilnærmede løsninger (numerikk)
     \end{itemize}

 \end{itemize}
 

 \subsection{Kap 3, Transportligningen}%
 \label{sub:kap_3_transportligningen}

 \[
 u_{t} + v u_{x} =0 \quad  \text{der } \quad  v\left( t,x \right) = 0 \quad  , u\left( 0,x \right) = u_{0} \left( x \right)   
 \] 
 Som kan omskrives til \[
   \begin{split}
 \frac{d u\left( t,x\left( t \right) \right)}{d t} &= u_{t} + \dot{x} u_{x} = 0  \\
 \text{dersom} \quad  & \dot{x} = v\left( t,x \right) \quad \text{har entydig løsning gitt}\quad x\left( 0 \right) = x_{0} \\
 \text{forutsatt} \quad   &  v \in C^{1} 
   \end{split}
 \] 
 derfor er $u\left( t, x\left( t \right) \right) = u\left( 0, x\left( 0 \right) \right) = u_{0} \left( x_{0} \right)$. La oss definere $X\left( t, x_{0} \right) = x\left( t \right)$ \[
 \text{dersom} \quad  x \quad \text{løser} \quad \begin{cases}
   \dot{x} &= v\left( t,x \right) \\
   x\left( 0 \right) &= x_{0}  
 \end{cases}  
 \] 

 La $u\left( t, X\left( t,x_{0} \right) \right) = u_{0} \left( x_{0} \right)$ . For a finne $u\left( t,x \right)$, løs $\left( X\left( t,x_{0} \right) \right)$ med hennold 
 pa $x_{0} $ og sett inn  
 
 \begin{example}
   \[
   u_{t} + \left( at +b \right)u_{x} = 0  \quad  a,b \quad \text{er kont}  
   \]
   Da er ligningen $\dot{x} + at = b$ slik at \[
   \begin{split}
     x &=  \frac{1}{2} at^2  + bt  +c\\
     x_{0} &=  x - \frac{1}{2  } at^2 - bt \\
     u\left( t,x \right) &= u_{0} \left( x-\frac{1}{2} at^2 -bt \right)\\
     u_{t} &= -\left( at +b \right)
   \end{split}
   \] 
 \end{example}



 
 


\bibliographystyle{plain}
\bibliography{references}
\end{document}

