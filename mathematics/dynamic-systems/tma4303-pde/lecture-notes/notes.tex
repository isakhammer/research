
\documentclass{article}
\usepackage[utf8]{inputenc}

\title{Partial Differential Equations}
\author{isakhammer }
\date{2020}

\usepackage{natbib}
\usepackage{graphicx}
\usepackage{amsmath}
\usepackage{amsthm}
\usepackage{amsfonts}
\usepackage{mathtools}
\usepackage{enumerate}
\usepackage{todonotes}


\usepackage{hyperref} 
\hypersetup{
  colorlinks=true, %set true if you want colored links
  linktoc=all,     %set to all if you want both sections and subsections linked
  linkcolor=blue,  %choose some color if you want links to stand out
} 
\hypersetup{linktocpage}


% inscape-figures
\usepackage{import}
\usepackage{pdfpages}
\usepackage{transparent}
\usepackage{xcolor}
\newcommand{\incfig}[2][1]{%
\def\svgwidth{#1\columnwidth}
\import{./figures/}{#2.pdf_tex} } \pdfsuppresswarningpagegroup=1

% Box environment
\usepackage{tcolorbox}
\usepackage{mdframed}
\newmdtheoremenv{definition}{Definition}[section]
\newmdtheoremenv{theorem}{Theorem}[section]
\newmdtheoremenv{lemma}{Lemma}[section]

\theoremstyle{remark}
\newtheorem*{remark}{Remark}
\newtheorem{example}{Example}


\begin{document}
\maketitle
\tableofcontents
\newpage

\newpage
\section{Lecture 1}%
\label{sec:lecture_1}
 \subsection{Praktiske Ting}%
 \label{sub:book}
 
 \begin{itemize}
   \item
 Borthwick, Introduction to Partial Differential Equations - Springer Link
 \item Ingen obligatoriske øvinger.

 \end{itemize}

 \subsection{Bevaring av Konserveringslov}%
 \label{sub:bevaring_av_konserveringslov}
 
 \begin{itemize}
   \item
\textbf{Konserveringslov}  \[
  \begin{split}
    \frac{\partial u}{\partial t} + \frac{\partial }{\partial x} f\left( u \right) &= 0\\
    u\left( t,x \right)  &  \quad \text{ukjent}  \\
    f \quad  & \text{er oppgit} 
  \end{split}
\] 
\item
\textbf{Hamilton Jacobi} \[
\frac{\partial u}{\partial t}  +  f \left( \frac{\partial u}{\partial x} \right)  = 0
\] 
\item \textbf{Bølgelingingen}  \[
\frac{\partial ^{2} u }{\partial t^{2}} - c^2 \frac{\partial ^2u}{\partial x^2}  = 0
\] 
\item \textbf{Varmeligningen} \[
\frac{\partial u}{\partial t} - \mathbb{H} \frac{\partial ^2}{\partial x^2}  = f\left( t,x \right) 
\] 
\item  \textbf{Possion lingingen} \[
    \begin{split}
- \frac{\partial ^2}{\partial x^2}  - \frac{\partial ^2}{\partial y^2}   & = f\left( x,y \right) \\
-\left( \frac{\partial ^2}{\partial x^2}  + \frac{\partial ^2}{\partial y^2}  \right)u &= f 
    \end{split}
\] 
\item \textbf{Korteweg - de vries} \[
\frac{\partial u}{\partial t}  + \frac{\partial ^{3}}{\partial x^{3}}  - 6u \frac{\partial u}{\partial x} = 0
\]  
\item \textbf{Navier Stokes} \[
\rho \left( \frac{\partial \mathbf{u}}{\partial t}  + \mathbf{u} \nabla \mathbf{u} \right) = - \nabla p + \mu \Delta \mathbf{u} + \rho \mathbf{g} 
\] 

 \end{itemize}

 \subsection{Notation}%
 \label{sub:notation}

 En generell pde kan beskrives som \[
 F\left( t, x, y, \ldots, u_{t}, u_{t} , \ldots , u_{y} , u_{xy}, \ldots  \right) = 0
 \] 

 og blir beskrevet som en partiell diffligning. \[
 u_{t} + f'\left( u \right) u_{x} = 0
 \] 

 En \textbf{klassisk løsning }  til en PDE av order $m$ er en $C^{m}$- funksjon som i ligningen. 

 \begin{example}
   Bølgeligningen $u_{tt} - c^2 u_{x x} = 0$ har en klassisk løsning $u\left( t,u \right) = f\left( x \pm ct \right)$ med $f \in C^2 $  der $\left[ f, f', f''  \right]$ er kontinuerlig. 
   \begin{align*}
     u_{t} &=  \pm c f'\left( x \pm ct \right) \\
     u_{t t} &=  f''\left( x \pm ct \right) \\
     u_{x x} &=  c^2 f''\left( x \pm ct \right) \\
     u_{t t} &=  f'' \left( x \pm ct \right) \\
     u_{tt} &=  c^2 u_{x x}     
   .\end{align*}
   Der av løsningen \[
   u\left( t,x \right) = f_{1}\left( x + ct \right) + f_{2}\left( x -ct \right)
   \] 
 \end{example}


 \subsection{PDE-Teori}%
 \label{sub:pde_teori}
 
 \begin{itemize}
   \item  Fine løsninger
   \item Analyse  
     \begin{itemize}
       \item Velstilt.
         \begin{itemize}
           \item Løsninger eksisteres
           \item De er entydige.
           \item De avhenger kontiuerlig av data. 
         \end{itemize}
       \item General opp oppførsel
         \[
           \begin{split}
         u_{t} - u_{x x}  &  \quad  t > 0 \\
         u\left( 0,x \right) &= u_{0} \left( x \right)
           \end{split}
         \] 
       \item Tilnærmede løsninger (numerikk)
     \end{itemize}

 \end{itemize}
 

 \subsection{Kap 3, Transportligningen}%
 \label{sub:kap_3_transportligningen}

 \[
 u_{t} + v u_{x} =0 \quad  \text{der } \quad  v\left( t,x \right) = 0 \quad  , u\left( 0,x \right) = u_{0} \left( x \right)   
 \] 
 Som kan omskrives til \[
   \begin{split}
 \frac{d u\left( t,x\left( t \right) \right)}{d t} &= u_{t} + \dot{x} u_{x} = 0  \\
 \text{dersom} \quad  & \dot{x} = v\left( t,x \right) \quad \text{har entydig løsning gitt}\quad x\left( 0 \right) = x_{0} \\
 \text{forutsatt} \quad   &  v \in C^{1} 
   \end{split}
 \] 
 derfor er $u\left( t, x\left( t \right) \right) = u\left( 0, x\left( 0 \right) \right) = u_{0} \left( x_{0} \right)$. La oss definere $X\left( t, x_{0} \right) = x\left( t \right)$ \[
 \text{dersom} \quad  x \quad \text{løser} \quad \begin{cases}
   \dot{x} &= v\left( t,x \right) \\
   x\left( 0 \right) &= x_{0}  
 \end{cases}  
 \] 

 La $u\left( t, X\left( t,x_{0} \right) \right) = u_{0} \left( x_{0} \right)$ . For a finne $u\left( t,x \right)$, løs $\left( X\left( t,x_{0} \right) \right)$ med hennold 
 pa $x_{0} $ og sett inn  
 
 \begin{example}
   \[
   u_{t} + \left( at +b \right)u_{x} = 0  \quad  a,b \quad \text{er kont}  
   \]
   Da er ligningen $\dot{x} + at = b$ slik at \[
   \begin{split}
     x &=  \frac{1}{2} at^2  + bt  +c\\
     x_{0} &=  x - \frac{1}{2  } at^2 - bt \\
     u\left( t,x \right) &= u_{0} \left( x-\frac{1}{2} at^2 -bt \right)\\
     u_{t} &= -\left( at +b \right)
   \end{split}
   \] 
 \end{example}



\newpage
\section{Lecture 2}%
\label{sec:lecture_2}

\subsection{Høy Dimensional Kalkulus}%
\label{sub:introduction}

Definition av funksjon
\[
\begin{split}
  f: & \mathbb{R} ^{2} \implies \mathbb{R} \\
  f \in & C^{1}\left( \mathbb{R} ^{n}; \mathbb{R} \mathbb{R}  \right) \quad  \text{betyr kontinuerlig}    \\ 
\end{split} 
\]  

For gradienter \[
\begin{split}
\nabla f &=  \partial _{x_{1}} f_{1}, \ldots, \partial _{x_{n}} f & 
&= \begin{pmatrix}
  \partial _{x_{1}} f \\
  \vdots \\
  \partial _{x_{n}} f
\end{pmatrix} 
\end{split} 
\] 

For vectorfelt \[
  \begin{split}
    \mathbf{F}:  & \mathbb{R} ^{n} \to \mathbb{R} ^{n} \\
    \mathbf{F} \in  &  C_{1} \left( \mathbb{R} ^{n} , \mathbb{R} ^{n} \right) \\
    \nabla \mathbf{F} &= \partial F_{1} + \ldots + \partial  F_{n}
  \end{split} 
\] 

\begin{definition}
  Divergensteoremet \[
  \int_{\omega }^{}   \nabla  \mathbf{F} d^{n} x = \int_{\partial \omega }^{}  \mathbf{F} \cdot \mathbf{v} dS 
  \] 
\end{definition}

% \begin{figure}[ht]
%     \centering
%     \incfig{divfigure}
%     \caption{divfigure}
%     \label{fig:divfigure}
% \end{figure}


\begin{proof}
  \[
  \int_{\mathbf{\sigma}}^{}  f dS = \int_{w}^{} f\left( \mathbf{\sigma}\left( \mathbf{y} \right) \right) det \left[ \frac{\partial \mathbf{v}}{\partial y_{1}}   \ldots , \frac{\partial \mathbf{\sigma}}{\partial y_{n-1}} , \mathbf{v} \right] dy_{1} \ldots dy_{n-1}
  \] 
\end{proof}, 

\begin{definition}
  Området er definert som åpen delmengde av $\mathbb{R} ^{n} $ sammenhengende. La $x_{0}$ være sentrum av sirkelen med radius $R$, da er \[
    \begin{split}
  B\left( \mathbf{x_{0}}; R \right)  & =  B_{R} \left( \mathbf{x_{0}} \right)\\
  &= \left\{ \mathbf{x_{0}} \in  \mathbb{R} ^{n}  \mid  \|\mathbf{x} - \mathbf{x_{0}}\| < R \right\} \\
  \\
    & \begin{cases}
       & \text{En omegn om } \mathbf{x_{0}} \text{er en mengde som inneholder  } B\left( \mathbf{x_{0}}; R \right) \text{ for en }  R > 0\\
       &\text{En åpent mengde som er en omegn }  \text{om alle sine punkter.}
  \end{cases}
    \end{split} 
  \] 
\end{definition}


\subsection{Konserveringslov}%
\label{sub:konserveringslov}

$u\left( t , \mathbf{x}\right)$ er en tetthet, det vil si \[
\int_{\omega }^{} u\left( t, \mathbf{x} \right) d^{n} \mathbf{x} .
\] 
Menger av $u$ inneholdt i $\omega $ vet tid $t^{n}$. \[
\frac{d }{d t} \int_{\omega }^{} u\left( t, \mathbf{x} \right) d ^{n} \mathbf{x} = \int_{\omega }^{} \partial _{t} u\left( t, \mathbf{x} \right) d^{n}x     
\] 

Derivasjon under integraltegniet. Ok dersom
\begin{itemize}
  \item $\omega $ er begrenset.
  \item $\partial _{t} u  \in  C \left( \overline{\omega } \right) $ der $\overline{\omega } = \omega \cup \partial \omega $
\end{itemize}

\[
  \begin{split}
    &  \frac{\int_{\Omega }^{} u\left( t + \Delta  t, \mathbf{x} \right) d^{n} \mathbf{x}}{\Delta t}  - \int_{\Omega }^{} u_{t} \left( t, \mathbf{y} \right) d^{n}    \mathbf{x}  \\
    &=  \int_{\Omega }^{} \left( \frac{u\left( t + \Delta t , \mathbf{x}\right) - u_{t}\left(t, \mathbf{x}  \right) }{\Delta t}   \right) d^{n} \mathbf{x}  \\
     &=   \int_{\Omega }^{} u_{t} \left( \theta \left( \mathbf{x}, \Delta t \right), \mathbf{x} \right) - u_{t}\left( t, \mathbf{x} \right) d^{n} \mathbf{x}  \\
      &  \theta \text{ mellom } t \text{ og } t + \Delta t \\
       &  \le \int_{-\infty}^{\infty}  
  \end{split} 
\] 

\todo[inline]{ Bruker alt for lang tid på å skrive dette. Må øve på ø-operator, fjerne default values på dint og begynne å skrive tegninger for å ta bilde av på mobil(?)}

\subsection{Flukstetthet}%
\label{sub:flukstetthet}

Fluks gjennom $S$ pr tidsenhet \[
\int_{S}^{} \mathbf{q} \left( t, \mathbf{x} \right) \mathbf{v} dS  
\] 

Der $\mathbf{v}$ er en enhetsnormal.
\par
\textbf{Bevaringslov.} \[
  \begin{split}
\frac{d }{d t} \int_{\Omega }^{}  u d ^{n} \mathbf{x} + \int_{\partial \Omega }^{} \mathbf{q} \cdot  \mathbf{v} dS  & = 0   .\\
\int_{\Omega }^{} u_{t} d^{n} \mathbf{x} + \int_{\Omega }^{} d ^{n} x  &= 0  \\
\int_{\Omega }^{} \left( u_{t} + \nabla \mathbf{q} \right) d^{n} \mathbf{x} = 0  
  \end{split} 
\]  
For et området $\Omega $ (begrenset ,  $C_{1} $ rand.)

\begin{definition}
  Generell bevaringslov på differensiell form \[
  u_{t} + \nabla \mathbf{q} = 0
  \] 
\end{definition}

\subsubsection{Spesialtilfelle}%
\label{ssub:spesialtilfelle}

\[
\begin{split}
\mathbf{q}  & = u \mathbf{v}\left( t, \mathbf{x} \right) \\
\nabla \cdot \mathbf{q}  &  = \nabla u \cdot \mathbf{v} + v \nabla \mathbf{v}\\
q_{1}  & = u v_{j} \\
\frac{\partial q_{j}}{\partial x_{j}}  &= u_{x_{j}} v_{j} + u \cdot v_{x_{j}} \\
u_{t} + \mathbf{v} \nabla u &=  -u \nabla \mathbf{v} \\
\nabla \mathbf{v}  & = 0 \text{ som gir } \\
u_{t} + \mathbf{v} \nabla u = 0 \\
\end{split} 
\]  

\begin{definition}
  Transportlingingen. \[
  u_{t} + v\cdot \nabla u = 0
  \] 
  For $n = 1$ er \[
  u_{t} + v u_{x} = 0
  \] 
\end{definition}

Karakteristisk kunne $\mathbf{x}\left( t \right)$ løse \[
\dot{\mathbf{x}} = \mathbf{v} \left( t, \mathbf{x} \right)
\] 

eller
\[
  \begin{split}
\dot{\mathbf{u}} \frac{Du}{Dt}   & = \frac{d}{dt}  u\left( t, \mathbf{x}\left( t \right) \right) = u_{t} + \dot{\mathbf{x}} u\\
&= u_{t} + \mathbf{v}\nabla u 
  \end{split} 
\] 

Mer generell variant kan skrives som \[
u_{t} + \mathbf{v} \nabla u = w\left( t, \mathbf{x}, u \right)
\]  

Nå blir \[
\frac{Du}{Dt}  = w\left( t,x\left( t \right), u \right)
\] 
Anta initialverdi \[
u\left( 0, x\left( 0 \right) \right) = u_{0} \left( x_{0} \right)
\] 
\begin{figure}[ht]
    \centering
    \incfig{figg}
    \label{fig:figg}
\end{figure}

Her er $x_{t} \left( t \right) = \mathbf{v}\left( t, \mathbf{x} \right)$ og $x\left( 0 \right) = x_{0}$. \par Løs \[
\begin{split}
  \dot{\hat{u}}   & = w\left( t, x\left( t \right), n \right) \\
  \hat{u} \left( 0 \right)  & = u_{0} \left( \mathbf{x}_{0} \right) \\
  \text{ og finne} \quad   & u\left( t, \mathbf{x}\left( t \right) \right) = \hat{u} \left( t \right) \\
  \mathbf{x}    \to \mathbf{X} &\left( t, \mathbf{x}_{0} \right) \quad \text{er kontinuerlig og one-to-one} 
\end{split} 
\] 

\subsection{Kvasilineære ligninger}%
\label{sub:kvasilineaere_ligninger}

\[
u_{t} + a\left( u \right) u_{x} = 0
\] 
Eksempel \[
  \begin{split}
u_{t} + f\left( u \right)_{x}   & = 0  \quad \implies  \quad  u_{t}  + f'\left( u  \right)  u_{x} = 0
  \end{split} 
\] 

\begin{figure}[ht]
    \centering
    \incfig{bil}
    \caption{Bil}
    \label{fig:bil}
\end{figure}

Kan. Ligning $\dot{x} = a\left( u\left( t,x \right) \right)$ \[
\frac{Du}{Dt}  = u_{t}  +  a\left( u \right) u_{x} = 0 \quad  \implies  \quad  \text{u er kontakt langt kovakttastikk.}
\] 
\[
  \frac{\partial x}{\partial x_{0}}  = 1 + t\left( a \circ u_{0} \right)' \left( x_{0} \right) \ge 1+ t\cdot  min\left( a_{0 } u_{0}  \right)'
  \begin{cases}
x\left( t \right)  & = x_{0} + t a\left( u \right) \\
&= x_{0} + t a\left( u_{0} \left( x_{0} \right) \right) 
  \end{cases}
\] 






\bibliographystyle{plain}
\bibliography{references}
\end{document}

