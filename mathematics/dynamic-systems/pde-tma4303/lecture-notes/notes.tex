
\documentclass{article}
\usepackage[utf8]{inputenc}

\title{Partial Differential Equations}
\author{isakhammer }
\date{2020}

%%%% DEPENDENCIES v1.3 %%%%%%

\usepackage{natbib}
\usepackage{graphicx}
\usepackage{amsmath}
\usepackage{amsthm}
\usepackage{amsfonts}
\usepackage{mathtools}
%\usepackage{enumerate}
\usepackage{enumitem}
\usepackage{todonotes}
\usepackage{esint}
\usepackage{float}


\usepackage{hyperref} 
\hypersetup{
  colorlinks=true, %set true if you want colored links
  linktoc=all,     %set to all if you want both sections and subsections linked
  linkcolor=blue,  %choose some color if you want links to stand out
} 
\hypersetup{linktocpage}


% inscape-figures
\usepackage{import}
\usepackage{pdfpages}
\usepackage{transparent}
\usepackage{xcolor}
\newcommand{\incfig}[2][1]{%
\def\svgwidth{#1\columnwidth}
\import{./figures/}{#2.pdf_tex} } \pdfsuppresswarningpagegroup=1

% Box environment
\usepackage{tcolorbox}
\usepackage{mdframed}
\newmdtheoremenv{definition}{Definition}[section]
\newmdtheoremenv{theorem}{Theorem}[section]
\newmdtheoremenv{lemma}{Lemma}[section]

% \DeclareMathOperator{\span}{span}

\theoremstyle{remark}
\newtheorem*{remark}{Remark}
%\newtheorem{example}{Example}

\newcommand{\newpara}
  {
  \vskip 0.4cm
  }

%%%%%%%%%%%%%%%%%%%%%%%%%%%%%%%%%%%%%%%%%%%%%%%%%%%%%%%%%%%%




\begin{document}
\maketitle
\tableofcontents
\newpage

\newpage
 \section{Lecture 1}%
 \label{sec:lecture_1}
  \subsection{Praktiske Ting}%
  \label{sub:book}

  \begin{itemize}
    \item
  Borthwick, Introduction to Partial Differential Equations - Springer Link
  \item Ingen obligatoriske øvinger.

  \end{itemize}

  \subsection{Bevaring av Konserveringslov}%
  \label{sub:bevaring_av_konserveringslov}

  \begin{itemize}
    \item
 \textbf{Konserveringslov}  \[
   \begin{split}
     \frac{\partial u}{\partial t} + \frac{\partial }{\partial x} f\left( u \right) &= 0\\
     u\left( t,x \right)  &  \quad \text{ukjent}  \\
     f \quad  & \text{er oppgit} 
   \end{split}
 \] 
 \item
 \textbf{Hamilton Jacobi} \[
 \frac{\partial u}{\partial t}  +  f \left( \frac{\partial u}{\partial x} \right)  = 0
 \] 
 \item \textbf{Bølgelingingen}  \[
 \frac{\partial ^{2} u }{\partial t^{2}} - c^2 \frac{\partial ^2u}{\partial x^2}  = 0
 \] 
 \item \textbf{Varmeligningen} \[
 \frac{\partial u}{\partial t} - \mathbb{H} \frac{\partial ^2}{\partial x^2}  = f\left( t,x \right) 
 \] 
 \item  \textbf{Possion lingingen} \[
     \begin{split}
 - \frac{\partial ^2}{\partial x^2}  - \frac{\partial ^2}{\partial y^2}   & = f\left( x,y \right) \\
 -\left( \frac{\partial ^2}{\partial x^2}  + \frac{\partial ^2}{\partial y^2}  \right)u &= f 
     \end{split}
 \] 
 \item \textbf{Korteweg - de vries} \[
 \frac{\partial u}{\partial t}  + \frac{\partial ^{3}}{\partial x^{3}}  - 6u \frac{\partial u}{\partial x} = 0
 \]  
 \item \textbf{Navier Stokes} \[
 \rho \left( \frac{\partial \mathbf{u}}{\partial t}  + \mathbf{u} \nabla \mathbf{u} \right) = - \nabla p + \mu \Delta \mathbf{u} + \rho \mathbf{g} 
 \] 

  \end{itemize}

  \subsection{Notation}%
  \label{sub:notation}

  En generell pde kan beskrives som \[
  F\left( t, x, y, \ldots, u_{t}, u_{t} , \ldots , u_{y} , u_{xy}, \ldots  \right) = 0
  \] 

  og blir beskrevet som en partiell diffligning. \[
  u_{t} + f'\left( u \right) u_{x} = 0
  \] 

  En \textbf{klassisk løsning }  til en PDE av order $m$ er en $C^{m}$- funksjon som i ligningen. 

  \textbf{Example.} 
    Bølgeligningen $u_{tt} - c^2 u_{x x} = 0$ har en klassisk løsning $u\left( t,u \right) = f\left( x \pm ct \right)$ med $f \in C^2 $  der $\left[ f, f', f''  \right]$ er kontinuerlig. 
    \begin{align*}
      u_{t} &=  \pm c f'\left( x \pm ct \right) \\
      u_{t t} &=  f''\left( x \pm ct \right) \\
      u_{x x} &=  c^2 f''\left( x \pm ct \right) \\
      u_{t t} &=  f'' \left( x \pm ct \right) \\
      u_{tt} &=  c^2 u_{x x}     
    .\end{align*}
    Der av løsningen \[
    u\left( t,x \right) = f_{1}\left( x + ct \right) + f_{2}\left( x -ct \right)
    \] 


  \subsection{PDE-Teori}%
  \label{sub:pde_teori}

  \begin{itemize}
    \item  Fine løsninger
    \item Analyse  
      \begin{itemize}
        \item Velstilt.
          \begin{itemize}
            \item Løsninger eksisteres
            \item De er entydige.
            \item De avhenger kontiuerlig av data. 
          \end{itemize}
        \item General opp oppførsel
          \[
            \begin{split}
          u_{t} - u_{x x}  &  \quad  t > 0 \\
          u\left( 0,x \right) &= u_{0} \left( x \right)
            \end{split}
          \] 
        \item Tilnærmede løsninger (numerikk)
      \end{itemize}

  \end{itemize}


  \subsection{Kap 3, Transportligningen}%
  \label{sub:kap_3_transportligningen}

  \[
  u_{t} + v u_{x} =0 \quad  \text{der } \quad  v\left( t,x \right) = 0 \quad  , u\left( 0,x \right) = u_{0} \left( x \right)   
  \] 
  Som kan omskrives til \[
    \begin{split}
  \frac{d u\left( t,x\left( t \right) \right)}{d t} &= u_{t} + \dot{x} u_{x} = 0  \\
  \text{dersom} \quad  & \dot{x} = v\left( t,x \right) \quad \text{har entydig løsning gitt}\quad x\left( 0 \right) = x_{0} \\
  \text{forutsatt} \quad   &  v \in C^{1} 
    \end{split}
  \] 
  derfor er $u\left( t, x\left( t \right) \right) = u\left( 0, x\left( 0 \right) \right) = u_{0} \left( x_{0} \right)$. La oss definere $X\left( t, x_{0} \right) = x\left( t \right)$ \[
  \text{dersom} \quad  x \quad \text{løser} \quad \begin{cases}
    \dot{x} &= v\left( t,x \right) \\
    x\left( 0 \right) &= x_{0}  
  \end{cases}  
  \] 

  La $u\left( t, X\left( t,x_{0} \right) \right) = u_{0} \left( x_{0} \right)$ . For a finne $u\left( t,x \right)$, løs $\left( X\left( t,x_{0} \right) \right)$ med hennold 
  pa $x_{0} $ og sett inn  

  \textbf{Example.} 
    \[
    u_{t} + \left( at +b \right)u_{x} = 0  \quad  a,b \quad \text{er kont}  
    \]
    Da er ligningen $\dot{x} + at = b$ slik at \[
    \begin{split}
      x &=  \frac{1}{2} at^2  + bt  +c\\
      x_{0} &=  x - \frac{1}{2  } at^2 - bt \\
      u\left( t,x \right) &= u_{0} \left( x-\frac{1}{2} at^2 -bt \right)\\
      u_{t} &= -\left( at +b \right)
    \end{split}
    \] 



 \newpage
 % \section{Lecture 2}%
 % \label{sec:lecture_2}

 % \subsection{Høy Dimensional Kalkulus}%
 % \label{sub:introduction}

 % Definition av funksjon
 % \[
 % \begin{split}
 %   f: & \mathbb{R} ^{2} \implies \mathbb{R} \\
 %   f \in & C^{1}\left( \mathbb{R} ^{n}; \mathbb{R} \mathbb{R}  \right) \quad  \text{betyr kontinuerlig}    \\ 
 % \end{split} 
 % \]  

 % For gradienter \[
 % \begin{split}
 % \nabla f &=  \partial _{x_{1}} f_{1}, \ldots, \partial _{x_{n}} f  
 % &= \begin{pmatrix}
 %   \partial _{x_{1}} f \\
 %   \vdots \\
 %   \partial _{x_{n}} f
 % \end{pmatrix} 
 % \end{split} 
 % \] 

 % For vectorfelt \[
 %   \begin{split}
 %     \mathbf{F}:  & \mathbb{R} ^{n} \to \mathbb{R} ^{n} \\
 %     \mathbf{F} \in  &  C_{1} \left( \mathbb{R} ^{n} , \mathbb{R} ^{n} \right) \\
 %     \nabla \mathbf{F} &= \partial F_{1} + \ldots + \partial  F_{n}
 %   \end{split} 
 % \] 

 % \begin{definition}
 %   Divergensteoremet \[
 %   \int_{\omega }^{}   \nabla  \mathbf{F} d^{n} x = \int_{\partial \omega }^{}  \mathbf{F} \cdot \mathbf{v} dS 
 %   \] 
 % \end{definition}

 % % \begin{figure}[ht]
 % %     \centering
 % %     \incfig{divfigure}
 % %     \caption{divfigure}
 % %     \label{fig:divfigure}
 % % \end{figure}


 % \begin{proof}
 %   \[
 %   \int_{\mathbf{\sigma}}^{}  f dS = \int_{w}^{} f\left( \mathbf{\sigma}\left( \mathbf{y} \right) \right) det \left[ \frac{\partial \mathbf{v}}{\partial y_{1}}   \ldots , \frac{\partial \mathbf{\sigma}}{\partial y_{n-1}} , \mathbf{v} \right] dy_{1} \ldots dy_{n-1}
 %   \] 
 % \end{proof}, 

 % \begin{definition}
 %   Området er definert som åpen delmengde av $\mathbb{R} ^{n} $ sammenhengende. La $x_{0}$ være sentrum av sirkelen med radius $R$, da er \[
 %     \begin{split}
 %   B\left( \mathbf{x_{0}}; R \right)  & =  B_{R} \left( \mathbf{x_{0}} \right)\\
 %   &= \left\{ \mathbf{x_{0}} \in  \mathbb{R} ^{n}  \mid  \|\mathbf{x} - \mathbf{x_{0}}\| < R \right\} \\
 %   \\
 %     & \begin{cases}
 %        & \text{En omegn om } \mathbf{x_{0}} \text{er en mengde som inneholder  } B\left( \mathbf{x_{0}}; R \right) \text{ for en }  R > 0\\
%        &\text{En åpent mengde som er en omegn }  \text{om alle sine punkter.}
 %   \end{cases}
 %     \end{split} 
 %   \] 
 % \end{definition}


 % \subsection{Konserveringslov}%
 % \label{sub:konserveringslov}

 % $u\left( t , \mathbf{x}\right)$ er en tetthet, det vil si \[
 % \int_{\omega }^{} u\left( t, \mathbf{x} \right) d^{n} \mathbf{x} .
 % \] 
 % Menger av $u$ inneholdt i $\omega $ vet tid $t^{n}$. \[
 % \frac{d }{d t} \int_{\omega }^{} u\left( t, \mathbf{x} \right) d ^{n} \mathbf{x} = \int_{\omega }^{} \partial _{t} u\left( t, \mathbf{x} \right) d^{n}x     
 % \] 

 % Derivasjon under integraltegniet. Ok dersom
 % \begin{itemize}
 %   \item $\omega $ er begrenset.
 %   \item $\partial _{t} u  \in  C \left( \overline{\omega } \right) $ der $\overline{\omega } = \omega \cup \partial \omega $
 % \end{itemize}

 % \[
 %   \begin{split}
 %     &  \frac{\int_{\Omega }^{} u\left( t + \Delta  t, \mathbf{x} \right) d^{n} \mathbf{x}}{\Delta t}  - \int_{\Omega }^{} u_{t} \left( t, \mathbf{y} \right) d^{n}    \mathbf{x}  \\
 %     &=  \int_{\Omega }^{} \left( \frac{u\left( t + \Delta t , \mathbf{x}\right) - u_{t}\left(t, \mathbf{x}  \right) }{\Delta t}   \right) d^{n} \mathbf{x}  \\
 %      &=   \int_{\Omega }^{} u_{t} \left( \theta \left( \mathbf{x}, \Delta t \right), \mathbf{x} \right) - u_{t}\left( t, \mathbf{x} \right) d^{n} \mathbf{x}  \\
 %       &  \theta \text{ mellom } t \text{ og } t + \Delta t \\
 %        &  \le \int_{-\infty}^{\infty}  
 %   \end{split} 
 % \] 

 % \todo[inline]{ Bruker alt for lang tid på å skrive dette. Må øve på ø-operator, fjerne default values på dint og begynne å skrive tegninger for å ta bilde av på mobil(?)}

 % \subsection{Flukstetthet}%
 % \label{sub:flukstetthet}

 % Fluks gjennom $S$ pr tidsenhet \[
 % \int_{S}^{} \mathbf{q} \left( t, \mathbf{x} \right) \mathbf{v} dS  
 % \] 

 % Der $\mathbf{v}$ er en enhetsnormal.
 % \par
 % \textbf{Bevaringslov.} \[
 %   \begin{split}
 % \frac{d }{d t} \int_{\Omega }^{}  u d ^{n} \mathbf{x} + \int_{\partial \Omega }^{} \mathbf{q} \cdot  \mathbf{v} dS  & = 0   .\\
 % \int_{\Omega }^{} u_{t} d^{n} \mathbf{x} + \int_{\Omega }^{} d ^{n} x  &= 0  \\
 % \int_{\Omega }^{} \left( u_{t} + \nabla \mathbf{q} \right) d^{n} \mathbf{x} = 0  
 %   \end{split} 
 % \]  
 % For et området $\Omega $ (begrenset ,  $C_{1} $ rand.)

 % \begin{definition}
 %   Generell bevaringslov på differensiell form \[
 %   u_{t} + \nabla \mathbf{q} = 0
 %   \] 
 % \end{definition}

% % \subsubsection{Spesialtilfelle}%
% % \label{ssub:spesialtilfelle}

% % \[
% % \begin{split}
% % \mathbf{q}  & = u \mathbf{v}\left( t, \mathbf{x} \right) \\
% % \nabla \cdot \mathbf{q}  &  = \nabla u \cdot \mathbf{v} + v \nabla \mathbf{v}\\
% % q_{1}  & = u v_{j} \\
% % \frac{\partial q_{j}}{\partial x_{j}}  &= u_{x_{j}} v_{j} + u \cdot v_{x_{j}} \\
% % u_{t} + \mathbf{v} \nabla u &=  -u \nabla \mathbf{v} \\
% % \nabla \mathbf{v}  & = 0 \text{ som gir } \\
% % u_{t} + \mathbf{v} \nabla u = 0 \\
% % \end{split} 
% % \]  

% % \begin{definition}
% %   Transportlingingen. \[
% %   u_{t} + v\cdot \nabla u = 0
% %   \] 
% %   For $n = 1$ er \[
% %   u_{t} + v u_{x} = 0
% %   \] 
% % \end{definition}

% % Karakteristisk kunne $\mathbf{x}\left( t \right)$ løse \[
% % \dot{\mathbf{x}} = \mathbf{v} \left( t, \mathbf{x} \right)
% % \] 

% % eller
% % \[
% %   \begin{split}
% % \dot{\mathbf{u}} \frac{Du}{Dt}   & = \frac{d}{dt}  u\left( t, \mathbf{x}\left( t \right) \right) = u_{t} + \dot{\mathbf{x}} u\\
% % &= u_{t} + \mathbf{v}\nabla u 
% %   \end{split} 
% % \] 

% % Mer generell variant kan skrives som \[
% % u_{t} + \mathbf{v} \nabla u = w\left( t, \mathbf{x}, u \right)
% % \]  

% % Nå blir \[
% % \frac{Du}{Dt}  = w\left( t,x\left( t \right), u \right)
% % \] 
% % Anta initialverdi \[
% % u\left( 0, x\left( 0 \right) \right) = u_{0} \left( x_{0} \right)
% % \] 
% % \begin{figure}[ht]
% %     \centering
% %     \incfig{figg}
% %     \label{fig:figg}
% % \end{figure}

% % Her er $x_{t} \left( t \right) = \mathbf{v}\left( t, \mathbf{x} \right)$ og $x\left( 0 \right) = x_{0}$. \par Løs \[
% % \begin{split}
% %   \dot{\hat{u}}   & = w\left( t, x\left( t \right), n \right) \\
% %   \hat{u} \left( 0 \right)  & = u_{0} \left( \mathbf{x}_{0} \right) \\
% %   \text{ og finne} \quad   & u\left( t, \mathbf{x}\left( t \right) \right) = \hat{u} \left( t \right) \\
% %   \mathbf{x}    \to \mathbf{X} &\left( t, \mathbf{x}_{0} \right) \quad \text{er kontinuerlig og one-to-one} 
% % \end{split} 
% % \] 

% % \subsection{Kvasilineære ligninger}%
% % \label{sub:kvasilineaere_ligninger}

% % \[
% % u_{t} + a\left( u \right) u_{x} = 0
% % \] 
% % Eksempel \[
% %   \begin{split}
% % u_{t} + f\left( u \right)_{x}   & = 0  \quad \implies  \quad  u_{t}  + f'\left( u  \right)  u_{x} = 0
% %   \end{split} 
% % \] 

% % \begin{figure}[ht]
% %     \centering
% %     \incfig{bil}
% %     \caption{Bil}
% %     \label{fig:bil}
% % \end{figure}

% % Kan. Ligning $\dot{x} = a\left( u\left( t,x \right) \right)$ \[
% % \frac{Du}{Dt}  = u_{t}  +  a\left( u \right) u_{x} = 0 \quad  \implies  \quad  \text{u er kontakt langt kovakttastikk.}
% % \] 
% % \[
% %   \frac{\partial x}{\partial x_{0}}  = 1 + t\left( a \circ u_{0} \right)' \left( x_{0} \right) \ge 1+ t\cdot  min\left( a_{0 } u_{0}  \right)'
% %   \begin{cases}
% % x\left( t \right)  & = x_{0} + t a\left( u \right) \\
% % &= x_{0} + t a\left( u_{0} \left( x_{0} \right) \right) 
% %   \end{cases}
% % \] 


% \newpage
\section{Lecture 26/08}%
\label{sec:lecture_n}

\subsection{ODE teori}%
\label{sub:ode_teori}

\begin{theorem}
  \[
\begin{split}
  \mathbf{\dot{x}}  & = \mathbf{f} \left( \mathbf{x} \right) \quad  \begin{cases}
    \dot{\mathbf{x}} \left( t \right)    & = \mathbf{f} \left( t, \mathbf{x}\left( t \right) \right)  \\
    \mathbf{x}\left( 0 \right) = \mathbf{x}_{0}
  \end{cases} 
\end{split} 
  \] 
  Anta ligningen er et apoent interval $o \in I$ slik at $\Omega \mathbb{R} ^{n}$ er et område slik at $\mathbf{f} \in  C ^{1} \left( I \times  \Omega  ; \mathbb{R} ^{n}  \right)$
  . Da finnes et største intervall $J$ $o \in J \subseteq I $  moden funcksjon $\mathbf{x} : I \to  \Omega $ som løser problemet. Videre er løsningen gitt. 
\end{theorem}

\begin{proof}
  Ideen er eksistense. Pcard iterasjon \[
  \mathbf{x}_{k+1}\left( t \right) = \mathbf{x}_{0} + \int_{0}^{t}  f\left( \tau , \mathbf{x}_{k} \left( \tau \right)  \right)  d \tau
  \] 
  Entydighet  
  \newpara
  Kontiuerlig /derivering avhenging av dote
  \newpara

\end{proof}

\begin{theorem}
  Anta gitt \[
  \dot{\mathbf{x} } = f\left( t, \mathbf{x} , \hat{c} \right)
  \] 
  der $\mathbf{x}\left( 0 \right) = \mathbf{a}$ 
  \newpara
  Dersom $f$ er $C^{k+1}$ , så vil $\mathbf{f}$ vare en $C^{k}$ funksjon av $\left( t, \mathbf{a}, \hat{c} \right)$
\end{theorem}


\begin{tcolorbox}
  \textbf{Autonome system}  \[
\dot{\mathbf{x}} = f\left( \mathbf{x} \right)
  \] 
  Løsningkurvene blander en en-dimensional foliering av $\Omega $. Har en kurve gjennom hvert punkt med ingen krysninger.
  \newpara
  \textbf{Ikke autonomt system} 
  \[
  \dot{\mathbf{x}} = f\left( t, \mathbf{x} \right)
  \] 
  Ekvivalent 
  \begin{align*}
     \dot{\tau } &= 1 \\  
     \dot{\mathbf{x}}  & = f\left( \tau ,  \right)\\
    \tau \left( 0 \right) &= 0 \\ 
    x^{(n)} &= f\left( t, x, \dot{x} , \ldots, x^{(n-1)} \right) \\
  .\end{align*}
  Hvis vi setter 
  \begin{align*}
    \mathbf{w} = &  \left( x, \dot{x}, \ldots, x^{(n-1)} \right) \\
    \dot{w_{1}} &= w_{2} \\
    \dot{w_{2}} &=  w_{3} \\
    \vdots  & \\
    \dot{w_{n} }  & = f\left( t, \mathbf{w} \right) \\
    \hline \\
    \dot{\mathbf{w}} &= F\left(t, \mathbf{w}  \right)
  .\end{align*}
\end{tcolorbox}

\subsection{Kvasilinær Ligning}%
\label{sub:kvasi_linear_lining}

\[
au_{x} + b u_{y} = c
\] 
$a,b,c$ er alle funksjoner av  $x,y, u\left( x,y \right)$

\begin{tcolorbox}
  \textbf{Grafen} til $u$ er \[
  \left\{ \left( x,y,z \right)  \mid  z = u\left( x,y \right) \right\}
  \] 
  Da antar vi en løsning $u$ , en kurve $\gamma $ i $\left( x,y \right)$ - planet.\[
  \left( x\left( \tau  \right) , y\left( \tau  \right)\right)
  \] 
  Git enn løsning $u\left( x,y \right)$ får vi en kurve i $T$ i $\mathbb{R} ^{3}$ i $\left( x\left( \tau  \right), y\left( \tau  \right), z\left( \tau  \right) \right)$.
  Da ender vi opp med \[
    \begin{split}
  z\left( \tau   \right)  & = u\left( x\left( \tau  \right), y\left( \tau  \right) \right) \\
  \dot{z} &=  \dot{x}u_{x}\left( x,y \right) + \dot{y}u_{y}\left( x,y \right) \\
  \text{anta} \quad   &  \begin{cases}
    \dot{x}\left( \tau  \right) = &  a \left( x,y, u\left( x,y \right) \right) = a\left( x,y,z \right) \\
    \dot{y}\left( \tau  \right) &= b\left( x,y, u\left( x,y \right) \right) = b\left( x,y,z \right) \\
  \end{cases} 
    \end{split} 
  \] 
  da blir \[
    \begin{split}
  \dot{z} &= a u_{x} + b u_{y} = x\left( x,y, u\left( x,y \right) \right) = c\left( x,y,z \right) \\
    \end{split} 
  \] 
  Vi får da \[
  \begin{cases}
    \dot{x} &= a\left( x,y,z \right) \\
    \dot{y} &=  b\left( x,y,z \right) \\
    \dot{z} &= c\left( x,y,z \right) \\
  \end{cases}
  \] 
  Som er kaldt de karakteristiske ligningene til \[
  a u_{x} + b u_{y} = c
  \] 
  Grafen til en løsning $u$ er an union av løsningkurven av de karakteristiske ligningene. (karakteristikk).
  
  \newpara
  Ikke-karakteristiske data for ligningen har formen \[
  u\left( x,y \right) = u_{0} \left( x,y \right) \quad \text{for} \quad  \left( x,y \right) \in \gamma   
  \] 
  der $\gamma $ er en kurve i $\mathbb{R} ^{2} $ , slik at \[
    \left\{ \left( x,y,z \right)  \mid  \left( x,y \right) \in  \gamma \quad \text{og} \quad  z = k\left( x,y  \right)    \right\}
  \] 
  Blir en kurve $\Gamma $ i $\mathbb{R} ^{3} $ ned en tangent som stiller en parabola med \[
  \left( a\left( x,y,z \right), b\left( x,y,z \right) , c\left( x,y,z  \right) \right)
  \] 

\end{tcolorbox}
\begin{figure}[ht]
    \centering
    \incfig{kurveiplan}
    \caption{kurveiplan}
    \label{fig:kurveiplan}
\end{figure}

\begin{figure}[ht]
    \centering
    \incfig{kurveiplan2}
    \caption{kurveiplan2}
    \label{fig:kurveiplan2}
\end{figure}




\newpage
\begin{theorem}
  Problemet har en entydig løsning $u\left( x,y \right)$ for $\left( x,y \right)$ i et åpent område som inneholder $\gamma $ . 
  
\end{theorem}


  \newpara
  Konkret: Anta $\gamma $ er gitt ved \[
  \left( \chi \left( \sigma  \right), \mu \left( \sigma  \right), \left( \dot{\chi }, \dot{\mu } \right) \neq 0,0 \right)
  \] 
  Sett $\chi \left( \sigma  \right) = u_{0} \left( \chi \left( \sigma  \right), \mu \left( \sigma  \right) \right)$ og $\Gamma $ gitt ved $\left( \chi , \mu , \zeta  \right)$.  Da løser vi $k$ ?? med initialdata $\left( \chi \left( \sigma  \right), \mu \left( \sigma  \right), \zeta \left( \sigma  \right) \right)$ . of kall resultatet \[
  \left( x\left( \sigma , \tau  \right), y\left( \sigma ,\tau  \right), z\left( \sigma ,\tau  \right) \right)
  \] 
  
  \newpara
  Vi skal ha \[
  u\left( x\left( \sigma ,\tau  \right), y\left( \sigma ,\tau  \right) \right) = z\left( \sigma ,\tau  \right)
  \] 
  Som er en implisitt løsning. Finn $\left( \sigma , \tau  \right) $ skal være en funksjojn av $\left( x,y \right)$.
\begin{figure}[ht]
    \centering
    \incfig{kdfkdkfd}
    \caption{kdfkdkfd}
    \label{fig:kdfkdkfd}
\end{figure}

  \textbf{Example.} \[
    \begin{split}
  u_{t} + a\left( t,u \right) u_{x}  & = c\left( t,u \right) \\
  u\left( 0,x \right) &=  u_{0}\left( x \right) \\
    \end{split} 
  \] 
  Kontuerlighet slik at \[
  \begin{split}
    \dot{t} &= 1, \quad  t\left( 0 \right) = 0  \quad  t\left( \tau  \right) = \tau   \\
    \dot{x} &= a\left( t,x,z \right) \\
    \dot{z} &= c\left( t,z \right) \\
  \end{split} 
  \] 
  som kan forenkles til \[
  .
  \begin{cases}
    \dot{x} &= a\left( t,x,z \right) \\
    \dot{z} &= c\left( t,z \right) \\
  \end{cases}
  \] 
  Spesialtilfeller er 
  
  \newpara
  Tramert $a = \left( t,x \right)$ \[
    \text{så} \quad  \begin{cases}
      \dot{x} &=  a\left( t,x \right) \\
      x\left( 0 \right) &=  \zeta  \\
    \end{cases} 
  \] 
  Kan løses hver for seg of så løser vi \[
  \begin{cases}
    \dot{z} =  &  c\left( t,x\left( t \right) , z \right) \\
    z\left( 0 \right) &=  u_{0}\left( \zeta  \right) \\
  \end{cases}
  \] 
  Slik at \[
  \frac{Du}{Dt}  = c\left( t,x,y \right) \implies  u\left( 0,\zeta  \right) = u_{0}\left( \zeta  \right)
  \] 

  
  \newpara
  Spesialtilfelle
  \[
  \begin{split}
    u_{t} + a\left( u \right) u_{x} &= 0 \\
    \dot{t}  & = 1, t\left( 0 \right) = 0, t = \tau  \\
    \dot{x} = &  a\left( z \right) \\
    \dot{z} &= 0 \\
  \end{split} 
  \] 
  $x = \zeta + t a\left( u_{0} \left( \zeta  \right) \right)$ slik at $u\left( t,x \right) = u_{0}\left( \zeta  \right)$

  
  \newpara
  Siste spesialtilfelle
  
  \newpara
  \[
  -y u u_{x} + xu u_{y} = 1  , \quad  u \left( x,0 \right) =0 
  \] 
  \[
  \begin{split}
    \dot{x} &=  -yz \quad  x\left( 0 \right) = \sigma   \\
    \dot{y} &= xz , \quad  y\left( 0 \right) = 0  \\
    \dot{z} &= 1 \quad z\left( 0 \right) = \sigma \quad  z\left( \tau  \right) = \sigma + \tau     \\
  \end{split} 
  \] 

  Et av resultatene er \[
  \frac{d }{d \tau }  \left( x^2 + y^2 \right) = 2x \dot{x} + 2y \dot{y} = 0
  \] 
  Slik at $x^2 + y^2 = \text{konstant} = \sigma ^2$ . 
  
  \newpara
  La oss skrive \[
  \begin{split}
    \begin{rcases}
    x &= \sigma  \cos \left( \phi \left( t \right) \right) \\
    y &=  \sigma \sin \left( \phi \left( t \right) \right) \\
    \end{rcases}
    \phi \left( 0 \right) = 0
  \end{split} 
  \] 
  da er \[
  \begin{split}
    \dot{x} &=  -\sigma \sin \left( \phi \left( \tau  \right) \right) \cdot \dot{\phi \left( \tau  \right)} = -y \dot{\phi \left( \tau  \right)} \\
    \dot{y} &= \sigma  \cos \left( \phi \left( \tau  \right) \right) \dot{\phi } = x \dot{\phi }\left( \tau  \right) \\
  \end{split} 
  \] 
  derfor er \[
    \dot{\phi } \left( \tau  \right) =  z = \sigma + \tau \\
    \phi = \sigma \tau  + \frac{1}{2} \tau ^2
  \] 

\begin{figure}[ht]
    \centering
    \incfig{polarfigure}
    \caption{polarfigure}
    \label{fig:polarfigure}
\end{figure}

   \[
   \frac{1}{2} \tau ^2 +  \sigma  \tau  - \phi = 0, \quad \tau = \frac{1}{2} \left( -\sigma \pm\sqrt{\sigma ^2 + 2\phi }  \right) 
   \] 





 \newpage

 \newpage
 \section{Lecture 02/09}%
 \label{sec:lecture_02_09}

 \subsection{Duhamds Prinsipp}%
 \label{sub:duhamds_prinsipp}

 \[
 \begin{split}
   u_{t} + c ^2 u_{xx} &=  f\left( t, x \right) \\
   u\left( 0,x \right) &=  0 \\
   u_{t} \left( 0,x \right) &=  0 \\
 \end{split} 
 \] 

\begin{figure}[ht]
    \centering
    \incfig{dkdkkd}
    \caption{dkdkkd}
    \label{fig:dkdkkd}
\end{figure}

La $\eta _{s} \left( t,x \right)$ løse \[
\begin{split}
  \eta _{s,tt} + c^2 \eta _{s,xx} &=  0 \\
  \eta _{s} \left( s, x \right) &=  0 \\
  \eta _{s,j} \left( s,x \right) &=  f\left( s,x \right) \\
\end{split} 
\] 
D. Alembtert \[
\eta _{s} \left( t,x \right) = \frac{1}{2c} \int_{x- x\left( t-s \right)}^{}  f\left( s, \chi  \right) d\chi  
\] 

Duhamed \[
u\left( t,x \right) = \int_{0}^{t}  \eta _{s}\left( t,x \right) ds 
\] 

\begin{theorem}
  Hvis $f \in  C_{1}$ så vil dette løse probleme.
\end{theorem}

\begin{proof}
  la $u\left( 0,x \right) = 0$. \[
   \begin{split}
  u_{t} &= \overbrace{\eta _{t}\left( t,x \right)}^{= 0  }  + \int_{0}^{t} \eta _{s,t} \left( t,x \right) ds   \\
  u_{t}\left( 0,x \right) &=  0 \\
  u_{tt} &=  \eta _{t,t} \left( t,x \right) + \int_{0}^{t}  \eta _{s,tt} \left( t,x \right) ds  \\
  &= f\left( t,x \right) - c ^2 \int_{0}^{t}  \eta _{s,xx} \left( t,x \right) ds  \\
   &  = f\left( t,x \right) -c ^2 \partial _{xx} \int_{0}^{t}  \eta _{s}\left( t,x \right) ds \\
   &= f\left( t,x \right) - c ^2  + u_{xx} \left( t,x \right)\\ 
   \end{split}  
  \] 
  We trenger $\eta \in  C^2$, $\eta _{s,tt} $ og $\eta _{s,xx}$ kontinuerlig mhp. $s,t,x$ . 
  \[
  \eta _{s,x} &= \frac{1}{2c} \left( f\left( x + c \left( t-s \right) - f\left( x - c\left( t-s \right) \right) \right) \right) \\
  \] 
  $\eta _{s,xx}, \eta _{s,tt}$ er kontinuerlig .  Merk at \[
  u\left( t,x \right) = \int_{0}^{t}  \overbrace{\int_{x - c\left( t -s \right)}^{x + c\left( t-s \right)} f\left( s, \chi  \right) d\chi   }^{ \eta _{s} \left( t,x \right)}  ds
  \] 
\end{proof}

\textbf{Eksempel.} 

\newpara
La $f\left( t,x \right) = \psi \left( t \right) \cdot  h\left( x \right) $ og \[
  \begin{split}
\eta _{s} \left( t,x \right)  & = \frac{1}{2c} \int_{x - x\left( t-s \right)}^{ x + c \left( t-s \right)}  \psi \left( s \right) h\left( \chi  \right) d\chi   \\
 &=   \\frac{\psi \left( s \right)}{ 2c}  \left(  H\left( x + c\left( t-s \right) \right) - H\left( x - c\left( t -s \right) \right) \right) \\
  \end{split} 
\] 
Der $\dot{H} \left( x \right) = h\left( x \right)$. 

\begin{figure}[ht]
    \centering
    \incfig{bølgeomraade}
    \caption{bølgeomraade}
    \label{fig:bølgeomraade}
\end{figure}

\[
  \begin{split}
\eta _{s}\left( t,x \right) &=  0 \quad \text{hvis} \quad  s > t-x   \\
\eta _{s}\left( t,s \right) &=  \frac{\psi \left( s \right)}{2c}  \left( x+t-s \right) \quad  0 < s <  t-x  \\
u\left( t,x \right) &=  \int_{0}^{t-x}  \frac{\cos s}{ 2}  \left( x+t -s \right) ds  \\
&= \frac{1}{2} \left[ -s \sin s - \cos s + \left( t-x \right) \sin s \right] _{s= 0} ^{s= t-x}\\
&=  \frac{1}{2} \left( 1- \cos \left( t- x \right) \right) \\
  \end{split} 
\] 
Her ble det brukt at \[
\int_{}^{} s \cos ^2 s ds = s \sin  s + \cos s 
% \] 
% \begin{figure}[ht]
%     \centering
%     \incfig{cossciz}
%     \caption{cossciz}
%     \label{fig:cossciz}
% \end{figure}

\subsection{Randverdier}%
\label{sub:randverdier}

Vi kan ta bølgeligningen \[
  \begin{rcases}
u_{tt} - c^2 u_{xx}  & = 0 \\
u\left( 0,x \right) &= g\left( x \right) \\
u_{t} \left( 0,x \right) &=  h\left( x \right) \\
u\left( t,0 \right) &=  0 \\
  \end{rcases}
  x > 0
\] 
En forutsetning for en klassisk løsning er $g\left( 0 \right) = 0$ or $h\left( 0 \right) = 0$. Hvis ikke er ikke initalialbetingelsene konsistente.

\begin{tcolorbox}
  \textbf{D`Alembert} 
  \[
  u\left( t,c \right) = \frac{g\left( x + ct \right) - g\left( x - ct \right)}{ 2}  = \frac{1}{2c} \int_{x - ct}^{ x+ ct}  h\left( \zeta  \right) d\zeta  
  \] 
\end{tcolorbox}

\textbf{Løsning.}  Utvid $g,h$ antisymmetrisk om $0$ \[
  \begin{split}
g\left( -x \right) &=  - g\left( x \right) \\
h\left( -x \right)  & = -h\left( x \right)
  \end{split} 
\] 
Konsenvensen av antisymmentrien er at \[
u\left( t, -x \right) = - u\left( t,x \right)
\] 
Alternativt til randbetingelsen, også kjent som\textbf{Neumann betingelse}  $u\left( t, 0 \right) = 0$ er \[
u_{x}\left( t,0 \right) = 0
\] 
Analogien er at man har en trå festet på en ring i et stivt rør som ikke har friksjon. Da må tråen være horisontal. 

Hvis vi ser på \[
  \begin{split}
  &  \left\{ \frac{d }{dx }    \left( g\left( x + ct \right) - g \left( x- ct \right)  \right)  \right\}_{x= 0}    \\
&= \dot{g} \left(  ct \right) + \dot{g}\left( -ct \right) \\
& \vdots  \\
\dot{g}\left( ct \right)  & = - \dot{g} \left( -ct \right) \\
g\left( x \right) &=  g\left( -x \right) \\
g,h  &  \quad \text{ symmetriske} 
  \end{split} 
\] 
\[
\begin{split}
  \frac{d }{d x}  h\left( \zeta  \right) d\zeta \quad  \text{for } \quad  x= 0   
\end{split} 
\] 


Utvid $g$ (og $h$ ) antisymmetrisk om $0, L$  \[
\begin{split}
  g\left( -x \right) &=  -g\left( x \right) \\
  g\left( L- x \right) &=  -g \left( L + x \right) \\
  g\left( L + x \right) &=  g\left( L -x \right) = g\left( x-L \right) \\
  g\left( x + L \right) &=  g\left( x-L \right) \quad  x \to  x+L  \\
  g\left( x + 2L \right) &=  g\left( x \right) \\
\end{split} 
\] 
\begin{tcolorbox}
  \textbf{Notater om kuler} 
  \[
    \begin{split}
  B_{r}\left( \mathbf{a} \right)  & = \left\{ x \in  \mathbb{R} ^{n}  \mid  \|\mathbf{x} - \mathbf{a}\|_{}^{} < r \right\} \\
  \overline{B} _{r} \left( \mathbf{a} \right)  & =  \left\{ x \in  \mathbb{R} ^{n}  \mid  \|\mathbf{x} - \mathbf{a}\|_{}^{}  \le r \right\} \\
  \partial B _{r} \left( \mathbf{a} \right)  & =  \left\{ x \in  \mathbb{R} ^{n}  \mid  \|\mathbf{x} - \mathbf{a}\|_{}^{}  = r \right\} \\
    \end{split} 
  \] 

  For integratler \[
  \int_{B_{r} \left( \mathbf{a} \right)}^{}  f\left( \mathbf{x} \right) d^{n}\mathbf{x} = \int_{0}^{r}  \int_{\partial B_{\rho  }\left( \mathbf{a} \right)}^{}  f\left( \mathbf{x} \right) dS dr  
  \] 
  Vi kan skrive $\mathbf{x = \mathbf{a} + \rho \mathbf{y} }$ der $\|\mathbf{y}\|_{}^{} = 1$ og $\mathbf{y} \in  S ^{n-1}$
  For $f = 1$  \[
    \begin{split}
  \int_{B_{r} \left( \mathbf{a} \right)}^{} d^{n}\mathbf{x}  & = \int_{0}^{r}  \underbrace{\int_{\partial B_{\rho }\left( s \right)}^{}  dS dr }_{ A_{n} \rho ^{r-n}}     \\
  &= A_{n} \int_{0}^{r}  \rho ^{n-1} d\rho = \frac{A_{n}}{n} r^{n}  \\
    \end{split} 
  \] 
  Der $ A_{n} $ er volumet av $S^{n-1} = \partial  B _{1} \left( \mathbf{o} \right) \subseteq  \mathbb{R} ^{n}$ \[
    A_{2} = 2 \pi  , \quad  A_{3} = 4 \pi , \ldots 
  \] 
  \[
  \begin{split}
    \int_{B_{r}\left( \mathbf{a} \right)}^{}  f\left( \mathbf{x} \right) d^{n}\mathbf{x} = \int_{0}^{r}  , \int_{S^{n-1}}^{}  f\left( \mathbf{a} + \rho \mathbf{y} \right) dS \rho ^{n-1} d\rho    
  \end{split} 
  \] 
  Integrasjon i generaliserte polarkoordinater.
  \[
    \frac{d }{d r}  \int_{B_{r}\left( \mathbf{a} \right)}^{}  f\left( \mathbf{x} \right) d^{n} \mathbf{x} = r^{n-1} \int_{S^{n-1}}^{}    f\left( \mathbf{a} + r \mathbf{y} \right) dS
  \] 
\end{tcolorbox}

\subsection{Darboux Formel}%
\label{sub:darboux_formel}

La oss ha en kule med sentrum $\mathbf{x}$ og radius $\rho $. Definer $\overline{f} \left( \mathbf{x}; \rho  \right) $ der \[
  \begin{split}
 \overline{f\left( \mathbf{x}; \rho  \right)}  & = \int_{\partial B_{\rho }\left( \mathbf{x} \right)}^{}  f\left( \mathbf{y} \right) dS\left( \mathbf{y} \right) := \frac{1}{A_{n} \rho ^{n-1}} \int_{\partial B_{\rho }\left( \mathbf{x} \right)}^{}  f\left( \mathbf{y} \right) dS  . \\
&=  \int_{ S^{n-1}}^{} f\left( \mathbf{x} + \rho  \mathbf{z} \right) dS\left( \mathbf{z} \right) := \frac{1}{A_{n}} \int_{S^{n-1}}^{} \ldots     \\
 \Delta \overline{f} \left( \mathbf{x}; \rho  \right)  & = \overline{\Delta f}  \left( \mathbf{x}; \rho  \right)
  \end{split} 
\] 
\textbf{Derbours Formel} 
\[
\left(\rho ^{n-1} \overline{f} _{\rho }  \right) _{\rho } = \rho ^{n-1} \Delta \overline{f} 
\] 


\newpage
\section{Lecture 04/09/29}%
\label{sec:lecture_04_09_29}


\begin{tcolorbox}
  \textbf{Bølgeligningen }  \[
  u_{tt} - \nabla u = 0
  \] 
  \[
  u\left( 0, \mathbf{x} \right) = g\left( \mathbf{x} \right), \quad u_{t}\left( 0, \mathbf{x} \right) = h\left( \mathbf{x} \right) 
  \] 
\end{tcolorbox}

 Darboux Formel. Anta $f: \mathbb{R}  ^{n} \to  \mathbb{R}  $ \[
   \begin{split}
   \overline{f} \left( \mathbf{x}, \rho  \right) &= \int_{\partial B_{\rho } \left( \mathbf{x} \right)}^{} f\left( \mathbf{y} \right) dS\left( \mathbf{y} \right)  \\
   &=  \int_{f\left( \mathbf{x} + \rho \mathbf{z} \right)}^{} dS\left( \mathbf{z} \right)  \\
   \end{split} 
 \]
   \[
   \begin{split}
     \int_{B_{\rho }\left( \mathbf{x} \right)}^{}  \Delta f\left( \mathbf{y} \right) d^{n} \mathbf{y} &=  \int_{\partial B_{\rho }\left( \mathbf{x} \right)}^{}  \mathcal{V}  \nabla f\left( \mathbf{y} \right)   \\ 
                                                                                                          &= A_{n} \rho ^{n-1} \int_{S^{n-1}}^{}  \mathcal{V} \nabla f\left( \mathbf{x} + \rho \mathbf{z} \right) dS\left( \xi  \right)  \\
                                                                                                          &=A_{n} \rho ^{n-1} \frac{\partial }{\partial \rho  } f\left( \mathbf{x} - \rho \mathbf{z} \right) dS\left( \mathbf{z} \right)  \\
                                                                                                          &= A_{n} \rho ^{n-1} \overline{f} _{\rho } \\
   \end{split} 
   \] 

   \begin{theorem}
     Darboux \[
     r^{n-1} \Delta \overline{f}  = \left( r^{n-1} \overline{f} _{\rho }  \right)_{\rho }.
     \] 
   \end{theorem}

   \textbf{Ide!}  Sett \[
   \overline{u} \left( t, \mathbf{x}, \rho  \right) = \fint_{\partial B_{\rho } \left( \mathbf{x} \right)}^{}  u\left( t,\mathbf{y} \right) dS\left( \mathbf{y} \right) = \fint_{S^{n-1}}^{}  u\left( t, \mathbf{x} + \rho \mathbf{z} \right) 
   \] 
   Merk at \[
   \begin{rcases}
     \overline{u} _{tt} - \Delta \overline{u}  &= 0 \\
     \Delta \overline{u}  &=  \frac{1}{\rho ^{n-1} } \left( \rho ^{n-1} u_{\rho }  \right)_{\rho}  \\
   \end{rcases}
   \underbrace{\overline{u} _{tt} - \frac{1}{\rho ^{n-1}} \left( \rho ^{n-1} \overline{u} _{\rho } \right)_{\rho } }_{\overline{u} _{\rho \rho } + \frac{n-1}{\rho }  \overline{u} _{\rho }} = 0
   \] 
   \[
     \begin{split}
   \left( \rho ^{k}\overline{u}  \right)_{\rho \rho }&= \left( \rho ^{k}\overline{u} _{\rho } + k \rho ^{k-1} \overline{u}  \right)_{\rho } \\
   &= \rho ^{k} \overline{u} _{\rho \rho } + 2 k \rho ^{k-1} \overline{u}  _{\rho } + k\left( k-1 \right) \rho ^{k-2} \overline{u}  \\
     \end{split} 
   \] 
   Da ender vi opp med \[
   \left( \rho \overline{u}  \right)_{\rho \rho } =  \rho  \overline{u} _{\rho \rho } + 2k \overline{u}  _{\rho } \\
   \] 
   
   La oss definere \[
   \begin{split}
     \hat{f}\left( \mathbf{x}, \rho  \right) &= \rho \overline{f} \left( \mathbf{x} \right) \\
     \hat{u}\left( t, \mathbf{x}, \rho  \right) &=  \rho \overline{u} \left( t, \mathbf{x} \right) \\
   \end{split} 
   \] 
   Som gjør vi ender opp i \[
     \begin{split}
   \hat{u}_{\rho \rho } &=  \rho \overline{u} _{\rho \rho } + 2 \overline{u} _{\rho } = \rho \left( \overline{u}  _{\rho \rho } + \left( n-1 \right) u_{p} \right) \\
   \rho \overline{u} _{tt} = \hat{u}_{tt } \\
   \rho \overline{u} _{tt} - \hat{u_{ \rho \rho }}  & = 0
     \end{split} 
   \] 

   Andre interessante omskriving er \[
     \begin{split}
  \hat{u} \left( t, \mathbf{x}, \partial  \right) &= \rho  \overline{u} \left( t, \mathbf{x}, \rho  \right) = \rho \fint_{S^{n-1}}^{} u\left( t, \mathbf{x} + \rho \mathbf{z} \right) dS\left( \mathbf{z} \right)  \\ 
  \text{NB!} \quad   & \overline{u} \left( t, \mathbf{x} , -\rho  \right) = \overline{u} \left( t, \mathbf{x} , \rho  \right) \quad  \implies  \quad  \hat{u}\left( t, \mathbf{x}, -\rho  \right) = - \hat{u}\left( t, \mathbf{x}, \rho  \right)   
  \hat{u}_{tt} - \Delta \hat{u}= 0
     \end{split} 
   \] 

   \begin{tcolorbox}
     \[
     \begin{split}
       \hat{u}_{tt} - \hat{u}_{\rho \rho } &= 0 \\
       \hat{u}\left( 0, \mathbf{x}, \rho  \right)  & = \hat{g}\left( \mathbf{x}, \rho  \right)\\
       \hat{u}_{t} \left( 0, \mathbf{x}, \rho  \right) &=  \hat{h} \left( \mathbf{x}, \rho  \right) \\
     \end{split} 
     \] 
   \end{tcolorbox}
   dAlembert
  \[
    \begin{split}
  \hat{u}\left( t, \mathbf{x}, \rho  \right) &= \frac{\hat{g}\left( \mathbf{x}, \rho  -t \right) + \hat{g} \left( \mathbf{x} , \rho +t \right)}{ 2}  \\
  & + \frac{1}{2} \int_{\rho -t}^{ \rho -t}  \hat{h}\left( \mathbf{x}, \sigma  \right)  d \sigma \\
    \end{split} 
  \] 
  Ved å ta gensene  \[
    \begin{split}
  u\left( t,x \right) &=  \lim_{\rho \to \infty}  \overline{u} \left( t, \mathbf{x}, \rho  \right) \\
  &= \lim_{\rho  \to  \infty }  \frac{\hat{u}\left( t, \mathbf{x}, \rho  \right)}{ \rho }  \\
  &= \lim_{\rho \to \infty}  \left( \frac{\hat{g}\left( \mathbf{x} , t+ \rho  \right) \hat{g}\left( \mathbf{x}, t - \rho  \right)}{2\rho } + \frac{1}{2\rho } \int_{t - \rho }^{t + \rho }  \hat{h}\left( \mathbf{x}, s \right) ds   \right) \\
  &= \hat{g}_{\rho }\left( \mathbf{x}, t \right) + \hat{h} \left( \mathbf{x},t \right) \\
  &= \partial _{t} \left( \hat{g}\left( \mathbf{x},t \right) \right) + \hat{h}\left( \mathbf{x},t \right) \\
    \end{split} 
  \] 
  \begin{theorem}
  Som er kjent som Kirchoffs Integralformel for $n=3$ \[
  u\left( t,x \right) = \hat{g}\left( \mathbf{x}, t \right)_{t} + \hat{h} \left( \mathbf{x}, t \right)
  \] 
  \end{theorem}
   .
   Method of descent 
   
   \newpara
   Fra $n=3$ til $n=2$. Problem  \[
     \begin{split}
   u_{tt} - \Delta u &= 0, \quad  t > 0 , x \in  \mathbb{R} ^2  \\
   \begin{rcases}
     u\left( 0, \mathbf{x} \right) &=  g\left( \mathbf{x} \right) \\
     u_{t} \left( 0, \mathbf{x} \right) &=  h\left( \mathbf{x} \right) \\
   \end{rcases}
    &  x \in  \mathbb{R} ^2
     \end{split} 
   \]  

   La \[
   \begin{split}
     u\left( t, \left( x_{1}, x_{2} , x_{3} \right) \right) &=  u\left( t, x_{1}, x_{2} \right) \\
     g\left( x_{1}, x_{2}, x_{3} \right) &=  g\left( x_{1},x_{2} \right) \\
     h\left( x_{1}, x_{2} ,x_{3} \right) &=  h\left( x_{1}, x_{2} \right) \\
   \end{split} 
   \] 
   Resultat \[
     u\left( t,x \right) &=  \overbrace{\hat{g}\left( x,t \right), \hat{h}\left( \mathbf{x},t \right)}^{ \text{Regulert i } \mathbb{R} ^{3}}  \\
   \] 

   \[
     \begin{split}
   \hat{g}\left( \mathbf{x},t \right) &=  t \fint_{S^{2}}^{} g\left( \mathbf{x} + \rho  \mathbf{z} \right) dS\left( \mathbf{z} \right)  \\
   &= \frac{1}{4\pi }  2t \int_{S^2}^{}  g\left( \mathbf{x} 0 t \mathbf{z} \right) \left[ z_{3} > 0 \right] dS\left( \mathbf{z} \right)  \\
   &=   \frac{1}{4\pi } \int_{D}^{} g\left( \mathbf{x + t \mathbf{z}} \right) \frac{dz_{1} dz_{2}}{ \sqrt{1- z_{1}^2 - z_{2}^2} }   \\
     \end{split} 
   \] 
   Parametrisert med \[
     \begin{split}
   z &= \left( z_{1}, z_{2} \right) \in  D \\
   dS &= \frac{dz_{1} dz_{2}}{ \sqrt{1 - z_{1} ^2 - z_{2} ^2} }  \\
   dS &= \sqrt{1 + \left( \frac{\partial z_{3}}{\partial z_{1}}  \right) ^2 \left( \frac{\partial z_{3}}{\partial z_{2}}  \right) ^2}  dz_{1} dz_{2}  \\
   \implies   &  z_{3} = \sqrt{ 1 - z_{1} ^2 + z_{2} ^2}  \\
   \left( \frac{\partial z_{3}}{\partial z_{1}}  \right) ^2 &=  \left( \frac{-z_{1}}{\sqrt{ 1- z_{1}^2 + z_{2} ^2}  }   \right)^2 = \frac{z _{1} ^2}{ 1- z_{1} ^2 + z_{2} ^2}  \\
   \implies   &  dS = \sqrt{1 + } 
     \end{split} 
   \] 
  
   \textbf{Poission Integralformel for n=2} 

   \[
   u\left( t,x \right) = \frac{\partial }{\partial t}  \left( \int_{D}^{} g\left( \mathbf{x} + t\mathbf{z} \right) \frac{1}{\sqrt{1 - \|z\|_{}^{2}} } dz_{1} dz_{2} \right)
   + \frac{t}{2\pi }  \int_{D}^{}  \frac{h\left( \mathbf{x} + t\mathbf{z} \right)}{ \sqrt{1  - \|z\|_{}^{2}} }  dz_{1} dz_{2} 
   \] 





 
\section{References}%
\label{sec:references}



\bibliographystyle{plain}
\bibliography{references}
\end{document}

