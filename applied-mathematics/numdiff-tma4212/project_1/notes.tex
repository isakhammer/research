\documentclass{article}
\usepackage[utf8]{inputenc}

\title{Notes}
\author{isakhammer }
\date{2020}


%
%%%% DEPENDENCIES v1.4 %%%%%%

\usepackage{natbib}
\usepackage{graphicx}
\usepackage{amsmath}
\usepackage{amsthm}
\usepackage{amsfonts}
\usepackage{mathtools}
%\usepackage{enumerate}
\usepackage{enumitem}
\usepackage{todonotes}
\usepackage{esint}
\usepackage{float}


\usepackage{hyperref}
\hypersetup{
    colorlinks=true, %set true if you want colored links
    linktoc=all,     %set to all if you want both sections and subsections linked
    linkcolor=blue,  %choose some color if you want links to stand out
}
\hypersetup{linktocpage}


% inscape-figures
\usepackage{import}
\usepackage{pdfpages}
\usepackage{transparent}
\usepackage{xcolor}
\newcommand{\incfig}[2][1]{%
\def\svgwidth{#1\columnwidth}
\import{./figures/}{#2.pdf_tex} } \pdfsuppresswarningpagegroup=1

% Box environment
\usepackage{tcolorbox}
\usepackage{mdframed}
\newmdtheoremenv{definition}{Definition}[section]
\newmdtheoremenv{theorem}{Theorem}[section]
\newmdtheoremenv{lemma}{Lemma}[section]

% \DeclareMathOperator{\span}{span}

\DeclareMathOperator{\atantwo}{atan2}
\DeclareMathOperator{\arctantwo}{arctan2}

\theoremstyle{remark}
\newtheorem*{remark}{Remark}
%\newtheorem{example}{Example}

\newcommand{\newpara}
    {
    \vskip 0.4cm
    }

%%%%%%%%%%%%%%%%%%%%%%%%%%%%%%%%%%%%%%%%%%%%%%%%%%%%%%%%%%%%

%

\begin{document}
\maketitle
\tableofcontents
\newpage

\newpage
\section{Notes}%
\label{sec:notes}


\subsection{Task 1}%
\label{sub:task_1}


We want to solve $u_{xx} = f\left( x \right)$ such that $u_{x}\left( 1 \right) = \sigma $ and $u _{x}\left( 1 \right) =
\sigma $. To validate the algorithm
will we assume that \[
f\left( x \right) = \cos \left( 2 \pi x \right),
\]
which can be easily be solved analytically by integrating $u$.\[
    \begin{split}
u\left( x \right)  & =  - \frac{\cos \left( 2\pi x \right)}{ 4 \pi ^{2}} +  \frac{x^{3}}{6} +  \left( \sigma  - \frac{1}{2} \right)x +  \alpha, \quad u_{}\left( 0 \right) = \alpha    \\
    u_{x}\left( x \right) &=   \frac{\sin \left( 2 \pi x \right)}{2 \pi } + \frac{x^{2}}{2} + \left( \sigma -
    \frac{1}{2} \right), \quad u_{x}\left( 1 \right) = \sigma      \\
    u_{xx} \left( x \right) &= \cos \left( 2 \pi x \right)  + x \\
    \end{split}
\]




\newpage
\section{References}%
\label{sec:references}

\bibliographystyle{plain}
\bibliography{references}
\end{document}

