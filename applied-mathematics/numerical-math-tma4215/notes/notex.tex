\documentclass{article}
\usepackage[utf8]{inputenc}

\title{Notes Numerical Maths}
\author{isakhammer }
\date{2020}

\usepackage{natbib}
\usepackage{graphicx}
\usepackage{amsmath}
\usepackage{amsthm}
\usepackage{amsfonts}
\usepackage{mathtools}
\usepackage{enumerate}
\usepackage{todonotes}
\usepackage{float}


\usepackage{hyperref} 
\hypersetup{
  colorlinks=true, %set true if you want colored links
  linktoc=all,     %set to all if you want both sections and subsections linked
  linkcolor=blue,  %choose some color if you want links to stand out
} 
\hypersetup{linktocpage}


% inscape-figures
\usepackage{import}
\usepackage{pdfpages}
\usepackage{transparent}
\usepackage{xcolor}
\newcommand{\incfig}[2][1]{%
\def\svgwidth{#1\columnwidth}
\import{./figures/}{#2.pdf_tex} } \pdfsuppresswarningpagegroup=1

% Box environment
\usepackage{tcolorbox}
\usepackage{mdframed}
\newmdtheoremenv{definition}{Definition}[section]
\newmdtheoremenv{theorem}{Theorem}[section]
\newmdtheoremenv{lemma}{Lemma}[section]

\theoremstyle{remark}
\newtheorem*{remark}{Remark}
%\newtheorem{example}{Example}

\newcommand{\newpara}
  {
  \vskip 0.4cm
  }


\begin{document}
\maketitle
\tableofcontents
\newpage
  
\newpage
\section{Singular Value Decomposition}%
\label{sec:singular_value_decomposition}

\begin{definition}
  Let $A \in  \mathbb{C} ^{m \times n }$, there exist two unitary matrices $U \in  C ^{m \times m }$ and $V \in \mathbb{C} ^{n \times n }$ such that \[
    U^{H} A V = \Sigma  = diag\left( \sigma _{1},\sigma _{2}, \ldots , \sigma _{p} \right) \in \mathbb{R} m \times n  \quad \text{with} \quad p = min\left( m,n \right)  
  \] 
  and $\sigma _{1}\ge \ldots \ge \sigma _{p} \ge 0$. The formula called SVD of $A$ and the numbers $\sigma _{i}$ or $\sigma_{i} \left( A \right)$ are called the singular values of $A$.
\end{definition}

\begin{definition}
  Suppose that $A \in  \mathbb{C} ^{m \times n } $ has rank equal to $r$ and that it admits a SVD of the type $U^{H} A V = \Sigma $.  The matrix $A^{\dagger} = V \Sigma ^{\dagger} H ^{H}$ is the \textbf{Moore-penrose pseudo inverse matrix }  being \[
  \Sigma ^{\dagger} = diag\left( \frac{1}{\sigma _{1}} ,\ldots, \frac{1}{\sigma _{r}}, 0, \ldots, 0 \right)
  \] 
  The matrix $A^{\dagger}$ is also called the \textbf{generalized inverse}  of $A$. Indeed, if $rank\left( A \right) = n < m$ , then $A^{\dagger} = \left( A^{T} A \right)^{-1} A^{T} $, while if $n =m = rank\left( A \right)$ , $A^{\dagger} = A^{-1}$. 
\end{definition}

\begin{definition}
  A matrix $A \in \mathbb{C} ^{n \times n }$ is called \textbf{hermition} of \textbf{self-adjoint}  if $A^{T} = \overline{A}$, that is if $A^{H} = A$, while it is called \textbf{unitary}  if $A^{H}A = A A^{H} = I$. Finally, if $A A^{H} = A^{H}A$, $A$ is called normal. 
\end{definition}
A unitary matrix is one such that $A^{-1}= A^{H}$ and is normal.


\newpage
\section{References}%
\label{sec:references}

\bibliographystyle{plain}
\bibliography{references}
\end{document}

