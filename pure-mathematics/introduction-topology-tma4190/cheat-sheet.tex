\documentclass{article}
\usepackage[utf8]{inputenc}

\title{Cheat Sheet}
\author{isakhammer }
\date{2020}

%
%%%% DEPENDENCIES v1.7 %%%%%%

\usepackage{natbib}
\usepackage{graphicx}
\usepackage{amsmath}
\usepackage{amsthm}
\usepackage{amsfonts}
\usepackage{mathtools}
%\usepackage{enumerate}
\usepackage{enumitem}
\usepackage{todonotes}
\usepackage{esint}
\usepackage{float}

\usepackage{mathrsfs}

\usepackage{hyperref}
\hypersetup{
    colorlinks=true, %set true if you want colored links
    linktoc=all,     %set to all if you want both sections and subsections linked
    linkcolor=blue,  %choose some color if you want links to stand out
}
\hypersetup{linktocpage}


% inscape-figures
\usepackage{import}
\usepackage{pdfpages}
\usepackage{transparent}
\usepackage{xcolor}
\newcommand{\incfig}[2][1]{%
\def\svgwidth{#1\columnwidth}
\import{./figures/}{#2.pdf_tex} } \pdfsuppresswarningpagegroup=1

% Box environment
\usepackage{tcolorbox}
\usepackage{mdframed}
\newmdtheoremenv{definition}{Definition}[section]
\newmdtheoremenv{theorem}{Theorem}[section]
\newmdtheoremenv{lemma}{Lemma}[section]
\newmdtheoremenv{corollary}{Corollary}[section] % Use theorem counter as

\DeclareMathOperator{\atantwo}{atan2}
\DeclareMathOperator{\arctantwo}{arctan2}

\theoremstyle{remark}
\newtheorem*{remark}{Remark}
%\newtheorem{example}{Example}

\newcommand{\newpara}
    {
    \vskip 0.4cm
    }

\usepackage{geometry}
%\usepackage{showframe} %This line can be used to clearly show the new margins

\newgeometry{vmargin={15mm}, hmargin={12mm,17mm}}

%

% \usepackage{multicol}
% \setlength{\columnsep}{1cm}%%%%%%%%%%%%%%%%%%%%%%%%%%%%%%%%%%%%%%%%%%%%%%%%%%%%%%%%%%%
%

\begin{document}
\maketitle
% \begin{multicols}{2}
% \tableofcontents
% \newpage

% $\mathscr{H}$
% $\mathscr{S}$


\section{ Introduction}%
\label{sec:Ch1}


\begin{theorem}[Brouwer fixed point theorem]
    Let $f: D^{n} \to D^{n} $ be continious map from the (unit) disk in $\mathbb{R} ^{n}$ to itself. Then $f$ has a
    fixed point, i.e., there is some point $x \in  D^{n}$ such that $f\left( x \right) = x$.
\end{theorem}

\begin{theorem}[The fundemental theorem of algebra]
    A polynomial equation \[
    z^{n} + a_{n-1}z^{n-1} + \ldots + a_{1}z + a_{0} = 0
    \]
\end{theorem}

\section{ Continious maps}%
\label{sec:continious_maps}

\subsection{Metric spaces}%
\label{sub:metric_spaces}

\begin{definition}[Metric spaces]
A metric psace $\left( X,d \right)$ is a non-empty set $X$ toghether with a map $d: X \times X  \to \mathbb{R} $ called
a metric such that the following properties hold:
\begin{enumerate}[label=(\roman*)]
    \item \textbf{M1} $d\left( x,y \right) \ge 0$ for all $x,y \in X$, and $d\left( x,y \right) = 0$ if and only if $x =
        y$ .
    \item \textbf{M2} $d\left( x,y \right) = d\left( y,x \right) $  for all $x,y \in  X$
    \item \textbf{M3} $d\left( x,z \right) \le d\left( x,y \right) + d\left( y,z \right)$  for all $x,y,z \in  X$ .
\end{enumerate}
\end{definition}


\subsection{Continious maps between metric spaces}%
\label{sub:continious_maps_between_spaces}

\begin{definition}[Continious maps between metric spaces]
    Let $d\left( X, d_{X} \right)$ and $\left( Y, d_{Y} \right)$ be two metric spaces. A map $f: X\to Y$ is continious
    at $p \in X$ if for all $\varepsilon > 0$  there exists a $\delta > 0 $ such that if $d_{X}\left( p,q \right) <
    \varepsilon $ then $d_{Y}\left( f\left( p \right), f\left( q \right) \right) < \varepsilon $.
    If $f$  is continious at every point $p \in  X$ , we say that $f$ is continious.
\end{definition}



\begin{definition}[Open and closed balls]
    Let $X\left( X,d \right)$ be a metric space, and let $a \in X$ and $r >0$ be real number. The open ball centered at
    $a$ with radius $r$  is the subset \[
        B\left( a;r \right) = \left\{ x \in X  \mid  d\left( x,a \right) < r \right\}
    \]
    of $X$. The closed ball centered at $a$ with radius $r$ is the subset \[
    \overline{B} \left( a;r \right) = \left\{ x \in X  \mid  d\left( x,a \right) \le r \right\}
    \]
    of $X$.

\end{definition}


\begin{definition}[Open and closed sets]
    Let $\left( X,d \right)$ be a metric space. A subset $A \subseteq X$ is open in $X$ if for every point $a \in A$,
    there exists an open ball $B\left( a;r \right)$ about $a$ contained in $A$. We say that $A$ is closed in $X$ if the
    complement $A^{c} = X \setminus A = \left\{ x \in X  \mid  x \not\in A  \right\}$ is open
\end{definition}





% \begin{definition}
%     [Open and closed sets]. Let $\left( X, \tau  \right)$, $U \subseteq X$
%     \begin{itemize}
%         \item
% \textbf{Open set}.  If $ U \in \tau $, then is $U$ open.
%     \item \textbf{Closed set.}  If $U^{c} = X- U \in \tau $, then is $U$ closed
%     \end{itemize}
% \end{definition}


\begin{remark}
Let $X = \left\{ a,b,c \right\}$ and let $U = \left\{ a,b \right\}$. Then if $\tau = \left\{ X, \emptyset  \right\}$,
$U$ is not open nor closed.
\end{remark}

\begin{lemma}
    Let $\left( X,d \right)$ be a metric space, $x \in X$ and $r >0$ a real number. Then the open ball $B\left( x;r
    \right) \subseteq X$ is open in $X$ , and the closed ball $\overline{B}\left( x;r \right) \subseteq X$ is closed in
    $X$ .
\end{lemma}

\begin{definition}[Neightbourhoods]
    Let $\left( X,d \right)$ be a metric space, $A$ a subset of $X$ and $x \in X$. We say that $A$ is a neighbourhood of
    $x$ if there is an open ball about $x$ contained in $A$. We say that $A$ is an open neighborhood (of $x$ ) if $A$
    itself is open.
\end{definition}

\begin{theorem}[Continuity of a point]
    Let $\left( X,d_{X} \right)$ and $\left( Y, d_{Y} \right)$ be two metric spaces and let $p \in X$ . A map $f: X \to
    Y$ is continious at $p$ if and only if for all neighbourhoods $B$ of $f\left( p \right)$, there is a neighbourhood
    $A$ of $p$ such that $f\left( A \right)\subseteq B$ .
\end{theorem}
% \begin{definition}[Neighbourhoods]
%     Let $X$ be a topological space,  $U$ a subset of $X$ and $x \in  X$. We say $ U$ is a neighborhood of $x$ if $ x \in
%     U $ and $U$ is open in $X$.
% \end{definition}
\begin{theorem}[Continious maps between metric spaces]
    Let $\left( X,d_{X} \right)$ and $\left( Y,d_{Y} \right)$ be two metric spaces. A map $f: X\to Y$ is continious if
    and only if for every subset $B \subseteq Y$ open in $Y$ , the preimage of $B$ under $f$ , \[
    f^{-1}\left( B \right) = \left\{ x \in X  \mid  f\left( x \right) \in B \right\} \subseteq X,
    \]
    is open in $X$ .

\end{theorem}

\section{ Topological spaces}%
\label{sec:Ch3}

\subsection{Definitions and examples}%
\label{sub:definitions_and_examples}



\begin{definition}[Topological spaces.] Recall that a topological space is a set $X$ together with a collection $Y$ of subsets
    of $X$ that are open in $X$ s.t.
    \begin{itemize}
        \item \textbf{T1}.  $ \emptyset , X \in  \tau $
        \item \textbf{T2.} $ \tau $ is closed under union if $U_{\lambda } \in  \tau $ for all $\lambda  \in \Lambda $, then \[
                \bigcup_{\lambda  \in  \Lambda }^{} U_{\lambda } \in \tau
        \]
    \item \textbf{T3.} $\tau  $ is under finite intersections if $U_{1}, U_{2}, \ldots, U_{n} \in \tau $ , then \[
    U_{1} \cap U_{2} \cap \ldots \cap  U_{n} \in \tau
    \]

    \end{itemize}
\end{definition}

\begin{theorem}[Metric spaces are topological spaces]

    Let $\left( X,d \right)$  be a metrix space. Let $\tau _{d}$ be the collection of subsets $U \subseteq X$ with the
    property that $U \in \tau _{d}$ if and only if for each $x \in U$  there is an $r > 0$ such that $B\left( x;r
    \right) \subseteq  U$. Then $\tau _{d}$  defines a topology on $X$.
\end{theorem}

\begin{theorem}
    Let $X$  be any set, and let $d_{1}$  and $d_{2}$ be two equivalent metrics on $X$ . Then \[
    \tau _{d_{1}} = \tau _{d_{2}} .
    \]
\end{theorem}

\begin{definition}[Comparable topologies]
    Let $X$  be a set and suppose that $\tau _{1}$  and $\tau _{2}$  are two topologies on $X$ . If $\tau _{1} \subseteq
    \tau _{2}$, we say that $\tau _{1 }$  is coarser than $\tau _{2}$ and  that $\tau _{2}$ is finar than $\tau _{1}$.
    We say that $\tau _{1} $ and $\tau _{2}$  are comparable if either $\tau _{1} \subseteq \tau _{2}$  or $\tau _{2}
    \subseteq  \tau _{1}$.

\end{definition}

\subsection{Continious maps .}%
\label{sub:continious_maps}

\begin{theorem}
    \textbf{Continuity between topological spaces.}
Let $X,Y$ be topological spaces. A map $f: X \to Y$ is said to be continious if preimages of open sets are open, i.e.,
if $V$ is an open set in $Y$ then the preimage $f^{-1} \left( V \right)$ of $V$ is open in $X$.
\end{theorem}

\begin{theorem}[Composition of continious maps]
    Let $X,Y $ and $ Z $ be topological sapces. If $f: X\to Y$  and $g: Y\to Z$  are continious maps, then the composite
    $g \circ f: X \to Z $ is continious.
\end{theorem}

\begin{definition}[Continuity at a point]
    Let $X$ and $Y$ be topological space, and let $x \in X$ . A map $f: X \to Y$  is continiousat $x$  if for all
    neighbourhoods $V$  of $f\left( x \right)$  there is a neighbourhood $U$ of $x$  such that \[
    f\left( U \right) \subseteq V
    \]

\end{definition}

\begin{theorem}
    Let $X$ and $Y$  be topological spaces. A map $f: X \to Y$ is continious if and only if it is continious at each $x
    \in X$ .
\end{theorem}

\subsection{Homeomorhpism}%
\label{sub:homeomorhpism}

\begin{definition}[Homeomorphism]
Let $X$  and $Y$  be topological spaces. A bijective map $f: \to Y$  with the property that both $f$ and $f^{-1}: Y \to
X $   are continious, is called a homeomorhpism. if there exists a homeomorhpism $f: X \to Y$, we say that $X$
and $Y$ are homeomorphic.
\end{definition}

\begin{theorem}
    Let $X,Y$  and $Z$ be topological spaces.
    \begin{enumerate}[label=(\roman*)]
        \item \textbf{Reflexivity} : The identity map: $id: X\to X$ (where the domain and the codomain are equipped
            witht the same topology) , given by $id\left( x \right) = x$ for $x \in X$ , is a homeomorphism.
        \item \textbf{Symmentry} : If $f: X\to Y$ is a homeomorphis, then $f^{-1}: Y \to X$ is also a homeomorphism.
        \item \textbf{Transitivity} : If $f: X\to Y$  and $g: Y\to Z$ are homeomorhism, then $g \circ f: X \to Z$ is
            also a homeomorphism.
    \end{enumerate}
\end{theorem}

\subsection{Closes sets}%
\label{sub:closes_sets}

\begin{definition}[Closed subsets]

    A subset $K$  of a topological space $X$  is closed in $X$  if and only if the complement \[
    K^{c} = X \setminus K
    \] is open in $X$ .
\end{definition}

\begin{theorem}
    Let $X$  be a topological space.
    \begin{enumerate}[label=(\roman*)]
        \item Both $\emptyset $  and $X$  are closed (as subsets) in $X$.
        \item The intersection of any subcollection of closed sets in $X$  is closed in $X$ .
        \item The union of any finite subcollection of closed sets in $X$  is closed in $X$ .
    \end{enumerate}
\end{theorem}

\begin{definition}[Closure]
    Let $X$  be a topological space, and let $A$  be a subset of $X$. The closure of $A$, written $\overline{A} $, is
    the intersection of all subsets of $X$  that constains A and which are closed in $X$ .
\end{definition}

\begin{definition}[Dense]
    Let $X$ be a topological space, and let $A$ be a subset of $X$ . We say that $A$ is dense in $X$ if $\overline{A}  =
    X$ .
\end{definition}

\begin{theorem}
    Let $f: X \to Y$  be a map between the topological spaces. Then the following are equivalent:
   \begin{enumerate}[label=(\roman*)]
       \item $f$  is continious.
       \item for everry subset $A$  of $X$, we have $f\left( \overline{A}  \right) \subseteq \overline{f\left( A
           \right)} $ .
       \item for every closed subset $B$ of $Y$, the preimage $f^{-1}\left( B \right)$ of $B$ under $f$  is closed in
           $X$ .
   \end{enumerate}
\end{theorem}

\section{ Generating topologies}%
\label{sec:chapter_4_generating_topologies}

\subsection{Generating topologies from subsets}%
\label{sub:generating_topologies_from_subsets}



\begin{theorem}[The intersection of two topologies is a topology]
    Let $X$ be a set, and let $\tau _{1}$ and $\tau _{2}$ be two topologies on $X$. Then $\tau _{1} \cap \tau _{2}$ is
    also a topology on $X$.
\end{theorem}

\begin{definition}[Topology generated by a collection of subsets]
    Let $X$ be a set, and let $ \mathscr{S} $ be a collection of subsets of $X$. The topology generated by
    $\mathscr{S}$ is the topology \[
    \left<\mathscr{S}  \right> = \bigcap_{\substack{\tau  \text{ topology} \\ S \subseteq \tau  } }^{}  \tau
    \]
\end{definition}

\subsection{Basis for a topology}%
\label{sub:basis_for_a_topology}

\begin{definition}[Basis]
    Let $X$ be a set. a \textbf{basis} for a topology on $X$ is a collection $ \mathscr{B} $ of subsets of $X$ such
    that
    \begin{itemize}
        \item \textbf{B1}: for each $x \in X$, there is a $B \in  \mathscr{B} $ such that $x \in B$
        \item \textbf{B1:} if $B_{1}, B_{2}$ and $x \in B_{1} \cap B_{2}$ , then there is a $B_{3} \in  B$ such that $x
            \in  B_{3} \subseteq B_{1} \cap B_{2}$.
    \end{itemize}


\end{definition}



% \begin{definition}[Basis]
%     A \textbf{basis} for a topology on $X$ is a collection of subsets of $X$ s.t.
% \begin{itemize}
%     \item \textbf{B1.} For each $x \in X$ there is a $B \in  \mathscr{B} $ s.t. $x \in  B$.
%     \item \textbf{B2.} $B_{1}, B_{2}$ and $x \in  B_{1} \cap B_{2} $ there is a $B_{3} \in  \mathscr{B}  $ s.t. $x \in
%         B_{3} \subseteq  B_{1} \cap B_{2} $
% \end{itemize}

% If $\mathscr{B}  $ satisfies these, the topology $\tau $ generated by $ \mathfrak{B} $ is defined as follows: A subset
% $U$ of $X$ is said to be open in $X$ if for each $x \in U$, there exists
% $B \in  \mathscr{B} $ such that $x \in B $ and $B \subset U $.

% \end{definition}

\begin{theorem}
Let $X$ be a set, and let $ \mathscr{B} $ be basis for a topology on $X$. The collection  $\tau $ generated by $
\mathscr{B} $ of subsets $U$ of  $X$ with the property that for each $x \in U$ there is a basis element $B \in
\mathscr{B} $ with $ x \in B \subseteq U$ is a topology on $X$.
\end{theorem}

\begin{theorem}
Let $X$ be a set, and let $\mathscr{B} $ be a basis for a topology $\tau $ on $X$. Then  $\tau $ is equal to the
collection of all unions of elements of $ \mathscr{B} $.
\end{theorem}

\begin{theorem}
Let $X$ be a set, and let  $ \mathscr{B} _{1} $ and $\mathscr{B} _{2}$ be bases for topologies $\tau _{1}$ and $\tau
_{2}$ , respectively , on $X$. Then the following are equivalent.
\begin{enumerate}[label=(\roman*)]
    \item $\tau _{2}$ is finer than $\tau _{1}$, i.e., $\tau _{} \subseteq  \tau _{2}$.
    \item For each $B_{1} \in \mathscr{B} _{1}$ and each $x \in B_{1}$, there is a $B_{2} \in \mathscr{B} _{2}$ such
        that $x \in  B_{2} \subseteq B_{1}$.
\end{enumerate}
\end{theorem}

\subsection{Subbasis for a topology}%
\label{sub:subbasis_for_a_topology}

\begin{definition}[Subbasis]

    Let $X$ be a set. A \textbf{subbasis } for a topology on $X$ is a collection $ \mathscr{S} $ whose union equals $X$.
\end{definition}

\begin{lemma}
Let $X$ be a set, and let $ \mathscr{S} $ be a subbasis for a topology on $X$. The collection $ \mathscr{B} $
consisting of all finite intersections of elements of $\mathscr{S} $ is a basis for a topology on $X$ and is called the
basis associated to $ \mathscr{S} $.
\end{lemma}


\begin{definition}[Standard topology]
    (Not in compendium.)
    The standard topology on $ \mathbb{R} $ is the topology generated by a basis consisting of all open intervals of $
    \mathbb{R} $.
\end{definition}

\begin{lemma}
Let $X$ be a set, and let $\mathscr{S} $ be a subbasis for a topology on $X$. The collection $\tau $ generated by $
\mathscr{S} $ consisting of all unions of all basis elements of the associated basis $\mathscr{B} $ is a topology on
$X$.
\end{lemma}

\begin{theorem}
    Let $X$ be a set, and let $ \mathscr{S} $ be a subbasis for a topology on $X$. Then there exists a unique topology
    $\left<\mathscr{S}  \right>$ generated by $ \mathscr{S} $ which is smaller than any other topology containing
    $\mathscr{S} $ , where \[
    \left<\mathscr{S}  \right> = \left\{ \bigcup_{\lambda  \in  \Lambda }^{} \bigcap_{i=1}^{n_{\lambda }} S_{\lambda
    ,i}  \mid  S_{\lambda ,i} \in  \mathscr{S}     \right\}
    \]
\end{theorem}

\begin{theorem}
    Let $X$ and $Y$ be topological spaces, and let $ \mathscr{B} $ (resp., $\mathscr{S} $ ) be a basis (resp.,
    subbasis) . Then a map $f: X \to Y  $ is continious if and only if for each $B \in \mathscr{B} $ (resp. $S \in
    \mathscr{S} $ ) the preimage $f^{-1} \left( B \right)$ (resp., $f^{-1} \left( S \right)$ ) is open in $X$   .
\end{theorem}

\section{ Constructing topological spaces}%
\label{sec:chapter_5_constructing_topological_spaces}

\subsection{Subspaces}%
\label{sub:subspaces}

\begin{definition}[Subspace topology]
    Let $X$ be a topological space, and let $A$ be a subset of $X$ . The collection \[
    \tau _{A} = \left\{ A \cap U  \mid   U \text{ is open in }X  \right\}
    \]
    of subets of $A$ is called the topology on $A$.
\end{definition}

\begin{lemma}
    Let $X$  be a topological space, and let $A$  be a subsets of $X$. Then the collection \[
        \tau _{A} = \left\{ A \cap  U   \mid  U \text{ is open in }X \right\}
    \]
    is a topology on A.
\end{lemma}

\begin{theorem}
    Let $X$ be a topological space, and let $\mathscr{B} $  be a basis for the topology on $X$. If A is a subset $X$ ,
    the collection \[
    \mathscr{B} _{A} = \left\{ A \cap  B  \mid  B \in \mathscr{B}  \right\}
    \]
    is a basis for the subsapace topology on A.
\end{theorem}

\begin{theorem}
    Let $X$  be a topological space, and let $A$ be a subset of $X$ . Then the subspace topology on $A$  is the only
    topology on $A$  witht the following universal property: for every topological space $Y$ and every map : \[
    f: Y \to A
    \]
    $f$ is continious if and only if $i \circ  f: Y \to X$  is continious where $i : A \to X$ is the inclusion map given
    by $i\left( x \right) = x$  for $x \in  A$ .

\end{theorem}

\subsection{Products}%
\label{sub:products}

\begin{definition}[Product topology]
    Let $X$  and $Y $  be topological spaces. THe product topology on $X \times Y$  is the topology generated by the
    basis \[
        \mathscr{B}  = \left\{ U \times V  \mid  U \text{ is open in } X \text{ and } V \text{ is open in } Y  \right\}
    \]
\end{definition}

\begin{lemma}
    Let $X$ and $Y$  be topological spaces. Then the collection \[
        \mathscr{B }  = \left\{ U \times  V  \mid  U \text{ is open in } X \text{ and } V \text{ is open in }Y \right\}
    \]
    is a basis for a topology on $ X \times Y $ .
\end{lemma}


\begin{theorem}
 Let $X$  and $Y$  be topological paces. If $\mathscr{B} _{X}$  is a basis for a the topology on $X$  and $\mathscr{B}
 _{Y}$  is a basis for the topology on $Y$, then the collection \[
 \mathscr{B} _{X \times Y} = \left\{ B_{X} \times B_{Y}  \mid  B_{X} \in  \mathscr{B} _{X} \text{ and } B_{Y} \in
 \mathscr{B} _{Y} \right\}
 \]
 is a bsis for the product topology on $X  \times  Y$.
\end{theorem}


\begin{theorem}
    Let $X$  and $Y$ be topological spaces. Let $\pi _{1}: X \times Y \to X$  and $\pi _{2} : X \times Y \to Y$ be the
    projections of $X \times Y$  onto its first and second factors, respectively. The product topology is the only
    topology on $X \times  Y$ with  the following universial property: for every topological space $Z$  and every map
    $f: Z  \to X \times Y$, $f $  is continious if and only if $\pi _{1} \circ f : Z \to X $ and $\pi _{2} \circ f: Z
    \to Y$ are continious.
\end{theorem}


\subsection{Quotient spaces}%
\label{sub:quotient_spaces}

\begin{definition}[Equivalence classes]
    Let $X$  be a set, and let $\sim $  be an equivalence relation on $X$ . The equivalence class of $x \in X$  is the
    subset \[
        \left[ x \right]= \left\{ y \in  X   \mid  x \sim y\right\}
    \]
    of $X$ . Let \[
        X / \sim = \left\{ \left[ x \right]  \mid  x \in  X \right\}
    \]

\end{definition}


\begin{lemma}
    Let $X$  and $A$ be sets, and let $\pi : X \to A $ be a surjective map. Then the map \[
    \phi : X / \sim \to A
    \]
    given by $\phi \left( \left[ x \right] \right) = \pi \left( x \right)$, where $x_{1} \sim x_{2}$ if and only if $\pi
    \left( x_{1} \right) = \pi \left( x_{2} \right)$ , is a bijection.
\end{lemma}


\begin{definition}[Quotient space]
    Let $X$ be a topological space, let $A$  be a set, and elt $\pi : X \to A $ be a surjective map. The quotient
    topology on $A$  induced by $\pi $  is the collection of subsets $U$  of $A$  such that $\pi ^{-1} \left( U \right)$
    is open in $X$ . We say that $\pi $  is a quotient map if $A$ is given the quotient topology, and we call $A$ the
    quotient space.

\end{definition}

\begin{lemma}
    Let $X$  be a topological space, let $A$ be a set, and let $\pi : X \to A$ be a surjective map. Then the quotient
    topology on $A$ induced by $\pi $ is a topology and it is the finest topology on $A$  such that $\pi  $ is
    continious.
\end{lemma}


\begin{definition}[Open and closed maps]
    Let $X$ and $Y$  be topological spaces, and let $f: X\to Y$  be a continious map. We say that $f$  is an open map
    for each sucbset $U$  of $X$ that is open in $X$  the image $f\left( U \right)$  is open in $Y$ . Likewise, we say
    that $f$  is a closed map if for each subset $V$ of $X$ that is closed in $X$ the image $f\left( V \right)$ is
    closed in $Y$ .

\end{definition}

\begin{lemma}
    Let $X$ and $Y$  be topological spaces, and let $\pi: X  \to Y  $ be a surjective continious map.
    \begin{enumerate}[label=(\roman*)]
        \item If $\pi $  is in addition open then it is a quotient map.
        \item If $\pi $  is in addition closed then it is a quotient map.
    \end{enumerate}
\end{lemma}


\begin{theorem}
    Let $X$ be a topological space, let $A$  be a set, and let $\pi : X \to A$ be a surjective map. The quotient
    topology is the only topology on $A$ with the following universal property: for every topological space $Y$ and
    every map $f: A \to Y$ , $f $ is continious if and only if $f \circ \pi : X \to Y$ is continious.
\end{theorem}






\section{ Topological properties}%
\label{sec:chapter_6_topological_properties}

\subsection{Connected spaces}%
\label{sub:connected_spaces}


\begin{definition}[Connected space]
Let $X$ be a topological space. A \textbf{seperation} of $X$ is a pair of non-empty subsets $U $ and $V$ that are open
in $X$, disjoint and whose union equal $X$. We say that $X$ is \textbf{connected} if there are no seperations of $X$.
Otherwise it is \textbf{disconnected.}
\end{definition}

\begin{theorem}[Closed and open subsets]
    Let $X$ be a topological space. Then $X$ is connected if and only if the are no non-empty proper subsets of $X$ that
    are both open and closed in $X$.

\end{theorem}

\begin{lemma}[Disconnectivity]
    Let $X$ be a disconnected space with seperation $U$ and $V$,  and et $A$ be a connected subspace of $X$. Then $A
    \subseteq U$ and $A \subseteq  V$.

\end{lemma}

\begin{theorem}[Collection connectivity]
    Let $X$ be a topological space,  and let $\left\{ A_{\lambda } \right\}_{\lambda \in \Lambda }$ be a collection of
    connected subspaces of $X$ such that $ \bigcap_{\lambda  \in \Lambda } A_{\lambda }$ is non-empty. Then $\bigcup
    _{\lambda \in  \Lambda } A_{\lambda }$ is connected.

\end{theorem}

\begin{definition}[Path connected space]
    Let $X$ be a topological space, and let $x,y \in X$. A path from $x$ to  $y$ is a continious map:  $f: \left[ a,b
    \right]  \to X$ .t. $f\left( a \right) = x$ and $f\left( b \right) = y$ where $\left[ a,b \right]$ is a subspace of
    $ \mathbb{R} $ with the standard topology. We say that $X$ is \textbf{path connected} if every pair of points of $X
    $ can be joined by a path in $X$.

\end{definition}


\begin{theorem}[ Connectivity in product spaces]
    Let $X_{1},  X_{2}, \ldots, X_{n}$ be connected spaces. Then the product space $X_{1} \times X_{2} \times  \ldots
    \times  X_{n}$ is connected.

\end{theorem}

\begin{theorem}[The real numbers are connected]
    Let $\mathbb{R} $ be the set of real numbers equipped with the standard topology. Then $\mathbb{R} $ is connected.

\end{theorem}

\begin{theorem}[Generalized intermediate value theorem]

    Let $X$ be a connected space and let $f: X \to  \mathbb{R} $ be a continious map where $ \mathbb{R} $ is given the
    standard topology.  If $a,b \in X $ and if $r$ is a real number that lies between $f\left(a  \right) $ and $f\left(
    b\right)$, there is a $c \in  X$ such that $f\left( c \right) = r$

\end{theorem}

\begin{theorem}[Connectivity]
    Let $X$ be a topological space. Then $X$ is connected if and only if the are no non-empty proper subsets of $X$ that
    are both open and closed.
\end{theorem}

\begin{theorem}[Path connectedness implies connectedness]
    Let $X$ be a path connectedness space. Then $X$ is connected.

\end{theorem}
\subsection{Hausdorff spaces}%
\label{sub:hausdorff_spaces}

\begin{definition}[Hausdorff]

    Let $X $ be a topological space. We say that $X$ is \textbf{Hausdorff} if for each part of points $x,y  \in X $ with
    $x\neq y$ , there are disjoint neighborhoods $U$ and $V$ of $x$ and $y$, respectively. In other words, for each pair
    of distinct point $x,y \in  X$ there are open subsets $U$ and $V$ of $X$ with $x \in U$ $y \in V $ where $U \cap V =
    \emptyset $
\end{definition}

\begin{theorem}
    Every metric space is Hausdorff

\end{theorem}

\begin{theorem}
    Let $X$ be a Hausdorff space. Then for each $x \in X$ the subset $\left\{ x  \right\}$ of $X$ is closed in  $X$.
\end{theorem}

\begin{theorem}
    Let $X_{1}, X_{2} \ldots, X_{n} $ be Hausdorff spaces.  Then the product space $X_{1} \times  X_{2}\times  \ldots
    \times X_{n}  $ is Hausdorff.

\end{theorem}


\begin{theorem}
Let $X$ be a topological space.  Then $X$ is Hausdorff if and only if the diagonal \[
    \Delta = \left\{ \left( x,x \right)  \mid  x \in  X \right\}
\]is closed in the product space $X \times  X$.
\end{theorem}

\subsection{Compact spaces}%
\label{sub:compace_spaces}

\begin{definition}[Cover of a space]
    Let $X$ be a topological space, and let $\mathscr{A} $ be the collection of subsets of $X$. We say that  $
    \mathscr{A} $ is a cover of $X$, or covering of X if $X = \bigcap_{A \in \mathscr{A} }^{} A$. If $A$ is also open in
     $X$ for each  $A \in  \mathscr{A} $, we say that $\mathscr{A} $ is an \textbf{open} cover of $X$, or open
     covering of $X$. We say that $ \mathscr{A}' $ is a subcover of $\mathscr{A} $ if $\mathscr{A} '  $ is another
     cover of $X$ that satisfies  $ \mathscr{A '} \subseteq \mathscr{A}  $.
\end{definition}


\begin{definition}[Compact spaces]
    Let $X$ be a topological space. We say that $X$ is \textbf{compact} if every open cover $ \mathscr{A} $ of $X$
    contains a finite subcover.

\end{definition}

\begin{definition}[Compact subspaces]
    Let $X$ be a topological space, and let $A$ be a subset of $X$. We say that $A$ is compact in $X$ if $A $ is compact
    in the subspace topology.
\end{definition}


\begin{lemma}
    Let $X$ be a topological space, and let $A$ be a subspace of $X$.  Then $A$ is compact in $X$ if and only if every
    cover of $A$ by open subsets of $X$ contains a finite subcollection that covers $A$.
\end{lemma}

\begin{theorem}
    Let $X$ be a compact space, and let $A$ be a closed subset of $X$.  Then $A $ is compact in $X$.
\end{theorem}

\begin{theorem}
Let $X$ be a Hausdorff space,  and let $K $ be a subset of $X$ which is compact in $X$. Then $K$ is closed in $X$.
\end{theorem}

\begin{theorem}
Let $X$ be a compact space, $Y$ a topological space and let $f: X \to Y$ be a surjective continious map. Then $Y$ is
compact.
\end{theorem}

\begin{lemma}[Tube lemma]
Let $X$ be a topological space, and let $Y$ be a compact space.  If $x \in  X $ and $U$ is an oppen set in the product
space $X \times Y$ containing $\left\{ x \right\} \times Y$ , then there is a neighborhood $W$ of $x$ in $X$ such that
$W \times  Y \subseteq U$
\end{lemma}


\begin{theorem}
Let $X_{1} , X_{2}, \ldots , X_{n}$ be compact spaces. Then the product space $X_{1} \times  X_{2} \times \ldots\times
X_{n}$ is compact.
\end{theorem}


\begin{theorem}
Let $\mathbb{R} $ be the set of real numbers equipped with the standard topology. Then every closed interval $\left[ a,b
\right] \in \mathbb{R} $ is compact in $ \mathbb{R} $.
\end{theorem}

\begin{definition}[Bounded subsets]

    Let $\left( X,d \right) $ be a metric space, and let $A$ be a subset of $X$. We say that $A$ is bounded if there is
    an $M  \in  \mathbb{R} $ such that $d\left( a_{1}, a_{2} \right) \le M$ for all $a_{1}, a_{2} \in  A$.

\end{definition}

\begin{theorem}[Heine- Borel]
    Let $\mathbb{R} ^{n}$ be given the (Euclidian) metric topology and the Euclidian metric. A subset $A$ of $\mathbb{R}
    ^{n}$ if and only if it is closed and bounded.

\end{theorem}


\begin{theorem}[Generalized extreme value theorem]

    Let $X$ be compact space, and let $f: X \to  \mathbb{R} $ be a continious map where $ \mathbb{R} $ is given the
    standard topology.  Then there are $m, M \in X $ such that \[
    f\left( m \right) \le f\left( x \right) \le f\left( M \right)
    \]
    for all $x \in   X$.
\end{theorem}
\section{ The fundamental group}%
\label{sec:the_fundamental_group}

\subsection{Homotopy of paths}%
\label{sub:homotopy_of_paths}



\begin{definition}[Homotopy]
    Let $X$  and $Y$ be topological spaces, and let $f_{0}, f_{1}: X \to Y$ be two continious maps. Furthermotre, let
    $\mathbb{R} $  be the set of real numbers with the standard topology, $I = \left[ 0,1 \right]$ be a subsapce of
    $\mathbb{R} $ , and let $X \times I$ be the given topology. We say that $f_{0}$ is homotopic to $f_{1}$ , written
    $ f_{0} \simeq f_{1}$, if there is a continious map \[
    H: X \times I \to Y
    \]
    such that $H\left( x,0 \right) = f_{0}\left( x \right)$  and $H\left( x,1 \right) = f_{1}\left( x \right)$ for all
    $x \in  X$. The map $H$ is called a homotopy between $f_{0}$  and $f_{1}$ . If $ f_{0} \simeq  f_{1} $ and $f_{1}$
    is a constant map, we say that $f_{0}$ is nullhomotopic.
\end{definition}


\begin{lemma}[Pasting lemma]
    Let $X = A \cup B$  be a topological space where $A$ and $B$  are closed in $X$. Furthermore, et $Y$ be a
    topological space, and assume that $f: A \to Y$  and $g: B \to Y$ are continious maps. If $f\left( x \right)  =
    g\left( x \right)$ for all $x \in A\cap B$, then the map $h: X\to Y$  given by \[
    h\left( x \right)
    \begin{cases}
        f\left( x \right),&  \quad x \in A \\
        g\left( x \right), &  \quad x \in  B
    \end{cases}
    \]
    is continious.
\end{lemma}

\begin{theorem}
    The relation $\simeq $ is an equivalence relation on the set of all continious maps from a topological space $X$ to
    a topological space $Y$ .
\end{theorem}

\begin{definition}[Homotopy classes]
    Let $X$ and $Y$ be topological spaces, and let $C\left(X,Y  \right)$  be the set of continious maps from $X$ to $Y$.
    The homotopy classes in $C\left( X,Y \right)$ are the equivalence classes under the relation $\simeq  $ . We write
    $\left[ f \right]$ for the homotopy class of $f \in C\left( X,Y \right)$ , i.e., \[
    \left[ f \right] = \left\{ g \in C\left( X,Y \right)  \mid f \simeq  g \right\}
    \]
    and we write $\left[ X,Y \right]$ for the set of homotopy classes of continious maps from $X$ to $Y$ , i.e., \[
        \left[ X,Y \right] ) C\left( X,Y \right) \setminus \simeq
    \]
\end{definition}

\begin{definition}[Path homotopy]
    Let $X$ be a topologicala space, and let $x_{0},x_{1} \in X$. We say that two paths $f,g: I \to X$ in $X$ from
    $x_{0}$ to $x_{1}$ are path homotopic, written $ f \simeq  _{p}g$, if there is a continious map $F: I \times I \to
    X$ such that \[
    H\left( s,0 \right) = f\left( s \right) \quad \text{and} \quad H\left( s,1 \right) = g\left( s \right)
    \]
    for all $s \in  I$, and \[
    H\left( 0,t \right) = x_{0} \quad \text{and} \quad H\left( 1,t \right) = x_{1}
    \]
    for all $t \in  I$. We call $H$  a path homotopy from $f$  to $g$ .
\end{definition}

\begin{theorem}
    Let $X$  be a topological space, and let $x_{0}, x_{1} \in  X$. Then the relation $\simeq _{p}$ is an equivalence
    relation on the set of all paths from $x_{0}$  to $x_{1}$ in $X$ .
\end{theorem}

\begin{definition}[Path homotopy classes]
    Let $X$ be a topological space, and let $x_{0}, x_{1} \in X$. If $f: I \to X$  is a path from $x_{0}$  to $x_{1}$,
    we write $\left[ f \right]$ for its path homotopy class, i.e., \[
    \left[ f \right] = \left\{ g: I \to X  \mid g \text{ is a path from }x_{0} \text{ to } x_{1} \text{ and } f \simeq
    _{p} g \right\}
    \]
\end{definition}

\begin{definition}[Product of paths]
    Let $X$ be a topologicala space, and let $x_{0}, x_{1}, x_{2} \in X$. If $f: I \to X$ is a path from $x_{0}$  to
    $x_{1}$ , and $g: I \to X$ is a is a path from $x_{1}$ to $x_{2}$, we define the product of $f$ and $g$  as the path
    $f * g: I \to X$ from $x_{0}$ to $x_{2}$ given by \[
    \left( f*g \right)\left( s \right) = \begin{cases}
        f\left( 2s \right) ,&  \quad 0\le s \le \frac{1}{2}\\
        g\left( 2s -1 \right),&  \quad  \frac{1}{2} \le s \le 1.
    \end{cases}
    \]
\end{definition}

\begin{lemma}
    Let $X$  be a topological space, and let $x_{0}, x_{1}, x_{2} \in  X$ . If $f: I \to X$ is a path from $x_{0}$ to
    $x_{1}$ and $g: I\to X$ is a path from $x_{1}$ to $x_{2}$, then the product $f * g$ induses a well-defined operation
    on path homotopy classes given by \[
    \left[ f \right] * \left[ g \right] = \left[ f * g \right]
    \]

\end{lemma}

\begin{theorem}
    Let $X$   be a topological space. Then the product paths, $*$  , has the following properties on the set of path
    homotopy classes.

    \begin{enumerate}[label=(\roman*)]
        \item \textbf{Associativity.} Let $x_{0}, x_{1}, x_{2}$   and $x_{3}$  be the points in $X$  . If $f_{0}: I \to X$
             is a path from $x_{0}$  to $x_{1}, f_{1}: I\to X$  is a path from $x_{1}$  to $x_{2}$, and $f_{2}: I \to X$  is
            a path from $x_{2}$  to $x_{3}$, then \[
            \left( \left[ f_{0} \right] * \left[ f_{1} \right] \right) * \left[ f_{2} \right] = \left[ f_{0} \right] *
            \left( \left[ f_{1} \right] * \left[ f_{2} \right] \right)
            \]
        \item \textbf{Left and right units} . For $x \in X$ , let $c_{x}: I\to X$ denote the constant path at $x$ ,
            given $c_{x}\left( s \right) = x$ for all $s \in  I$. If $f: I \to X$ is a path from $x_{0}$ to $x_{1}$
            then \[
            \left[ c_{x_{0}} \right] * \left[ f \right] = \left[ f \right] = \left[ f \right] * \left[ c_{x_{0}} \right]
            \]
        \item \textbf{inverse} If $f: I\to X$  is a path from $x_{0}$ to $x_{1}$, let $\overline{f} : I \to X$ be the
            reverrse path from $x_{1}$  to $x_{0}$ , given by $\overline{f} \left( s \right) = f\left( 1-s \right)$  for
            all $s \in I$ . Then \[
            \left[ f \right] * \left[ \overline{f}  \right] = \left[ c_{x_{0}} \right] \text{ and } \quad \left[
            \overline{f}  \right] * \left[ f \right] = \left[ c_{x_{1}} \right]
            \]
    \end{enumerate}

\end{theorem}
\subsection{Definition and elementary properties of the fundamental group}%
\label{sub:definition_and_elementary_properties_of_the_fundamental_group}

\begin{definition}[The fundemental group]
    Let $\left( X, x_{0} \right)$ be a based space. A path $f: I \to X$  from $x_{0}$  to $x_{0}$ is called a loop in
    $X$  based at $x_{0}$ . Let \[
        \pi _{1}\left( X, x_{0} \right) = \left\{ \left[ f \right]  \mid  f \text{ is a loop in } X \text{ based at
        }x_{0}  \right\}
    \]
    be the set of path hotopy classes of loops in $X$ at $x_{0}$ . We say that $\pi _{1} \left( X,x_{0} \right)$ is the
    fundemental group of $X$  based at $x_{0}$ .
\end{definition}

\begin{theorem}
    Let $\left( X, X_{0} \right)$ be a based space. THen the fundementa group $\pi _{1} \left( X, x_{0} \right)$ of $X$
    based at $x_{0}$  is, in fact, a group with product paths, $*$ , as its binary operation. The identity element $e$
    is equal to the path homotopy class of the constant path at $x_{0}$, $e = \left[ c_{x_{0}} \right]$ , and the
    inverse of $\left[ f \right]$ is $\left[ f \right]^{-1}= \left[ \overline{f}  \right]$, where $\overline{f} $ is the
    reverse path of $f$ .
\end{theorem}

\begin{theorem}
    Let $X$ be a path connected space, and let $x_{0}, x_{1} \in X$ . Then $\pi _{1}\left( X, x_{0} \right)$ is
    isomorphic to $\pi _{1}\left( X, x_{1} \right)$ .
\end{theorem}

\begin{definition}[Simply connected spaces]
    Let $X$  be a path connected sace. We say that $X$ is simply connected if $\pi _{1}\left( X, x_{0} \right)$ is the
    trivial group for some $x_{0} \in  X$, and hence, for all $x_{0} \in X$ .
\end{definition}

\begin{definition}[Based maps]
    Let $\left( X, x_{0} \right)$ and $\left( Y, y_{0} \right)$ be based spaces. A based map \[
    h: \left( X, x_{0} \right) \to \left( Y,y_{0} \right)
    \] is a continious map $h: X \to Y$ such that $h\left( x_{0} \right) = y_{0}$ .
\end{definition}

\begin{definition}[Homomorphism induced by based maps]
    Let $\left( X, x_{0} \right) $  and $\left( Y, y_{0} \right)$ be based spaces, and let $h: \left( X, x_{0} \right)
    \to  \left( Y , y_{0} \right)$ be based map.  The map \[
        h_{*} : \pi _{1}\left( X, x_{0} \right)  \to \pi _{1}\left( Y , y_{0} \right)
    \] given by \[
    h_{*}\left( \left[ f \right] \right) = \left[ h \circ f \right]
    \] is called the homomorphism indeiced by h.
\end{definition}

\begin{lemma}
    Let $\left( X, x_{0}\right)$   and $\left( Y ,y_{0} \right)$  be based spaces, and let $h: \left( X, x_{0} \right) \to
    \left( Y ,y_{0} \right)$  be a based map. The map \[
    h_{*} : \pi _{1} \left( X, x_{0} \right)  \to \pi _{2} \left( Y ,y_{0} \right).
    \]
    given by \[
    h_{*} \left( \left[ f \right] \right) =  \left[ h \circ f \right]
    \] is a homomorphism.
\end{lemma}

\begin{theorem}[Functoriality]
   Let $\left( X, x_{0} \right)$ , $\left( Y ,y_{0} \right)$ and $\left( Z ,z_{0} \right)$ be based spaces, and let
   $h_{2}: \left( X, x_{0} \right) \to \left( Y, y_{0} \right)$ and $h_{2}\left( Y, y_{0} \right) \to  \left( Z, z_{0}
   \right)$ be based maps. Then \[
   \left( h_{2} \circ h_{1} \right) = \left( h_{2} \right)_{*} \circ \left( h_{1} \right)_{*}.
   \]
   If $id_{X} : X \to X$ is the identity map, then $\left( id_{X} \right)_{*}$ is the identity automorphism of $\pi
   _{1}\left( X,x_{0} \right)$ .
\end{theorem}

\begin{corollary}
    Let $\left( X, x_{0} \right)$  and $\left( Y ,y_{0} \right)$ be based spaces. If $h: X \to Y$ is a homeomorphism
    such that $h\left( x_{0} \right) = y_{0}$, then \[
        h_{*} : \pi _{1}\left( X, x_{0} \right) \to \pi _{1} \left( Y, y_{0} \right)
    \]
    is an isomorphism.
\end{corollary}

\begin{theorem}
    Let $\left( X,  x_{0} \right) $ and $\left( Y, y_{0} \right)$ be based spaces. Then $\pi _{1}\left( X \times Y,
    \left( x_{0}, y_{0} \right) \right)$ is isomorphic to the direct product $\pi _{1} \left( X, x_{0} \right) \times
    \pi _{1}\left( Y ,y_{0} \right)$ .
\end{theorem}

\subsection{Homotopy type}%
\label{sub:homotopy_type}

\begin{lemma}
    Let $\left( X, x_{0} \right)$ and $\left( Y,y_{0} \right)$ be based spacaes, and let $h: \left( X ,x_{0} \right) \to
    \left( Y,y_{0} \right)$ and $k: \left( X,x_{0} \right) \to \left( Y, y_{0} \right)$ be based mmaps. If there is a
    homotopy $H: X \times I \to Y$  from $h$  to $k$ such that $H\left( X_{0}, t \right) =  y_{0}$ . for all $t \in  I$,
    then the homomorphism $h_{*} : \pi _{1} \left( X, x_{0} \right) \to \pi _{1}\left( Y, y_{0} \right)$  and $k_{* } :
    \pi _{1} : \left( X ,x_{0} \right) \to \pi \left( Y, y_{0} \right) $ induced by $h$ and $k$ , respectively , are
    equal.
\end{lemma}

\begin{definition}[Retractions]
    Let $X$ be topological space, and let $A$ be subspace of $X$.  We say that a continious map $r: X \to A$ is a
    retraction of $X$  onto $A$ if $r\left( a \right) = a$ for each $a \in  A$. If there is a retraction of $X$ onto
    $A$, we say that $A$ is retract of $X$ .
\end{definition}

\begin{lemma}
    Let $X$ be topological space, and let $A$ be a subspace of $X$. If $x_{0} \in  A$ and $A $ is a retract of $X$. THen
    the homomorphism $i_{* } : \pi  _{1} \left( A, x_{0} \right) \to \pi _{1}\left(X, x_{0}  \right)$ induced by the
    inclusion map $i: A \to X$ is a monomorphism.
\end{lemma}

\begin{definition}[Deformation retracts]
    Let $X$ be a topological space, and let $A$ be a subspace of $X$. A homotopy \[
    H: X \times I \to X
    \]
    is called deformation retraction of $X$  onto $A$  if $H\left( x, 0 \right) = x$ and $H\left( x,1 \right) \in  A$
    for all $x \in X$, and $H\left(  a,t \right) = a$  for all $a \in  A$ and all $t \in  I$. We say that $A$ is a
    deformation tract of $X$ .
\end{definition}

\begin{theorem}
    Let $X$ be a topological space, and let $A$ be a subspace of $X$ . If $x_{0} \in  A$ and $A$ is a deformation
    retract of $X$, then the homomorphism $i_{*}: \pi _{1}\left( A, x_{0} \right) \to \pi \left( X , x_{0} \right)$
    indeuced by the inclusion map $i: A \to X$ is an isomorphism.
\end{theorem}

\begin{definition}[Homotopy equivalences]
    Let $X$  and $Y$ be topological spaces. If $f: X\to Y$ and $g: Y \to X$ are continious maps such that $g \circ f$ is
    homotopic to the identity map of $X$, $id_{X}$, and $f \circ g$ is homotopic to the identity map of $Y$, $id_{Y}$ ,
    we say that $f$ and $g$  are homotopy equivalences. We say that each of $f$ and $g$  is a homotopy inverse of the
    other.
\end{definition}

\begin{definition}[Homotopy types]
    Let $X $ and $Y$ be topological spaces. We say that $X$ and $Y$ have the same homotopy type if there is a homotopy
    equivalence $f: X \to Y$ .
\end{definition}

\begin{lemma}
    Let $X$  and $Y$ be topoplogical spaces, and let $f :X \to Y$ and $g: X \to Y$ be continious maps such that $f\left(
    x_{0}\right) =  y_{0}$ and $g\left( x_{0} \right) = y_{1}$. if $H: X \times I \to Y  $ is a homotopy from $f$  to
    $g$, there is a path $\alpha : I \to Y$ in $Y$ from $y_{0}$ to $y_{1}$ fiven by $\alpha \left( t \right) = H\left(
    x_{0}, t \right)$ such that $g _{*} = \hat{\alpha }\circ f_{*}$
\end{lemma}

\begin{theorem}
    Let $X$ and $Y$ be topological spaces, and let $f: X \to Y$ be homotopy equivalence such that $f\left( x_{0} \right)
    0 y_{0}$. tThen \[
    f_{*}: \pi _{1}\left( X, x_{0} \right) \to  \pi _{1}\left( Y, y_{0} \right)
    \]
    is an isomorphism.
\end{theorem}

\section{ The fundamental group of the circle}%
\label{sec:the_fundamental_group_of_the_circle}



% \end{multicols}

\newpage
\section{References}%
\label{sec:references}

\bibliographystyle{plain}
\bibliography{references}
\end{document}

