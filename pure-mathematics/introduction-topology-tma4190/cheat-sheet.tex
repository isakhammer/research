\documentclass{article}
\usepackage[utf8]{inputenc}

\title{Cheat Sheet}
\author{isakhammer }
\date{2020}

%
%%%% DEPENDENCIES v1.5 %%%%%%

\usepackage{natbib}
\usepackage{graphicx}
\usepackage{amsmath}
\usepackage{amsthm}
\usepackage{amsfonts}
\usepackage{mathtools}
%\usepackage{enumerate}
\usepackage{enumitem}
\usepackage{todonotes}
\usepackage{esint}
\usepackage{float}

\usepackage{mathrsfs}

\usepackage{hyperref}
\hypersetup{
    colorlinks=true, %set true if you want colored links
    linktoc=all,     %set to all if you want both sections and subsections linked
    linkcolor=blue,  %choose some color if you want links to stand out
}
\hypersetup{linktocpage}


% inscape-figures
\usepackage{import}
\usepackage{pdfpages}
\usepackage{transparent}
\usepackage{xcolor}
\newcommand{\incfig}[2][1]{%
\def\svgwidth{#1\columnwidth}
\import{./figures/}{#2.pdf_tex} } \pdfsuppresswarningpagegroup=1

% Box environment
\usepackage{tcolorbox}
\usepackage{mdframed}
\newmdtheoremenv{definition}{Definition}[section]
\newmdtheoremenv{theorem}{Theorem}[section]
\newmdtheoremenv{lemma}{Lemma}[section]

\DeclareMathOperator{\atantwo}{atan2}
\DeclareMathOperator{\arctantwo}{arctan2}

\theoremstyle{remark}
\newtheorem*{remark}{Remark}
%\newtheorem{example}{Example}

\newcommand{\newpara}
    {
    \vskip 0.4cm
    }

%%%%%%%%%%%%%%%%%%%%%%%%%%%%%%%%%%%%%%%%%%%%%%%%%%%%%%%%%%%%

%

\begin{document}
\maketitle
% \tableofcontents
% \newpage

% $\mathscr{H}$
% $\mathscr{S}$


\section{ Introduction}%
\label{sec:Ch1}

\section{ Continious maps}%
\label{sec:continious_maps}

\section{ Topological spaces}%
\label{sec:Ch3}

\begin{definition}[Topological spaces.] Recall that a topological space is a set $X$ together with a collection $Y$ of subsets
    of $X$ that are open in $X$ s.t.
    \begin{itemize}
        \item \textbf{T1}.  $ \emptyset , X \in  \tau $
        \item \textbf{T2.} $ \tau $ is closed under union if $U_{\lambda } \in  \tau $ for all $\lambda  \in \Lambda $, then \[
                \bigcup_{\lambda  \in  \Lambda }^{} U_{\lambda } \in \tau
        \]
    \item \textbf{T3.} $\tau  $ is under finite intersections if $U_{1}, U_{2}, \ldots, U_{n} \in \tau $ , then \[
    U_{1} \cap U_{2} \cap \ldots \cap  U_{n} \in \tau
    \]

    \end{itemize}
\end{definition}

\begin{definition}
    [Open and closed sets]. Let $\left( X, \tau  \right)$, $U \subseteq X$
    \begin{itemize}
        \item
\textbf{Open set}.  If $ U \in \tau $, then is $U$ open.
    \item \textbf{Closed set.}  If $U^{c} = X- U \in \tau $, then is $U$ closed
    \end{itemize}
\end{definition}


\begin{remark}
Let $X = \left\{ a,b,c \right\}$ and let $U = \left\{ a,b \right\}$. Then if $\tau = \left\{ X, \emptyset  \right\}$,
$U$ is not open nor closed.
\end{remark}

\begin{definition}[Neighbourhoods]
    Let $X$ be a topological space,  $U$ a subset of $X$ and $x \in  X$. We say $ U$ is a neighborhood of $x$ if $ x \in
    U $ and $U$ is open in $X$.
\end{definition}

\begin{theorem}
    \textbf{Continuity between topological spaces.}
Let $X,Y$ be topological spaces. A map $f: X \to Y$ is said to be continious if preimages of open sats are open, i.e.,
if $V$ is an open set in $Y$ then the preimage $f^{-1} \left( V \right)$ of $V$ is open in $X$.
\end{theorem}

\section{ Generating topologies}%
\label{sec:chapter_4_generating_topologies}

\subsection{Generating topologies from subsets}%
\label{sub:generating_topologies_from_subsets}



\begin{theorem}[The intersection of two topologies is a topology]
    Let $X$ be a set, and let $\tau _{1}$ and $\tau _{2}$ be two topologies on $X$. Then $\tau _{1} \cap \tau _{2}$ is
    also a topology on $X$.
\end{theorem}

\begin{definition}[Topology generated by a collection of subsets]
    Let $X$ be a set, and let $ \mathscr{S} $ be a collection of subsets of $X$. The topology generated by
    $\mathscr{S}$ is the topology \[
    \left<\mathscr{S}  \right> = \bigcap_{\substack{\tau  \text{ topology} \\ S \subseteq \tau  } }^{}  \tau
    \]

\end{definition}

\subsection{Basis for a topology}%
\label{sub:basis_for_a_topology}

\begin{definition}[Basis]
    Let $X$ be a set. a \textbf{basis} for a topology on $X$ is a collection $ \mathscr{B} $ of subsets of $X$ such
    that
    \begin{itemize}
        \item \textbf{B1}: for each $x \in X$, there is a $B \in  \mathscr{B} $ such that $x \in B$
        \item \textbf{B1:} if $B_{1}, B_{2}$ and $x \in B_{1} \cap B_{2}$ , then there is a $B_{3} \in  B$ such that $x
            \in  B_{3} \subseteq B_{1} \cap B_{2}$.
    \end{itemize}


\end{definition}



% \begin{definition}[Basis]
%     A \textbf{basis} for a topology on $X$ is a collection of subsets of $X$ s.t.
% \begin{itemize}
%     \item \textbf{B1.} For each $x \in X$ there is a $B \in  \mathscr{B} $ s.t. $x \in  B$.
%     \item \textbf{B2.} $B_{1}, B_{2}$ and $x \in  B_{1} \cap B_{2} $ there is a $B_{3} \in  \mathscr{B}  $ s.t. $x \in
%         B_{3} \subseteq  B_{1} \cap B_{2} $
% \end{itemize}

% If $\mathscr{B}  $ satisfies these, the topology $\tau $ generated by $ \mathfrak{B} $ is defined as follows: A subset
% $U$ of $X$ is said to be open in $X$ if for each $x \in U$, there exists
% $B \in  \mathscr{B} $ such that $x \in B $ and $B \subset U $.

% \end{definition}

\begin{theorem}
Let $X$ be a set, and let $ \mathscr{B} $ be basis for a topology on $X$. The collection  $\tau $ generated by $
\mathscr{B} $ of subsets $U$ of  $X$ with the property that for each $x \in U$ there is a basis element $B \in
\mathscr{B} $ with $ x \in B \subseteq U$ is a topology on $X$.
\end{theorem}

\begin{theorem}
Let $X$ be a set, and let $\mathscr{B} $ be a basis for a topology $\tau $ on $X$. Then  $\tau $ is equal to the
collection of all unions of elements of $ \mathscr{B} $.
\end{theorem}

\begin{theorem}
Let $X$ be a set, and let  $ \mathscr{B} _{1} $ and $\mathfrak{B} _{2}$ be bases for topologies $\tau _{1}$ and $\tau
_{2}$ , respectively , on $X$. Then the following are equivalent.
\begin{enumerate}[label=(\roman*)]
    \item $\tau _{2}$ is finer than $\tau _{1}$, i.e., $\tau _{} \subseteq  \tau _{2}$.
    \item For each $B_{1} \in \mathscr{B} _{1}$ and each $x \in B_{1}$, there is a $B_{2} \in \mathfrak{B} _{2}$ such
        that $x \in  B_{2} \subseteq B_{1}$.
\end{enumerate}
\end{theorem}

\subsection{Subbasis for a topology}%
\label{sub:subbasis_for_a_topology}

\begin{definition}[Subbasis]

    Let $X$ be a set. A \textbf{subbasis } for a topology on $X$ is a collection $ \mathscr{S} $ whose union equals $X$.
\end{definition}

\begin{lemma}
Let $X$ be a set, and let $ \mathscr{S} $ be a subbasis for a topology on $X$. The collection $ \mathfrak{B} $
consisting of all finite intersections of elements of $\mathscr{S} $ is a basis for a topology on $X$ and is called the
basis associated to $ \mathscr{S} $.
\end{lemma}


\begin{definition}[Standard topology]
    (Not in compendium.)
    The standard topology on $ \mathbb{R} $ is the topology generated by a basis consisting of all open intervals of $
    \mathbb{R} $.
\end{definition}

\begin{lemma}
Let $X$ be a set, and let $\mathscr{S} $ be a subbasis for a topology on $X$. The collection $\tau $ generated by $
\mathscr{S} $ consisting of all unions of all basis elements of the associated basis $\mathfrak{B} $ is a topology on
$X$.
\end{lemma}

\begin{theorem}
    Let $X$ be a set, and let $ \mathscr{S} $ be a subbasis for a topology on $X$. Then there exists a unique topology
    $\left<\mathscr{S}  \right>$ generated by $ \mathfrak{S} $ which is smaller than any other topology containing
    $\mathscr{S} $ , where \[
    \left<\mathscr{S}  \right> = \left\{ \bigcup_{\lambda  \in  \Lambda }^{} \bigcap_{i=1}^{n_{\lambda }} S_{\lambda
    ,i}  \mid  S_{\lambda ,i} \in  \mathscr{S}     \right\}
    \]
\end{theorem}

\begin{theorem}
    Let $X$ and $Y$ be topological spaces, and let $ \mathscr{B} $ (resp., $\mathfrak{S} $ ) be a basis (resp.,
    subbasis) . Then a map $f: X \to Y  $ is continious if and only if for each $B \in \mathscr{B} $ (resp. $S \in
    \mathscr{S} $ ) the preimage $f^{-1} \left( B \right)$ (resp., $f^{-1} \left( S \right)$ ) is open in $X$   .
\end{theorem}

\section{ Constructing topological spaces}%
\label{sec:chapter_5_constructing_topological_spaces}

\section{ Topological properties}%
\label{sec:chapter_6_topological_properties}

\subsection{Connected spaces}%
\label{sub:connected_spaces}


\begin{definition}[Connected space]
Let $X$ be a topological space. A \textbf{seperation} of $X$ is a pair of non-empty subsets $U $ and $V$ that are open
in $X$, disjoint and whose union equal $X$. We say that $X$ is \textbf{connected} if there are no seperations of $X$.
Otherwise it is \textbf{disconnected.}
\end{definition}

\begin{theorem}[Closed and open subsets]
    Let $X$ be a topological space. Then $X$ is connected if and only if the are no non-empty proper subsets of $X$ that
    are both open and closed in $X$.

\end{theorem}

\begin{lemma}[Disconnectivity]
    Let $X$ be a disconnected space with seperation $U$ and $V$,  and et $A$ be a connected subspace of $X$. Then $A
    \subseteq U$ and $A \subseteq  V$.

\end{lemma}

\begin{theorem}[Collection connectivity]
    Let $X$ be a topological space,  and let $\left\{ A_{\lambda } \right\}_{\lambda \in \Lambda }$ be a collection of
    connected subspaces of $X$ such that $ \bigcap_{\lambda  \in \Lambda } A_{\lambda }$ is non-empty. Then $\bigcup
    _{\lambda \in  \Lambda } A_{\lambda }$ is connected.

\end{theorem}

\begin{definition}[Path connected space]
    Let $X$ be a topological space, and let $x,y \in X$. A path from $x$ to  $y$ is a continious map:  $f: \left[ a,b
    \right]  \to X$ .t. $f\left( a \right) = x$ and $f\left( b \right) = y$ where $\left[ a,b \right]$ is a subspace of
    $ \mathbb{R} $ with the standard topology. We say that $X$ is \textbf{path connected} if every pair of points of $X
    $ can be joined by a path in $X$.

\end{definition}


\begin{theorem}[ Connectivity in product spaces]
    Let $X_{1},  X_{2}, \ldots, X_{n}$ be connected spaces. Then the product space $X_{1} \times X_{2} \times  \ldots
    \times  X_{n}$ is connected.

\end{theorem}

\begin{theorem}[The real numbers are connected]
    Let $\mathbb{R} $ be the set of real numbers equipped with the standard topology. Then $\mathbb{R} $ is connected.

\end{theorem}

\begin{theorem}[Generalized intermediate value theorem]

    Let $X$ be a connected space and let $f: X \to  \mathbb{R} $ be a continious map where $ \mathbb{R} $ is given the
    standard topology.  If $a,b \in X $ and if $r$ is a real number that lies between $f\left(a  \right) $ and $f\left(
    b\right)$, there is a $c \in  X$ such that $f\left( c \right) = r$

\end{theorem}

\begin{theorem}[Connectivity]
    Let $X$ be a topological space. Then $X$ is connected if and only if the are no non-empty proper subsets of $X$ that
    are both open and closed.
\end{theorem}

\begin{theorem}[Path connectedness implies connectedness]
    Let $X$ be a path connectedness space. Then $X$ is connected.

\end{theorem}
\subsection{Hausdorff spaces}%
\label{sub:hausdorff_spaces}

\begin{definition}[Hausdorff]

    Let $X $ be a topological space. We say that $X$ is \textbf{Hausdorff} if for each part of points $x,y  \in X $ with
    $x\neq y$ , there are disjoint neighborhoods $U$ and $V$ of $x$ and $y$, respectively. In other words, for each pair
    of distinct point $x,y \in  X$ there are open subsets $U$ and $V$ of $X$ with $x \in U$ $y \in V $ where $U \cap V =
    \emptyset $
\end{definition}

\begin{theorem}
    Every metric space is Hausdorff

\end{theorem}

\begin{theorem}
    Let $X$ be a Hausdorff space. Then for each $x \in X$ the subset $\left\{ x  \right\}$ of $X$ is closed in  $X$.
\end{theorem}


\begin{theorem}
    Let $X_{1}, X_{2} \ldots, X_{n} $ be Hausdorff spaces.  Then the product space $X_{1} \times  X_{2}\times  \ldots
    \times X_{n}  $ is Hausdorff.

\end{theorem}


\begin{theorem}
Let $X$ be a topological space.  Then $X$ is Hausdorff if and only if the diagonal \[
    \Delta = \left\{ \left( x,x \right)  \mid  x \in  X \right\}
\]is closed in the product space $X \times  X$.
\end{theorem}

\subsection{Compact spaces}%
\label{sub:compace_spaces}

\begin{definition}[Cover of a space]
    Let $X$ be a topological space, and let $\mathscr{A} $ be the collection of subsets of $X$. We say that  $
    \mathscr{A} $ is a cover of $X$, or  \textbf{X} if $X = \bigcap_{A \in \mathfrak{A} }^{} A$. If $A$ is also open in
     $X$ for each  $A \in  \mathscr{A} $, we say that $\mathfrak{A} $ is an \textbf{open} cover of $X$, or open
     covering of $X$. We say that $ \mathscr{A}' $ is a subcover of $\mathfrak{A} $ if $\mathfrak{A} '  $ is another
     cover of $X$ that satisfies  $ \mathscr{A '} \subseteq \mathfrak{A}  $.

\end{definition}


\begin{definition}[Compact spaces]
    Let $X$ be a topological space. We say that $X$ is \textbf{compact} if every open cover $ \mathscr{A} $ of $X$
    contains a finite subcover.

\end{definition}

\begin{definition}[Compact subspaces]
    Let $X$ be a topological space, and let $A$ be a subset of $X$. We say that $A$ is compact in $X$ if $A $ is compact
    in the subspace topology.
\end{definition}


\begin{lemma}
    Let $X$ be a topological space, and let $A$ be a subspace of $X$.  Then $A$ is compact in $X$ if and only if every
    cover of $A$ by open subsets of $X$ contains a finite subcollection that covers $A$.
\end{lemma}

\begin{theorem}
    Let $X$ be a compact space, and let $A$ be a closed subset of $X$.  Then $A $ is compact in $X$.
\end{theorem}

\begin{theorem}
Let $X$ be a Hausdorff space,  and let $K $ be a subset of $X$ which is compact in $X$. Then $K$ is closed in $X$.
\end{theorem}

\begin{theorem}
Let $X$ be a compact space, $Y$ a topological space and let $f: X \to Y$ be a surjective continious map. Then $Y$ is
compact.
\end{theorem}

\begin{lemma}[Tube lemma]
Let $X$ be a topological space, and let $Y$ be a compact space.  If $x \in  X $ and $U$ is an oppen set in the product
space $X \times Y$ containing $\left\{ x \right\} \times Y$ , then there is a neighborhood $W$ of $x$ in $X$ such that
$W \times  Y \subseteq U$
\end{lemma}


\begin{theorem}
Let $X_{1} , X_{2}, \ldots , X_{n}$ be compact spaces. Then the product space $X_{1} \times  X_{2} \times \ldots\times
X_{n}$ is compact.
\end{theorem}


\begin{theorem}
Let $\mathbb{R} $ be the set of real numbers equipped with the standard topology. Then every closed interval $\left[ a,b
\right] \in \mathbb{R} $ is compact in $ \mathbb{R} $.
\end{theorem}

\begin{definition}[Bounded subsets]

    Let $\left( X,d \right) $ be a metric space, and let $A$ be a subset of $X$. We say that $A$ is bounded if there is
    an $M  \in  \mathbb{R} $ such that $d\left( a_{1}, a_{2} \right) \le M$ for all $a_{1}, a_{2} \in  A$.

\end{definition}

\begin{theorem}[Heine- Borel]
    Let $\mathbb{R} ^{n}$ be given the (Euclidian) metric topology and the Euclidian metric. A subset $A$ of $\mathbb{R}
    ^{n}$ if and only if it is closed and bounded.

\end{theorem}


\begin{theorem}[Generalized extreme value theorem]

    Let $X$ be compact space, and let $f: X \to  \mathbb{R} $ be a continious map where $ \mathbb{R} $ is given the
    standard topology.  Then there are $m, M \in X $ such that \[
    f\left( m \right) \le f\left( x \right) f\left( M \right)
    \]
    for all $x \in   X$
    \textbf{}
    mb
    $\bigcap_{i \in  I}^{} $
\end{theorem}
\section{ The fundamental group}%
\label{sec:the_fundamental_group}




\section{ The fundamental group of the circle}%
\label{sec:the_fundamental_group_of_the_circle}

 Hello I am having a great time \textbf{} \textbf{} \textbf{}



\newpage
\section{References}%
\label{sec:references}

\bibliographystyle{plain}
\bibliography{references}
\end{document}

