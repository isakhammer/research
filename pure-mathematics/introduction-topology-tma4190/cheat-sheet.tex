\documentclass{article}
\usepackage[utf8]{inputenc}

\title{Cheat Sheet}
\author{isakhammer }
\date{2020}

%
%%%% DEPENDENCIES v1.4 %%%%%%

\usepackage{natbib}
\usepackage{graphicx}
\usepackage{amsmath}
\usepackage{amsthm}
\usepackage{amsfonts}
\usepackage{mathtools}
%\usepackage{enumerate}
\usepackage{enumitem}
\usepackage{todonotes}
\usepackage{esint}
\usepackage{float}


\usepackage{hyperref}
\hypersetup{
    colorlinks=true, %set true if you want colored links
    linktoc=all,     %set to all if you want both sections and subsections linked
    linkcolor=blue,  %choose some color if you want links to stand out
}
\hypersetup{linktocpage}


% inscape-figures
\usepackage{import}
\usepackage{pdfpages}
\usepackage{transparent}
\usepackage{xcolor}
\newcommand{\incfig}[2][1]{%
\def\svgwidth{#1\columnwidth}
\import{./figures/}{#2.pdf_tex} } \pdfsuppresswarningpagegroup=1

% Box environment
\usepackage{tcolorbox}
\usepackage{mdframed}
\newmdtheoremenv{definition}{Definition}[section]
\newmdtheoremenv{theorem}{Theorem}[section]
\newmdtheoremenv{lemma}{Lemma}[section]

% \DeclareMathOperator{\span}{span}

\DeclareMathOperator{\atantwo}{atan2}
\DeclareMathOperator{\arctantwo}{arctan2}

\theoremstyle{remark}
\newtheorem*{remark}{Remark}
%\newtheorem{example}{Example}

\newcommand{\newpara}
    {
    \vskip 0.4cm
    }

%%%%%%%%%%%%%%%%%%%%%%%%%%%%%%%%%%%%%%%%%%%%%%%%%%%%%%%%%%%%

%

\begin{document}
\maketitle
\tableofcontents
\newpage

\newpage
\section{Definitions}%
\label{sec:Bases}

\begin{definition}[Topological spaces.] Recall that a topological space is a set $X$ together with a collection $Y$ of subsets
    of $X$ that are open in $X$ s.t.
    \begin{itemize}
        \item \textbf{T1}.  $ \emptyset , X \in  \tau $
        \item \textbf{T2.} $ \tau $ is closed under union if $U_{\lambda } \in  \tau $ for all $\lambda  \in \Lambda $, then \[
                \bigcup_{\lambda  \in  \Lambda }^{} U_{\lambda } \in \tau
        \]
    \item \textbf{T3.} $\tau  $ is under finite intersections if $U_{1}, U_{2}, \ldots, U_{n} \in \tau $ , then \[
    U_{1} \cap U_{2} \cap \ldots \cap  U_{n} \in \tau
    \]

    \end{itemize}
\end{definition}

\begin{definition}[Basis]
    A \textbf{basis} for a topology on $X$ is a collection of subsets of $X$ s.t.
\begin{itemize}
    \item \textbf{B1.} For each $x \in X$ there is a $B \in  \mathfrak{B} $ s.t. $x \in  B$.
    \item \textbf{B2.} $B_{1}, B_{2}$ and $x \in  B_{1} \cap B_{2} $ there is a $B_{3} \in  \mathfrak{B}  $ s.t. $x \in
        B_{3} \subseteq  B_{1} \cap B_{2} $
\end{itemize}

\end{definition}

\begin{definition}
    [Open and closed sets]. Let $\left( X, \tau  \right)$, $U \subseteq X$
    \begin{itemize}
        \item
\textbf{Open set}.  If $ U \in \tau $, then is $U$ open.
    \item \textbf{Closed set.}  If $U^{c} = X- U \in \tau $, then is $U$ closed
    \end{itemize}
\end{definition}


\begin{remark}
Let $X = \left\{ a,b,c \right\}$ and let $U = \left\{ a,b \right\}$. Then if $\tau = \left\{ X, \emptyset  \right\}$,
$U$ is not open nor closed.
\end{remark}

\begin{theorem}
    \textbf{Continuity between topological spaces.}
Let $X,Y$ be topological spaces. A map $f: X \to Y$ is said to be continious if preimages of open sats are open, i.e.,
if $V$ is an open set in $Y$ then the preimage $f^{-1} \left( V \right)$ of $V$ is open in $X$.
\end{theorem}

\begin{definition}[Neighbourhoods]
    Let $X$ be a topological space,  $U$ a subset of $X$ and $x \in  X$. We say $ U$ is a neighborhood of $x$ if $ x \in
    U $ and $U$ is open in $X$.
\end{definition}

\newpage
\section{References}%
\label{sec:references}

\bibliographystyle{plain}
\bibliography{references}
\end{document}

