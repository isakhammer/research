
\documentclass{article}
\usepackage[utf8]{inputenc}

\title{Solutions}
\author{isakhammer }
%
%%%% DEPENDENCIES v1.5 %%%%%%

\usepackage{natbib}
\usepackage{graphicx}
\usepackage{amsmath}
\usepackage{amsthm}
\usepackage{amsfonts}
\usepackage{mathtools}
%\usepackage{enumerate}
\usepackage{enumitem}
\usepackage{todonotes}
\usepackage{esint}
\usepackage{float}

\usepackage{mathrsfs}

\usepackage{hyperref}
\hypersetup{
    colorlinks=true, %set true if you want colored links
    linktoc=all,     %set to all if you want both sections and subsections linked
    linkcolor=blue,  %choose some color if you want links to stand out
}
\hypersetup{linktocpage}


% inscape-figures
\usepackage{import}
\usepackage{pdfpages}
\usepackage{transparent}
\usepackage{xcolor}
\newcommand{\incfig}[2][1]{%
\def\svgwidth{#1\columnwidth}
\import{./figures/}{#2.pdf_tex} } \pdfsuppresswarningpagegroup=1

% Box environment
\usepackage{tcolorbox}
\usepackage{mdframed}
\newmdtheoremenv{definition}{Definition}[section]
\newmdtheoremenv{theorem}{Theorem}[section]
\newmdtheoremenv{lemma}{Lemma}[section]

\DeclareMathOperator{\atantwo}{atan2}
\DeclareMathOperator{\arctantwo}{arctan2}

\theoremstyle{remark}
\newtheorem*{remark}{Remark}
%\newtheorem{example}{Example}

\newcommand{\newpara}
    {
    \vskip 0.4cm
    }

%%%%%%%%%%%%%%%%%%%%%%%%%%%%%%%%%%%%%%%%%%%%%%%%%%%%%%%%%%%%

%

%

\begin{document}
\maketitle
\tableofcontents
\newpage

\newpage
\section{Chapter 4}%
\label{sec:chapter_4}


\subsection{Exercise 4.6}%
\label{sub:exercise_4_6}

Let $\mathscr{B} $ be collection of all subsets on the form $A_{a,b} = \left\{ az + b  \mid  z \in \mathbb{Z}  \right\}$
of $\mathbb{Z} $ , where $a,b \in \mathbb{Z}  $ and $ a \neq 0 $ . (The set $A_{a,b}$ is known as as an arithmetic
progression )

\begin{itemize}
    \item Show that $\mathscr{B} $ is a basis for a topology on $\mathbb{Z} $.
        \begin{tcolorbox}
            \textbf{Answer.}
            \begin{itemize}
                \item \textbf{B1:}  For every $n \in  \mathbb{Z} $, there is an \[
                        a \in \mathbb{Z} \setminus \left\{ 0 \right\}
                \]
                such that \[
                n \in A_{a,n} = \left\{ az + n  \mid  z \in \mathbb{Z}  \right\}
                \]
                Hence, \textbf{B1} holds.
            \item \textbf{B2:} Let $B_{1} = A_{a_{1},b_{1}}$ , $B_{2} = A_{a_{2}, b_{2}}$  be two basis elements.   Let
                $x \in  B_{1 } \cap B_{2}$ . Then  \[
                    \begin{split}
                x \in B_{1}&  = A_{a_{1}, b_{1}} = \left\{ a_{1} z + b_{1} \right\} \\
                & = \left\{ \ldots, -2a_{1} + b_{1}, -a_{1} + b_{1}, b_{1}, a_{1} + b_{1}, \ldots  \right\} \\
                    \end{split}
                \]
                Let $ x \in  B_{2} = A_{a_{2}, b_{2}} = A_{a_{2}, x}$  . Thus if \[
                B_{3} = A_{a_{1}, a_{2}} , x = \left\{ a_{1} a_{2} z + x  \mid z \in  z \right\}
                \]
                then $x \in  B_{3} \subseteq  B_{1} \cap  B_{2}$. Hence \textbf{B2}  holds.
            \end{itemize}


        \end{tcolorbox}

    \item Show that there are infinitely many primes by using the topology generated by $\mathscr{B} $ . (This topology
        is known as the arithmetic progression on $\mathbb{Z} $)
        \begin{tcolorbox}
            \textbf{Answer.} We observe that $A_{a,b}$  is both open and closed: it is clearly open as it is a basis
            element and it is closed since \[
            A^{c} _{a,b} = \mathbb{Z} \setminus A_{a,b}
            \]
            is open: for $x \in A^{c} _{a,b}$ , we have \[
            A_{a,x} \subseteq  A_{a,b}^{c}
            \] .

            \newpara
            Assume there are finitel many primes. Then \[
            \bigcup_{P \text{ primes}}^{} A_{p,a}  = \mathbb{Z} \setminus \left\{ -1,1 \right\}
            \]
            is closed as it is the union of finitely many closed sets.
            Hence, $\left\{ -1,1 \right\}$  must be open which is a contradiction: Every non-empty open set in this
            space is infinite.


            \newpara
            Thus there are infinitely many prims.

        \end{tcolorbox}
\end{itemize}





\newpage
\section*{Chapter 5}%
\label{sec:chapter_5}

\subsection*{Ex 5.1}%
\label{sub:ex_5_1}

$\mathbb{R} $. $\mathbb{R} $  with the standard topology \[
\begin{split}
    X & = \left( a,b \right) \subseteq  \quad  \mathbb{R}  \text{is subpace} \\
    Y &= \left( -1,1 \right) \subseteq \quad   \mathbb{R} \text{is subspace} \\
    X  &  \simeq   Y
\end{split}
\]

Let  \begin{align*}
    f:  X &\longrightarrow  Y \\
     x&\longmapsto fX(x) =2  \frac{ x-a }{b-a}  - 1
.\end{align*}

Then $f$  is a bijectictive continious map with inverse \begin{align*}
    f^{-1}Y: X &\longrightarrow Y \\
    y &\longmapsto f^{-1}Y(y) = a + \left( b-a \right) \frac{y+1}{2}
.\end{align*}
Which is continious. Thus $f$  is a homeomorphism.

\subsection*{Ex 5.2}%
\label{ssub:ex_5_2}

Let \begin{align*}
    g: Y &\longrightarrow \mathbb{R}  \\
    y &\longmapsto g(y) = \tan \left( \frac{\pi}{2}  y \right)
.\end{align*}

Then $g$   is a bijective continious map with inverse \begin{align*}
    g^{-1}: \mathbb{R}  &\longrightarrow Y \\
     t &\longmapsto g^{-1}(t) = \frac{2}{\pi } \arctan \left( t \right)
.\end{align*}

From calculus we know that $g^{-1}$  is continious. Hence, $g$  is homeomorphism \[
x \simeq \mathbb{R}
\]

Let \begin{align*}
    h: X &\longrightarrow \mathbb{R}  \\
    x &\longmapsto h(x) = \left( g \cdot f\left( x \right) \right) = g\left( f\left( x \right) \right) \\
 &  \downarrow \\
    g\left( f\left( x \right) \right) &=  \left( \frac{2\left( x-a \right)}{b-a} -1  \right)  \\
    &= \tan \left( \frac{\pi }{2} \left( \frac{2\left( x-a \right)}{b-a} -1 \right) \right) \\
.\end{align*}

Then $h$  is a homeomorphism as it is the composition of $f$  and $g$ .


\subsection*{Ex 5.3}%
\label{ssub:5_3}

\begin{itemize}
    \item $X$  be topological space.
    \item Let $Y \subseteq X$,    $A \subseteq Y$  be subsets .
    \item
 $\tau _{X_{A}} $  subspace topology on $A$  inherited from $X$ .
\item  $\tau _{Y_{A}}$ subspace topology on $A$  inherited from $Y$ .
\end{itemize}

Let $\tau $  be the topology on $X$, and let $\tau _{Y}$  be the subspace topology on $Y$ . Thus \[
\begin{split}
   \tau_{Y} &=  \left\{ Y \cap U  \mid U \subseteq X \text{ is open}   \right\}\\
   \tau _{X_{A}} &=  \left\{ A \cap V  \mid V \subseteq X \text{ is open} \right\} \\
   \tau _{Y_{A}}  &  = \left\{ A \cap  \mid w \in \tau _{Y} \right\} \\
\end{split}
\]

\begin{enumerate}[label=(\roman*)]
    \item $\tau _{X_{A}} \subseteq  \tau _{Y_{A}}$ : Let $A\cap V \in \tau _{X_{A}} $. Then \[
    Y \cap V \in \tau _{Y}
    \]
    and so \[
    A \cap V \in \tau _{Y_{A}}
    \]
    Hence  $\tau _{X_{A}} \subseteq  \tau _{Y_{A}}$

\item $\tau _{Y_{A}}  \subseteq  \tau_{X_{A}}  $: Let $A \cap W \in \tau _{Y_{A}}$. Then there is a $V \in \tau $ s.t.
   \[
   W = Y \cap V
   \]
   Hence, \[
       \begin{split}
       A \cap  W &= A \cap \left( Y \cap V \right) \\
       &= A \cap V \in \tau _{X_{A}} \\
       \end{split}
   \]
\end{enumerate}

$\tau _{X_{A}} = \tau _{Y_{A}}$


\subsection*{Ex 5.4}%
\label{sub:ex_4}

Let $\emptyset  \subseteq  \mathbb{R} $ be a subspace. \[
A = \left\{ x \in  \emptyset   \mid  - \sqrt{5}  < x < \sqrt{5}  \right\}
\]

\textbf{Fact.}
\begin{enumerate}[label=(\roman*)]
    \item $X $  topological space
    \item $S \subseteq X$  subspace
\end{enumerate}

$K \subseteq S$  is closed \[
\implies
\]
There is a $L \subseteq  X$  closed in $X$  with $K = S \cap L$


\begin{proof}
    \begin{enumerate}[label=(\roman*)]
        \item $\implies :$  Assume that $K \subseteq  S$  is closed in $S$ . Then \[
        V  = S \setminus K
        \]
        is open in $S$ . Thus there is some $U \subseteq  X$  open with \[
        V = S \cap U.
        \] Furthermore \[
        L = X \setminus  U
        \]
        is closed in $X$. Also \[
            \begin{split}
        S \cap  L  & = S \cap \left( X \setminus U \right) = S \setminus \left( S \cap U \right) \\
        &= S \cap V = K \\
            \end{split}
        \]
    \item $\impliedby $: By asumption. $U = X \setminus L$ is open in $X$. Thus $V = S\cap U $ is open in $S$ . Since \[
            \begin{split}
            S \setminus K &= S \setminus \left( S \cap L \right)  \\
            &= S \cap \left( X \setminus L \right) \\
            &= S \cap U \\
            &= V \\
            \end{split}
    \]
    Thus $K$  is closed. Since $\left( - \sqrt{5}, \sqrt{5}   \right) \subseteq \mathbb{R} $ is open and \[
    A = \left( -\sqrt{5} , \sqrt{5}  \right) \cap \mathbb{Q}
    \]
    $A $  is open in $\mathbb{Q} $ . Similarly as \[
    \left[ - \sqrt{5} , \sqrt{5}  \right] \subseteq \mathbb{R}
    \]
    is closed and  \[
    A = \left( -\sqrt{5}  , \sqrt{5}  \right) \cap  \mathbb{Q}
    \]
    $A$  is closed in $\mathbb{Q} $ .
    \end{enumerate}
\end{proof}


\subsection*{5.5}%
\label{sub:5_5}

$X,Y$ topological spaces. $A \subseteq  X, B \subseteq Y$  subspaces. $\tau _{A \times B}$  product topology on $A
\times B$ . $\tau _{\left( X \times Y \right)_{A \times  B}}$  the subspace topology on $A \times  B$  inherited from $X
\times  Y$ .
\newpara
We will show that \[
\tau _{A \times  B} = \tau _{\left( X \times Y \right)_{A \times B}}.
\]

Let $\tau _{A}$ be the subspace topology on $A$ , i.e., \[
\tau _{A} = \left\{ A \subseteq  U  \mid  U \subseteq  X \text{ is open} \right\}
\]
Similarly , \[
\tau _{B} = \left\{ B \subseteq  V  \mid  V \subseteq Y \text{ is open} \right\}
\]

Let $\mathscr{B} _{X}$ be the basis for the topology on $X$ . Let $\mathscr{B} _{Y}$ be the basis for the topology on
$Y$ .  Then \[
\mathscr{B}_{A \times B} = \left\{ \left( A \subseteq U _{X} \right) \times \left( B \cap U_{Y} \right)  \mid  U_{X} \in
\mathscr{B} _{X}, U_{Y} \in  \mathscr{B}  _{Y}\right\}
\] is a basis for $\tau _{A \times  B}$ . From the fact that \[
\mathscr{B}  _{X \times Y} = \left\{ B_{x} \times  B_{Y}  \mid  B_{X} \in  \mathscr{B} _{X}, B_{Y} \in \mathscr{B} _{Y} \right\}
\]
is a basis for $\tau _{X \times Y}$ , it follows that \[
\mathscr{B} _{\left( X \times Y \right)_{A \times B}} = \left\{ \left( A \cap B_{X} \right) \cap \left( B_{x} \times
B_{Y} \right)  \mid  B_{X} \in \mathscr{B} _{X} , B_{Y} \in \mathscr{B} _{Y} \right\}
\]
is a basis for $\tau _{\left( X \times Y \right)_{A \times B}}$ . Since \[
\left( A \times B \right) \subseteq \left( B_{X} \times  B_{Y} \right) = \left( A \cap B_{X}  \right) \times  \left( B
\subseteq B_{Y} \right)
\]
We have \[
    \mathscr{B} _{A\times B} = \mathscr{B} _{\left( X \times Y \right)_{A \times B}}
\]

is a basis for $\tau _{A  \times B}$.

\subsection*{5.6}%
\label{sub:5_6}

$X,Y$  topological spaces  Let \[
    \begin{split}
\pi _{1}:&  X \times Y \to X \\
\pi _{2} : &  X \times Y \to y
    \end{split}
\]
be projection maps. Let $\tau _{X \times Y}$ be the product topology $X \times Y$. By definition of the product
topology, $\tau _{1}$  and $\tau _{2}$  are contiions.

\newpara

Assume that $\tau $  is some topology on $X \times  Y$ s.t. $\pi _{1}$  and $\pi _{2}$ are continious. Then, \[
\begin{split}
    \pi ^{-1}\left( U \right) &=  U \times Y \in \tau  \\
    \pi _{2}^{-1} \left( V \right) &=  X \times  V \in  \tau \\
\end{split}
\]

For $U \subseteq X$ is open, $V \subseteq Y$is open. Since $\tau $ is a topology \[
    \left( U \times Y \right) \cap \left( X \times Y \right) = U \times V \in  \tau
\]
Hence ,
\[
\tau _{X \times Y} \in \tau
\]

\subsection*{Ex 5.8}%
\label{sub:ex_5_8}

Let
\begin{itemize}
    \item $\mathbb{R} :$   $\mathbb{R} $   with standard topology.
    \item $\pi $   $\mathbb{R} \to \mathbb{Z} $
    \item \begin{align*}
        x: x, \quad  & x \in X  \\
          n , \quad&   x \in \left( n-1, n+1 \right) , \quad  n \text{ odd integer}
    .\end{align*}
\end{itemize}


\[
\tau ^{\pi } = \left\{ U \subseteq \mathbb{Z} , \quad \pi ^{-1} \left( U \right) \text{ is open in }\mathbb{R}  \right\}
\]

For $n$  an odd integet, we have \[
    \pi ^{-1}\left( \left\{ n \right\} \right) = \left( n-1, n+1 \right) \subseteq \mathbb{R}
\]
is open. For $n$  an even integer , $\pi ^{-1} \left( \left\{ n \right\} \right) = \left\{ n \right\}$ which is not
open.
\newpara

The smallest open subset of $\mathbb{Z} $ that contains $n$  is $\left\{ n-1, n, n+1 \right\}$ as \[
\pi ^{-1} \left( \left\{ n-1, n, n+1 \right\} = \left( n-2, n+2 \right) \right)
\]
is open in $\mathbb{R} $ .

Hence, $\tau ^{\pi }$  is the same as the digital line topology.





\newpage

\section{References}%
\label{sec:references}

\bibliographystyle{plain}
\bibliography{references}
\end{document}

