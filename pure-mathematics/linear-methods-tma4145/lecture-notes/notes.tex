\documentclass{article}
\usepackage[utf8]{inputenc}

\title{Linear Methods Lecture}
\author{isakhammer }
\date{2020}

\usepackage{natbib}
\usepackage{graphicx}
\usepackage{amsmath}
\usepackage{amsthm}
\usepackage{amsfonts}
\usepackage{mathtools}
\usepackage{enumerate}
\usepackage{todonotes}


\usepackage{hyperref} 
\hypersetup{
  colorlinks=true, %set true if you want colored links
  linktoc=all,     %set to all if you want both sections and subsections linked
  linkcolor=blue,  %choose some color if you want links to stand out
} 
\hypersetup{linktocpage}


% inscape-figures
\usepackage{import}
\usepackage{pdfpages}
\usepackage{transparent}
\usepackage{xcolor}
\newcommand{\incfig}[2][1]{%
\def\svgwidth{#1\columnwidth}
\import{./figures/}{#2.pdf_tex} } \pdfsuppresswarningpagegroup=1

% Box environment
\usepackage{tcolorbox}
\usepackage{mdframed}
\newmdtheoremenv{definition}{Definition}[section]
\newmdtheoremenv{theorem}{Theorem}[section]
\newmdtheoremenv{lemma}{Lemma}[section]

\theoremstyle{remark}
\newtheorem*{remark}{Remark}
\newtheorem{example}{Example}


\begin{document}
\maketitle
\tableofcontents
\newpage

\newpage
\section{Lecture 1}%
\label{sec:lecture_1}

\subsection{Set Theory}%
\label{sub:set_theory}

\begin{definition}
  A \textbf{set} is a collection of distinct objects, its elements. \[
  x \in X \quad  x \text{ is a element of the set } X 
  \] 
  and similary 
  \[
  x \not\in X \quad  \text{ x is not an element of X} 
  \] 

  \par
   Two sets are identical $X=Y$ , if \[
   x \in X \leftrightarrow x \in Y
   \] 
   for any element  $x$ .
\end{definition}

\begin{definition}
  $Y$ is a subset of $X$, $Y \mathbb{C}  X$ if for all $y \in X$. If $Y \subset X $ and $Y \neq X$, we write $y \subset X$ (or $Y \not \subset X$). $Y$ is then a proper subset of $X$ .
  Showing to sets are equal, 
  \begin{itemize}
    \item $x \in X \leftrightarrow x \in Y$
    \item $x \subset Y$ and $ y \subset X$
  \end{itemize}
  The empty set are denoted by null.
\end{definition}

\begin{example}
  \begin{itemize}
    \item $\mathbb{N}  = \left\{ 1,2,3,4,5, \ldots \right\}$
    \item $\mathbb{Z}  = \left\{ \ldots, -1,0,1,\ldots \right\}$
    \item $\mathbb{Q}  = \left\{ \frac{p}{q}: p,q \in \mathbb{Z} , q \neq0 \right\}$ 
    \item $\mathbb{R}  = \text{reals}$ 
    \item $\mathbb{C}: \text{Complex numbers} \quad  a + ib  $ 
    \item Finite set $\left\{ 3,4,5,6 \right\}$ 
    \item Intervals in $\mathbb{R} $ For real numbers $a < b < \infty$
      \begin{align*}
        & (a,b)\\
        & \left[ a,b \right] \\
        & (a,b] , \quad  [a,b) 
      .\end{align*}
  \end{itemize} 
\end{example}


\begin{definition}
  Let $X$ and $Y$ be two sets then
  \begin{itemize}
    \item Union. $X \cup Y = \left\{ z  \mid z \in X \quad \text{or} \quad  z \in Y     \right\}$ 
      \[
      \bigcup_{i \in  I}  X_{i} = \left\{ z  \mid z \in X_{i} \quad   \text{ for some } \quad   i \in I \right\}
      \] 
    \item Intersection if $\bigcap_{i \in  I}  = \{ z  \mid z \in X_i \quad \text{For every} \quad  i \in I  \} $
    \item Complement if $S$ is a subset of $X$ , then the complement of $S$ is \[
    X \setminus S = S^{c} = \{ x \in X: x \not\in S\} .
    \] 
  \item Cartesian product \[
  X \times  Y = \{ \left( x,y \right) : x \in X , \quad   y \in Y\}  
  \] 
  \end{itemize}
\end{definition}


\begin{lemma}
  \begin{itemize}
    \item $x \cap \left( Y \cup Z \right) = \left( X \cap Y \right) \cup \left( X \cap Z \right) \quad   $  and \[
    X \cup \left( Y \cap Z \right) = \left( X \cup Y \right) \cap \left( X \cup Z \right)
    \] 
  \item $\left( X \cup Y \right)^{c} = X^{c} \cap Y^{c}$
  \item $\left( X \cap Y \right)^{c} = X^{c} \cup Y^{c}$
  \item Demo organs law 

    \begin{align*}
      X \setminus \left( Y \cup Z \right) &= \left( X \setminus Y\right) \cap \left( X \setminus Z \right) \\
    .\end{align*}
  \item $\left( X^{c} \right)^{c} = X$
  \end{itemize}
\end{lemma}

\begin{proof}
  Proof of $\left( X \cup Y \right)^{c} ) =  X^{c} \cap Y^{c}$
  \begin{equation*}
    \begin{split}
      x \in \left( X \cup Y \right)^{c} & \to x \in X \cup U\\
       & x \not\in X \quad \text{and} \quad  x \not\in Y \\
       & x \in X^{c} \quad \text{and} \quad  x \in Y \\
       & x \in X^{c} \cap Y^{c}
    \end{split}
  .\end{equation*}
\end{proof}

\subsection{Functions}%
\label{sub:functions}

Let $X,Y$ be sets. A function $f$ from $X$ to $Y$, denoted $f: X \to Y$ , is defined by a set $G$ of ordered pairs $\left( x,y \right) $, where $x \in X, \quad  y \in Y  $ and with the property that;
\par
For each set is there a unique $y \in Y \quad  \text{ s.t.} \quad  \left( x,y \right) \in G $. We write $f\left( x \right) = y$. 
\begin{itemize}
  \item We say that $X$ is the domain and $Y$ is the codomain.
  \item The (direct) image of a set $ A \subset X$ under f is \[
  f\left( A \right) = \{f\left( t \right): t \in A\} \subset Y
  \] 
\item The \textbf{inverse image}  of a set $B \subset Y$ under f is \[
    f^{-1} \left(  B \right)  = \{x \in X  \mid f\left( x \right) \in B\} \subset X
\] 
\item The \textbf{range} if $f$ is the image of its domain $X$ is \[
    ran\left( f \right) = f\left( X \right) = \{f\left( t \right): t \in X\} 
\] 
\end{itemize}

\begin{example}
  Let $f: \mathbb{R}  \to \mathbb{R} $ given by \[
  f\left( x \right) = max\left\{ x,0 \right\} = x^{+}
  \] 
  Then is the $ran\left( f \right) = [0, \infty)$.  The inverse is $f^{-1} \left( \left\{ y \right\} \right) = \left\{ y \right\}$ and $f^{-1}\left( \left\{ 0 \right\} \right) = (- \infty , 0]$  and \[
  f^{-1} \left( \left\{ y \right\} \right) = \text{NULL} \quad \text{if} \quad  y < 0  
  \] 
\end{example}


\begin{definition}
  Let $f: X \to Y$ be a function
  \begin{itemize}
    \item $f$ is \textbf{injective}  or \textbf{one-to-one}  if $f\left( x_{1} \right) \to x_{1} = x_{1}$ 
    \item $f$ is \textbf{surjective}  or \textbf{onto}  if $ran\left( f \right) = y$ 
    \item $f$ is \textbf{bijective}  if it is both surjective and injective.
  \end{itemize}
\end{definition}

\begin{example}
  Lets continue the example.
  \begin{itemize}
    \item
  Let $f: \mathbb{R} \to \mathbb{R} $ , $f\left( x \right) = max \left\{ x,0 \right\}$. Injective? No; $f\left( x_{1} \right) = \underbrace{f\left( x_{2} \right)}_\text{$= 0$}  $ for any two $x_{1}, x_{1} < 0$ .
\item  A \textbf{bijection}  $f:  X \in Y$ has a \textbf{inverse}  function $f^{-1} : Y \to X$, defined by $f^{-1} \left( y \right) = x$ if $f\left( x \right) = y$ . \par
  THe inverse function $f^{-1} $ is also a bijection. 
  \end{itemize}
\end{example}
\begin{remark}
  Not to be confused with the inverse image of a set $f^{-1} \left( B \right) $ introduced earlier.
\end{remark}

\newpage
\section{Lecture 2}%
\label{sec:lecture_2}

\subsection{Recall}%
\label{sub:recall}

Let $f: X \to Y$ then is 
\begin{itemize}
  \item[i)] Inective: $f\left( x_{1} \right) = f\left( x_{2} \right) \to \quad x_{1} = x_{2} $ 
  \item [ii)] Surjective:  For all $y$ in $Y$ there is a $x$ in $X$ s.t. $f\left( x \right) = y$. 
  \item [iii)] Bijective if i) and ii) holds.
\end{itemize}


\begin{itemize}
  \item If $F: X \to Y$ is a bijective then there is an inverse \[
      f^{-1} : Y \to X
  \] 
  Given by \[
  f^{-1} \left( y \right) = x \quad  \text{if} \quad  f\left( x \right)   = y
  \] 
\item Identify function/map 
  \begin{itemize}
    \item id: $X \to X$
    \item $id_{x}\left( x \right) = x$ for all $x \in X$
  \end{itemize}
\item The composition of a function \[
    g: Y \to Z \quad  \text{with} \quad  f: X \to X  
\] is the function $g\cdot f : X \to Y$   defined by \[
\left( g\cdot f \right)\left( x \right) = g\left( f\left( x \right) \right) \quad  \text{for} \quad  x \in X  
\] 
\end{itemize}

\begin{definition}
  Anternative version. Given a bijection $f: X\to Y$ the inverse function $f^{-1}: Y \to X$ is the unique function satisftying $f^{-1} \cdot  f = id_{x}$ and $f\cdot f^{-1} = id_{y} $
\end{definition}
\begin{example}
  $\frac{d }{d x}: C^{1}\left( \mathbb{R} , \mathbb{R}  \right) \to C \left( \mathbb{R} ,\mathbb{R}  \right) $. Inverse? no. 
  \par
  Let $g \in C^{1} \left( \mathbb{R} ,\mathbb{R}  \right)$. Then is \[
  \frac{d \left( g + c \right)}{d x} = \frac{d g}{d x} \quad \text{where c is the constant.} 
  \] 
  It is surjective because given any $f \in C\left( \mathbb{R} ,\mathbb{R}  \right)$
  we can define $F \in C^{1}\left( \mathbb{R} ,\mathbb{R}  \right)$ by \[
  F: X \to \int_{0}^{x} f\left( t \right)dt
  \] 
  and \[
  \frac{d F}{d x} = f \quad  \text{fundamental theorem of calculus.} 
  \] 
\end{example}

\subsection{Cardinality}%
\label{sub:cardinality}

Cardinality is a tool for comparing the sizes of sets. 
\begin{definition}
  We say that two sets $A$ and $B$ has the same cardinality if there exist a bijection between $A$ and $B$. 
\end{definition}

\begin{tcolorbox}
  \textbf{Example.}
  \par
  \begin{itemize}
    \item [i)] The two inervals $\left[ 0,2 \right]$ and $\left[ 0,1 \right]$ have the same cardinality. \[
    \begin{split}
       &  f:\left[ 0,2 \right] \to \left[ 0,1 \right]  \\
        &  f\left( t \right) = \frac{t}{2}
    \end{split}
    \] 
  \item [ii)] Let $\mathbb{N}  = \{ 1,2,3,4, \ldots \} $ and $\mathbb{N} \setminus \{1\} = \{2,3,4,5, \ldots\}  $ have the same cardinality \[
  f\left( n \right) = n+1
  \] 
\item [iii)] $n$ is finite integer. Then there is no bijection \[
    f: \{1,2,3, \ldots , n\}  \to \mathbb{N} 
\] 
These two sets \textbf{do not}  have the same cardinality. 
  \end{itemize}

\end{tcolorbox}

\begin{definition}
  Let $X$ be a set. We say $X$ is \textbf{finite} if either $X = \text{NULL} $ or there exist $n \in \mathbb{N} $ s. T. $X$ has the same cardinality as $\{ 1,2,3,4, \ldots, n\} $ if \[
 \text{ There exist} \quad  f: \{1,2,3, \ldots m b\}  \to X \quad \text{for some} \quad  n    
  \] 
  $X$ is \textbf{infinite }  if it is not finite.
\end{definition}

\newpage 
\begin{definition}
  A set $X$ is 
  \begin{itemize}
    \item Countable infinite if it has the same cardinality as $\mathbb{N} $. \[
    \exists \text{bijection} \quad  f: X \to \mathbb{N}  
    \] 
  \item Countable if it is either countably infinite or finite. or equivalently 
    \begin{itemize}
      \item if $\exists$ injection $f: X \to \mathbb{N} $ 
      \item $\exists$ surjection $f: \mathbb{N}  \to X$
    \end{itemize}
  \item Uncountable if it is not countable.
  \end{itemize}
\end{definition}

\begin{tcolorbox}
  \textbf{Example.} 
  \begin{itemize}
    \item Any finitie set is, e.g. $\{ 2,5,9\} $ 
    \item $X = \{ 1,4,9,16, \ldots , n^2 , \ldots\} $  such that \[
    f: \mathbb{N}  \to X, \quad  f\left( n \right) = n^2 
    \] 
  \item $\mathbb{N} \times  \mathbb{N}  $ is countable ; \par
    We arrange $N\times N $ in a table. \[
    \begin{split}
         f: \mathbb{N}  &\to \mathbb{N} \times  \mathbb{N}  \\
        f\left( 1 \right) &=  \left( 1,1 \right) \\
        f\left( 2 \right) &=  \left(2,1  \right)  \\
        f\left( 3 \right) &=  \left( 1,2 \right) \\
        f\left( 4 \right) &=  \left( 3,1 \right) \\
        \vdots   
    \end{split} 
    \]  
  \item $\mathbb{Z} $ and $\mathbb{Q} $ are countable (Prob set 1).
  \item If $X$ and $Y$ are countable, then so is $X \cup  Y$ .
  \end{itemize}
\end{tcolorbox}

\subsection{Schroeder Bernstein Theorem}%
\label{sub:schroeder_bernstein_theorem}

Let $X$ and $Y$ by two be two sets. Suppose there are injective maps $f: X \to Y$ and $g: Y \to X$. Then there exists a bijection between $X$ and $Y$. 

\begin{tcolorbox}
  \textbf{Example.} The interval $\left( 0,1 \right) \subseteq  \mathbb{R} $. Claim it is uncountable.  
  \begin{proof}
    The Cantor diagonalization argument. Suppose that $\left( 0,1 \right) $ is countable. \[
      \begin{split}
        \left( 0,1 \right) = &\{x_{1}, x_{2}, x_{3}, x_{4} , \ldots\} \\
         & f\left( 1 \right), f\left( 2 \right) , f\left( 3 \right), \ldots\\
         \\
      f: \mathbb{N}  & \to \left( 0,1 \right) \\
      x_{i} = 0,  & x_{i1}, x_{i2} , x_{i3}, \ldots 
      \end{split} 
    \] 
    Now let \[
    a = 0, a_{1}, a_{2}, a_{3}, a_{4}, a_{5}, \ldots
    \] 
    where \[
    a_{i} = \begin{cases}
      3 &  \text{if} \quad   x_{ii} \neq 3 \\
      1  & \text{if} \quad   x_{ii} =3  
    \end{cases}
    \] 
    Then $a_{i} \neq x_{ii} $ , so by construction $a \neq x_{i}$ for all $i$. Moreover, we must have $a \in \left( 0,1 \right)$. This is a contradiction, so $\left( 0,1 \right) $ cannot be countable. 
  \end{proof}
\end{tcolorbox}


\begin{tcolorbox}
  \textbf{Example.} The set of all binary sequences $ X = \left\{ \left( x_{1}, x_{2}, x_{3} , \ldots \right) \right\} : \quad  x_{i} \in \left\{ 0,1 \right\}  $
  is uncountable . 

  \begin{proof}
    Problem set 2.
  \end{proof}
\end{tcolorbox}

\newpage
  \begin{lemma}
    Let $X$ and $Y$ be sets. Then 
    \begin{itemize}
      \item If $X$ is countable and $ Y \subseteq  X$ , then $Y$ is also countable. \[
      \left\{ 1,2,3,4,5, \ldots \right\} \to \{x_{1}, x_{2} , x_{3}, x_{4} , \ldots\} 
      \] 
    \item If $X$ is uncountable and $X \subseteq  Y$, then $Y$ is uncountable. 
    \item If $X$ is countable and there is an injection \[
    f: Y \to X   
    \] 
    then $Y$ is countable.
  \item If $X$ is uncountable and \[
  \exists \quad  \text{injective} \quad  f: X \to Y,   
  \] 
  then $Y$ is uncountable.
    \end{itemize}
  \end{lemma}
   \begin{tcolorbox}
     \textbf{Example.} Have proved formally that $\left( 0,1 \right) \subseteq  \mathbb{R} $ is countable $\overbrace{\to}^\text{ii)}  \mathbb{R} $ must be uncountable \[
       R \subset \mathbb{C}  \overbrace{\longrightarrow}^\text{ii)} \mathbb{C} \quad \text{is uncountable} 
     \] 
   \end{tcolorbox}

   \begin{tcolorbox}
     \textbf{Example.} $R = \mathbb{Q}  \cup \mathbb{I}$. Know: $\mathbb{Q} $ countable.  \par
     Assume $\mathbb{I}$ countable. Then $R \cup \mathbb{I}$ which is a contradiction. So $\mathbb{I}$ is uncountable
   \end{tcolorbox}

\newpage
\section{Lecture 3}%
\label{sec:lecture_3}

\subsection{Sequences}%
\label{sub:sequences}

Fixed set $J$ and set $X$ with elements $x_{j} \in X $ for $j \in J$. $J$ is a \textbf{index set} , $x_{j}$ is the $j$-th component of the sequence $\left\{ x_{j \in J} \right\}_{j}$.  

\begin{remark}
  $\left( x_{j} \right) $ is equivalent to $\left( x_{j} \right)_{j}$. More technically $x: J \to  X$ s.t. $x_{\left( j \right)} = x_{j}$.
\end{remark}

\subsection{Infima and Suprema}%
\label{sub:infima_and_suprema}

\begin{definition}
  Suppose $A \subseteq \mathbb{R} $ is nonempty.
  \begin{enumerate}
    \item $A$ is \textbf{bounded}  if  \[
    \exists \quad   M \in  \mathbb{R}  \quad \text{s.t.} \quad a \le M \quad \text{for all} \quad a \in  A    
    \] 
  \item $A$ is \textbf{bounded below}  if \[
  \exists \quad   m \in \mathbb{R} \quad \text{s.t.} a \ge m \quad \text{for all } a \in  A  
  \] 
\item $A$ is \textbf{bounded}  is $1.,2.$
\item  $v$ is a \textbf{maximal element } of $A$ if $v \in A$ and $a \le v$ for every $a \in  A$. We write $v = \max \left( A \right)$
\item $v$ is a \textbf{minimal element }  of $A$ if $v \in  A$ and $a \ge v$ for every $a \in  A$. We write $v = \min \left( A \right)$
  \end{enumerate}
\end{definition}

\begin{definition}[Infimum and supremum]
  Suppose $A \subseteq  \mathbb{R} $ is nonemtpy. 
  \begin{enumerate}
    \item We say that $M \in \mathbb{R}  $ is the \textbf{ supreme} or \textbf{least upper bound} of $A$ if 
      \begin{enumerate}
        \item $M$ is a upper bound of $A$ , i.e. $ a \le M$ for every $a \in  A$. 
        \item All other upper bounds $M^{'} $ of $A$ satisfied $M^{'} \ge M$. We write $M = sup \left( A \right)$ (and if it exists a max element $u \in  A$, then $u = sup\left( A \right) = max \left( A \right)$
      \end{enumerate}
    \item $m \in  \mathbb{R} $  is the \textbf{infimum} or the \textbf{greated lower bound}  of $A$ if 
      \begin{enumerate}
        \item It is a lower bound, $a \ge m \forall a \in  A$ 
        \item All other lower bounds $m^{'}$ are smaller $m ^{'} < m$
      \end{enumerate}
  \end{enumerate}
  
\end{definition}
\begin{tcolorbox}
  \textbf{Example.} \[
  A = \begin{pmatrix}
  0  & 1
  \end{pmatrix} 
  \to  
  \begin{cases}
     & inf\left( A \right) = 0 \\
      &  sup\left( A \right) = 1
  \end{cases}
  \] 
\end{tcolorbox}


\begin{remark}
  \begin{itemize}
    \item If $A \subset \mathbb{R} $ is not bounded from above, we write $sup \left( A \right) = \infty$
    \item If $A \subset  \mathbb{R} $ is not bounded below,  we write $inf \left( A \right) = - \infty$
  \end{itemize}
\end{remark}

\begin{lemma}
  $A \subseteq  \mathbb{R} $  is nonemtpy. 
  \begin{enumerate}
    \item Say $A$ is bounded above. Then $M \in  \mathbb{R} $ is the sup of $A$ if 
      \begin{enumerate}
        \item $a \ll  M \quad  \forall \quad  a \in  A  $ 
        \item $\forall \epsilon > 0 \quad  \exists \quad   a \in  A \quad  \text{s.t.} \quad a > M  - \epsilon    $
      \end{enumerate}
    \item Say $A$ is bounded from below.  Then $m \in  \mathbb{R} $ is the inf of $A$ if 
      \begin{enumerate}
        \item $a \ge m \quad  \forall \quad  a \in  A  $
        \item $ \forall \epsilon  > 0 \quad  \exists a \in  A \quad  \text{s.t.} \quad  a < m +\epsilon    $
      \end{enumerate}
  \end{enumerate}
\end{lemma}
   
\begin{tcolorbox}
  \textbf{Example.} Let $A = \left\{ \frac{1}{n} : n \in \mathbb{N}   \right\} = \left\{ 1, \frac{1}{2} , \frac{1}{3} , \ldots \right\}$ then is 
  \begin{enumerate}
    \item $\inf \left( A \right) = 0$, since $\frac{1}{n} \ge 0 \quad  \forall n \in  \mathbb{N}  $, and for any $\epsilon  > 0$ we can find $N$ s.t. $\frac{1}{N} < \epsilon $
    \item $sup\left( A \right) = 1$ , since $\frac{1}{n} \le 1$ for all $n \in  \mathbb{N} $ and for any $\epsilon  > 0$ we have $1 > 1- \epsilon $ which concludes \[
    \max \left( A \right) = sup \left( A \right) = 1
    \] 
  \end{enumerate}
\end{tcolorbox}

\begin{itemize}
  \item From our definitnion, it follows that \[
      \inf \left( A \right) \ge sup \left( A \right)
  \] 
\item If $A = \left( a_{n} \right)_{n}$  then we usually write \[
{sup}_{\cap } \left( a_{n} \right)
\] 
\item If we have a function $f: X \to  Y$ then \[
    {sup}_{x} f = sup \left\{ f\left( x \right) : x \in  X \right\}
\] 
\end{itemize}

\begin{definition}[Dilate Set]
We define the \textbf{dilate}  by $ c \in  \mathbb{R} $ of a set $A \subseteq  \mathbb{R} $  by \[
cA = \left\{  b \in  \mathbb{R} : \quad  b = ca , \quad  a \in A   \right\}
\] 
\end{definition}

\begin{lemma}[Properties of dilates, subsets, sums]
  Let $A, B \subseteq  \mathbb{R} $ be nonempty and bounded. 
  \begin{enumerate}
    \item if $c > 0$ , then $\sup cA = c \sup A$  and $\inf cA = c \inf A$
    \item If $ c < 0$ , then $\sup  cA = c \inf A$  and $ \inf cA = c\sup A$
    \item $\sup \left( A +B \right) = \sup A + \sup  B $  and $\inf \left( A + B \right) = \inf A + \inf B$
    \item If $ B \subset  A$ , then is $ \inf B \ge \inf A$ and $\sup B \le \sup A$
   \end{enumerate}
  
\end{lemma}

\begin{proof}
  We want to show that $\sup cA = c \sup A$ for $c > 0$.  Let $\sup  A = M$. Then is $\forall a \in  A$ , $a \le M \implies  ca \ge cM$ and $\sup  cA \le cM$. Moroever, for every $\epsilon  > 0$  does exist $a \in  A$ s.t. $a \ge M - \frac{\epsilon}{c}$. This can be rewritten such that \[
  \begin{split}
     ca   & \ge cM - \epsilon    
     \implies  \sup cA =  cM 
  \end{split} 
  \]  
\end{proof}
 \begin{tcolorbox}
   \textbf{Example.} Let $X = \left\{ g \in  C\left[ 0,2 \right]: \left| g \right| < M \right\}$ and  \[
   \begin{split}
       & f: X \to  \mathbb{R}   \\
        & g: \to  \int_{0}^{2}  g\left( x \right)dx \\
           \sup _{x}f &= \sup  \left\{ f\left( g \right) : g \in  X \right\}  \\
         &=  \sup \left\{ \int_{0}^{2}  g\left( x \right)dx:  g \in  X  \right\} 
   \end{split} 
   \] 
   We can show that \[
     \int_{0}^{2}  g\left( x \right)dx  \le \overbrace{ \sup _{x \in  \left[ 0,2 \right] } g\left( x \right) }^{< M} \underbrace{\int_{0}^{2} dx }_{= 2} \le 2M
   \] 
   Claim that $\sup _{x} f = 2M$. And then is the task: For any $ \epsilon  > 0$ , find $g \in X$ s.t. \[
   \int_{0}^{2} g\left( x \right)dx > 2M - \epsilon  
   \] 
 \end{tcolorbox}

   
 \subsection{Known material (self-study)}%
 \label{sub:known_material_self_study_}
 \begin{itemize}
   \item 1.7 :  Convergent sequenxes of numbers.
     \begin{itemize}
       \item Say $\left( x_{n} \right)_{n \in \mathbb{N} } $ sequenc of real/complex numbers. $\left( x_{n} \right)$ converges if $\exists$ some $x$ in $  \frac{\mathbb{R} }{\mathbb{C} } $ s.t. \[
       \forall \epsilon > 0 \exists N \in  \mathbb{N}  \quad \text{s.t.} n \ge N \quad  \implies  \|x_{n} - x \| < \epsilon   
       \] 
       We write $x_{n} \to  x$ , $\lim _{n \to  \infty} x_{n} = x$ , $\lim x_{n} = x$ 
     \item If $\left( x_{n} \right)$ sequence of real numbers, then we say that $\left( x_{n} \right) $ diverges to $\infty$ if \[
     \forall R > 0 \exists N> 0 \quad \text{s.t.} \quad  x_{n}> R \forall n >N  
     \] 
     We write $\lim_{n\to \infty} = \infty$ , $\lim x_{n} \to  \infty$ , $x_{n} \to  \infty$
     \end{itemize}
   \item 1.8: Infinite Series of numbers
     \begin{itemize}
       \item $\displaystyle \sum_{n= 1}^{\infty}  c_{n}$ series of real/complex numbers converges if the sequence of partial sums \[
       s_{n} = \sum_{n=1}^{N}  c_{n}
       \] 
       Converges as $ N \to  \infty$ . Say $ S_{N} \to  S$ . We then write $\sum_{i=1}^{\infty} c_{i} = s$ . 
     \item Recall if $\sum_{i=1}^{\infty}  c_{i} $ converges, then $\lim _{i \to  \infty} c_{i } = 0$
\item Recall if $\sum_{i=1}^{\infty}  c_{i} $ converges, then $\lim _{N\to \infty} \left( \sum_{i=N}^{} c_{i}  \right) = 0$ 
     \end{itemize}
\item  Concerning 1.9 $\to $ read it!
 \end{itemize}

 \newpage
 \section{Lecture 27. Aug}%
 \label{sec:lecture_27_aug}

 \subsection{VEctor spaces}%
 \label{sub:vector_spaces}

 Let $ V$ be a set such that the scalar field $F$: this (always) means  $F = \mathbb{R} $ or $F =\mathbb{C} $. 
 \begin{definition}
   A vecotr space over a scalar field $F$, is a set $V$ that satisfies the following conditions. 
   \begin{enumerate}
     \item Vector addition: Given any two $x,y \in V$, there is a unique element $x+y \in V$ , the \textbf{sum} of $x$ and $ya$.
       \todo{ Write sum in math mode }
     \item Scalar multiplication: Given $x \in V$ and a scalar $x \in F$, there is a unique element $cx \in V$, the \textbf{product}  of $x$ and $ x$ .
     \item Cummative property: $x+y = y +x \quad  \forall x,y \in  V $, 
     \item $\left( x+y \right) + z = x + \left( y+z \right) \quad  \forall \quad x,y,z \in  V  $
     \item Additive identity: $\exists$ an element $ 0 \in  V$ s.t. \[
     0 + x = x \quad  \forall x \in  V  \quad   
     \] 
   \item Additive inverse. $\forall x \in  V \exists$ an element $\left( -x \right) \in  V$ s.t. \[
   x+ \left( -x \right) = 0
   \] 
 \item  $\left( ab \right)x = a\left( bx \right)  \quad  \forall a,b \in F \quad  , x \in  V   $ associativity.
 \item Multidentity: Scalar multiplied by $1$ leaves element unchanged.
   \todo[inline]{ Does it exist cases where this is not satisfied? }
 \item $ c\left( x+y \right) = cx + xy \quad  \forall c \in  F , \quad    x,y \in  V $
 \item $\left( a + b \right)x = ax + bx \quad  \forall a,b  \in  F , \quad   x \in  V $

   \end{enumerate}
 \end{definition}
 
 \begin{remark}
   \textbf{Few notes aboth the definitions.} 
   \begin{itemize}
     \item if $F = \mathbb{R} $ : real vector space
     \item $ =  \mathbb{C} $: complex vectorspace.
     \item $F$ : the scalar field of the vector space. Elements of $F$ are scalars.
     \item Elements of $V$ are \textbf{vectors}.
     \item Vector space $=$ linear space. \[
     \begin{rcases}
       v_1, \ldots , v_{n}   &  \in     V \\
       c_1, \ldots  c_{n}  &  \in     F \\
       c_{1} v_{1} + \ldots + c_{n} v_{n}  &  \in    V
     \end{rcases}
     \] 
   \end{itemize}
 \end{remark}
 
 \begin{tcolorbox}
   \textbf{Example.}  
   \begin{enumerate}
     \item  $\left( \mathbb{R} , + , \cdot  \right)$
     \item $\mathbb{R} ^{n} = \left\{ \left( x_{1},  \ldots, x_{n} \right) : x_i \in  \mathbb{R}  \quad \forall i  \right\}$ with componentwise addition and scalar multiplication. \[
         \begin{split}
     \left( x_{1}, \ldots, x_{n} \right) + \left(  y_{1} , \ldots , y_{n}\right) &= \left( x_{1} + y_{1}, \ldots , x_{n} + y_{n} \right) \\
     c\left( x_{1}, \ldots, x_{n}\right) &=  \left( cx_{n} , \ldots, cx_{n}  \right) \\
         \end{split} 
     \] 
   \item Spaces of sequences \[
   S = \left\{ \left( x_{n} \right)_{n \in \mathbb{N} } : x_{i} \in  \mathbb{R}  \forall i \right\}
   \] 
   with componentwise addition and sclar multiplication. 
 \item Spaces of functions: Let 
   \[
   \mathcal{F} \left( \left[ 0,1 \right] \right) = \left\{ f: \left[ 0,1 \right] \to  \mathbb{F} : \quad f \text{ is a function on } \left[ 0,1 \right]  \right\}
   \] 
   and let \[
     \begin{split}
   \left( f+g \right)\left( t \right) &=  f\left( t \right) + g\left( t \right) , t \in  \left[ 0,1 \right] \\
   \left( xf \right)\left( t \right) &=  c f\left( t \right) \\
     \end{split} 
   \] 
   The zero vector $0$ is the zero function.
   
   \end{enumerate}
 \end{tcolorbox}

 \begin{definition}
   A subset $Y$ of $V$ is called a \textbf{subspace} of $V$ if is a vectorspace with the inherited addition and scalar multiplication. MOre precisely, iff (if and only if)
   \begin{enumerate}
     \item $y_{1} + y_{2} \in  Y$  for all $y _{1} , y_{2} \in  Y$
     \item $cy \in  Y$ for all $y \in  Y, \quad   c \in  \mathbb{F} $
   \end{enumerate}
 \end{definition}
   
   \textbf{Example.} Let $V = \mathbb{R} ^2$  
\begin{figure}[ht]
    \centering
    \incfig{rsquare}
    \caption{Rsquare}
    \label{fig:rsquare}
\end{figure}
Any line $L_{1}$ through the origin is a subspace. 

\newpara
Any line $L_{2}$ is not a subspace.


\newpara
\textbf{Example} 
Let  \[
C\left[ 0,1 \right] = \left\{ f: \left[ 0,1 \right] \to  \mathbb{F}  : \quad  f \text{ is continious on } \left[ 0,1 \right]  \right\}
\] 
This is a nonempty, proper subspace of $\mathcal{F}  \left( \left[ 0,1 \right] \right)$ \[
C\left[ 0,1 \right]\not \subseteq  \mathcal{F\left( \left[ 0,1 \right] \right)} 
\] 

\newpara
\textbf{Example} . Let $ I = \left( -1,1 \right)$ and \[
C^{1} \left( I \right) = \left\{ f: I \to  \mathbb{R}  : \quad f \quad \text{and} \quad f^{'} \quad \text{are continious functions on } I     \right\}
\] 
Then $C^{1} \left( I \right) \not \subseteq  C\left( I \right)$ .

\newpara
Proper: E.g. $F\left( t \right) = \left\lvert t \right\rvert  \in  C\left( I \right)$ but $f \not\in C^{1} \left( I \right)$. Likewise , \[
  \begin{split}
C^{2} \left( I \right)  & = \left\{ f: I \to  \mathbb{R}  : f , f^{'} \quad \text{and} \quad f^{''} \quad \text{are contionious on} \quad  I \quad , C^{2} \not \subseteq  C^{1} \left( I \right)      \right\} \\
  & \vdots \\
  C^{\infty}  & = \left( I \right) =  \left\{ f: I \to  \mathbb{R}  : f \quad  \text{is infinitely many times contionous differentiable}  \right\} , \quad \text{ex. } \quad  f\left( t \right) = e^{t}  
  \end{split} 
\] 
\[
  F\left( I \right)  \supset C\left( I \right) \supset C^{2} \left( I \right) \supset C^{\infty}\left( I \right) \supset  \mathcal{P}  \ldots 
\] 
Where \[
\mathcal{P}  = \left\{ \sum_{k=0}^{\infty}  c_{k} t^{k} : \quad  c_{k} \in  \mathbb{R}  , \quad  N \ge 0   \right\}
\] 
\subsection{Span and independence}%
\label{sub:span_and_independence}

  \textbf{Linear combination } is a vector space V \[
  v = \sum_{i=1}^{\infty}  c_{i} v_{i} = v_{1} c_{1} + \ldots +  c_{n} v_{n}
  \]  
  where $c_{1}, \ldots , c_{n} \in  \mathbb{F} $ and $v_{1}, \ldots, v_{n} \in  V$ 
\begin{definition}
  Let $A \subseteq  V$ be a nonemptu subset. the \textbf{finite linear span}  of $A$ is defined as \[
  span\left( A \right)  = \left\{ \sum_{i=1}^{N} c_{i} x_{i} \quad  N>0 , \quad  c_{i } \in  \mathbb{F}  , \quad x_{i } \in  A    \right\}
  \] 
  If $A = \emptyset $ then we declare $span\left( \emptyset  \right) = \left\{ 0 \right\}$  
  
  \newpara
  If $ A = \left\{ x_{1, \ldots, x_{n}}  \right\} $ is finite then \[
  span A = \left\{ c_{1}, + \ldots + c_{n} x_{n} : \quad  c_{1} \in  \mathbb{F}  \forall i  \right\}
  \] 
\end{definition}

\textbf{Example} . Consider the space $\mathcal{P} $ Let \[
\mathcal{M}  = \left\{ 1, t, t^{2} , \ldots  \right\} = \left\{ t^{n} \right\}_{n=0}^{\infty}
\] 
Then $span\left( \mathcal{M}  \right) = \mathcal{P} $ any $f \in  \mathcal{P}  $ is of the form $f = \sum_{n=0}^{N} c_{n} t^{n}$ for some $N>0$ and $c_{n} \in  \mathbb{F} $ .
\begin{definition}
  A nonempty subset $A$ of a vectorspace $V$ is \textbf{finetely linearly independent}  if  given any $N >0$ and any distinct elemnts $x_{1},  \ldots, x_{N} \in A$ and $c_{1} , \ldots, c_{n} \in  \mathbb{F} $, then \[
  c_{1} x_{1} + \ldots + c_{N} x_{N} \quad \leftrightarrow \quad c_{1} = \ldots = c_{N} = 0  
  \] 
  We declare $\emptyset $ to be linearly independent
\end{definition}

\begin{definition}
  Let $V$ be a nontrivial vectorspace (not containing only zero). Then a set of vectors $\mathcal{B}  \subset V$ is a \textbf{hamel basis} for $V$ if
  \begin{enumerate}
    \item $\mathcal{B} $ is linearly independent.
    \item $span\left( \mathcal{B}  \right) = V$
  \end{enumerate}
\end{definition}
 \begin{remark}
   Two hamel bases for the same space $V$ must have the same cardinality.
 \end{remark}


\newpage
\section{References}%
\label{sec:references}


\bibliographystyle{plain}
\bibliography{references}
\end{document}

